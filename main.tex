%%%%%%%%%%%%%%%%%%%%%%%%%%%%%%%%%%%%%%%%%
% Doctoral Thesis 
% LaTeX Template
% Version 2.5 (27/8/17)
%
% This template was downloaded from:
% http://www.LaTeXTemplates.com
%
% Version 2.x major modifications by:
% Vel (vel@latextemplates.com)
%
% This template is based on a template by:
% Steve Gunn (http://users.ecs.soton.ac.uk/srg/softwaretools/document/templates/)
% Sunil Patel (http://www.sunilpatel.co.uk/thesis-template/)
%
% Template license:
% CC BY-NC-SA 3.0 (http://creativecommons.org/licenses/by-nc-sa/3.0/)
%
%%%%%%%%%%%%%%%%%%%%%%%%%%%%%%%%%%%%%%%%%

%----------------------------------------------------------------------------------------
%	PACKAGES AND OTHER DOCUMENT CONFIGURATIONS
%----------------------------------------------------------------------------------------
%\PassOptionsToPackage{english,greek}{babel}
\documentclass[
11pt, % The default document font size, options: 10pt, 11pt, 12pt
%oneside, % Two side (alternating margins) for binding by default, uncomment to switch to one side
english, % ngerman for German
singlespacing, % Single line spacing, alternatives: onehalfspacing or doublespacing
%draft, % Uncomment to enable draft mode (no pictures, no links, overfull hboxes indicated)
%nolistspacing, % If the document is onehalfspacing or doublespacing, uncomment this to set spacing in lists to single
liststotoc, % Uncomment to add the list of figures/tables/etc to the table of contents
toctotoc, % Uncomment to add the main table of contents to the table of contents
%parskip, % Uncomment to add space between paragraphs
%nohyperref, % Uncomment to not load the hyperref package
headsepline, % Uncomment to get a line under the header
%chapterinoneline, % Uncomment to place the chapter title next to the number on one line
%consistentlayout, % Uncomment to change the layout of the declaration, abstract and acknowledgements pages to match the default layout
]{MastersDoctoralThesis} % The class file specifying the document structure

\usepackage[utf8]{inputenc} % Required for inputting international characters
\usepackage[T1]{fontenc} % Output font encoding for international characters
%\usepackage{textgreek}
%\usepackage[main=english,greek]{babel}

\usepackage{mathpazo} % Use the Palatino font by default

\usepackage[
backend=biber,
style=authoryear,
sorting=nyt]{biblatex} % Use the bibtex backend with the authoryear citation style (which resembles APA)

\addbibresource{references.bib} % The filename of the bibliography
% try to mention my works
\DeclareSourcemap{
  \maps[datatype=bibtex]{
    \map{
      \step[fieldsource=author,
            match=Paragkamian,
            final]
      \step[fieldset=keywords, fieldvalue=own]
    }
  }
}


\usepackage[autostyle=true]{csquotes} % Required to generate language-dependent quotes in the bibliography

% Here you can load other packages or provide your own definitions
% ----------------------------------------------------------------
% Finally, hyperref is used for PDF files.
\usepackage[pdfusetitle, colorlinks, citecolor, plainpages=false]{hyperref}
\usepackage{xcolor}
%----------------------------------------------------------------------------------------
% Colors
%----------------------------------------------------------------------------------------
\hypersetup{
linkcolor=teal
,citecolor=Green
}

% tables
\usepackage{graphicx}
% To have multirows in tables
\usepackage{multirow}


% To have rotated tables
\usepackage{rotating} 
\usepackage{float}

\usepackage{changepage}
% To have check boxes
\usepackage{pifont}
\usepackage{amsmath}
\usepackage{multirow}
% For lemas
\usepackage{amsthm}
\usepackage{amssymb}

% To have todo notes
%\usepackage{todonotes}

% To have part A, B etc in a figure
\usepackage{subcaption}
\usepackage{caption}

% to have multiline equations
\usepackage{mathtools}

%----------------------------------------------------------------------------------------
%	MARGIN SETTINGS
%----------------------------------------------------------------------------------------

\geometry{
	paper=a4paper, % Change to letterpaper for US letter
	inner=2.5cm, % Inner margin
	outer=3.8cm, % Outer margin
	bindingoffset=.5cm, % Binding offset
	top=1.5cm, % Top margin
	bottom=1.5cm, % Bottom margin
	%showframe, % Uncomment to show how the type block is set on the page
}

%----------------------------------------------------------------------------------------
%	THESIS INFORMATION
%----------------------------------------------------------------------------------------

\thesistitle{Deciphering the relation of the microbiome interactome with ecosystem function and biogeochemical processes} % Your thesis title, this is used in the title and abstract, print it elsewhere with \ttitle
\supervisor{Prof. Panagiotis F. \textsc{Sarris}} % Your supervisor's name, this is used in the title page, print it elsewhere with \supname
\examiner{} % Your examiner's name, this is not currently used anywhere in the template, print it elsewhere with \examname
\degree{Doctor of Philosophy} % Your degree name, this is used in the title page and abstract, print it elsewhere with \degreename
\author{Savvas \textsc{Paragkamian}} % Your name, this is used in the title page and abstract, print it elsewhere with \authorname
\addresses{} % Your address, this is not currently used anywhere in the template, print it elsewhere with \addressname

\subject{Microbial Ecology} % Your subject area, this is not currently used anywhere in the template, print it elsewhere with \subjectname
\keywords{ecosystem functioning, macroecology, microbial networks, conservation} % Keywords for your thesis, this is not currently used anywhere in the template, print it elsewhere with \keywordnames
\university{\href{https://www.uoc.gr}{University of Crete}} % Your university's name and URL, this is used in the title page and abstract, print it elsewhere with \univname
\department{\href{https://www.biology.uoc.gr}{University of Crete, Department of Biology}} % Your department's name and URL, this is used in the title page and abstract, print it elsewhere with \deptname
\group{\href{https://imbbc.hcmr.gr}{Institute of Marine Biology, Biotechnology and Aquaculture (IMBBC) - HCMR}} % Your research group's name and URL, this is used in the title page, print it elsewhere with \groupname
\faculty{\href{https://www.biology.uoc.gr/en/content/faculty-members}{Faculty, Department of Biology}} % Your faculty's name and URL, this is used in the title page and abstract, print it elsewhere with \facname

\AtBeginDocument{
\hypersetup{pdftitle=\ttitle} % Set the PDF's title to your title
\hypersetup{pdfauthor=\authorname} % Set the PDF's author to your name
\hypersetup{pdfkeywords=\keywordnames} % Set the PDF's keywords to your keywords
}

\begin{document}

\frontmatter % Use roman page numbering style (i, ii, iii, iv...) for the pre-content pages

\pagestyle{plain} % Default to the plain heading style until the thesis style is called for the body content
\hypersetup{linkcolor=teal}
%----------------------------------------------------------------------------------------
%	TITLE PAGE
%----------------------------------------------------------------------------------------

\begin{titlepage}
\begin{center}
\begin{minipage}{4cm}
\begin{flushleft}
    \raggedleft
\includegraphics{figures/uoc-logo-1.jpg} % University/department logo - uncomment to place it
\end{flushleft}
\end{minipage}
\begin{minipage}{6cm}
\begin{flushright}
%\vspace{1cm}
\LARGE \univname
%\vspace{1.0cm} % University name
\end{flushright}
\end{minipage}

\vspace{1cm}
\textsc{\Large Doctoral Thesis}\\[0.5cm] % Thesis type

\HRule \\[0.4cm] % Horizontal line
{\huge \bfseries \ttitle\par}\vspace{0.4cm} % Thesis title
\HRule \\[1.0cm] % Horizontal line
 
\begin{minipage}[t]{0.4\textwidth}
\begin{flushleft} \large
\emph{Author:}\\
\href{http://www.johnsmith.com}{\authorname} % Author name - remove the \href bracket to remove the link
\end{flushleft}
\end{minipage}
\begin{minipage}[t]{0.5\textwidth}
\begin{flushright} \large
\emph{Doctoral Advisory Committee:} \\
\href{https://www.imbb.forth.gr/imbb-people/en/sarris-members/item/2695-dr-panagiotis-f-sarris}{Prof. Panagiotis F. \textsc{Sarris}}\\ % Supervisor name - remove the \href bracket to remove the link  
\href{http://lab42open.hcmr.gr/people/evangelospafilis/}{Dr Evangelos \textsc{Pafilis}}\\ % Supervisor name - remove the \href bracket to remove the link  
\href{https://users.auth.gr/~iantonio/MEMBERSAntoniou.html}{Prof. Ioannis \textsc{Antoniou}}\\ % Supervisor name - remove the \href bracket to remove the link  
\emph{Examination Committee:} \\
\href{https://dinalika.weebly.com}{Prof. Dina \textsc{Lika}}\\ % Supervisor name - remove the \href bracket to remove the link  
\href{https://www.biology.uoc.gr/index.php/en/personnel/faculty-members?view=article&id=246:ladoukakis-emmanuel&catid=27:dep-en-gb}{Prof. Emmanuel \textsc{Ladoukakis}}\\ % Supervisor name - remove the \href bracket to remove the link  
\href{http://pop-gen.eu/wordpress/people}{Prof. Pavlos \textsc{Pavlidis}}\\ % Supervisor name - remove the \href bracket to remove the link  
\href{https://www.medschool.umaryland.edu/profiles/holm-johanna/}{Prof. Johanna \textsc{Holm}}\\ % Supervisor name - remove the \href bracket to remove the link  
\end{flushright}
\end{minipage}\\[1cm]
 

\large \textit{A thesis submitted in fulfillment of the requirements\\ for the degree of \degreename}\\[0.3cm] % University requirement text
\textit{in the}\\[0.4cm]
\deptname\\[0.4cm] 

\textit{and the}\\[0.4cm]
\groupname\\[0.5cm] % Research group name and department name
 
%%
\vspace{1cm}
{\large \today}\\[4cm] % Date
 
\end{center}
\end{titlepage}

%----------------------------------------------------------------------------------------
%	DECLARATION PAGE
%----------------------------------------------------------------------------------------

\begin{declaration}
\addchaptertocentry{\authorshipname} % Add the declaration to the table of contents
\noindent I, \authorname, declare that this thesis titled, \enquote{\ttitle} and the work presented in it are my own. The work of three chapters is 
published in peer reviewed journals with impact factor and I am the first author with shared co-authoriship, see \hyperref[app:publications]{Publications}. I confirm that:

\begin{itemize} 
    \item This work was done wholly or mainly while in candidature for a research degree at this University.
    \item No part of this thesis has previously been submitted for a degree or any other qualification at this University or any other institution.
    \item Where I have consulted the published work of others, this is always clearly attributed.
    \item Where I have quoted from the work of others, the source is always given. With the exception of such quotations, this thesis is entirely my own work.
    \item I have acknowledged all main sources of help.
    \item Where the thesis is based on work done by myself jointly with others, I have made clear exactly what was done by others and what I have contributed myself.
    \item This thesis has scored 20\% similarity score using Turnitin account of University of Crete on 26th of June 2024 using the settings "exclude bibliography, exclude quoted material".
\end{itemize}
% 
\noindent Signed:\\
\rule[0.5em]{25em}{0.5pt} % This prints a line for the signature
 
\noindent Date:\\
\rule[0.5em]{25em}{0.5pt} % This prints a line to write the date
\end{declaration}
%
%\cleardoublepage

%----------------------------------------------------------------------------------------
%	QUOTATION PAGE
%----------------------------------------------------------------------------------------

%\vspace*{0.2\textheight}

%\noindent\enquote{\itshape Thanks to my solid academic training, today I can write hundreds of words on virtually any topic without possessing a shred of information, which is how I got a good job in journalism.}\bigbreak

%\hfill Dave Barry

%----------------------------------------------------------------------------------------
%	ABSTRACT PAGE
%----------------------------------------------------------------------------------------

\begin{abstract}
\addchaptertocentry{\abstractname} % Add the abstract to the table of contents

To comprehend ecosystem functioning, it's imperative to discern
the processes occurring in various environments (where) and the organisms
responsible for them (who). Ecosystems form complex associations of taxa,
abiotic parameters and physical characteristics of the materials. For example,
the soil ecosystem holds 10 billion cells in each gram. The texture and the 
chemistry of the soil affects and is affected by the existing taxa across the
tree of life, plants, arthropods, bacteria, fungi etc. Regarding microbes and
metagenomics there is a wealth of data in open omics databases and in the 
literature. Yet, all this information is inhomogeneous and spread in separate 
resources. PREGO, a comprehensive knowledge base, amalgamates
text mining and data integration techniques to extract these what-where-who
associations from scattered scientific literature and omics repositories. It
identifies microorganisms, biological processes, and environmental types,
mapping them to ontology terms. Through text and metagenomics data analysis,
PREGO extracts associations, assigning confidence levels via a scoring scheme.
With 364,508 microbial taxa, 1090 environmental types, 15,091 biological processes,
and 7,971 molecular functions, PREGO aims to aid researchers in experiment
design and interpretation. Additionally, it facilitates exploration of
environment-process-microbe associations.

Furthermore, to understand the current status of ecosystems is important to 
know their past conditions. Historical biodiversity documents are crucial for
long-term data cycles but pose challenges in data curation due to their historical
context. The data rescue process involves document digitization, transcription, information extraction
using text mining tools, and publication to standardized formats. Information
Extraction (IE) tools, evolved over the years, recognize entities in text, aiding in curation.
To this end, this work expanded on IE from a marine historical biodiversity perspective,
orchestrating tools to provide a unified methodology. The classification of
tools enables curators to choose based on their needs. A new tool, DECO, is
introduced, aiming to enhance the data rescue process in biodiversity research.

Studying ecosystems past and current conditions has made apparent the need for 
conservation actions because there are under multiple threats.
An example of ecosystem collapse is the arthropod decline, a globally documented
trend that remains insufficiently handled by the society. In
Crete, a biodiversity hotspot, research on its arthropod fauna dates back centuries.
Here we investigate the conservation status of the Cretan Arthropods using
Preliminary Automated Conservation Assessments (PACA) and the overlap of Cretan
Arthropod distributions with the Natura 2000 protected areas. Moreover we
investigate their endemicity hotspots and propose candidate Key Biodiversity Areas.
In order to perform these analyses, we assembled occurrences of the endemic
Arthropods in Crete located in the collections of the Natural History Museum of
Crete together with literature data. Conservation status assessments using PACA
reveal potential threats to 75\% of endemic arthropods.
Endemic taxa hotspots and candidate Key Biodiversity Areas are mainly in
mountainous regions, often overlapping with Natura 2000 protected areas.
However, human activities impact these areas, and some taxa lack sufficient
protection.

Apart from plants and fauna, microbes significantly influence soil ecosystem
functioning, yet synthesizing microbial biodiversity data remains challenging.
Island biogeography presents a unique opportunity, to advance the understanding
of microbiome diversity in the soil environment.
Crete, as a continental island, has a distinct natural and evolutionary history
with extreme contrasts in vegetation cover, climatic conditions and geology. 
It has been studied for centuries and because of its location in southeastern
Mediterranean is important for climate change studies.
To this end, the first island-wide soil microbiome study conducted on the
island of Crete in 2016, assessed the microbiome diversity drivers and determined that
Crete microbial diversity is driven by pH and soil moisture along elevation
gradient. Together, a team of researchers and citizen scientists collected 435 soil samples
from 72 sites across four distinct ecozones in a single day. The Island Sampling Day Crete 2016
project, further integrates microbial data with soil, spatial and remote sensing data,
to decipher ecosystem function drivers on the island. High-altitude areas harbor diverse
microorganisms, mirroring patterns seen in other faunal groups like arthropods.
Spatial data integration identifies warning signals in pristine and grazing
ecosystems, aiding in identifying biodiversity drivers and evaluating ecosystem threats.

In summary, studying ecosystems ecology with contemporary tools and data poses unique 
opportunities. Cumulative work across the centuries is becoming integrated to the 
digital and holistic representation of ecosystems. Multiple steps are still required 
to this realisation but there is a ground opportunity to unlock novel insights 
through the integration. Conservation and basic research are occurring simultaneously 
because the window of action in the face of ecosystem services collapse is narrow.

%\textgreek{
%Για να κατανοήσουμε τη λειτουργία των οικοσυστημάτων, είναι ανάγκη να διακρίνουμε
%τις διεργασίες που συμβαίνουν σε διάφορα περιβάλλοντα και τους οργανισμούς
%που τις επιτελούν. Τα οικοσυστήματα αποτελούνται από πολύπλοκες αλληλεπιδράσεις τάξων,
%αβιοτικών παραμέτρων και τα φυσικά χαρακτηριστικά των υλικών. Για παράδειγμα,
%το οικοσύστημα του εδάφους περιέχει 10 δισεκατομμύρια κύτταρα σε κάθε γραμμάριο. Η υφή και η
%η χημεία του εδάφους επηρεάζει και επηρεάζεται από τα υπάρχοντα τάξα από όλο το
%δέντρο της ζωής, φυτά, αρθρόποδα, βακτήρια, μύκητες κλπ. Όσον αφορά τα μικρόβια
%υπάρχει πληθώρα δεδομένων σε ανοιχτές βάσεις δεδομένων και στη
%βιβλιογραφία. Ωστόσο, όλες αυτές οι πληροφορίες είναι ανομοιογενείς και αποθηκευμένες σε 
%βάσεις δεδομένων ως σιλό. Το PREGO, αποτελεί μια συγκεντρωτική βάση γνώσης. Μέσω
%τεχνικών εξόρυξης κειμένου και ενσωμάτωσης δεδομένων συγχωνεύει 
%συσχετίσεις οργανισμών, περιβάλλοντος και λειτουργιών από
%από την επιστημονική βιβλιογραφία και βάσεις δεδομένων. 
%Οι μικροοργανισμοί, βιολογικές διεργασίες και περιβαλλοντικοί τύποι αντιστοιχούνται
%με όρους οντολογίας και κάθε συσχέτιση κατηγοριοποιείται με ένα σύστημα αξιολόγησης 
%βαθμών εμπιστοσύνης.
%Με 364.508 μικροβιακά τάξα, 1090 περιβαλλοντικούς τύπους, 15.091 βιολογικές διεργασίες,
%και 7.971 μοριακές λειτουργίες, το PREGO στοχεύει να βοηθήσει τους ερευνητές στο 
%σχεδιασμό και ερμηνεία πειραμάτων αλλά και την διατύπωση νέων υποθέσων εργασίας. Επιπλέον, διευκολύνει την εξερεύνηση
%συσχετίσεις περιβάλλοντος-διαδικασιών-μικροβίων.
%
%Είναι σημαντικό να κατανοήθεί η τρέχουσα κατάσταση των οικοσυστημάτων. Σε αυτό είναι
%απαραίτητη η ιστορική γνώση που υπάρχει από προηγούμενες μελέτες.
%Τα ιστορικά έγγραφα βιοποικιλότητας είναι ζωτικής σημασίας για
%μακροπρόθεσμες αναλύσεις. Όμως η διάσωση και η επιμέλεια δεδομένων ιστορικών δεδομένων είναι δύσκολη λόγω του ιστορικού τους
%πλαισίου. Η διαδικασία διάσωσης δεδομένων περιλαμβάνει ψηφιοποίηση εγγράφων, μεταγραφή, εξαγωγή πληροφοριών
%χρησιμοποιώντας εργαλεία εξόρυξης κειμένου και τη μετέπειτα δημοσίευση σε βάσεις δεδομένων.
%Τα εργαλεία εξόρυξης γνώσης, που έχουν εξελιχθεί ραγδαία τα τελευταία χρόνια,
%αναγνωρίζουν οντότητες στο κείμενο, βοηθώντας στην επιμέλεια.
%Για το σκοπό αυτό, αυτή η εργασία εστίασε στην διάσωση γνώσης θαλάσσιας ιστορικής βιοποικιλότητας και την
%αξιολόγηση των εργαλείων για τον σχεδιασμό μιας ενοποιημένης μεθοδολογίας επιμέλειας. 
%Πηγαίνοντας ένα βήμα παραπέρα, ένα νέο εργαλείο, το DECO, αναπτύχθηκε
%με στόχο την ενίσχυση της διαδικασίας διάσωσης δεδομένων στην έρευνα για τη βιοποικιλότητα.
%
%Η μελέτη των οικοσυστημάτων του παρελθόντος και των σημερινών συνθηκών έχει κάνει εμφανή την ανάγκη για
%δράσεις διατήρησης επειδή βρίσκονται υπό πολλαπλές απειλές.
%Ένα παράδειγμα είναι η κατάρρευση των αρθροπόδων, μια παγκόσμια τεκμηριωμένη
%τάση που έχει αντιμετωπιστεί ανεπαρκώς από την κοινωνία.
%Η Κρήτη είναι ένα hotspot βιοποικιλότητας και η έρευνα για την πανίδα των αρθροπόδων της χρονολογείται από τον 19ο αιώνα.
%Στην παρούσα εργασία αξιολογήθηκε η κατάσταση της διατήρησης των ενδημικών αρθρόποδων της Κρήτης χρησιμοποιώντας
%μια αυτοματοποιημένη αξιολόγηση (PACA) και η επικάλυψη των κατανομών των αρθρόποδων
%στις προστατευόμενες περιοχές Natura 2000. Επιπλέον 
%να διερευνήθηκαν τα hotspot ενδημικότητάς τους και προτάθηκαν υποψήφιες Βασικές Περιοχές Βιοποικιλότητας.
%Για αυτές τις αναλύσεις, συγκεντρώθηκαν παρουσίες του ενδημικών αρθροπόδων της Κρήτης
%που βρίσκονται στις συλλογές του Μουσείου Φυσικής Ιστορίας της
%Κρήτη μαζί με στοιχεία από την εκτενή βιβλιογραφία. Η αξιολόγηση της κατάστασης διατήρησης με την μέθοδο PACA
%κατέταξε το 75\% των ενδημικών αρθροπόδων ως πιθανά απειλούμενα.
%Τα ενδημικά hotspots και οι υποψήφιες βασικές περιοχές βιοποικιλότητας βρίσκονται κυρίως σε
%ορεινές περιοχές, που συχνά επικαλύπτονται με προστατευόμενες περιοχές Natura 2000.
%Ωστόσο, οι ανθρώπινες δραστηριότητες επηρεάζουν αυτές τις περιοχές και ορισμένα αρθρόποδα δεν 
%προστατεύονται επαρκώς.
%
%Εκτός από την χλωρίδα και την πανίδα, τα μικρόβια επηρεάζουν σημαντικά το οικοσύστημα του εδάφους.
%Όμως η αποσαφήνιση των λειτουργιών της μικροβιακής βιοποικιλότητας παραμένει πρόκληση.
%Η νησιωτική βιογεωγραφία προσφέρει μια μοναδική ευκαιρία, για να προωθηθεί η κατανόηση
%της ποικιλότητας των μικροβιωμάτων στο εδαφικό περιβάλλον.
%Η Κρήτη, ως ηπειρωτικό νησί, έχει μια ξεχωριστή φυσική και εξελικτική ιστορία
%με ακραίες αντιθέσεις στη βλάστηση, τις κλιματικές συνθήκες και τη γεωλογία.
%Έχει μελετηθεί για αιώνες και λόγω της θέσης του στα νοτιοανατολικά
%της Μεσογείου είναι σημαντική για τις μελέτες για την κλιματική αλλαγή.
%Για το σκοπό αυτό, η πρώτη ολοκληρωμένη μελέτη μικροβιώματος εδάφους σε νησί διεξήχθη στο
%νησί της Κρήτης το 2016. Αξιολογήθηκαν οι δείκτες ποικιλότητας του μικροβιώματος και ως αποτέλεσμα
%φαίνεται να είναι ότι η μικροβιακή ποικιλότητα της Κρήτης επηρεάζεται από το pH και
%την υγρασία του εδάφους, σε σχέση και με το υψόμετρο. Αυτή η έρευνα συντονίστηκε από μια ομάδα ερευνητών η οποία συνέλεξε 435 δείγματα εδάφους
%από 72 τοποθεσίες σε τέσσερις διακριτές οικολογικές ζώνες της Κρήτης σε μια μέρα. The Island Sampling Day Κρήτη 2016
%ενσωματώνει επίσης, μικροβιακά δεδομένα με δεδομένα εδάφους, χωρικά και τηλεπισκόπησης.
%Βασικός στόχος είναι η ψηφιακή αναπαράσταση του εδάφους της Κρήτης για να αποκρυπτογραφήθούν
%οι λειτουργίες των ποικίλων οικοσυστημάτων του νησιού. Οι περιοχές με μεγάλο υψόμετρο φιλοξενούν μεγάλη βιοποικιλότητα
%μικροοργανισμών, αντικατοπτρίζοντας μοτίβα που παρατηρούνται σε άλλες ομάδες πανίδας όπως τα αρθρόποδα.
%Η ενοποίηση χωρικών δεδομένων ανέδειξε πιθανά σήματα αρνητικής ανθρωπογενούς επίδρασης σε παρθένα οικοσυστήματα αλλά και 
%βοσκοτόπια βοηθώντας στον εντοπισμό των παραγόντων της βιοποικιλότητας και στην αξιολόγηση των απειλών.
%
%Συνοψίζοντας, η μελέτη της οικολογίας των οικοσυστημάτων με σύγχρονα εργαλεία και δεδομένα
%έχει σημαντικές προοπτικές. Η συσσώρευση δεδομένων καταστεί εφικτή την
%ψηφιακή και ολιστική αναπαράσταση των οικοσυστημάτων. Απαιτούνται ακόμη πολλά βήματα
%για την υλοποίηση αυτή όμως υπάρχουν ευκαιρίες για να νέα ερωτήματα και νέες ιδέες
%μέσω της ενσωμάτωσης. Η προστασία, η διατήρηση και η βασική έρευνα πλεόν χρειάζεται να 
%γίνονται ταυτόχρονα επειδή τα περιθώρια δράσης ενόψει της κατάρρευσης των
%οικοσυστημάτων είναι στενά.
%}

\end{abstract}

\textbf{Keywords}

\keywordnames

%----------------------------------------------------------------------------------------
%	ACKNOWLEDGEMENTS
%----------------------------------------------------------------------------------------

\begin{acknowledgements}
\addchaptertocentry{\acknowledgementname} % Add the acknowledgements to the table of contents


First, I am truly thankful to Dr Kotoulas for introducing me to his HCMR colleagues and
made the connection to begin my PhD.
His true interest in deep scientific and philosophical questions is uplifting. He is constantly
joining people together around the globe while valuing most the creative working environment,
the new generations of scientists. Working with openness and without borders is an
important lesson he taught me that made possible this PhD.

I want to thank Dr Pafilis for having me in the PREGO project team, the Island Sampling
Day and teaching me how to make the most out
of previous work and implementing big projects in small steps, one step at a time.

Many thanks go to Prof Panos Sarris for sharing his academic values and excellence and helping me in making
decisions. His deep knowledge in molecular microbiology and thirst for debate inspire for
a collaborating and creative environment.

Prof Ioannis Antoniou thank you for being a concise and sincere mentor. His amazing ability to focus
on the important questions and hold on to information that's relevant, removing the clutter, 
is remarkable. Really looking forward to being able to collaborating in ecological 
projects through his thermodynamics, network and dynamics expertise!

I thank the super EMODnet team of HCMR, Dr Gerobasileiou, Dr Arvanitidis, Ms Mavraki and
Ms Georgia Sarafidou. What a great project and nice collaboration about Historical 
biodiversity literature.

Special thanks go to the Arthropods Lab of the Natural History Museum, to
Dr Apostolos Trichas and Ms Giannis Bolanakis for the long hours working and discussing together.
Thank you for contacting and sharing your 
precious data and exchanging movies, music and great scientific texts! Working with you is a joy!

My gratitude and respect go to all the people of Island Sampling Day project and especially
Lynn Schriml and Johanna Holm. I have learned and continue to learn so much from 
you. Working together in a hobby project and giving our care means a lot to me and your
generosity is so humbling.

Haris Zafeiropoulos, my friend, I look forward to many collaborations in the future. Seeing you code and work
is inspiring! Most of all, feeling your positive spirit in people has made me more
open that I thought I could be, life is more fun this way!

This is an acknowledgement section for the people that we worked together. Can't 
thank enough my family and friends in this section, yet I want to share my deepest love.

\end{acknowledgements}

%----------------------------------------------------------------------------------------
%	PREFACE
%----------------------------------------------------------------------------------------

\addcontentsline{toc}{chapter}{Preface}
\chapter*{Preface}

Being an bioinformatician means spending a most of my time in the office. This is a 
stark contrast in what I do in other aspects of my life. Canyoning, Climbing, Caving 
outdoor activities in heights using ropes is truly what brings me peace. Yet the 
amazing questions that come from ecology and the origin of life require theoretical 
knowledge, mathematics and programming. So one of a kind of important realisations
during my PhD was that I felt sheer joy when combining these two extremes. This was
of course during the microbial sampling of Crete and of the deepest cave of Greece, Gourgouthakas
and in capturing raptor birds in the islands of the Aegean sea. Bringing data from 
extreme environments and then, after the incredible bench work of colleagues, analyse 
these data made my heart sing.

Another aspect of being a bioinformatician is working on different projects with 
multiple people. This realisation was a revelation to me and this PhD is the outcome 
of deep collaboration with four different groups of people and many new friendships. 

The chapters of this PhD cover many disciplines, similar to most bioinformatic PhDs,
yet they all converge to integrating multiple types of data to investigate the multifaceted
and complex nature and functioning of ecosystems. The main focus is the soil ecosystems and 
their microbiome, with the island of Crete being the field of application.

The chapters are :

\begin{enumerate}
    \item Island Sampling Day Crete project in Chapter \ref{cha:isd-crete-soil}
    \item PREGO project in Chapter \ref{cha:prego}
    \item DECO workflow in Chapter \ref{cha:deco}
    \item Endemic arthropods of Crete island in Chapter \ref{cha:arthropods}
    \item The data integration and the soil microbiome diversity in Chapter \ref{cha:crete-idea} and \ref{cha:crete-soil}
\end{enumerate}

The Chapter \ref{cha:isd-crete-soil} is about the Island Sampling Day (ISD), a large scale study for the microbial diversity of the
topsoil of Crete. It the first large scale microbiome sampling in Crete and I analysed
all this wealth of data to describe the hidden biodiversity. ISD is a 16s rRNA amplicon study designed to put standards into action. 
ISD is a citizen science standardisation effort, voluntarily designed and
operated by the participants of the 18th Genome Standards Consortium
Workshop and the Hellenic Centre for Marine Research in June 2016.
This project adds a new chapter to the study of biodiversity in Crete,
the soil microbial diversity. But these data and metadata alone are 
insufficient to understand ecosystem function and even the drivers 
of microbial community structures. How to enrich these valuable data 
with existing available information? 

Regarding the microbes there a wealth of information hidden in text and metagenomic resources.
To address this issue, I co-developed PREGO, a global data and text mining metagenomic resourse, designed to assist microbiologists
to enrich their knowledge about microbes, the processes they perform and the environments 
they appear. We developed PREGO's knowledge base with both a web interface and an Application Programming Interface.
PREGO is a meta platform, which is based upon open data and standards of metagenomics. 
It brings the wealth of taxonomic, environmental and biological functions of 
microbial data and literature all in one place. Multiple platforms are harvested and the data 
and metadata are harmonised using text mining. In addition, academic texts contain 
valuable information that is missing from databases and text mining solutions unearth 
this information. Hence, the PREGO approach allowed to bring the heterogeneous
data together in a single knowledge base. 

Yet, another biodiversity pillar of the Cretan ecosystems is arthropods which 
interact with microbiomes. Based on a novel dataset
that contained historical data and specimens of the Natural History Museum of Crete
we assessed the conservation status of the endemic arthropods of Crete. Doing a assessment of 
conservation of the endemic arthropods was an obvious priority because to have the ability to 
study them first the have to be conserved from their documented collapse. Hence, this work contained 
work from the 19th century along with the 4 decades long samplings of the NHMC. 
The overlap of Cretan Arthropod distributions with the Natura 2000 protected areas
were investigated along with their endemicity hotspots and proposed candidate Key Biodiversity Areas.
This work is serves as an example to further highlight the importance of historical data along with contemporary.

Upon realising the struggles and the benefits of rescuing historical biodiversity data, I
developed a workflow to accelerate the process. Because, apart from the contemporary data and metagenomics, important biodiversity data is 
included in historical biodiversity documents. This information from 
previous centuries reports, books and expeditions notes is an important glimpse into
the past. These data due to their legacy format and other complexities, for example 
multiple languages, different taxa names and approximate coordinates, require 
enormous effort to rescue in databases. Curators are indispensable in this process 
yet there are multiple tools that assist them. Using text mining and other tools 
for Object Character Recognition we demonstrated a workflow to rescue such documents.
This workflow is named DECO and is a software container that can handle multiple 
documents and also combines multiple tools for the steps of the data rescue process.

How to bring all this information from literature, biodiversity and maps in one place? Focusing on Crete
provided a unique opportunity to use it as a model of the digital representation of ecosystems. 
As an island has a confined space and also has been studied for thousands of years.
Being able to collect all the these data in one place allows to perform analyses on the soil ecosystems of Crete.
Hence, the chapter \ref{cha:crete-idea} is about the total environment of Crete.
In this chapter I address the questions of 
\begin{itemize}
    \item How many documents are containing valuable information?
    \item What kind of samplings have been carried out?
    \item Which spatial data about Crete are available? 
\end{itemize}

Bringing all this information together poses unique challenges of different types of data, standards and code. 
But with the establishment of this digital description of the open knowledge about Crete allows the generation of novel questions and hypotheses.
Additionally, a literature and biodiversity data summary is 
provided to set the basis for future integrative studies and for the documentation of the current status of 
research regarding the island.
The soil microbiome is core of the soil ecosystems, and using the aforementioned data I build the Crete Data Cube.
In the final chapter, \ref{cha:crete-soil} of this PhD, I integrate the soil biodiversity, it's functional potential and 
the Crete Data Cube to decipher the soil ecosystems of concern, vulnerable soils and one health related taxa. The results 
of this analysis provide a new perspective of some well known ecosystems of Crete.
But also sets an example of the importance of integration of prior knowledge and data to address current challenges.

Academic writing and publishing is important, albeit difficult, and I have learned a
lot during the PhD. The first 3 chapters are accepted/published in journals as
mentioned in each chapter. The last chapter will be 2 back to back submissions.
There is so much production of papers and journals that is difficult to decide where to publish. 
So except from the common flagship journals I discovered the work that shaped my 
PhD in themed issues of Philosophical Transactions of the Royal Society B.
A journal that really stands out and it was established on 1887! This was an
important realisation during the PhD that there are journals launched only 10 years
ago with incredible high impact factor but there are journals in ecology
with rich history and groundbreaking articles.

Most notably:

\begin{itemize}
    \item Conceptual challenges in microbial community ecology \href{https://doi.org/10.1098/rstb.2019.0241}{Volume 375, Issue 1798}
    \item Ecological complexity and the biosphere: the next 30 years \href{https://doi.org/10.1098/rstb.2021.0376}{Volume 377, Issue 1857}
    \item Integrative research perspectives on marine conservation \href{https://doi.org/10.1098/rstb.2019.0444}{Volume 375, Issue 1814}
\end{itemize}

Having these issues as guides covered all the different fields that I am interested 
in working on. Microbial ecology is a great field, here I focused on soil mostly, that
combines all aspects of numerical ecology. There are many things that scientists 
haven't deciphered yet, like how the above ground biodiversity influences the below
ground biodiversity. Ecological complexity is what inspired me to continue to a 
master's degree and I still am looking forward to apply and learn the amazing 
theoretical mathematical models. And lastly is the integration of multiple methods 
and data to describe an ecosystem. This is what I worked mostly on my PhD and 
harvesting different types of data, i.e sequences, spatial data, occurrence data, 
text etc, has it's unique challenges. Each type of data requires different tools and has 
different objects to handle them. Furthermore, types of data come from different 
fields with different scientific communities and standards. Incorporating and even 
discovering these data is time consuming and mind bending. Yet, utilising diverse 
published data to analyse an ecosystem can lead to novel insights. 

The projects that are presented in the following chapters are about 
bringing together contrasts. Starting with biodiversity, we investigated taxa that 
have been studied for centuries, e.g coleoptera, as well as new undiscovered ones, the soil microbiome of 
Crete. These works span from basic research to conservation analyses. 
Spatial analysis brought together the micro scale of microbes with the macro
scale of the island of Crete.
Microbes also unlocked some processes and functions they facilitate in their environments assisting in
categorising them and finding possible one health concerns.
Using text mining we incorporated historical texts and contemporary literature.
Lastly, in the Island Sampling Day project we covered
the whole spectrum of the analysis from large scale field sampling to informatics.

All this work was made possible because of the adoption of open data and standards
and open source software. Following these values, all the data and software produced
in this work are documented and deposited in platforms with open source licences. 
Working also with the Genome Standards Consortium was a great experience to witness
great scientists devoting a lot of time to design and promote standards across the 
community. This openness is both inspiring and humbling and I hope to contribute and follow 
their lead.

\addcontentsline{toc}{chapter}{Funding}
\chapter*{Funding}

The first two years of this PhD were about the PREGO project, discussed in Chapter \ref{cha:prego}. My work in this 
project was supported by the Hellenic Foundation for Research and
Innovation (HFRI) and the General Secretariat for Research and Innovation (GSRI),
under Grant No. 241, PREGO project.

The last two years were supported by the 3rd HFRI. Scholarships for PHD
Candidates (no. 5726) leading to the Chapters \ref{cha:isd-crete-soil}, \ref{cha:arthropods} and \ref{cha:crete-soil}.

   \begin{figure}[h]
      \centering
      \includegraphics[width=0.30\columnwidth]{figures/ELIDEK_Logo_EN_and_GSRI_logo_EN.png}
      \caption[HFRI and GSRI funding]{
          This PhD was supported by HFRI - GSRI grands. 
      }
      \label{fig:hfri_logo}
   \end{figure}


%----------------------------------------------------------------------------------------
%	LIST OF CONTENTS/FIGURES/TABLES PAGES
%----------------------------------------------------------------------------------------

\tableofcontents % Prints the main table of contents

\listoffigures % Prints the list of figures

\listoftables % Prints the list of tables

%----------------------------------------------------------------------------------------
%	ABBREVIATIONS
%----------------------------------------------------------------------------------------

\begin{abbreviations}{ll} % Include a list of abbreviations (a table of two columns)

\textbf{API} & \textbf{A}pplication \textbf{P}rogramming \textbf{I}nterface\\
\textbf{ASV} & \textbf{A}mplicon \textbf{S}equencing \textbf{V}ariants\\
\textbf{BHL} & \textbf{B}iodiversity \textbf{H}eritage \textbf{L}ibrary\\
\textbf{BOM} & \textbf{B}iodiversity \textbf{O}bservations \textbf{M}iner\\
\textbf{BODC} & \textbf{B}ritish \textbf{O}ceanographic \textbf{D}ata \textbf{C}entre\\
\textbf{CLI} & \textbf{C}ommand \textbf{L}ine \textbf{I}nterface\\
\textbf{CTI} & \textbf{C}ommunity \textbf{T}emperature \textbf{I}ndex\\
\textbf{DECO} & Bio\textbf{D}iv\textbf{E}rsity Data \textbf{C}uration w\textbf{O}rkflow\\
\textbf{DPI} & \textbf{D}ots \textbf{P}er \textbf{I}nch\\
\textbf{DwC-A} & \textbf{D}arwin \textbf{C}ore \textbf{A}rchive\\
\textbf{EM} & \textbf{E}ntity \textbf{M}apping\\
\textbf{EMODnet} & \textbf{E}uropean \textbf{M}arine \textbf{O}bservation and \textbf{D}ata \textbf{Net}work\\
\textbf{ENVO} & \textbf{Env}ironment \textbf{O}ntology\\
\textbf{EOL} & \textbf{E}ncyclopedia \textbf{O}f \textbf{L}ife\\
\textbf{FAIR} & \textbf{F}indable \textbf{A}ccessible \textbf{I}nteroperable \textbf{R}eusable\\
\textbf{GBIF} & \textbf{G}lobal \textbf{B}iodiversity \textbf{I}nformation \textbf{F}acility\\
\textbf{GLOBI} & \textbf{GLO}bal \textbf{B}iotic \textbf{I}nteractions\\
\textbf{GNA} & \textbf{G}lobal \textbf{N}ames \textbf{A}rchitecture\\
\textbf{GNRD} & \textbf{G}lobal \textbf{N}ames \textbf{R}ecognition and \textbf{D}iscovery\\
\textbf{GUI} & \textbf{G}raphical \textbf{U}ser \textbf{I}nterface\\
\textbf{HTML} & \textbf{H}yper\textbf{T}ext \textbf{M}arkup \textbf{L}anguage\\
\textbf{HTTP} & \textbf{H}yper\textbf{T}ext \textbf{T}ransfer \textbf{P}rotocol\\
\textbf{ICZN} & \textbf{I}nternational \textbf{C}ommission on \textbf{Z}oological \textbf{N}omenclature\\
\textbf{ID} & \textbf{I}dentifier\\
\textbf{IE} & \textbf{I}nformation \textbf{E}xtraction\\
\textbf{IPT} & \textbf{I}ntegrated \textbf{P}ublishing \textbf{T}oolkit\\
\textbf{IR} & \textbf{I}nformation \textbf{R}etrieval\\
\textbf{JSON} & \textbf{J}ava\textbf{S}cript \textbf{O}bject \textbf{N}otation\\
\textbf{LSID} & \textbf{L}ife \textbf{S}cience \textbf{I}dentifier\\
\textbf{MedOBIS} & \textbf{Med}iterranean node of the \textbf{O}cean \textbf{B}iodiversity \textbf{I}nformation \textbf{S}ystem\\
\textbf{NCBI} & \textbf{N}ational \textbf{C}enter for \textbf{B}iotechnology \textbf{I}nformation\\
\textbf{NER} & \textbf{N}amed \textbf{E}ntity \textbf{R}ecognition\\
\textbf{NERC} & \textbf{N}atural \textbf{E}nvironment \textbf{R}esearch \textbf{C}ouncil\\
\textbf{NLP} & \textbf{N}atural \textbf{L}anguage \textbf{P}rocess\\
\textbf{OBIS} & \textbf{O}cean \textbf{B}iodiversity \textbf{I}nformation \textbf{S}ystem\\
\textbf{OCR} & \textbf{O}ptical \textbf{C}haracter \textbf{R}ecognition\\
\textbf{OS} & \textbf{O}perating \textbf{S}ystem\\
\textbf{OTU} & \textbf{O}perational \textbf{T}axonomic \textbf{U}nit\\
\textbf{PDF} & \textbf{P}ortable \textbf{D}ocument \textbf{F}ormat\\
\textbf{PNG} & \textbf{P}ortable \textbf{N}etwork \textbf{G}raphics\\
\textbf{REST} & \textbf{RE}presentational \textbf{S}tate \textbf{T}ransfer\\
\textbf{SI} & \textbf{I}nternational \textbf{S}ystem of \textbf{U}nits\\
\textbf{URL} & \textbf{U}niform \textbf{R}esource \textbf{L}ocator\\
\textbf{WoRMS} & \textbf{Wo}rld \textbf{R}egister of \textbf{M}arine \textbf{S}pecies\\
\end{abbreviations}

%----------------------------------------------------------------------------------------
%	PHYSICAL CONSTANTS/OTHER DEFINITIONS
%----------------------------------------------------------------------------------------

%\begin{constants}{lr@{${}={}$}l} % The list of physical constants is a three column table
%
%% The \SI{}{} command is provided by the siunitx package, see its documentation for instructions on how to use it
%
%Speed of Light & $c_{0}$ & \SI{2.99792458e8}{\meter\per\second} (exact)\\
%%Constant Name & $Symbol$ & $Constant Value$ with units\\
%
%\end{constants}
%
%%----------------------------------------------------------------------------------------
%	SYMBOLS
%----------------------------------------------------------------------------------------

%\begin{symbols}{lll} % Include a list of Symbols (a three column table)
%
%$a$ & distance & \si{\meter} \\
%$P$ & power & \si{\watt} (\si{\joule\per\second}) \\
%%Symbol & Name & Unit \\
%
%\addlinespace % Gap to separate the Roman symbols from the Greek
%
%$\omega$ & angular frequency & \si{\radian} \\
%
%\end{symbols}

%----------------------------------------------------------------------------------------
%	DEDICATION
%----------------------------------------------------------------------------------------

%\dedicatory{For/Dedicated to/To my\ldots} 

%----------------------------------------------------------------------------------------
%	THESIS CONTENT - CHAPTERS
%----------------------------------------------------------------------------------------

\mainmatter % Begin numeric (1,2,3...) page numbering

\pagestyle{thesis} % Return the page headers back to the "thesis" style

% Include the chapters of the thesis as separate files from the Chapters folder
% Uncomment the lines as you write the chapters

% --------------------------------------------------
% 
% This chapter is the introduction
% 
% --------------------------------------------------


\chapter{Introduction}
\label{cha:intro}

% Integration of the general approach of the PhD

\section{Ecosystem ecology}
\label{sec:intro-ecosystem}

Comprehension of ecosystem function is one of the pinnacles of ecology and
requires quantitative and conceptual advances \parencite{Chapin_Matson_Vitousek_2011}.
Ecosystem function refers to the processes that occur in an ecosystem, their
interconnectance, the agents that carry them out
and their relationship with the environment \parencite{Chapin_Matson_Vitousek_2011}. Biodiversity
\parencite{hooperEFFECTSBIODIVERSITYECOSYSTEM2005, loreau2001Biodiversity}
and community structure and dynamics \parencite{gonze2018Microbial,morris2020linking}
are known parameters of ecosystem function. This holistic approach must be taken
into account when tackling complex systems such as ecosystems. Studying and
combining these parameters covers multiple fields of ecological science;
metabolic ecology, biogeochemistry, community and population ecology and
environmental ecology. Adding to this complexity, ecosystem behaviour cannot be
explained or predicted with just vast amounts of data because interconnectance
and interdependence lead to emergent phenomena in spatial and temporal scales.
Currently we are experiencing a shift in understanding ecosystem function
because the accumulated knowledge from ecological theories is quantified and
supported with experimental data.

Large scale -omics experiments have introduced a vast quantity of data from the
environment regarding sequences, proteins and small molecules \parencite{shaffer2022Standardized}.
System biology has been established as a means to study community ecology interactions spanning
from the molecular level to the community level \parencite{raes2008Molecular}.
One area that has emerged from omics analyses is microbial ecology and
consequently the metabolic processes of ecosystems \parencite{perez_garcia2016Metabolic}
which unify all living scales, their diversity and complex dynamics \parencite{smith2016Origin}.
Microbes are claimed to be one of the major drivers of ecosystem metabolic
processes of all kinds \parencite{falkowski2008microbial,hall2018understanding} and
their percentage in total biomass is enormous, comparable to plants \parencite{bar2018biomass}.
These discoveries make microbes and microbiomes perfect candidates when
studying ecosystem function \parencite{klitgord2011Ecosystems,widder2016Challenges}.

Microbial ecology brought together all scales of ecosystem functioning, from
molecules to biogeochemical cycles \parencite{hall2018understanding,kempes2012Growth,raes2011molecular}.
Several large scale sampling projects have been launched like the Earth
Microbiome Project, Tara Oceans, Global Ocean Sampling Expedition (GOS), and
many small ones.
This explosion of metagenomic and microbiome data was followed by the creation
of many repositories and databases, some project specific and some that
collected many different sources. The metadata richness of metagenomic data
varies depending on the sampling and it may include location data, habitat type
and environmental parameters like pH, humidity and nutrients. Metadata
deficiency of data is a limiting step for the ecosystem function analysis
which, in the short-term can be resolved with data integration solutions from
alternative sources (for example satellite data for the sampling region). In
the long-run, initiatives like data FAIRification will improve their
interoperability \parencite{wilkinson2016the-fair}. Nevertheless, microbiome data can be integrated with microbe
specific data of metabolism and pathways using ontologies like KEGG, Gene
Ontology and REACTOME and with habitat description from the environment
ontology (EnvO) \parencite{buttigieg2016environment}. Bringing together all the above requires web technologies and
data enrichment methodologies which are only recently starting to be
implemented in microbial ecology \parencite{jiang2016Microbiome}. This integration
will create maps of microbes, metabolism and environments. 

However, this mapping derived from well structured data forms like databases
and ontologies will cover only a small part of the current knowledge. Most
scientific information is scattered in journal publications which consist of
text that is unstructured. Hence, text mining is needed to synthesize
information from publications \parencite{jensen2006Literature}. Microbiome and
human diseases research as stated in \parencite{badal2019Challenges} have been
greatly advanced by text mining. This is still in its infancy in microbial
ecology. Bringing together text mining associations and maps derived from
databases will provide a huge knowledge base consisting of microbes, their
metabolism and their habitat. The creation of the knowledge base is the first
step towards relation deciphering between these data types.

Biodiversity and ecosystems contain life across the tree of life, from
bacteria, to nematoda, to arthropods, to plants, to birds, to mammals to 
humans. Network analysis based on the knowledge base will quantify the relations
between microbes, their metabolism and ecosystem
function \parencite{graham2016Microbes,muller2018Using, perez_garcia2016Metabolic}.
Microbial interactions can be inferred from the aforementioned data as shown
in \parencite{machado2021Polarization} based on their habitat and metabolism.
These community dynamics can be used to build ecosystem level metabolic
networks \parencite{perez_garcia2016Metabolic} that influence nutrient cycling in
ecosystems \parencite{bauer2018Network}. Another approach is by using species as a
container of pathways and then examining the functions accumulated in the
ecosystem \parencite{loucaDecouplingFunctionTaxonomy2016}. Both approaches have
been applied to specific ecosystems with applicable results but relying on a
global knowledge base will expand ecosystem function analysis into many
different ecosystems that have been sampled.

Microbes are known for their versatility, abundance and influence on ecosystem
functions. Yet, for a more complete analysis all forms of life must be included,
especially plants \parencite{thompson2012Food}. But viruses as well because they
change the metabolic processes of microbes and are highly important in
ecosystem functions \parencite{hurwitz2016Viral}. Another restriction is the fact
that the bulk of analyses regarding microbial data are using associations
through different categories of data. Associations are very useful and can
provide important insights but more rigorous and explanatory methodologies
regarding bioenergetics \parencite{kempes2012Growth}, population dynamics
\parencite{gonze2018Microbial}, ecosystem stability \parencite{berdugo2020Global} and
metabolic ecology \parencite{brown2004METABOLIC} should be incorporated as well to
advance our understanding of ecosystem function.

\section{Integration}
\label{sec:integration}

The integration of biodiversity knowledge in one place is a longstanding
goal in ecological research \parencite{Walter_2012}. The synthesis of multiple
data types and datasets across the globe has enabled 
holistic approaches to crucial scientific and societal questions \parencite{heberling_j_mason_data_2021}.
Yet there remain some conceptual challenges in microbial community ecology \parencite{prosser2020Conceptual}.
There multiple roads taken in order to decipher the microbiomes from a macroecological \parencite{Mascarenhas2020}, 
eco-evolutionary \parencite{martiny2023Investigating, loreau2023Opportunities} and synthetic biology \parencite{Leggieri2021} approaches.

Current biodiversity knowledge is in multiple digital forms. 
Taxa names, traits and metadata are mainly in matrix data
structures. Taxa spatial distributions are in spatial data
forms like polygons or points. In additions taxa information
is available in sequences in fasta format. Abiotic data like 
temperature, pH, nutrients are also stored in matrix
and spatial formats. Audiovisual files contain information
of species and ecosystems in images, audio and video formats.
In addition, mathematical models contain mechanistic information
about many different aspects of ecology and algorithms, in the 
form of code, as well. All the above are communicated and disseminated through the
literature in the form of text. A user friendly interface web portal can provide all this wealth of data
regarding the multifaceted ecosystem variables. There is a need for policy regarding earth system for the global
sustainability \parencite{reid2010earth}. Communities like GEO BON (Biodiversity Observation Network) and Soil BON
have pushed towards these goals in the past decade.

In addition to biodiversity data, the spatial dimension is shaping the biodiversity and with the remote sensing revolution now
it's possible to use these data to monitor resilience \parencite{Lenton2022resilience}. 
The concept of Digital Earth, first coined in Al Gore’s book entitled 
“Earth in the Balance” (Gore 1992), was further developed in a speech
written for Gore at the opening of the California Science Center in 1998. Other words with
similar meaning are Digital Twin, Island Digital Ecosystem Avatars (IDEA), Earth system
and Global Earth Observation System of Systems (GEOSS).

Currently there are many global databases with distinct and overlapping 
scope like GBIF, OBIS, iDigBio, EOL etc. \parencite{feng2022Review}. 
The contents of these databases covers taxonomy, spatial extend, traits, 
connections to specimens, species literature, distributions,
IUCN status. These portals in order to remain relevant and updated require 
continuous funding and development. Thus, Research Infrastructures (RI) are 
the most suitable organisational models to sustain these portals. Examples 
of such RIs are :

\begin{itemize}

    \item eLTER: European long-term ecosystem, critical zone and socio-ecological systems research infrastructure 
    \item LifeWatch ERIC
    \item EMBRC

\end{itemize}

Yet, a synthesised knowledge base of biodiversity, in terms of ecological and
remote-sensing data remains a major challenge \parencite{feng2022Review}. The challenge
is mainly on the different data standards used but also in the implementation of 
data exchange models and synchronisation.

Apart from data and their hosting, ecosystem holistic understanding requires
integrative methods. Macrosystems approach uses interactions of local parameters
such as biological, geophysical and sociocultural and explores their influence in
the macro scale \parencite{heffernan2014}. Combining multiple omics data and community ecology and causality has
been an important method to integrate \parencite{jurburg2022community}. Useful are also the network inference methods 
and their analysis for biogeochemical cycling \parencite{jameson2023Network}. Going further, multilayer networks \parencite{marine-multilayers}
are the way to include multiple omics data in specific environmental communities. 
Metabolism is a scale independent biological process spanning from chemical reactions
to biogeochemical cycles \parencite{hall2018understanding}. Bioenergetic models on food 
webs can provide information about energy exchanges in the food webs \parencite{valdovinos2023bioenergetic}.
Last but not least, information theory is a conceptual framework to bring together agents,
their interactions and the flow of information through these interactions \parencite{oconnor-information-ecology}.

\section{Soil ecosystems}
\label{sec:soil_ecosystems}

Soil ecosystems are the cornerstone of terrestrial functioning.
Biodiversity interactions are between all domains of life which form
multilevel associations. Bacteria \parencite{Delgado-Baquerizo-atlas}, archaea,
unicellular eukaryotes, nematodes \parencite{vandenHoogen2019},
springtails \parencite{potapov2023Globally}, earthworms \parencite{Phillips2021},
arthropods \parencite{milo-arthropods}, molluscs, plants, mammals; all occur in unison and 
influence the ecosystems they inhabit with their abundance, biomass \parencite{bar2018biomass} and metabolism.
The plant-insect-soil ecosystem is starting to be studied as a whole to discover
important associations with practical implications such as plant resistance 
to insect attack \parencite{plant-insect-soil2023}.
All of these life forms occur side by side and influence on another. This has been shown in the 
top of the mountains \parencite{winkler2018side}, in plant traits and soil microbiome interaction \parencite{beugnon2022Abiotic} and others. 

Many worldwide studies regarding soil ecosystems are being implemented, yet
there are many gaps to cover the complexity of functioning and biodiversity
\parencite{guerra2020Blind}. Knowledge is lacking for specific taxonomic groups and
ecosystems for example the Sahara desert and below ground fauna.
These gaps are crucial to be studied in order to
improve our understanding of soil ecosystems \parencite{cameron2018Global}. 
Islands can be important case studies for this integration.

Ecosystem functioning is an integral part of the sciences of climate change
and conservation \parencite{cavicchioli2019scientists}. These fields undertake the
sustainability of ecosystems such as coral reefs and tropical forests, when
faced with habitat loss, biodiversity loss, pollution and generally any change
in the natural environment conditions. Monitoring these ecosystems is crucial
and environmental data analysed using network theory will facilitate real time
inspection and inspire immediate action \parencite{derocles2018Biomonitoring}.
Ecosystem services, on the other hand, aim at utilising ecosystem resources
and functions for all sorts of human activities ranging from goods supplied
from agriculture and farming \parencite{alvarez-silva2017Compartmentalized} to
recreational and educational purposes. In addition, the one health concept has
shown that microbiomes are the for human health and prosperity
\parencite{banerjee2023Soil, lehmann2020concept}. Yet, the pressure poised from human activities
is reducing the ecosystem services \parencite{rillig2023Increasing}.

Last but not least, terraformation
projects explore ecosystem function with a bottom-up approach for engineering
systems to become habitable for humans, like planet Mars, or transforming
Earth's collapsed ecosystems to new habitable states
\parencite{conde-pueyo2020Synthetic}. Using the data and methods described in this
article will enhance our understanding in each of these areas. Knowledge
integration from data, metadata and text sources combined with ecological and
metabolic theory will expand our knowledge and eventually lead to new
knowledge regarding the aforementioned fields. A soil monitoring framework 
is overdue to track and inform the society about the health of 
soil ecosystems \parencite{guerra2021tracking}.

\section{Aim of this research}
\label{sec:aim}

The research presented here touches on multiple disciplines shown in Figure \ref{fig:phd-one-slide}, 
biodiversity, microbes and integration. 
First, regarding biodiversity as species occurrences, two projects were implemented.
Using historical biodiversity literature, the methods of data rescue were 
investigated using text mining tools. In addition, to further demonstrate the 
value of historical data along with contemporary data from the field sampling, a 
novel assessment of the endemic arthropods of the island of Crete was carried out.
Second, regarding the microbial biodiversity, the island sampling day project 
topsoil cores of the island of Crete were analysed. This adds a new chapter to 
the soil biodiversity of the island with interesting implications to one health.
And third, the integration of the existing knowledge regarding the metagenomic 
data and literature as well as the available spatial datasets about Crete. 
An additional aim of this work is to demonstrate that the wealth of available open data
and open source tools can inspire novel projects and integrative approaches that lead
to new knowledge.

   \begin{figure}[ht]
      \centering
      \includegraphics[width=\textwidth,height=\textheight,keepaspectratio]{figures/2024_phd_one_slide_en.png}
      \caption[Graphical abstract of this PhD]{The different sections of this PhD. There are 3 disciplines, Biodiversity, Microbiology and Data Integration. Each one contains different data sources, field applications and analysing methodologies.}
      \label{fig:phd-one-slide}
   \end{figure}



% --------------------------------------------------
% 
% This chapter is for PREGO
% 
% --------------------------------------------------


\chapter{PREGO: a literature- and data-mining resource to associate microorganisms, biological processes, and environment types}
\label{cha:prego}


\textbf{Citation:} \\ 
Zafeiropoulos H, Paragkamian S, Ninidakis S, Pavlopoulos GA, Jensen LJ, Pafilis E. PREGO: A Literature and Data-Mining Resource to Associate Microorganisms, Biological Processes, and Environment Types. Microorganisms. 2022; 10(2):293. \\ 
DOI: \href{https://doi.org/10.3390/microorganisms10020293}{10.3390/microorganisms10020293}\footnote{
   For author contributions and supplementary material please refer to the relevant sections. 
   This is a modified version of the published version,
   in terms of relevance, coherence and formatting.
   }



% PREGO ABSTRACT
\section{Abstract}

   To elucidate ecosystem functioning, it is fundamental to recognize \textit{what} processes occur in which environments (\textit{where}) and which microorganisms carry them out (\textit{who}). 
   Here, we present PREGO, a one-stop-shop knowledge base providing such associations. 
   PREGO combines text mining and data integration techniques to mine such what-where-who associations from data and metadata scattered in the scientific literature and in public omics repositories. 
   Microorganisms, biological processes, and environment types are identified and mapped to ontology terms from established community resources. 
   Analyses of comentions in text and co-occurrences in metagenomics data/metadata are performed to extract associations and a level of confidence is assigned to each of them thanks to a scoring scheme. 
   The PREGO knowledge base contains associations for 364,508 microbial taxa, 1090 environmental types, 15,091 biological processes, and 7,971 molecular functions with a total of almost 58 million associations. 
   These associations are available through a web interface (\href{https://prego.hcmr.gr}{https://prego.hcmr.gr}), an Application Programming Interface (API), and bulk download. 
   By exploring environments and/or processes associated with each other or with microbes, PREGO aims to assist researchers in design and interpretation of experiments and their results. 
   To demonstrate PREGO's capabilities, a thorough presentation of its web interface is given along with a meta-analysis of experimental results from a lagoon-sediment study of sulfur-cycle related microbes.

% RPEGO INTRODUCTION
\section{Introduction}
\label{sec:prego-intro}

   Microbes are omnipresent and impact global ecosystem functions \cite{falkowski2008microbial} through their abundance \cite{bar2018biomass}, versatility \cite{delgado2016microbial}, and interactions \cite{rottjers2018hairballs}. 
   These facts have inspired microbiologists from diverse scientific fields to study their genotype and phenotype \cite{morris2020linking}, their metabolism \cite{biggs2015metabolic}, and their interactions with the environment \cite{hall2018understanding}. 
   All this work has resulted in a wealth of knowledge available in the forms of literature and experimental data. Literature contains vast amounts of information in the free text form that overwhelms researchers. 
   Advanced text mining methods \cite{jensen2006literature} have been developed to assist this issue. 
   Experimental data and their metadata require mining \cite{delmont2011metagenomic} as well for their integration, mostly through metagenomic mining from online repositories. 
   Hence, the combination of this knowledge about microbial life (who), their metabolic functions (what), and the environment they influence (where) is an important step to study ecosystem function \cite{raes2008molecular}.

   High Throughput Sequencing (HTS) has turned the page on microbial ecology studies \cite{nilsson2019mycobiome}. 
   Over the past 20 years, both the taxonomic and the functional profiles of microbial communities from both local and large-scale regions (e.g., Tara Oceans \cite{pesant2015open}, Earth Microbiome \cite{gilbert2014earth}) are being accumulated at a higher and higher rate. 
   Extreme environments, i.e., areas with high salinity, low pH, etc., are being studied, providing us with unprecedented insight \cite{shu2021microbial}. 
   Both amplicon and shotgun metagenomics studies have played a crucial part in this development. Latest technological breakthroughs, such as Metagenome-Assembled Genomes (MAGs) and Single Amplified Genomes (SAGs), are enhancing the assessment of the taxonomic and functional repertoire of microbiomes even further. 
   However, the mass use of these technologies and their consequent data have led to a number of needs and challenges, with metadata curation being among the most crucial ones.

   Standards-promoting communities, like \href{https://gensc.org/}{Genomic Standards Consortium (GSC)}\footnote{
      \href{https://gensc.org/}{https://gensc.org/}
   }, their efforts, like Minimum Information about any (x) Sequence (MIxS) \cite{yilmaz2011minimum}, and projects endorsing those, like National Microbiome Data Collaborative (NMDC) \cite{wood2020national, vangay2021microbiome}, offer guidelines and best-practices to assist the annotation of microbial ecology samples. 
   Controlled vocabularies and ontologies contribute to these efforts as they describe each subject area with formal terms \cite{walls2014semantics}. 
   Environment types, for example, are described by the Environment Ontology (ENVO) \cite{buttigieg2016environment}. 
   Other key biological aspects that have been captured include molecular functions (Gene Ontology Molecular Function (GOmf) \cite{ashburner2000gene, gene2021gene}, Enzyme Commission nomenclature \cite{noauthor_1999}, etc.), and the pathways carrying out different biological processes (GO Biological Process (GObp), MetaCyc \cite{caspi2020metacyc}, etc.). 
   These knowledge structures, along with taxonomic centralized resources like the National Center for Biotechnology Information (NCBI) Taxonomy \cite{schoch2020ncbi}
   and LPSN (List of Prokaryotic names with Standing in Nomenclature) \cite{parte2020list}, 
   provide the means for a standardized representation of, for example, environments, process-oriented terms, and microbial taxa, respectively. Global-scale public resources (like MGnify \cite{mitchell2020mgnify}, 
   JGI/IMG \cite{chen2021img}, 
   MG-RAST \cite{wilke2015restful}) 
   combine some of the aforementioned resources to support the collection, analysis, and distribution of multiple types of HTS data (e.g., amplicon, metagenomics, metatranscriptomics, etc.).

   Besides the data and the analyses \textit{per se}, the related scientific literature stores valuable information in billions of text lines. 
   PubMed \cite{schoch2020ncbi} and PubMed Central (PMC) \cite{roberts2001pubmed} are gateways to relationships among microbes (\textit{who}), the environments they live in (\textit{where}) and their associated processes and functions (\textit{what}) hidden in text \cite{harmston2010papers}. 
   Text mining (on both literature and metadata) can serve the extraction of these relationships. Named Entity Recognition (NER) can, for example, locate organism names \cite{pafilis2013species}, ENVO and GO terms \cite{pafilis2016extract} mentioned in text and map them to their corresponding identifiers. 
   Association statistics, like co-mention analysis, can subsequently suggest ranked association among such entities \cite{von2005string, franceschini2012string}. 
   The new era of omics has been interwoven with data integration \cite{gomez2014data} by bringing together scattered and fragmented pieces of information.

   The time is ripe for tools that integrate all this knowledge and henceforth assist researchers to tackle major challenges like climate change \cite{cavicchioli2019scientists}, 
   sustainability \cite{d2021microbiome}, 
   and synthetic ecology \cite{conde2020synthetic}.
   Many resources have emerged in this realm \cite{baltoumas2021biomolecule}, 
   each one serving a specific purpose, such as BacDive \cite{reimer2019bac}. 
   BacDive is a large-scale curated database with prokaryotic information about phenotypic, morphological, and metabolic information. 
   Other resources like Microbe Directory \cite{shaaban2018microbe}, 
   Web of Microbes (WoM) \cite{kosina2018web}, 
   and Microbial Interaction Network Database (\href{http://www.microbialnet.org/mind_home.html}{MIND}\footnote{
      \href{http://www.microbialnet.org/mind_home.html}{http://www.microbialnet.org/mind\_home.html}
   }) focus on microbial environmental conditions, metabolite interactions with microbes and microbe-microbe interactions, respectively. 
   In addition, taking advantage of aforementioned resources, novel pipelines, e.g., \cite{tang2020tripartite}, are emerging with the aim to explore the network associations of who (microbial taxa) is performing what (microbial processes) and where (environments) directly using graph theory \cite{koutrouli2020guide}. 
   These analyses and resources are important because microbiologists can enrich their data to explore hypotheses but also to identify potential gaps in knowledge regarding these associations \cite{li2021microbial}.

   Here, we present PREGO, a hypothesis generation web resource that is designed to be useful to microbiologists—in particular microbial ecologists and environmental microbiologists. 
   Its specific aims include: 
   (a) the gathering of source data, metadata, and literature followed by the extraction of microorganism, process, environment associations contained therein, 
   (b) making such a mined knowledge base available to life sciences researchers via an easy to use and explore web portal. 
   As such, PREGO can be useful also to system microbiologists and large-scale data analysts through bulk download and programming access. 
   We document the principles, analysis methodology, and contents behind PREGO. 
   Last but not least, we demonstrate PREGO's capabilities for researcher-support related to the above through a case study involving sulfate-reducing microorganisms.



% RPEGO METHODS
\section{Methods \& Implementation}
\label{sec:prego-methods}

   PREGO is a resource designed to assist molecular ecologists in acquiring a single point overview of what-where-who process–environment–organism associations. The system is comprised of two main parts: (a) a server that periodically harvests data and extracts process-environment-organism associations from the scientific literature, environmental samples, and genome annotation sequences (Figure~\ref{fig:prego-pipeline}, step 1 to 5) and (b) a web-based interface as well as an Application Programming Interface (API) that provides users and programmers with a friendly way to extract and navigate across the process–environment–organism associations (Figure~\ref{fig:prego-pipeline}, step 6).

   \begin{figure}[h]
      \centering
      \includegraphics[width=0.98\columnwidth]{figures/prego_analysis_horizontal.png}
      \caption[PREGO analysis methodology]{
         PREGO analysis methodology: PREGO periodically retrieves three distinct types of data from open access resources. 
         Scientific text, environmental sample data, and genomic annotations are handled with respective methodologies in order to standardize their entities. 
         Named Entity Recognition and Comention/Co-occurrence analysis is the common framework in order to build a weighted association network with nodes being the entity identifiers. Lastly, all these associations are available through a Web interface and an API. 
         All these steps have been implemented in an autonomous way with regular cycles of updates (see Appendix~\ref{app:B}). 
         Icons used from the Noun Project released under CC BY: Books by Shakeel Ch., Bacteria by Maxim Kulikov, ftp by DinosoftLab, Mountain by Diane, Ship on Sea by farra nugraha, River by Chanut is Industries.
      }
      \label{fig:prego-pipeline}
   \end{figure}



   \subsection{Entity Types, Channels, and Associations}
   \label{subsec:prego-terms}

   PREGO supports three entity types: \textit{Process}, \textit{Environment}, and \textit{Organism}. 
   For interoperability and consistency, an ontology or taxonomy is adopted for each type of entity.
   Processes are represented as Gene Ontology (GO) terms and are grouped either as Biological processes (GObp) or as Molecular functions (GOmf). 
   In addition, Environments are represented by terms from the Environmental Ontology. 
   Organisms are represented by the microbial NCBI Taxonomy Ids (Bacteria, Archaea, and unicellular eukaryotes). For the unicellular eukaryotes, a custom list was populated with the unicellular eukaryotic taxa using a curated list.
   PREGO's contents are mainly divided into three distinct channels of information based on data origin and format (Figure~\ref{fig:prego-pipeline}, step 1). 
   The \textit{Literature} channel exploits scientific publications, i.e., abstracts and full text open access scientific publications (Table~\ref{table:prego1} and Section~\ref{subsec:prego-tm}). 
   Through the \textit{Annotated Genomes and Isolates} channel, PREGO retrieves genome annotations and their accompanying metadata (Table~\ref{table:prego1} and Section~\ref{subsec:prego-isolates}). 
   Finally, the \textit{Environmental Samples} channel supports the integration of metagenomic analyses from both amplicon and shotgun studies. 
   These include taxonomic and functional profiles along with their corresponding metadata (Table~\ref{table:prego1}, more details in Section~\ref{subsec:prego-envsamples}).


   % SOURCE DATABASES TABLE table:prego1
   \begin{table}[h]
      \begin{tabular}{@{}ccccc@{}}
      \toprule
      \textbf{Source} & \textbf{\# items} & \textbf{Data type} & \textbf{Metadata} & \textbf{License} \\ \midrule
      \begin{tabular}[c]{@{}c@{}}MEDLINE and \\ PubMed\end{tabular} & 33 million & abstracts (text) & no & NLM Copyright \\
      \begin{tabular}[c]{@{}c@{}}PubMed Central \\ OA Subset\end{tabular} & 2.7 million & full article (text) & no & \begin{tabular}[c]{@{}c@{}}CC for Commercial, \\ non-commercial\end{tabular} \\
      JGI IMG & 9,644 & \begin{tabular}[c]{@{}c@{}}Isolates Annotated \\ genomes\end{tabular} & yes & JGI Data Policy \\
      Struo & 21,276 & Annotated genomes & no & MIT, CC BY-SA 4.0 \\
      BioProject & 18,752 & \begin{tabular}[c]{@{}c@{}}Annotated genomes \\ with abstracts (text)\end{tabular} & yes & INSDC policy \\
      \multirow{2}{*}{MG-RAST} & 16,096 & markergene samples & yes & CC0 \\
      & 7,965 & metagenomic samples & yes & CC0 \\
      MGnify & 10,500 & markergene samples & yes & CC-BY, CC0 \\ \bottomrule
      \end{tabular}
      \caption[Source databases integrated in PREGO and the number of items retrieved]{
         Source databases that are integrated in PREGO and the number of items retrieved. 
         The Open Access subset of PubMed Central has a Creative Commons license available for commercial and noncommercial use. 
         JGI has its own license, the same applies for BioProject, MEDLINE®, and PubMed® as well.
         }
         \label{table:prego1}
   \end{table}


   In cases in which the retrieved data and metadata are in text form, they are standardized to the aforementioned identifiers and taxonomies using Named Entity Recognition (NER) tools, namely the EXTRACT tagger \cite{pafilis2016extract, jensen2016one}. 
   In cases where data contain KEGG Orthology terms and/or Uniref identifiers, they are mapped to the respective GOmf using the mapping files available from the UniProt (see Appendix~\ref{app:A}). 
   Associations are extracted after the mapping and standardization of the entities from each resource (Figure~\ref{fig:prego-pipeline}, step 3).
   The association extraction pipeline is distinct for each channel and resource because of differences in the data type origin (see \texttt{prego\_gathering\_data} in the Availability of Supporting Source Codes section). 
   By the means of navigation, the large number of associations returned to the user require a type of sorting; 
   ideally, one that ranks the most trustworthy associations at the top. 
   For those reasons, each channel of PREGO has a dedicated scoring scheme bounded within the (0,5] space for consistency. 
   In Appendix ~\ref{app:A} , the scoring scheme of each channel is elaborated.




   \subsection{Text Mining of Scientific Literature}
   \label{subsec:prego-tm}

   PREGO implements a text mining methodology to extract associations of the aforementioned entities from literature. 
   The origin of text mining is a corpus that comprises scientific abstracts and full text articles from MEDLINE® and PubMed® and PubMed Central® Open Access Subset (PMC OA Subset) \cite{sayers2021database}, respectively. 
   The building and periodic update of the corpus is possible through the NCBI File Transfer Protocol (FTP) services. 
   PREGO also has a dedicated text-mining dictionary (see Availability of Supporting Source Codes section) that contains the entities ids, names, synonyms, and neglected words (stop words). 
   PREGO dictionary incorporates the ORGANISMS \cite{pafilis2013species} and ENVIRONMENTS \cite{pafilis2015environments} evaluated dictionaries as well as the experimental dictionaries of Gene Ontology Biological Process and Molecular Function.
   Text mining is subsequently performed on the corpus using the dictionary through the EXTRACT tagger \cite{pafilis2016extract, jensen2016one}. 
   The tagger recognizes the entities of the dictionary in each abstract and full text article and assigns their co-mentions with a score. 
   The score is sensitive to the text structural level of co-mention; higher to lower scoring when co-mention appears in the same sentence, then, in the same paragraph, and lastly in the same article. 
   All these are integrated and normalized to a single score for each association, as implemented in STRING 9.1 \cite{franceschini2012string} (see Appendix ~\ref{app:C} for more details). 
   In addition, the tagger extracts each mention in every article to provide the origin of each association it extracts.


   \subsection{Annotated Genomes and Isolates}
   \label{subsec:prego-isolates}

   Annotated genomes and isolates comprise the most trustworthy data in PREGO's knowledge base because they refer to a single species/strain and also have manually curated metadata. 
   Among other data types, JGI-IMG \cite{chen2021img, mukherjee2021genomes} includes millions of genes from isolated genomes (isolates), SAGs and MAGs. Such annotations, along with their corresponding metadata, were collected using web-parsing technologies. Their metadata, describing their related environment/ecosystem, were tagged using the EXTRACT tagger to infer organisms—environments associations. The annotated KEGG terms were mapped to GOmf terms (see Appendix A). The GOmf terms were then used to extract organisms—processes associations.
   
   The Struo pipeline \cite{de2020struo} and its outcome when using the Genome Taxonomy DataBase (GTDB) (v.03-RS86) \cite{parks2020complete} was exploited to enrich organisms—processes associations. 
   A set of 21,276 representative genomes, accompanied by UniRef50 annotations, was retrieved using the provided FTP server. The annotations were then mapped to GOmf terms (see Appendix~\ref{app:A}). 
   Related GTDB genomes were mapped to their corresponding NCBI taxa (see Appendix~\ref{app:A}). 
   All associations extracted from these resources were assigned arbitrarily a confidence level of four out of five. 
   This score choice reflects the high-quality of these data and metadata.
   
   In addition, BioProject data were integrated to PREGO using the NCBI FTP/e-utils services \cite{sayers2021database}. 
   The BioProject ids that were integrated are the ones that have been assigned a PubMed abstract, a unicellular taxon, and Genome sequencing as data type. Then, using the text mining pipeline, associations were extracted connecting the assigned taxon with the rest of the entities that appear in the abstracts. This method resulted in associations that were assigned a confidence level of three (out of five) because of the combined method of curated data with text mining.


   \subsection{Environmental Samples}
   \label{subsec:prego-envsamples}

   MGnify \cite{mitchell2020mgnify} and MG-RAST \cite{wilke2015restful} repositories provide a great number of public metagenomic records. 
   In the PREGO framework, both amplicon and shotgun metagenomic analyses are retrieved periodically along with their corresponding metadata. 
   Data retrieval from these resources is possible from their Application Programming Interfaces (APIs). Marker gene analyses are retrieved and by measuring
   the co-occurrence of taxa present in the various environmental types (e.g., biomes, materials, features, etc.) organisms—environments associations are extracted. 
   These associations emerge when a taxon is reported together with a certain environmental type, being mentioned in the metadata of a sample (metadata based co-occurrence). 
   Similarly, analyses of metagenomic samples along with their corresponding metadata and annotations are also retrieved and organisms—environments, organisms—processes and processes—environments are extracted. 
   The processes—environments associations are possible through co-occurrence of the functional annotations of metagenomes with the environmental metadata of the samples.
   
   In all cases, the EXTRACT tagger is used on the microorganism names and the corresponding metadata of each sample to identify their identifiers (NCBI ids, ENVO terms, GOmf, GObp). 
   All associations in this channel are scored based on the number of samples the entity of interest co-occurs with specific sample metadata (e.g., environmental type) or annotations (functional annotations or taxonomic annotations). 
   The same scoring scheme was implemented across the channel resources (see Appendix~\ref{app:C} for more details), which ranks these associations with a value in the (0,5] space.


   \subsection{Sequence Search}
   \label{subsec:prego-seq-search}

   In the case of organisms, PREGO enables sequence-based queries, meaning a sequence (amplicon) can be used as an entry point like it was a taxon name. 
   To this end, a custom database was built using a set of reference custom databases for four commonly used marker genes. For 16S and 18S rRNA, the SILVA database (v.138) \cite{quast_silva_2013} and the PR2 database (version\_4.14.0) \cite{guillou2012protist, del2018eukref} were used. 
   Cytochrome c oxidase I (COI) \cite{suter2021capturing} is another commonly used marker gene; 
   for this reason, Midori 2 (version GB243) \cite{leray2018midori} was integrated in PREGO's custom database. 
   Finally, for the Internal transcribed spacer (ITS), common in studies focusing on Fungi, the Unite (version 8.3, accessed 10.05.2021) \cite{nilsson2019unite} database was added.



   \subsection{Back-End Server and Front-End Implementation}
   \label{subsec:prego-back-end}

   PREGO is a multi-tier web-based application. 
   It is hosted on a 64 GB RAM DELL R540, 20 core, Debian server. 
   Custom API clients (written in Python) are responsible for retrieving the data and metadata from each source (Figure~\ref{fig:prego-pipeline}, step 2). 
   These clients as well as the subsequent methodology (Figure~\ref{fig:prego-pipeline}, step 3 to 6) are updated in regular cycles using custom daemons (see Appendix~\ref{app:B}, Figure~\ref{fig:devops}). 
   The \textit{mamba/blackmamba} web framework underlies communication to the Postgres association-holding database and the client-side communication. 
   HTML 5, Ajax, JQuery, and custom Javascript enhance the user web experience. 
   PREGO supports widely used browsers (e.g., Chrome, Firefox, Safari, Edge) in various operating systems, such as Windows 10, Linux (Ubuntu 18), and MacOS (10.12, 11).


% RPEGO RESULTS
\section{Results \& Validation}
\label{sec:prego-results}

   \subsection{The PREGO Web Resource}
   \label{subsec:prego-web-resource}

   Users can access the PREGO contents through its web User Interface (UI) (Figures~\ref{fig:prego_ui} and \ref{fig:prego_ui_resources}), its Application Programming Interface (API) (Figure~\ref{fig:prego_api_schema}), or bulk download of all associations (Appendix D). 
   The User Interface comes with two search fields: a plain text search and a sequence search (Figure~\ref{fig:prego_ui}a). 
   The latter is used when the user wants to search for a taxon sequence (see Section~\ref{subsec:prego-seq-search} for supported sequence databases). 
   The plain text search supports three types of entry points; the user can search for a taxon name, e.g., \textit{Methanosarcina mazei}, an environmental type, e.g., lagoon, or a biological process e.g., methanogenesis. 
   In all entry points, PREGO returns an overview page consisting of tabs with associations of the entity of interest with the entities of the two other types (Figure~\ref{fig:prego_ui}b–d) as well as Documents and Downloads tabs (Figure~\ref{fig:prego_ui}e,f).


   \begin{figure}[h]
      \centering
      \includegraphics[width=0.98\columnwidth]{figures/prego_ui.png}
      \caption[PREGO web user interface]{
         PREGO web user interface. 
         (a) There are two search fields, plain text and taxa sequences. 
         (b-d) three associations tabs each one presenting associations of the querred entity with the respective entities, Environments (b), Biological Process (c ) and Molecular Function (d). 
         Three channels of information are distinguishing the associations based on the original data. 
         (e) Documents tab presents the scientific articles that mention the queried entity highlighted with color. 
         (f) Downloads tab provides the associations of each channel (when available) to be downloaded in JSON and TSV format.
      }
      \label{fig:prego_ui}
   \end{figure}


   Regarding the association tabs, when a taxon is used as a query, PREGO returns an overview page consisting of tabs for environments, biological processes, and molecular functions. When an environmental type is used as input, PREGO returns the organisms that have been found to be related to it, as well as the Biological Processes observed in the given environment. 
   Lastly, if a biological process is under study, PREGO returns a tab with the organisms along with another tab with the Environments related to the process. 
   Notably, only the associations with scores higher than $0.5$ are presented in the web platform and are sorted in descending order based on their score. 
   The score is visualized with a five-star system (see Appendix~\ref{app:C} for the scoring scheme).
   Every association tab contains three tables with associations derived from the PREGO channels (see Section~\ref{sec:prego-methods}) along with their supported evidence. 
   The user can both search and scroll through these tables, which makes knowledge extraction easier in cases where, for example, Isolate data contain hundreds of associations. 
   In the \textit{Literature} channel, each association is supported by the scientific articles with text-mining identified co-mentions. 
   When a user clicks on an association, a popup window appears. This window displays abstracts or excerpts of full text with the associated entities highlighted (Figure~\ref{fig:prego_ui_resources}a). 
   Additionally, the Environmental Samples and Genome annotations and Isolates channels support evidence for each association by providing links to more detailed information. 
   In the former channel, when the users click on an association, they are redirected to pertinent sample pages of MGnify (Figure~\ref{fig:prego_ui_resources}b). 
   Similarly, the latter redirects users to JGI and NCBI genomes when the associations originated from JGI—IMG and Struo, respectively (Figure~\ref{fig:prego_ui_resources}c).


   \begin{figure}[h]
      \centering
      \includegraphics[width=0.85\textwidth]{figures/prego_ui_resources.png}
      \caption[PREGO in action - examples]{ 
         Each association is supported by original data. 
         (a) Literature channel has a pop-up functionality that displays the scientific articles that each specific association occurs with highlighted color. 
         (b) Environmental Samples channel redirects to the samples that support the specific association (currently only is supported MGnify). 
         (c) Annotated Genomes channel similarly redirects to the isolates ids that each association is based on (both Struo and JGI IMG are supported).
      }
      \label{fig:prego_ui_resources}
   \end{figure}


   The \textit{Documents} tab includes a list of scientific publications where the queried entity is mentioned. 
   Through the \textit{Downloads} tab, users are able to get all of the PREGO associations found for their query, per entity type (e.g., all the environments found related to an organism) and per channel (e.g., all the Environments found related to an organism through the \textit{Literature} channel). 
   This data retrieval functionality is also available via the PREGO API (syntax described in Figure~\ref{fig:prego_api_schema}). 
   Finally, all PREGO associations are available for bulk download from each channel (see Table~\ref{table:pregoA1}).



   \begin{figure}[h]
      \centering
      \includegraphics[width=135mm]{figures/figure_4_PREGO API.png}
      \caption[The PREGO API schema]{The PREGO API schema.}
      \label{fig:prego_api_schema}
   \end{figure}


   \subsection{PREGO in Action}
   \label{subsec:prego-action}

   To demonstrate PREGO's potential, we present four different ways that PREGO can assist molecular ecologists. 
   The demo focuses on the sulfate-reducing microorganisms (SRMs) as well as the processes and environments that relate to sulfate reduction. 
   Through this demo, we highlight how the different channels may provide complementary insights regarding different taxonomic levels and different association types.

   \subsection*{\textit{Which Environments Are Related to a Taxon?}}
   \label{subsec:envo-taxa}

   Based on Pavloudi et al. (2017) \cite{pavloudi2017diversity}, several bacterial and archaeal SRM were found in lagoonal sediments, after amplifying and sequencing the dissimilatory sulfite reductase $\beta$-subunit (dsrB). 
   Using PREGO for the case of Desulfobacteraceae, the family in which the majority of the observed OTUs of the study belonged to, several environmental types similar to lagoons were retrieved from both the \textit{Literature} and the \textit{Environmental samples} channels (Figure~\ref{fig:prego_ui_resources}a,b). Moreover, most of them had a high $z$-score, such as \textit{"sediment"}, \textit{"sludge"}, and \textit{"activated sludge"}. 
   Several dissimilar environmental types were associated with Desulfobacteraceae, e.g., \textit{"oil reservoir"} indicating them as potential environments where sulfate reduction takes place. 
   However, the presence of taxa within that family in different environments, from \textit{"sea water"} to \textit{"forest"} and \textit{"Wastewater treatment plant"}, may suggest that this family has ubiquitous representatives in diverse conditions.

   Searching for \textit{Desulfatiglans anilini} (\href{https://prego.hcmr.gr/example1}{example1}\footnote{\href{https://prego.hcmr.gr/example1}{https://prego.hcmr.gr/example1}}, accessed on 24 December 2021) at the species level, 
   the most abundant species in Pavloudi et al. (2017) and, 
   for \textit{Desulfatiglans anilini} DSM 4660 strain ({\href{https://prego.hcmr.gr/example2}{example 2}\footnote{\href{https://prego.hcmr.gr/example2}{https://prego.hcmr.gr/example2}}, accessed on 24 December 2021), 
   PREGO provides associations with the \textit{"Anaerobic sediment"}, \textit{"Marine oxygen minimum zone"}, and \textit{"Anaerobic digester sludge"} terms. 
   These associations further corroborate the relationship between the species and sulfate reduction. 
   More specifically, the \textit{"sulfur spring"} ENVO term was retrieved from the Environmental samples channel as well.


   \subsection*{\textit{Which Biological Processes and Molecular Functions Are Related to a Taxon?}}
   \label{subsec:proc-taxa}

   According to Pavloudi et al. (2017), \textit{Desulfatiglans anilini} plays an important role in sulfate reduction. 
   The Biological Processes provided by PREGO's Literature channel are the GO terms \textit{"Sulfate reduction"}, \textit{"Sulfide oxidation"}, and \textit{"Sulfide ion homeostasis"}, which support this claim. 
   In addition, the \textit{"Denitrification pathway"} term was also retrieved. 
   This is rather interesting as it is in line with what Pavloudi et al. (2017) discussed about the SRMs and their ability to use various electron acceptors, e.g., nitrate and nitrite.

   Furthermore, PREGO's Molecular Function tab provides more insight on this example. Several GO terms related to sulfate reduction (e.g., terms related to \textit{"sulfite reductase"}) were associated with DSM 4660 strain and Desulfatiglans anilini species in multiple channels. Interestingly, in the case of the strain query, the Annotated Genomes channel returned many GO terms related to the nitrogen fixation, e.g., \textit{"nitric oxide dioxygenase activity"}.


   \subsection*{\textit{Which Taxa Are Related to a Biological Process?}}
   \label{subsec:taxa-proces}

   PREGO can be also used to report organisms that relate to a certain biological process. Searching for \textit{"dissimilatory sulfate reduction"} associations with taxa (\href{https://prego.hcmr.gr/example3}{example 3}\footnote{\href{https://prego.hcmr.gr/example3}{https://prego.hcmr.gr/example3}}, accessed on 24 December 2021) resulted in several taxa that were mentioned in the Pavloudi et al. (2017) study. 
   For example, taxa such as Thermodesulfobacteria and Thermodesulfovibrio were found among the entries with the highest score (e.g.,) based on the Literature channel. 
   The other two channels did not contain any associations. 
   Using the \textit{"Sulfate assimilation"} (\href{https://prego.hcmr.gr/example4}{example 4}\footnote{\href{https://prego.hcmr.gr/example4}{https://prego.hcmr.gr/example4}}, accessed on 24 December 2021) as the biological process input, PREGO results showed several genera that were missing from PREGO results concerning the \textit{"dissimilatory sulfate reduction"}. 
   Hence, manual search of GObp terms that describe the actual biological process of interest is more insightful.


   \subsection*{\textit{Are There Any Associations between Environments and Biological Processes?}}
   \label{subsec:envo-proc}

   Are there other environmental types, except the lagoonal sediments, in which sulfate assimilation occurs? In that question, and in \textit{"dissimilatory sulfate reduction"} (\href{https://prego.hcmr.gr/example3}{example 3}) in particular, PREGO assigns the highest score to “sediment” while, among others, \textit{"anoxic water"}, \textit{"oil reservoir"}, \textit{"mud volcano"}, and \textit{"basalt"} are potentially associated with environments related to sulfate reduction.
   
   Inversely, PREGO is insightful about occurring processes in a specific environmental type. 
   For example, searching for the biological processes that take place in \textit{"basalt"} (\href{https://prego.hcmr.gr/example5}{example 5}\footnote{\href{https://prego.hcmr.gr/example5}{https://prego.hcmr.gr/example5}}, accessed on 24 December 2021), processes like \textit{"Nitrogen fixation"} and \textit{"Reactive nitrogen species metabolic process"} stand out. 
   However, sulfate reduction is not among the associations. 
   However, when asking for \textit{"Mafic lava"} (\href{https://prego.hcmr.gr/example6}{example 6}\footnote{\href{https://prego.hcmr.gr/example6}{https://prego.hcmr.gr/example6}}, accessed on 24 December 2021), both the \textit{"nitrogen fixation"} and \textit{"Sulfur compound metabolic process"} terms are returned. 
   This highlights the suggestions of Pavloudi et al. (2017), regarding the potential use of various electron acceptors from the different strains present in different environmental types.


   \subsection{PREGO Contents}
   \label{subsec:prego-contents}

   PREGO contains the literature, environmental samples, and genome annotations of the resources shown in Table~\ref{table:prego1}. 
   The extracted contents of these resources have resulted to a knowledge base with ~364 K distinct taxonomic groups (out of a pool of ~$620$K Bacteria, Archaea, and microbial eukaryotes, based on NCBI Taxonomy) from which ~$258$K are at the species level (Table~\ref{table:prego2}). 
   These taxa are associated with ~1 K Environment Ontology terms, ~15 K GObp terms, and with ~7.9 K GOmf terms. 
   Combining the above, PREGO maintains a knowledge base of entities and associations between them that form a multipartite network with entities as nodes and scored associations between them as weighted links.


   % PREGO ENTITIES AFTER NER AND MAPPING - table:prego2
   \begin{table}[h]
      \begin{tabular}{@{}cccrccc@{}}
      \toprule
      \textbf{Channel} & \textbf{Source} & \multicolumn{2}{c}{\textbf{Taxonomy}} & \textbf{\begin{tabular}[c]{@{}c@{}}Environ- \\ ments\end{tabular}} & \textbf{\begin{tabular}[c]{@{}c@{}}Biological \\ Processes\end{tabular}} & \textbf{\begin{tabular}[c]{@{}c@{}}Molecular \\ Functions\end{tabular}} \\ \midrule
      \multirow{3}{*}{Literature} & \multirow{3}{*}{\begin{tabular}[c]{@{}c@{}}MEDLINE \\ PubMed - \\ PMC OA\end{tabular}} & Strains & 8,929 & \multirow{3}{*}{1,077} & \multirow{3}{*}{15,079} & \multirow{3}{*}{7,318} \\
      &  & Species & 240,377 &  &  &  \\
      &  & Total & 342,506 &  &  &  \\
      \multirow{9}{*}{\begin{tabular}[c]{@{}c@{}}Environ- \\ mental \\ samples\end{tabular}} & \multirow{3}{*}{\begin{tabular}[c]{@{}c@{}}MG-RAST \\ amplicon\end{tabular}} & Strains & 1,392 & \multirow{3}{*}{162} & \multirow{3}{*}{-} & \multirow{3}{*}{-} \\
      &  & Species & 4,324 &  &  &  \\
      &  & Total & 5,859 &  &  &  \\
      & \multirow{3}{*}{\begin{tabular}[c]{@{}c@{}}MG-RAST \\ metagenome\end{tabular}} & Strains & 2,522 & \multirow{3}{*}{258} & \multirow{3}{*}{-} & \multirow{3}{*}{3,839} \\
      &  & Species & 4,406 &  &  &  \\
      &  & Total & 7,157 &  &  &  \\
      & \multirow{3}{*}{\begin{tabular}[c]{@{}c@{}}MGnify \\ amplicon\end{tabular}} & Strains & 2 & \multirow{3}{*}{216} & \multirow{3}{*}{11} & \multicolumn{1}{l}{\multirow{3}{*}{-}} \\
      &  & Species & 1,471 &  &  & \multicolumn{1}{l}{} \\
      &  & Total & 2,955 &  &  & \multicolumn{1}{l}{} \\
      \multirow{9}{*}{\begin{tabular}[c]{@{}c@{}}Annotated\\Genomes \&\\Isolates\end{tabular}} & \multirow{3}{*}{JGI IMGisolates} & Strains & 2,398 & \multirow{3}{*}{241} & \multirow{3}{*}{-} & \multirow{3}{*}{3,670} \\
      &  & Species & 11,203 &  &  &  \\
      &  & Total & 13,849 &  &  &  \\
      & \multirow{3}{*}{STRUO} & Strains & 6 & \multirow{3}{*}{-} & \multirow{3}{*}{-} & \multirow{3}{*}{2,789} \\
      &  & Species & 19,289 &  &  &  \\
      &  & Total & 19,325 &  &  &  \\
      & \multirow{3}{*}{BioProject} & Strains & 5,754 & \multirow{3}{*}{309} & \multirow{3}{*}{626} & \multirow{3}{*}{-} \\
      &  & Species & 3,373 &  &  &  \\
      &  & Total & 9,393 &  &  &  \\
      \multirow{3}{*}{Total} & \multirow{3}{*}{All} & Strains & 12,840 & \multirow{3}{*}{1,090} & \multirow{3}{*}{15,091} & \multirow{3}{*}{7,971} \\
      &  & \multicolumn{1}{l}{} & \multicolumn{1}{l}{} &  &  &  \\
      &  & \multicolumn{1}{l}{} & \multicolumn{1}{l}{} &  &  &  \\ \bottomrule
      \end{tabular}

      \caption[The entities of PREGO after the NER and mapping of every source]{
         The entities of PREGO after the NER and mapping of every source. 
         Counts of distinct entities of Taxa, Environments (ENVO terms), Biological Processes (Gene Ontology Biological process) and Molecular Function (Gene Ontology Molecular Function).
      }
      \label{table:prego2}

   \end{table}


   As shown in Figure~\ref{fig:prego-entities}, in its current version (December 2021), PREGO knowledge base covers $157$ bacterial phyla ($107$ are Candidatus), $23$ phyla from archaea ($18$ are Candidatus), and $22$ unicellular eukaryotic phyla described in the NCBI Taxonomy database. 
   The number of bacterial taxa present among the associations of each phylum ranges from the order of $10$s, as in the case of \textit{Candidatus Coatesbacteria}, to hundreds of thousands, e.g., Actinobacteriae. 
   The number of environmental types, found among the PREGO associations for each phylum, ranges from just a few to up to $1000$. 
   Similarly, the number of biological processes that have been related to the various phyla may range from less than a dozen, e.g., Yanofskybacteria to up to several thousands, e.g., Bacteroidetes. On the contrary, the number of molecular functions found to be related to taxa of each phylum is rather constant in all three domains.

   \begin{figure}
      \centering
      \includegraphics[width=0.85\textwidth]{figures/figure_5_all_channels_ranks.png}  
      \caption[Summary of the unique entities per phylum for each of the four entity types on PREGO]{Summary of all the unique entities per phylum for each of the four entity types (in log10 scale) that appear in PREGO. Phyla are grouped based on their superkingdom (in log10 scale). Only phyla for which associations are available in the PREGO platform are mentioned.
      }
      \label{fig:prego-entities}
   \end{figure}




% RPEGO DISCUSSION
\section{Discussion}
\label{sec:prego-discussion}

   \subsection{PREGO Contents}
   \label{subsec:prego-contents-disc}

   On its current version and according to the NCBI Taxonomy that it is based on, PREGO manages to cover a great range of microbial taxa, as most (if not all phyla) are present in the knowledge base (Figure~\ref{fig:prego-entities}). 
   The different number of organisms' entities per phylum highlights the diverse number of the members of the various phyla. On the contrary, the similar number of molecular functions in all cases indicates the robustness of the main metabolic processes required for life. 
   With respect to biological processes, their number per phylum varies to some extent, especially for the case of Bacteria and Archaea. 
   That could be observed as, in many cases, phyla that have been recently described using molecular techniques have not been studied extensively yet, e.g., Candidatus Delongbacteria. 
   As expected, the number of environmental types that have been associated with members of each phylum varies, as a phylum may be universally present, while others could be strongly niche-specific (e.g., Hydrothermarchaeota).

   Because of its three different channels, PREGO manages to extract associations both in the species and higher taxonomic levels. The Isolates channel supports explicit associations at the species level (Table~\ref{table:prego3} and Figure S3). 
   Interestingly, the number of such genomes seems to have reached a plateau for now, as PREGO-like platforms include the same order of magnitude. 
   The \textit{Literature} channel, on the other hand, promotes the extraction of associations at higher taxonomic levels (Table~\ref{table:prego3} and Figure S1). 
   This also applies to environment—organisms associations derived from the Environmental Samples channel (Table~\ref{table:prego3} and Figure S2). Associations regarding biological processes, though, are strongly enhanced by the Literature channel and the massive increase of literature.


   % ASSOCIATIOS TABLE - table:prego3
   \begin{sidewaystable}
      \begin{tabular}{@{}cccccrcr@{}}
      \toprule
      \textbf{Channel} & \textbf{Source} & \textbf{\begin{tabular}[c]{@{}c@{}}Environments \\ - \\ Processes\end{tabular}} & \textbf{\begin{tabular}[c]{@{}c@{}}Environments \\ - \\ Functions\end{tabular}} & \textbf{Taxonomy} & \multicolumn{1}{c}{\textbf{\begin{tabular}[c]{@{}c@{}}Taxa \\ -\\  Environments\end{tabular}}} & \textbf{\begin{tabular}[c]{@{}c@{}}Taxa \\ - \\ Processes\end{tabular}} & \multicolumn{1}{c}{\textbf{\begin{tabular}[c]{@{}c@{}}Taxa \\ -\\ Function\end{tabular}}} \\ \midrule
      \multirow{3}{*}{Literature} & \multirow{3}{*}{\begin{tabular}[c]{@{}c@{}}MEDLINE \\ PubMed - \\ PMC OA\end{tabular}} & \multirow{3}{*}{883,997} & \multirow{3}{*}{422,579} & Strains & 69,968 & \multicolumn{1}{r}{590,630} & 384,079 \\
      &  &  &  & Species & 778,877 & \multicolumn{1}{r}{3,501,635} & 1,961,920 \\
      &  &  &  & Total & 1,669,608 & \multicolumn{1}{r}{7,969,310} & 4,613,827 \\
      \multirow{9}{*}{\begin{tabular}[c]{@{}c@{}}Environmental \\ samples\end{tabular}} & \multirow{3}{*}{\begin{tabular}[c]{@{}c@{}}MG-RAST \\ amplicon\end{tabular}} & \multirow{3}{*}{-} & \multirow{3}{*}{-} & Strains & 13,645 & \multirow{3}{*}{-} & \multicolumn{1}{c}{\multirow{3}{*}{-}} \\
      &  &  &  & Species & 39,007 &  & \multicolumn{1}{c}{} \\
      &  &  &  & Total & 53,439 &  & \multicolumn{1}{c}{} \\
      & \multirow{3}{*}{\begin{tabular}[c]{@{}c@{}}MG-RAST \\ metagenome\end{tabular}} & \multirow{3}{*}{-} & \multirow{3}{*}{620,846} & Strains & 262,106 & \multirow{3}{*}{-} & 8,626,328 \\
      &  &  &  & Species & 103,913 &  & 10,715,548 \\
      &  &  &  & Total & 372,301 &  & 19,950,096 \\
      & \multirow{3}{*}{\begin{tabular}[c]{@{}c@{}}MGnify \\ amplicon\end{tabular}} & \multirow{3}{*}{-} & \multirow{3}{*}{-} & Strains & 18 & - & \multicolumn{1}{l}{} \\
      &  &  &  & Species & 30,122 & \multicolumn{1}{r}{351} & \multicolumn{1}{c}{-} \\
      &  &  &  & Total & 111,976 & \multicolumn{1}{r}{2,097} & \multicolumn{1}{l}{} \\
      \multirow{9}{*}{\begin{tabular}[c]{@{}c@{}}Annotated Genomes \\ and Isolates\end{tabular}} & \multirow{3}{*}{\begin{tabular}[c]{@{}c@{}}JGI IMG\\ isolates\end{tabular}} & \multirow{3}{*}{-} & \multirow{3}{*}{-} & Strains & 8,229 & \multirow{3}{*}{-} & 3,461,693 \\
      &  &  &  & Species & 42,141 &  & 13,216,559 \\
      &  &  &  & Total & 50,888 &  & 16,821,850 \\
      & \multirow{3}{*}{STRUO} & \multirow{3}{*}{-} & \multirow{3}{*}{-} & Strains & \multicolumn{1}{c}{\multirow{3}{*}{-}} & \multirow{3}{*}{-} & 1,803 \\
      &  &  &  & Species & \multicolumn{1}{c}{} &  & 4,070,195 \\
      &  &  &  & Total & \multicolumn{1}{c}{} &  & 4,079,312 \\
      & \multirow{3}{*}{BioProject} & \multirow{3}{*}{-} & \multirow{3}{*}{-} & Strains & 3,263 & \multicolumn{1}{r}{7,473} & \multicolumn{1}{l}{} \\
      &  &  &  & Species & 4,187 & \multicolumn{1}{r}{4,294} & \multicolumn{1}{l}{} \\
      &  &  &  & Total & 7,641 & \multicolumn{1}{r}{12,169} & \multicolumn{1}{l}{} \\
      \multirow{3}{*}{Total} & \multirow{3}{*}{All} & \multirow{3}{*}{883,997} & \multirow{3}{*}{1,043,425} & Strains & 357,229 & \multicolumn{1}{r}{598,103} & 12,473,903 \\
      &  &  &  & Species & 998,247 & \multicolumn{1}{r}{3,506,280} & 29,964,222 \\
      &  &  &  & Total & 2,265,853 & \multicolumn{1}{r}{7,983,576} & 45,465,085 \\ \cmidrule(l){5-8} 
      \end{tabular}
      \caption[Associations among the PREGO entities]{
         The associations between entities of PREGO after co-occurrence analysis: The supported entity types of associations are Environments—Biological Processes, Environments—Molecular Functions, Taxa—Environments, Taxa—Biological Processes, Taxa—Molecular Functions.
      }
      \label{table:prego3}
   \end{sidewaystable}

   Additionally, the text mining methodology of the Literature channel has retrieved most of the entities present in PREGO knowledge base (Table~\ref{table:prego2}). 
   A significant contribution to the taxa with associations is due to the PMC OA processing by the text mining pipeline of the Literature channel. 
   This is in-line with reports in other applications of text mining when using full text articles \cite{westergaard2018comprehensive}. 
   However, the resulting associations are suggestive because of the text mining nature, and therefore subject for further review by the users.

   \subsection{Related Tools' Functionality and Content}
   \label{subsec:prego-similar-platforms}

   There is an emerging niche for tools similar to PREGO to bring forward microbe associations and metadata. 
   Table~\ref{table:prego4} summarizes the common and different features of BacDive, WoM, NMDC data portal, and PREGO. 
   All of them commonly share the environmental associations and biological/metabolic processes with the microbes.

   BacDive is a well-established platform with a focus on phenotype and cultivation information for about 100,000 prokaryotes, bacteria, and archaea. 
   It has a high level of curation for most of its input types, like literature, internal databases, and personal collections. 
   The NMDC data portal has published the scheme, the user interface, and a demonstrative collection of samples that will be populated later on. Standout features are the spatial visualization with coordinates and the detailed information of the samples, e.g., sequencing instruments and methodology. 
   An alternative approach is facilitated by WoM, which aims to bind chemistry to microbes. An environment, in particular, is defined as the starting metabolite pool that is transformed by an organism.
   Another tool is The Microbe Directory that contains fully curated metadata for about 8000 microbes from all superkingdoms. This tool focuses on conditions of growth and on host taxa.

   Complementary to these tools, PREGO contains associations of bacteria, archaea, and eukaryotes. Distinctive features are the associations of environments with processes/functions and the large-scale literature integration with text mining. Most importantly, most of the tools are complementary to each other with minimum overlap, an indication of the opportunities for further innovative synergies.


   % PREGO COMPARISONS - table:prego4
   \begin{table}[h]

      \begin{adjustwidth}{-0.75cm}{}

      \begin{tabular}{@{}lllll@{}}
      
      \toprule
      Functionality & BacDive & Web of Microbes & NMDC & PREGO \\ \midrule
      manual curation & high & high & intermediate & low \\ 

      literature integration & limited & no & no & yes \\

      environment—taxa associations & yes & yes & yes & yes \\

      \begin{tabular}[c]{@{}l@{}}environment—process/\\ function associations\end{tabular} & no & no & no & yes \\

      process/function—taxa associations & yes & yes & yes & yes \\
      phenotypic data & yes & no & no & no \\

      data origin & \begin{tabular}[c]{@{}l@{}} original \\integration \end{tabular} & original & \begin{tabular}[c]{@{}l@{}} original \\integration \end{tabular} & integration \\

      spatial coordinates & yes & no & yes & no \\

      application programming interface & yes & no & yes & yes \\

      bulk download & limited & yes & yes & yes \\ \bottomrule

      \end{tabular}
      \end{adjustwidth}
      \caption[Feature comparison between PREGO and other similar platforms]{Feature comparison among platforms that facilitate knowledge discovery and integration of microbial data.}
      \label{table:prego4}
   
   \end{table}      

   \subsection{PREGO Next Steps}
   \label{subsec:prego-next-steps}

   PREGO is a user-friendly association mining and sharing platform. 
   Its modular web-architecture grants it the flexibility for further improvements in the aforementioned aspects, namely: 
   source datasets, user interface, entity, and association scope expansion. Regarding datasets, additional data, such as transcriptomes from MGnify and other records annotated with metadata from studies in \href{https://ebi-metagenomics.github.io/blog/2021/11/17/Publication-Annotations/}{EuroPMC}, accessed on 24 December 2021) \cite{ferguson2021europe}, could be incorporated. 
   Similarly, the \href{https://data.microbiomedata.org/}{NMDC data platform standards-compliant annotated records}\footnote{\href{https://data.microbiomedata.org/}{https://data.microbiomedata.org/}} (accessed on 24 December 2021) could serve as an additional resource with its high-quality metadata \cite{wood2020national, vangay2021microbiome}. 
   Reciprocally, if requested, pertinent literature and association summaries could be programmatically offered to interested third parties.

   Furthermore, the entity types supported by the PREGO system could be expanded. For example, GOmf terms could be upgraded as a search-entry point to the system. 
   In addition, disease and tissue describing terms, already supported by the PREGO-underlying EXTRACT system \cite{pafilis2016extract}, could enter the PREGO ecosystem of associated entities. 
   From a statistics perspective, the calculation of a combined association score, when an association is reported by more than one channel of information, could be another feature to add.

   The user interface can be enhanced to support multiple entity and/or sequence queries, instead of single ones. 
   Sequences can be processed by taxonomy assignment pipelines (e.g., PEMA \cite{zafeiropoulos2020pema}) and be converted into searching PREGO for associations. 
   In addition, network visualization tools, like Arena3Dweb \cite{karatzas2021arena3dweb}, could allow interactive browsing of associations through multi-layered graphs. 
   Enrichment analyses, like those performed by OnTheFly2.0 \cite{baltoumas2021onthefly2} or Flame \cite{thanati2021flame}, 
   can be incorporated. 
   Omics data analysis pipelines, like MiBiOmics \cite{zoppi2021mibiomics}, environment associations with sequences using SeqEnv \cite{sinclair2016seqenv} and biogeochemical associations with metagenomic data with DiTing~\cite{xue_diting_2021} could be enabled by comparing the associations pertinent to different groups of entities. 
   The computationally intensive tasks of multiple queries, taxonomy assignments to sequences and enrichment analysis could be offered by our in-house High Performance Computing facility (https://hpc.hcmr.gr/, accessed on 24 December 2021) \cite{zafeiropoulos_0s_2021} in synergy with pertinent Research Infrastructures like \href{https://elixir-europe.org}{ELIXIR}\footnote{\href{https://elixir-europe.org}{https://elixir-europe.org}} (accessed on 24 December 2021) and \href{https://www.lifewatch.eu/}{LifeWatch ERIC}\footnote{\href{https://www.lifewatch.eu/}{https://www.lifewatch.eu/}} (accessed on 24 December 2021).



   \subsection*{Availability of Supporting Source Codes:} 
   The PREGO software modules are available under BSD 2-Clause “Simplified” License. Scripts, where additional libraries have been used, are subject to their individual licenses. More information on each module can be found as listed below:
   
   \begin{itemize}
      \item prego\_gathering\_data 
      \href{https://github.com/lab42open-team/prego_gathering_data}{github.com/lab42open-team/prego\_gathering\_data}
      \item prego\_daemons \href{https://github.com/lab42open-team/prego_daemons}{github.com/lab42open-team/prego\_daemons}
      \item prego\_mappings \href{https://github.com/lab42open-team/prego_mappings}{github.com/lab42open-team/prego\_mappings} 
      \item prego\_statistics \href{https://github.com/lab42open-team/prego_statistics}{github.com/lab42open-team/prego\_statistics}
   \end{itemize}

   Additional software and curated lists along with their individual license are:
   \begin{itemize}
      \item tagger:	\href{https://github.com/larsjuhljensen/tagger}{https://github.com/larsjuhljensen/tagger}, BSD 2-Clause "Simplified" License
      \item mamba: \href{https://github.com/larsjuhljensen/mamba}{https://github.com/larsjuhljensen/mamba}, BSD 2-Clause "Simplified" License 
      \item tagger dictionary:  \href{https://download.jensenlab.org/}{https://download.jensenlab.org/} and there in: \\
      \href{https://download.jensenlab.org/prego_dictionary.tar.gz}{https://download.jensenlab.org/prego\_dictionary.tar.gz}, CC-BY 4.0 license
   \end{itemize}


% --------------------------------------------------
% 
% This chapter is for PREGO
% 
% --------------------------------------------------


\chapter{Harmonising the literature and metagenomic resources to infer microbial, environmental and functional relations}
\label{cha:prego}


\textbf{Citation:} \\ 
Zafeiropoulos H, Paragkamian S, Ninidakis S, Pavlopoulos GA, Jensen LJ, Pafilis E. PREGO: A Literature and Data-Mining Resource to Associate Microorganisms, Biological Processes, and Environment Types. Microorganisms. 2022; 10(2):293. \\ 
Shared co-first authorship.

\textbf{Journal:} Microorganisms

This article belongs to the Special Issue Selected Papers from the 9th Conference of the Hellenic Scientific Society MIKROBIOKOSMOS \textit{Beneficial Microbes at the Heart of Mikrobiokosmos}

DOI: \href{https://doi.org/10.3390/microorganisms10020293}{10.3390/microorganisms10020293}\footnote{
   For author contributions and supplementary material please refer to the relevant sections. 
   This is a modified version of the published version,
   in terms of relevance, coherence and formatting.}



% PREGO ABSTRACT
%\section{Abstract}
%
%   To elucidate ecosystem functioning, it is fundamental to recognize \textit{what} processes occur in which environments (\textit{where}) and which microorganisms carry them out (\textit{who}). 
%   Here, we present PREGO, a one-stop-shop knowledge base providing such associations. 
%   PREGO combines text mining and data integration techniques to mine such what-where-who associations from data and metadata scattered in the scientific literature and in public omics repositories. 
%   Microorganisms, biological processes, and environment types are identified and mapped to ontology terms from established community resources. 
%   Analyses of comentions in text and co-occurrences in metagenomics data/metadata are performed to extract associations and a level of confidence is assigned to each of them thanks to a scoring scheme. 
%   The PREGO knowledge base contains associations for 364,508 microbial taxa, 1090 environmental types, 15,091 biological processes, and 7,971 molecular functions with a total of almost 58 million associations. 
%   These associations are available through a web interface (\href{https://prego.hcmr.gr}{https://prego.hcmr.gr}), an Application Programming Interface (API), and bulk download. 
%   By exploring environments and/or processes associated with each other or with microbes, PREGO aims to assist researchers in design and interpretation of experiments and their results. 
%   To demonstrate PREGO's capabilities, a thorough presentation of its web interface is given along with a meta-analysis of experimental results from a lagoon-sediment study of sulfur-cycle related microbes.
%
% RPEGO INTRODUCTION
\section{Introduction}
\label{sec:prego-intro}

   Microbes are omnipresent and impact global ecosystem functions \parencite{falkowski2008microbial} through their abundance \parencite{bar2018biomass}, versatility \parencite{delgado2016microbial}, and interactions \parencite{rottjers2018hairballs}. 
   These facts have inspired microbiologists from diverse scientific fields to study their genotype and phenotype \parencite{morris2020linking}, their metabolism \parencite{biggs2015metabolic}, and their interactions with the environment \parencite{hall2018understanding}. 
   All this work has resulted in a wealth of knowledge available in the forms of literature and experimental data. Literature contains vast amounts of information in the free text form that overwhelms researchers. 
   Advanced text mining methods \parencite{jensen2006Literature} have been developed to assist this issue. 
   Experimental data and their metadata require mining \parencite{delmont2011metagenomic} as well for their integration, mostly through metagenomic mining from online repositories. 
   Hence, the combination of this knowledge about microbial life (who), their metabolic functions (what), and the environment they influence (where) is an important step to study ecosystem function \parencite{raes2008Molecular}.

   High Throughput Sequencing (HTS) has turned the page on microbial ecology studies \parencite{nilsson2019mycobiome}. 
   Over the past 20 years, both the taxonomic and the functional profiles of microbial communities from both local and large-scale regions (e.g., Tara Oceans \parencite{pesant2015open}, Earth Microbiome \parencite{gilbert2014earth}) are being accumulated at a higher and higher rate. 
   Extreme environments, i.e., areas with high salinity, low pH, etc., are being studied, providing us with unprecedented insight \parencite{shu2021microbial}. 
   Both amplicon and shotgun metagenomics studies have played a crucial part in this development. Latest technological breakthroughs, such as Metagenome-Assembled Genomes (MAGs) and Single Amplified Genomes (SAGs), are enhancing the assessment of the taxonomic and functional repertoire of microbiomes even further. 
   However, the mass use of these technologies and their consequent data have led to a number of needs and challenges, with metadata curation being among the most crucial ones.

   Standards-promoting communities, like \href{https://gensc.org/}{Genomic Standards Consortium (GSC)}\footnote{
      \href{https://gensc.org/}{https://gensc.org/}
   }, their efforts, like Minimum Information about any (x) Sequence (MIxS) \parencite{yilmaz2011minimum}, and projects endorsing those, like National Microbiome Data Collaborative (NMDC) \parencite{wood2020national, vangay2021microbiome}, offer guidelines and best-practices to assist the annotation of microbial ecology samples. 
   Controlled vocabularies and ontologies contribute to these efforts as they describe each subject area with formal terms \parencite{walls2014semantics}. 
   Environment types, for example, are described by the Environment Ontology (ENVO) \parencite{buttigieg2016environment}. 
   Other key biological aspects that have been captured include molecular functions (Gene Ontology Molecular Function (GOmf) \parencite{ashburner2000gene, gene2021gene}, Enzyme Commission nomenclature \parencite{noauthor_1999}, etc.), and the pathways carrying out different biological processes (GO Biological Process (GObp), MetaCyc \parencite{caspi2020metacyc}, etc.). 
   These knowledge structures, along with taxonomic centralized resources like the National Center for Biotechnology Information (NCBI) Taxonomy \parencite{schoch2020ncbi}
   and LPSN (List of Prokaryotic names with Standing in Nomenclature) \parencite{parte2020list}, 
   provide the means for a standardized representation of, for example, environments, process-oriented terms, and microbial taxa, respectively. Global-scale public resources (like MGnify \parencite{mitchell2020mgnify}, 
   JGI/IMG \parencite{chen2021img}, 
   MG-RAST \parencite{wilke2015restful}) 
   combine some of the aforementioned resources to support the collection, analysis, and distribution of multiple types of HTS data (e.g., amplicon, metagenomics, metatranscriptomics, etc.).

   Besides the data and the analyses \textit{per se}, the related scientific literature stores valuable information in billions of text lines. 
   PubMed \parencite{schoch2020ncbi} and PubMed Central (PMC) \parencite{roberts2001pubmed} are gateways to relationships among microbes (\textit{who}), the environments they live in (\textit{where}) and their associated processes and functions (\textit{what}) hidden in text \parencite{harmston2010papers}. 
   Text mining (on both literature and metadata) can serve the extraction of these relationships. Named Entity Recognition (NER) can, for example, locate organism names \parencite{pafilis2013species}, ENVO and GO terms \parencite{pafilis2016extract} mentioned in text and map them to their corresponding identifiers. 
   Association statistics, like co-mention analysis, can subsequently suggest ranked association among such entities \parencite{von2005string, franceschini2012string}. 
   The new era of omics has been interwoven with data integration \parencite{gomez2014data} by bringing together scattered and fragmented pieces of information.

   The time is ripe for tools that integrate all this knowledge and henceforth assist researchers to tackle major challenges like climate change \parencite{cavicchioli2019scientists}, 
   sustainability \parencite{d2021microbiome}, 
   and synthetic ecology \parencite{conde2020synthetic}.
   Many resources have emerged in this realm \parencite{baltoumas2021biomolecule}, 
   each one serving a specific purpose, such as BacDive \parencite{reimer2019bac}. 
   BacDive is a large-scale curated database with prokaryotic information about phenotypic, morphological, and metabolic information. 
   Other resources like Microbe Directory \parencite{shaaban2018microbe}, 
   Web of Microbes (WoM) \parencite{kosina2018web}, 
   and Microbial Interaction Network Database (\href{http://www.microbialnet.org/mind_home.html}{MIND}\footnote{
      \href{http://www.microbialnet.org/mind_home.html}{http://www.microbialnet.org/mind\_home.html}
   }) focus on microbial environmental conditions, metabolite interactions with microbes and microbe-microbe interactions, respectively. 
   In addition, taking advantage of aforementioned resources, novel pipelines, e.g., \parencite{tang2020tripartite}, are emerging with the aim to explore the network associations of who (microbial taxa) is performing what (microbial processes) and where (environments) directly using graph theory \parencite{koutrouli2020guide}. 
   These analyses and resources are important because microbiologists can enrich their data to explore hypotheses but also to identify potential gaps in knowledge regarding these associations \parencite{li2021microbial}.

   Here, we present PREGO, a hypothesis generation web resource that is designed to be useful to microbiologists—in particular microbial ecologists and environmental microbiologists. 
   Its specific aims include: 
   (a) the gathering of source data, metadata, and literature followed by the extraction of microorganism, process, environment associations contained therein, 
   (b) making such a mined knowledge base available to life sciences researchers via an easy to use and explore web portal. 
   As such, PREGO can be useful also to system microbiologists and large-scale data analysts through bulk download and programming access. 
   We document the principles, analysis methodology, and contents behind PREGO. 
   Last but not least, we demonstrate PREGO's capabilities for researcher-support related to the above through a case study involving sulfate-reducing microorganisms.



% RPEGO METHODS
\section{Methods \& Implementation}
\label{sec:prego-methods}

   PREGO is a resource designed to assist molecular ecologists in acquiring a single point overview of what-where-who process–environment–organism associations. The system is comprised of two main parts: (a) a server that periodically harvests data and extracts process-environment-organism associations from the scientific literature, environmental samples, and genome annotation sequences (Figure~\ref{fig:prego-pipeline}, step 1 to 5) and (b) a web-based interface as well as an Application Programming Interface (API) that provides users and programmers with a friendly way to extract and navigate across the process–environment–organism associations (Figure~\ref{fig:prego-pipeline}, step 6).

   \begin{figure}[h]
      \centering
      \includegraphics[width=0.98\columnwidth]{figures/prego_analysis_horizontal.png}
      \caption[PREGO analysis methodology]{
         PREGO analysis methodology: PREGO periodically retrieves three distinct types of data from open access resources. 
         Scientific text, environmental sample data, and genomic annotations are handled with respective methodologies in order to standardize their entities. 
         Named Entity Recognition and Comention/Co-occurrence analysis is the common framework in order to build a weighted association network with nodes being the entity identifiers. Lastly, all these associations are available through a Web interface and an API. 
         All these steps have been implemented in an autonomous way with regular cycles of updates (see Appendix~\ref{app:B}). 
         Icons used from the Noun Project released under CC BY: Books by Shakeel Ch., Bacteria by Maxim Kulikov, ftp by DinosoftLab, Mountain by Diane, Ship on Sea by farra nugraha, River by Chanut is Industries.
      }
      \label{fig:prego-pipeline}
   \end{figure}



   \subsection{Entity Types, Channels, and Associations}
   \label{subsec:prego-terms}

   PREGO supports three entity types: \textit{Process}, \textit{Environment}, and \textit{Organism}. 
   For interoperability and consistency, an ontology or taxonomy is adopted for each type of entity.
   Processes are represented as Gene Ontology (GO) terms and are grouped either as Biological processes (GObp) or as Molecular functions (GOmf). 
   In addition, Environments are represented by terms from the Environmental Ontology. 
   Organisms are represented by the microbial NCBI Taxonomy Ids (Bacteria, Archaea, and unicellular eukaryotes). For the unicellular eukaryotes, a custom list was populated with the unicellular eukaryotic taxa using a curated list.
   PREGO's contents are mainly divided into three distinct channels of information based on data origin and format (Figure~\ref{fig:prego-pipeline}, step 1). 
   The \textit{Literature} channel exploits scientific publications, i.e., abstracts and full text open access scientific publications (Table~\ref{table:prego1} and Section~\ref{subsec:prego-tm}). 
   Through the \textit{Annotated Genomes and Isolates} channel, PREGO retrieves genome annotations and their accompanying metadata (Table~\ref{table:prego1} and Section~\ref{subsec:prego-isolates}). 
   Finally, the \textit{Environmental Samples} channel supports the integration of metagenomic analyses from both amplicon and shotgun studies. 
   These include taxonomic and functional profiles along with their corresponding metadata (Table~\ref{table:prego1}, more details in Section~\ref{subsec:prego-envsamples}).


   % SOURCE DATABASES TABLE table:prego1
   \begin{table}[ht]
      \begin{tabular}{@{}ccccc@{}}
      \toprule
      \textbf{Source} & \textbf{\# items} & \textbf{Data type} & \textbf{Metadata} & \textbf{License} \\ \midrule
      \begin{tabular}[c]{@{}c@{}}MEDLINE and \\ PubMed\end{tabular} & 33 million & abstracts (text) & no & NLM Copyright \\
      \begin{tabular}[c]{@{}c@{}}PubMed Central \\ OA Subset\end{tabular} & 2.7 million & full article (text) & no & \begin{tabular}[c]{@{}c@{}}CC for Commercial, \\ non-commercial\end{tabular} \\
      JGI IMG & 9,644 & \begin{tabular}[c]{@{}c@{}}Isolates Annotated \\ genomes\end{tabular} & yes & JGI Data Policy \\
      Struo & 21,276 & Annotated genomes & no & MIT, CC BY-SA 4.0 \\
      BioProject & 18,752 & \begin{tabular}[c]{@{}c@{}}Annotated genomes \\ with abstracts (text)\end{tabular} & yes & INSDC policy \\
      \multirow{2}{*}{MG-RAST} & 16,096 & markergene samples & yes & CC0 \\
      & 7,965 & metagenomic samples & yes & CC0 \\
      MGnify & 10,500 & markergene samples & yes & CC-BY, CC0 \\ \bottomrule
      \end{tabular}
      \caption[Source databases integrated in PREGO and the number of items retrieved]{
         Source databases that are integrated in PREGO and the number of items retrieved. 
         The Open Access subset of PubMed Central has a Creative Commons license available for commercial and noncommercial use. 
         JGI has its own license, the same applies for BioProject, MEDLINE®, and PubMed® as well.
         }
         \label{table:prego1}
   \end{table}


   In cases in which the retrieved data and metadata are in text form, they are standardized to the aforementioned identifiers and taxonomies using Named Entity Recognition (NER) tools, namely the EXTRACT tagger \parencite{pafilis2016extract, jensen2016one}. 
   In cases where data contain KEGG Orthology terms and/or Uniref identifiers, they are mapped to the respective GOmf using the mapping files available from the UniProt (see Appendix~\ref{app:A}). 
   Associations are extracted after the mapping and standardization of the entities from each resource (Figure~\ref{fig:prego-pipeline}, step 3).
   The association extraction pipeline is distinct for each channel and resource because of differences in the data type origin (see \texttt{prego\_gathering\_data} in the Availability of Supporting Source Codes section). 
   By the means of navigation, the large number of associations returned to the user require a type of sorting; 
   ideally, one that ranks the most trustworthy associations at the top. 
   For those reasons, each channel of PREGO has a dedicated scoring scheme bounded within the (0,5] space for consistency. 
   In Appendix ~\ref{app:A} , the scoring scheme of each channel is elaborated.




   \subsection{Text Mining of Scientific Literature}
   \label{subsec:prego-tm}

   PREGO implements a text mining methodology to extract associations of the aforementioned entities from literature. 
   The origin of text mining is a corpus that comprises scientific abstracts and full text articles from MEDLINE® and PubMed® and PubMed Central® Open Access Subset (PMC OA Subset) \parencite{sayers2021database}, respectively. 
   The building and periodic update of the corpus is possible through the NCBI File Transfer Protocol (FTP) services. 
   PREGO also has a dedicated text-mining dictionary (see Availability of Supporting Source Codes section) that contains the entities ids, names, synonyms, and neglected words (stop words). 
   PREGO dictionary incorporates the ORGANISMS \parencite{pafilis2013species} and ENVIRONMENTS \parencite{pafilis2015environments} evaluated dictionaries as well as the experimental dictionaries of Gene Ontology Biological Process and Molecular Function.
   Text mining is subsequently performed on the corpus using the dictionary through the EXTRACT tagger \parencite{pafilis2016extract, jensen2016one}. 
   The tagger recognizes the entities of the dictionary in each abstract and full text article and assigns their co-mentions with a score. 
   The score is sensitive to the text structural level of co-mention; higher to lower scoring when co-mention appears in the same sentence, then, in the same paragraph, and lastly in the same article. 
   All these are integrated and normalized to a single score for each association, as implemented in STRING 9.1 \parencite{franceschini2012string} (see Appendix ~\ref{app:C} for more details). 
   In addition, the tagger extracts each mention in every article to provide the origin of each association it extracts.


   \subsection{Annotated Genomes and Isolates}
   \label{subsec:prego-isolates}

   Annotated genomes and isolates comprise the most trustworthy data in PREGO's knowledge base because they refer to a single species/strain and also have manually curated metadata. 
   Among other data types, JGI-IMG \parencite{chen2021img, mukherjee2021genomes} includes millions of genes from isolated genomes (isolates), SAGs and MAGs. Such annotations, along with their corresponding metadata, were collected using web-parsing technologies. Their metadata, describing their related environment/ecosystem, were tagged using the EXTRACT tagger to infer organisms—environments associations. The annotated KEGG terms were mapped to GOmf terms (see Appendix A). The GOmf terms were then used to extract organisms—processes associations.
   
   The Struo pipeline \parencite{de2020struo} and its outcome when using the Genome Taxonomy DataBase (GTDB) (v.03-RS86) \parencite{parks2020complete} was exploited to enrich organisms—processes associations. 
   A set of 21,276 representative genomes, accompanied by UniRef50 annotations, was retrieved using the provided FTP server. The annotations were then mapped to GOmf terms (see Appendix~\ref{app:A}). 
   Related GTDB genomes were mapped to their corresponding NCBI taxa (see Appendix~\ref{app:A}). 
   All associations extracted from these resources were assigned arbitrarily a confidence level of four out of five. 
   This score choice reflects the high-quality of these data and metadata.
   
   In addition, BioProject data were integrated to PREGO using the NCBI FTP/e-utils services \parencite{sayers2021database}. 
   The BioProject ids that were integrated are the ones that have been assigned a PubMed abstract, a unicellular taxon, and Genome sequencing as data type. Then, using the text mining pipeline, associations were extracted connecting the assigned taxon with the rest of the entities that appear in the abstracts. This method resulted in associations that were assigned a confidence level of three (out of five) because of the combined method of curated data with text mining.


   \subsection{Environmental Samples}
   \label{subsec:prego-envsamples}

   MGnify \parencite{mitchell2020mgnify} and MG-RAST \parencite{wilke2015restful} repositories provide a great number of public metagenomic records. 
   In the PREGO framework, both amplicon and shotgun metagenomic analyses are retrieved periodically along with their corresponding metadata. 
   Data retrieval from these resources is possible from their Application Programming Interfaces (APIs). Marker gene analyses are retrieved and by measuring
   the co-occurrence of taxa present in the various environmental types (e.g., biomes, materials, features, etc.) organisms—environments associations are extracted. 
   These associations emerge when a taxon is reported together with a certain environmental type, being mentioned in the metadata of a sample (metadata based co-occurrence). 
   Similarly, analyses of metagenomic samples along with their corresponding metadata and annotations are also retrieved and organisms—environments, organisms—processes and processes—environments are extracted. 
   The processes—environments associations are possible through co-occurrence of the functional annotations of metagenomes with the environmental metadata of the samples.
   
   In all cases, the EXTRACT tagger is used on the microorganism names and the corresponding metadata of each sample to identify their identifiers (NCBI ids, ENVO terms, GOmf, GObp). 
   All associations in this channel are scored based on the number of samples the entity of interest co-occurs with specific sample metadata (e.g., environmental type) or annotations (functional annotations or taxonomic annotations). 
   The same scoring scheme was implemented across the channel resources (see Appendix~\ref{app:C} for more details), which ranks these associations with a value in the (0,5] space.


   \subsection{Sequence Search}
   \label{subsec:prego-seq-search}

   In the case of organisms, PREGO enables sequence-based queries, meaning a sequence (amplicon) can be used as an entry point like it was a taxon name. 
   To this end, a custom database was built using a set of reference custom databases for four commonly used marker genes. For 16S and 18S rRNA, the SILVA database (v.138) \parencite{quast_silva_2013} and the PR2 database (version\_4.14.0) \parencite{guillou2012protist, del2018eukref} were used. 
   Cytochrome c oxidase I (COI) \parencite{suter2021capturing} is another commonly used marker gene; 
   for this reason, Midori 2 (version GB243) \parencite{leray2018midori} was integrated in PREGO's custom database. 
   Finally, for the Internal transcribed spacer (ITS), common in studies focusing on Fungi, the Unite (version 8.3, accessed 10.05.2021) \parencite{nilsson2019unite} database was added.



   \subsection{Back-End Server and Front-End Implementation}
   \label{subsec:prego-back-end}

   PREGO is a multi-tier web-based application. 
   It is hosted on a 64 GB RAM DELL R540, 20 core, Debian server. 
   Custom API clients (written in Python) are responsible for retrieving the data and metadata from each source (Figure~\ref{fig:prego-pipeline}, step 2). 
   These clients as well as the subsequent methodology (Figure~\ref{fig:prego-pipeline}, step 3 to 6) are updated in regular cycles using custom daemons (see Appendix~\ref{app:B}, Figure~\ref{fig:devops}). 
   The \textit{mamba/blackmamba} web framework underlies communication to the Postgres association-holding database and the client-side communication. 
   HTML 5, Ajax, JQuery, and custom Javascript enhance the user web experience. 
   PREGO supports widely used browsers (e.g., Chrome, Firefox, Safari, Edge) in various operating systems, such as Windows 10, Linux (Ubuntu 18), and MacOS (10.12, 11).


% RPEGO RESULTS
\section{Results \& Validation}
\label{sec:prego-results}

   \subsection{The PREGO Web Resource}
   \label{subsec:prego-web-resource}

   Users can access the PREGO contents through its web User Interface (UI) (Figures~\ref{fig:prego_ui} and \ref{fig:prego_ui_resources}), its Application Programming Interface (API) (Figure~\ref{fig:prego_api_schema}), or bulk download of all associations (Appendix D). 
   The User Interface comes with two search fields: a plain text search and a sequence search (Figure~\ref{fig:prego_ui}a). 
   The latter is used when the user wants to search for a taxon sequence (see Section~\ref{subsec:prego-seq-search} for supported sequence databases). 
   The plain text search supports three types of entry points; the user can search for a taxon name, e.g., \textit{Methanosarcina mazei}, an environmental type, e.g., lagoon, or a biological process e.g., methanogenesis. 
   In all entry points, PREGO returns an overview page consisting of tabs with associations of the entity of interest with the entities of the two other types (Figure~\ref{fig:prego_ui}b–d) as well as Documents and Downloads tabs (Figure~\ref{fig:prego_ui}e,f).


   \begin{figure}[h]
      \centering
      \includegraphics[width=0.98\columnwidth]{figures/prego_ui.png}
      \caption[PREGO web user interface]{
         PREGO web user interface. 
         (a) There are two search fields, plain text and taxa sequences. 
         (b-d) three associations tabs each one presenting associations of the querred entity with the respective entities, Environments (b), Biological Process (c ) and Molecular Function (d). 
         Three channels of information are distinguishing the associations based on the original data. 
         (e) Documents tab presents the scientific articles that mention the queried entity highlighted with color. 
         (f) Downloads tab provides the associations of each channel (when available) to be downloaded in JSON and TSV format.
      }
      \label{fig:prego_ui}
   \end{figure}


   Regarding the association tabs, when a taxon is used as a query, PREGO returns an overview page consisting of tabs for environments, biological processes, and molecular functions. When an environmental type is used as input, PREGO returns the organisms that have been found to be related to it, as well as the Biological Processes observed in the given environment. 
   Lastly, if a biological process is under study, PREGO returns a tab with the organisms along with another tab with the Environments related to the process. 
   Notably, only the associations with scores higher than $0.5$ are presented in the web platform and are sorted in descending order based on their score. 
   The score is visualized with a five-star system (see Appendix~\ref{app:C} for the scoring scheme).
   Every association tab contains three tables with associations derived from the PREGO channels (see Section~\ref{sec:prego-methods}) along with their supported evidence. 
   The user can both search and scroll through these tables, which makes knowledge extraction easier in cases where, for example, Isolate data contain hundreds of associations. 
   In the \textit{Literature} channel, each association is supported by the scientific articles with text-mining identified co-mentions. 
   When a user clicks on an association, a popup window appears. This window displays abstracts or excerpts of full text with the associated entities highlighted (Figure~\ref{fig:prego_ui_resources}a). 
   Additionally, the Environmental Samples and Genome annotations and Isolates channels support evidence for each association by providing links to more detailed information. 
   In the former channel, when the users click on an association, they are redirected to pertinent sample pages of MGnify (Figure~\ref{fig:prego_ui_resources}b). 
   Similarly, the latter redirects users to JGI and NCBI genomes when the associations originated from JGI—IMG and Struo, respectively (Figure~\ref{fig:prego_ui_resources}c).


   \begin{figure}[h]
      \centering
      \includegraphics[width=0.85\textwidth]{figures/prego_ui_resources.png}
      \caption[PREGO in action - examples]{ 
         Each association is supported by original data. 
         (a) Literature channel has a pop-up functionality that displays the scientific articles that each specific association occurs with highlighted color. 
         (b) Environmental Samples channel redirects to the samples that support the specific association (currently only is supported MGnify). 
         (c) Annotated Genomes channel similarly redirects to the isolates ids that each association is based on (both Struo and JGI IMG are supported).
      }
      \label{fig:prego_ui_resources}
   \end{figure}


   The \textit{Documents} tab includes a list of scientific publications where the queried entity is mentioned. 
   Through the \textit{Downloads} tab, users are able to get all of the PREGO associations found for their query, per entity type (e.g., all the environments found related to an organism) and per channel (e.g., all the Environments found related to an organism through the \textit{Literature} channel). 
   This data retrieval functionality is also available via the PREGO API (syntax described in Figure~\ref{fig:prego_api_schema}). 
   Finally, all PREGO associations are available for bulk download from each channel (see Table~\ref{table:pregoA1}).



   \begin{figure}[h]
      \centering
      \includegraphics[width=135mm]{figures/figure_4_PREGO API.png}
      \caption[The PREGO API schema]{The PREGO API schema.}
      \label{fig:prego_api_schema}
   \end{figure}


   \subsection{PREGO in Action}
   \label{subsec:prego-action}

   To demonstrate PREGO's potential, we present four different ways that PREGO can assist molecular ecologists. 
   The demo focuses on the sulfate-reducing microorganisms (SRMs) as well as the processes and environments that relate to sulfate reduction. 
   Through this demo, we highlight how the different channels may provide complementary insights regarding different taxonomic levels and different association types.

   \subsection*{\textit{Which Environments Are Related to a Taxon?}}
   \label{subsec:envo-taxa}

   Based on Pavloudi et al. (2017) \parencite{pavloudi2017diversity}, several bacterial and archaeal SRM were found in lagoonal sediments, after amplifying and sequencing the dissimilatory sulfite reductase $\beta$-subunit (dsrB). 
   Using PREGO for the case of Desulfobacteraceae, the family in which the majority of the observed OTUs of the study belonged to, several environmental types similar to lagoons were retrieved from both the \textit{Literature} and the \textit{Environmental samples} channels (Figure~\ref{fig:prego_ui_resources}a,b). Moreover, most of them had a high $z$-score, such as \textit{"sediment"}, \textit{"sludge"}, and \textit{"activated sludge"}. 
   Several dissimilar environmental types were associated with Desulfobacteraceae, e.g., \textit{"oil reservoir"} indicating them as potential environments where sulfate reduction takes place. 
   However, the presence of taxa within that family in different environments, from \textit{"sea water"} to \textit{"forest"} and \textit{"Wastewater treatment plant"}, may suggest that this family has ubiquitous representatives in diverse conditions.

   Searching for \textit{Desulfatiglans anilini} (\href{https://prego.hcmr.gr/example1}{example1}\footnote{\href{https://prego.hcmr.gr/example1}{https://prego.hcmr.gr/example1}}, accessed on 24 December 2021) at the species level, 
   the most abundant species in Pavloudi et al. (2017) and, 
   for \textit{Desulfatiglans anilini} DSM 4660 strain ({\href{https://prego.hcmr.gr/example2}{example 2}\footnote{\href{https://prego.hcmr.gr/example2}{https://prego.hcmr.gr/example2}}, accessed on 24 December 2021), 
   PREGO provides associations with the \textit{"Anaerobic sediment"}, \textit{"Marine oxygen minimum zone"}, and \textit{"Anaerobic digester sludge"} terms. 
   These associations further corroborate the relationship between the species and sulfate reduction. 
   More specifically, the \textit{"sulfur spring"} ENVO term was retrieved from the Environmental samples channel as well.


   \subsection*{\textit{Which Biological Processes and Molecular Functions Are Related to a Taxon?}}
   \label{subsec:proc-taxa}

   According to Pavloudi et al. (2017), \textit{Desulfatiglans anilini} plays an important role in sulfate reduction. 
   The Biological Processes provided by PREGO's Literature channel are the GO terms \textit{"Sulfate reduction"}, \textit{"Sulfide oxidation"}, and \textit{"Sulfide ion homeostasis"}, which support this claim. 
   In addition, the \textit{"Denitrification pathway"} term was also retrieved. 
   This is rather interesting as it is in line with what Pavloudi et al. (2017) discussed about the SRMs and their ability to use various electron acceptors, e.g., nitrate and nitrite.

   Furthermore, PREGO's Molecular Function tab provides more insight on this example. Several GO terms related to sulfate reduction (e.g., terms related to \textit{"sulfite reductase"}) were associated with DSM 4660 strain and Desulfatiglans anilini species in multiple channels. Interestingly, in the case of the strain query, the Annotated Genomes channel returned many GO terms related to the nitrogen fixation, e.g., \textit{"nitric oxide dioxygenase activity"}.


   \subsection*{\textit{Which Taxa Are Related to a Biological Process?}}
   \label{subsec:taxa-proces}

   PREGO can be also used to report organisms that relate to a certain biological process. Searching for \textit{"dissimilatory sulfate reduction"} associations with taxa (\href{https://prego.hcmr.gr/example3}{example 3}\footnote{\href{https://prego.hcmr.gr/example3}{https://prego.hcmr.gr/example3}}, accessed on 24 December 2021) resulted in several taxa that were mentioned in the Pavloudi et al. (2017) study. 
   For example, taxa such as Thermodesulfobacteria and Thermodesulfovibrio were found among the entries with the highest score (e.g.,) based on the Literature channel. 
   The other two channels did not contain any associations. 
   Using the \textit{"Sulfate assimilation"} (\href{https://prego.hcmr.gr/example4}{example 4}\footnote{\href{https://prego.hcmr.gr/example4}{https://prego.hcmr.gr/example4}}, accessed on 24 December 2021) as the biological process input, PREGO results showed several genera that were missing from PREGO results concerning the \textit{"dissimilatory sulfate reduction"}. 
   Hence, manual search of GObp terms that describe the actual biological process of interest is more insightful.


   \subsection*{\textit{Are There Any Associations between Environments and Biological Processes?}}
   \label{subsec:envo-proc}

   Are there other environmental types, except the lagoonal sediments, in which sulfate assimilation occurs? In that question, and in \textit{"dissimilatory sulfate reduction"} (\href{https://prego.hcmr.gr/example3}{example 3}) in particular, PREGO assigns the highest score to “sediment” while, among others, \textit{"anoxic water"}, \textit{"oil reservoir"}, \textit{"mud volcano"}, and \textit{"basalt"} are potentially associated with environments related to sulfate reduction.
   
   Inversely, PREGO is insightful about occurring processes in a specific environmental type. 
   For example, searching for the biological processes that take place in \textit{"basalt"} (\href{https://prego.hcmr.gr/example5}{example 5}\footnote{\href{https://prego.hcmr.gr/example5}{https://prego.hcmr.gr/example5}}, accessed on 24 December 2021), processes like \textit{"Nitrogen fixation"} and \textit{"Reactive nitrogen species metabolic process"} stand out. 
   However, sulfate reduction is not among the associations. 
   However, when asking for \textit{"Mafic lava"} (\href{https://prego.hcmr.gr/example6}{example 6}\footnote{\href{https://prego.hcmr.gr/example6}{https://prego.hcmr.gr/example6}}, accessed on 24 December 2021), both the \textit{"nitrogen fixation"} and \textit{"Sulfur compound metabolic process"} terms are returned. 
   This highlights the suggestions of Pavloudi et al. (2017), regarding the potential use of various electron acceptors from the different strains present in different environmental types.


   \subsection{PREGO Contents}
   \label{subsec:prego-contents}

   PREGO contains the literature, environmental samples, and genome annotations of the resources shown in Table~\ref{table:prego1}. 
   The extracted contents of these resources have resulted to a knowledge base with ~364 K distinct taxonomic groups (out of a pool of ~$620$K Bacteria, Archaea, and microbial eukaryotes, based on NCBI Taxonomy) from which ~$258$K are at the species level (Table~\ref{table:prego2}). 
   These taxa are associated with ~1 K Environment Ontology terms, ~15 K GObp terms, and with ~7.9 K GOmf terms. 
   Combining the above, PREGO maintains a knowledge base of entities and associations between them that form a multipartite network with entities as nodes and scored associations between them as weighted links.


   % PREGO ENTITIES AFTER NER AND MAPPING - table:prego2
   \begin{table}[ht]
      \begin{tabular}{@{}cccrccc@{}}
      \toprule
      \textbf{Channel} & \textbf{Source} & \multicolumn{2}{c}{\textbf{Taxonomy}} & \textbf{\begin{tabular}[c]{@{}c@{}}Environ- \\ ments\end{tabular}} & \textbf{\begin{tabular}[c]{@{}c@{}}Biological \\ Processes\end{tabular}} & \textbf{\begin{tabular}[c]{@{}c@{}}Molecular \\ Functions\end{tabular}} \\ \midrule
      \multirow{3}{*}{Literature} & \multirow{3}{*}{\begin{tabular}[c]{@{}c@{}}MEDLINE \\ PubMed - \\ PMC OA\end{tabular}} & Strains & 8,929 & \multirow{3}{*}{1,077} & \multirow{3}{*}{15,079} & \multirow{3}{*}{7,318} \\
      &  & Species & 240,377 &  &  &  \\
      &  & Total & 342,506 &  &  &  \\
      \multirow{9}{*}{\begin{tabular}[c]{@{}c@{}}Environ- \\ mental \\ samples\end{tabular}} & \multirow{3}{*}{\begin{tabular}[c]{@{}c@{}}MG-RAST \\ amplicon\end{tabular}} & Strains & 1,392 & \multirow{3}{*}{162} & \multirow{3}{*}{-} & \multirow{3}{*}{-} \\
      &  & Species & 4,324 &  &  &  \\
      &  & Total & 5,859 &  &  &  \\
      & \multirow{3}{*}{\begin{tabular}[c]{@{}c@{}}MG-RAST \\ metagenome\end{tabular}} & Strains & 2,522 & \multirow{3}{*}{258} & \multirow{3}{*}{-} & \multirow{3}{*}{3,839} \\
      &  & Species & 4,406 &  &  &  \\
      &  & Total & 7,157 &  &  &  \\
      & \multirow{3}{*}{\begin{tabular}[c]{@{}c@{}}MGnify \\ amplicon\end{tabular}} & Strains & 2 & \multirow{3}{*}{216} & \multirow{3}{*}{11} & \multicolumn{1}{l}{\multirow{3}{*}{-}} \\
      &  & Species & 1,471 &  &  & \multicolumn{1}{l}{} \\
      &  & Total & 2,955 &  &  & \multicolumn{1}{l}{} \\
      \multirow{9}{*}{\begin{tabular}[c]{@{}c@{}}Annotated\\Genomes \&\\Isolates\end{tabular}} & \multirow{3}{*}{JGI IMGisolates} & Strains & 2,398 & \multirow{3}{*}{241} & \multirow{3}{*}{-} & \multirow{3}{*}{3,670} \\
      &  & Species & 11,203 &  &  &  \\
      &  & Total & 13,849 &  &  &  \\
      & \multirow{3}{*}{STRUO} & Strains & 6 & \multirow{3}{*}{-} & \multirow{3}{*}{-} & \multirow{3}{*}{2,789} \\
      &  & Species & 19,289 &  &  &  \\
      &  & Total & 19,325 &  &  &  \\
      & \multirow{3}{*}{BioProject} & Strains & 5,754 & \multirow{3}{*}{309} & \multirow{3}{*}{626} & \multirow{3}{*}{-} \\
      &  & Species & 3,373 &  &  &  \\
      &  & Total & 9,393 &  &  &  \\
      \multirow{3}{*}{Total} & \multirow{3}{*}{All} & Strains & 12,840 & \multirow{3}{*}{1,090} & \multirow{3}{*}{15,091} & \multirow{3}{*}{7,971} \\
      &  & \multicolumn{1}{l}{} & \multicolumn{1}{l}{} &  &  &  \\
      &  & \multicolumn{1}{l}{} & \multicolumn{1}{l}{} &  &  &  \\ \bottomrule
      \end{tabular}

      \caption[The entities of PREGO after the NER and mapping of every source]{
         The entities of PREGO after the NER and mapping of every source. 
         Counts of distinct entities of Taxa, Environments (ENVO terms), Biological Processes (Gene Ontology Biological process) and Molecular Function (Gene Ontology Molecular Function).
      }
      \label{table:prego2}

   \end{table}


   As shown in Figure~\ref{fig:prego-entities}, in its current version (December 2021), PREGO knowledge base covers $157$ bacterial phyla ($107$ are Candidatus), $23$ phyla from archaea ($18$ are Candidatus), and $22$ unicellular eukaryotic phyla described in the NCBI Taxonomy database. 
   The number of bacterial taxa present among the associations of each phylum ranges from the order of $10$s, as in the case of \textit{Candidatus Coatesbacteria}, to hundreds of thousands, e.g., Actinobacteriae. 
   The number of environmental types, found among the PREGO associations for each phylum, ranges from just a few to up to $1000$. 
   Similarly, the number of biological processes that have been related to the various phyla may range from less than a dozen, e.g., Yanofskybacteria to up to several thousands, e.g., Bacteroidetes. On the contrary, the number of molecular functions found to be related to taxa of each phylum is rather constant in all three domains.

   \begin{figure}[h]
      \centering
      \includegraphics[width=0.85\textwidth]{figures/figure_5_all_channels_ranks.png}  
      \caption[Summary of the unique entities per phylum for each of the four entity types on PREGO]{Summary of all the unique entities per phylum for each of the four entity types (in log10 scale) that appear in PREGO. Phyla are grouped based on their superkingdom (in log10 scale). Only phyla for which associations are available in the PREGO platform are mentioned.}
      \label{fig:prego-entities}
   \end{figure}


% RPEGO DISCUSSION
\section{Discussion}
\label{sec:prego-discussion}

   \subsection{PREGO Contents}
   \label{subsec:prego-contents-disc}

   On its current version and according to the NCBI Taxonomy that it is based on, PREGO manages to cover a great range of microbial taxa, as most (if not all phyla) are present in the knowledge base (Figure~\ref{fig:prego-entities}). 
   The different number of organisms' entities per phylum highlights the diverse number of the members of the various phyla. On the contrary, the similar number of molecular functions in all cases indicates the robustness of the main metabolic processes required for life. 
   With respect to biological processes, their number per phylum varies to some extent, especially for the case of Bacteria and Archaea. 
   That could be observed as, in many cases, phyla that have been recently described using molecular techniques have not been studied extensively yet, e.g., Candidatus Delongbacteria. 
   As expected, the number of environmental types that have been associated with members of each phylum varies, as a phylum may be universally present, while others could be strongly niche-specific (e.g., Hydrothermarchaeota).

   Because of its three different channels, PREGO manages to extract associations both in the species and higher taxonomic levels. The Isolates channel supports explicit associations at the species level (Table~\ref{table:prego3} and Figure S3). 
   Interestingly, the number of such genomes seems to have reached a plateau for now, as PREGO-like platforms include the same order of magnitude. 
   The \textit{Literature} channel, on the other hand, promotes the extraction of associations at higher taxonomic levels (Table~\ref{table:prego3} and Figure S1). 
   This also applies to environment—organisms associations derived from the Environmental Samples channel (Table~\ref{table:prego3} and Figure S2). Associations regarding biological processes, though, are strongly enhanced by the Literature channel and the massive increase of literature.


   % ASSOCIATIOS TABLE - table:prego3
   \begin{sidewaystable}
      \begin{tabular}{@{}cccccrcr@{}}
      \toprule
      \textbf{Channel} & \textbf{Source} & \textbf{\begin{tabular}[c]{@{}c@{}}Environments \\ - \\ Processes\end{tabular}} & \textbf{\begin{tabular}[c]{@{}c@{}}Environments \\ - \\ Functions\end{tabular}} & \textbf{Taxonomy} & \multicolumn{1}{c}{\textbf{\begin{tabular}[c]{@{}c@{}}Taxa \\ -\\  Environments\end{tabular}}} & \textbf{\begin{tabular}[c]{@{}c@{}}Taxa \\ - \\ Processes\end{tabular}} & \multicolumn{1}{c}{\textbf{\begin{tabular}[c]{@{}c@{}}Taxa \\ -\\ Function\end{tabular}}} \\ \midrule
      \multirow{3}{*}{Literature} & \multirow{3}{*}{\begin{tabular}[c]{@{}c@{}}MEDLINE \\ PubMed - \\ PMC OA\end{tabular}} & \multirow{3}{*}{883,997} & \multirow{3}{*}{422,579} & Strains & 69,968 & \multicolumn{1}{r}{590,630} & 384,079 \\
      &  &  &  & Species & 778,877 & \multicolumn{1}{r}{3,501,635} & 1,961,920 \\
      &  &  &  & Total & 1,669,608 & \multicolumn{1}{r}{7,969,310} & 4,613,827 \\
      \multirow{9}{*}{\begin{tabular}[c]{@{}c@{}}Environmental \\ samples\end{tabular}} & \multirow{3}{*}{\begin{tabular}[c]{@{}c@{}}MG-RAST \\ amplicon\end{tabular}} & \multirow{3}{*}{-} & \multirow{3}{*}{-} & Strains & 13,645 & \multirow{3}{*}{-} & \multicolumn{1}{c}{\multirow{3}{*}{-}} \\
      &  &  &  & Species & 39,007 &  & \multicolumn{1}{c}{} \\
      &  &  &  & Total & 53,439 &  & \multicolumn{1}{c}{} \\
      & \multirow{3}{*}{\begin{tabular}[c]{@{}c@{}}MG-RAST \\ metagenome\end{tabular}} & \multirow{3}{*}{-} & \multirow{3}{*}{620,846} & Strains & 262,106 & \multirow{3}{*}{-} & 8,626,328 \\
      &  &  &  & Species & 103,913 &  & 10,715,548 \\
      &  &  &  & Total & 372,301 &  & 19,950,096 \\
      & \multirow{3}{*}{\begin{tabular}[c]{@{}c@{}}MGnify \\ amplicon\end{tabular}} & \multirow{3}{*}{-} & \multirow{3}{*}{-} & Strains & 18 & - & \multicolumn{1}{l}{} \\
      &  &  &  & Species & 30,122 & \multicolumn{1}{r}{351} & \multicolumn{1}{c}{-} \\
      &  &  &  & Total & 111,976 & \multicolumn{1}{r}{2,097} & \multicolumn{1}{l}{} \\
      \multirow{9}{*}{\begin{tabular}[c]{@{}c@{}}Annotated Genomes \\ and Isolates\end{tabular}} & \multirow{3}{*}{\begin{tabular}[c]{@{}c@{}}JGI IMG\\ isolates\end{tabular}} & \multirow{3}{*}{-} & \multirow{3}{*}{-} & Strains & 8,229 & \multirow{3}{*}{-} & 3,461,693 \\
      &  &  &  & Species & 42,141 &  & 13,216,559 \\
      &  &  &  & Total & 50,888 &  & 16,821,850 \\
      & \multirow{3}{*}{STRUO} & \multirow{3}{*}{-} & \multirow{3}{*}{-} & Strains & \multicolumn{1}{c}{\multirow{3}{*}{-}} & \multirow{3}{*}{-} & 1,803 \\
      &  &  &  & Species & \multicolumn{1}{c}{} &  & 4,070,195 \\
      &  &  &  & Total & \multicolumn{1}{c}{} &  & 4,079,312 \\
      & \multirow{3}{*}{BioProject} & \multirow{3}{*}{-} & \multirow{3}{*}{-} & Strains & 3,263 & \multicolumn{1}{r}{7,473} & \multicolumn{1}{l}{} \\
      &  &  &  & Species & 4,187 & \multicolumn{1}{r}{4,294} & \multicolumn{1}{l}{} \\
      &  &  &  & Total & 7,641 & \multicolumn{1}{r}{12,169} & \multicolumn{1}{l}{} \\
      \multirow{3}{*}{Total} & \multirow{3}{*}{All} & \multirow{3}{*}{883,997} & \multirow{3}{*}{1,043,425} & Strains & 357,229 & \multicolumn{1}{r}{598,103} & 12,473,903 \\
      &  &  &  & Species & 998,247 & \multicolumn{1}{r}{3,506,280} & 29,964,222 \\
      &  &  &  & Total & 2,265,853 & \multicolumn{1}{r}{7,983,576} & 45,465,085 \\ \cmidrule(l){5-8} 
      \end{tabular}
      \caption[Associations among the PREGO entities]{
         The associations between entities of PREGO after co-occurrence analysis: The supported entity types of associations are Environments—Biological Processes, Environments—Molecular Functions, Taxa—Environments, Taxa—Biological Processes, Taxa—Molecular Functions.
      }
      \label{table:prego3}
   \end{sidewaystable}

   Additionally, the text mining methodology of the Literature channel has retrieved most of the entities present in PREGO knowledge base (Table~\ref{table:prego2}). 
   A significant contribution to the taxa with associations is due to the PMC OA processing by the text mining pipeline of the Literature channel. 
   This is in-line with reports in other applications of text mining when using full text articles \parencite{westergaard2018comprehensive}. 
   However, the resulting associations are suggestive because of the text mining nature, and therefore subject for further review by the users.

   \subsection{Related Tools' Functionality and Content}
   \label{subsec:prego-similar-platforms}

   There is an emerging niche for tools similar to PREGO to bring forward microbe associations and metadata. 
   Table~\ref{table:prego4} summarizes the common and different features of BacDive, WoM, NMDC data portal, and PREGO. 
   All of them commonly share the environmental associations and biological/metabolic processes with the microbes.

   BacDive is a well-established platform with a focus on phenotype and cultivation information for about 100,000 prokaryotes, bacteria, and archaea. 
   It has a high level of curation for most of its input types, like literature, internal databases, and personal collections. 
   The NMDC data portal has published the scheme, the user interface, and a demonstrative collection of samples that will be populated later on. Standout features are the spatial visualization with coordinates and the detailed information of the samples, e.g., sequencing instruments and methodology. 
   An alternative approach is facilitated by WoM, which aims to bind chemistry to microbes. An environment, in particular, is defined as the starting metabolite pool that is transformed by an organism.
   Another tool is The Microbe Directory that contains fully curated metadata for about 8000 microbes from all superkingdoms. This tool focuses on conditions of growth and on host taxa.

   Complementary to these tools, PREGO contains associations of bacteria, archaea, and eukaryotes. Distinctive features are the associations of environments with processes/functions and the large-scale literature integration with text mining. Most importantly, most of the tools are complementary to each other with minimum overlap, an indication of the opportunities for further innovative synergies.


   % PREGO COMPARISONS - table:prego4
   \begin{table}[ht]

      \begin{adjustwidth}{-0.75cm}{}

      \begin{tabular}{@{}lllll@{}}
      
      \toprule
      Functionality & BacDive & Web of Microbes & NMDC & PREGO \\ \midrule
      manual curation & high & high & intermediate & low \\ 

      literature integration & limited & no & no & yes \\

      environment—taxa associations & yes & yes & yes & yes \\

      \begin{tabular}[c]{@{}l@{}}environment—process/\\ function associations\end{tabular} & no & no & no & yes \\

      process/function—taxa associations & yes & yes & yes & yes \\
      phenotypic data & yes & no & no & no \\

      data origin & \begin{tabular}[c]{@{}l@{}} original \\integration \end{tabular} & original & \begin{tabular}[c]{@{}l@{}} original \\integration \end{tabular} & integration \\

      spatial coordinates & yes & no & yes & no \\

      application programming interface & yes & no & yes & yes \\

      bulk download & limited & yes & yes & yes \\ \bottomrule

      \end{tabular}
      \end{adjustwidth}
      \caption[Feature comparison between PREGO and other similar platforms]{Feature comparison among platforms that facilitate knowledge discovery and integration of microbial data.}
      \label{table:prego4}
   
   \end{table}      

   \subsection{PREGO Next Steps}
   \label{subsec:prego-next-steps}

   PREGO is a user-friendly association mining and sharing platform. 
   Its modular web-architecture grants it the flexibility for further improvements in the aforementioned aspects, namely: 
   source datasets, user interface, entity, and association scope expansion. Regarding datasets, additional data, such as transcriptomes from MGnify and other records annotated with metadata from studies in \href{https://ebi-metagenomics.github.io/blog/2021/11/17/Publication-Annotations/}{EuroPMC}, accessed on 24 December 2021) \parencite{ferguson2021europe}, could be incorporated. 
   Similarly, the \href{https://data.microbiomedata.org/}{NMDC data platform standards-compliant annotated records}\footnote{\href{https://data.microbiomedata.org/}{https://data.microbiomedata.org/}} (accessed on 24 December 2021) could serve as an additional resource with its high-quality metadata \parencite{wood2020national, vangay2021microbiome}. 
   Reciprocally, if requested, pertinent literature and association summaries could be programmatically offered to interested third parties.

   Furthermore, the entity types supported by the PREGO system could be expanded. For example, GOmf terms could be upgraded as a search-entry point to the system. 
   In addition, disease and tissue describing terms, already supported by the PREGO-underlying EXTRACT system \parencite{pafilis2016extract}, could enter the PREGO ecosystem of associated entities. 
   From a statistics perspective, the calculation of a combined association score, when an association is reported by more than one channel of information, could be another feature to add.

   The user interface can be enhanced to support multiple entity and/or sequence queries, instead of single ones. 
   Sequences can be processed by taxonomy assignment pipelines (e.g., PEMA \parencite{zafeiropoulos2020pema}) and be converted into searching PREGO for associations. 
   In addition, network visualization tools, like Arena3Dweb \parencite{karatzas2021arena3dweb}, could allow interactive browsing of associations through multi-layered graphs. 
   Enrichment analyses, like those performed by OnTheFly2.0 \parencite{baltoumas2021onthefly2} or Flame \parencite{thanati2021flame}, 
   can be incorporated. 
   Omics data analysis pipelines, like MiBiOmics \parencite{zoppi2021mibiomics}, environment associations with sequences using SeqEnv \parencite{sinclair2016seqenv} and biogeochemical associations with metagenomic data with DiTing~\parencite{xue_diting_2021} could be enabled by comparing the associations pertinent to different groups of entities. 
   The computationally intensive tasks of multiple queries, taxonomy assignments to sequences and enrichment analysis could be offered by our in-house High Performance Computing facility (https://hpc.hcmr.gr/, accessed on 24 December 2021) \parencite{zafeiropoulos_0s_2021} in synergy with pertinent Research Infrastructures like \href{https://elixir-europe.org}{ELIXIR}\footnote{\href{https://elixir-europe.org}{https://elixir-europe.org}} (accessed on 24 December 2021) and \href{https://www.lifewatch.eu/}{LifeWatch ERIC}\footnote{\href{https://www.lifewatch.eu/}{https://www.lifewatch.eu/}} (accessed on 24 December 2021).



   \subsection*{Availability of Supporting Source Codes:} 
   The PREGO software modules are available under BSD 2-Clause “Simplified” License. Scripts, where additional libraries have been used, are subject to their individual licenses. More information on each module can be found as listed below:
   
   \begin{itemize}
      \item prego\_gathering\_data 
      \href{https://github.com/lab42open-team/prego_gathering_data}{github.com/lab42open-team/prego\_gathering\_data}
      \item prego\_daemons \href{https://github.com/lab42open-team/prego_daemons}{github.com/lab42open-team/prego\_daemons}
      \item prego\_mappings \href{https://github.com/lab42open-team/prego_mappings}{github.com/lab42open-team/prego\_mappings} 
      \item prego\_statistics \href{https://github.com/lab42open-team/prego_statistics}{github.com/lab42open-team/prego\_statistics}
   \end{itemize}

   Additional software and curated lists along with their individual license are:
   \begin{itemize}
      \item tagger:	\href{https://github.com/larsjuhljensen/tagger}{https://github.com/larsjuhljensen/tagger}, BSD 2-Clause "Simplified" License
      \item mamba: \href{https://github.com/larsjuhljensen/mamba}{https://github.com/larsjuhljensen/mamba}, BSD 2-Clause "Simplified" License 
      \item tagger dictionary:  \href{https://download.jensenlab.org/}{https://download.jensenlab.org/} and there in: \\
      \href{https://download.jensenlab.org/prego_dictionary.tar.gz}{https://download.jensenlab.org/prego\_dictionary.tar.gz}, CC-BY 4.0 license
   \end{itemize}


% --------------------------------------------------
% 
% This chapter is for Cretan endemic Arthropods
% 
% --------------------------------------------------


\chapter{The conservation status of the Cretan Endemic Arthropods under Natura 2000 network}
\label{cha:arthropods}


\textbf{Citation:} \\ 
Giannis Bolanakis, Savvas Paragkamian, Maria Chatzaki, Nefeli Kotitsa, Liubitsa Kardaki and Apostolos Trichas.\\

Shared co-first authorship.

DOI: \href{https://doi.org/10.21203/rs.3.rs-2671168/v1}{10.21203/rs.3.rs-2671168/v1}\footnote{
   For author contributions and supplementary material please refer to the relevant sections. 
   This is a modified version of the published version,
   in terms of relevance, coherence and formatting.
   }



% ABSTRACT
\section{Abstract}

Arthropod decline has been globally and locally documented, yet they are still
not sufficiently protected. Crete (Greece), a Mediterranean biodiversity
hotspot, is a continental island renowned for its diverse geology, ecosystems
and endemicity of flora and fauna, with continuous research on its Arthropod
fauna dating back to the 19th century. Here we investigate the conservation
status of the Cretan Arthropods using Preliminary Automated Conservation
Assessments (PACA) and the overlap of Cretan Arthropod distributions with the
Natura 2000 protected areas. Moreover we investigate their endemicity hotspots
and propose candidate Key Biodiversity Areas. In order to perform these
analyses, we assembled occurrences of the endemic Arthropods in Crete located
in the collections of the Natural History Museum of Crete together with
literature data. These assessments resulted in 75\% of endemic Arthropods as
potentially threatened. The hotspots of endemic taxa and the candidate Key
Biodiversity Areas are distributed mostly on the mountainous areas where the
Natura 2000 protected areas have great coverage. Yet human activities have
significant impact even in those areas, while some taxa are not sufficiently
covered by Natura 2000. These findings call for countermeasures and conservation actions.


% INTRODUCTION
\section{Introduction}
\label{sec:arthropods-intro}

In the Anthropocene, the need to tackle biodiversity loss is urgent \parencite{johnson2017biodiversity, meng2021biodiversity}.
Arthropods include more than 78\% of the described animal taxa \parencite{zhang2013animal},
numbering approximately 7 million terrestrial species \parencite{stork2018how-many}. Many recent
studies highlight the decline of Insect \parencite{cardoso2020scientists, wagner2020insect, raven2021agricultural},
Spider \parencite{potapov2020functional, branco2020an-expert-based} and Myriapod biodiversity \parencite{karam-gemael2018why-be-red-listed, iniesta2023where}.
For instance, \textcite{hallmann2017more} estimate a 75\% reduction of flying Insect
biomass in Germany in the last 27 years. \textcite{klink2020meta-analysis} yielded a 9\% per
decade decline in Insect abundance. \textcite{sanchez-bayo2019worldwide} estimate
the possible extinction of 40\% of Insect species in the near future (but see
\textcite{wagner2019global} for a critique). Consequently \textcite{samways2019insect} speaks about the “Fall of Insects”.
Yet, the actions taken for their conservation are deemed as insufficient in
global and local scale \parencite{cardoso2012the-underrepresentation, damen2013protected, chowdhury2023three-quarters}.

Numbering approximately 7,000 species [extrapolated from Fauna Europaea
(Jong et al. 2014) and Legakis et al. (2018)], the Arthropods of Crete, Greece,
have been studied for almost two centuries (Anastasiou et al. 2018).
Only 135 of these species (1.9\%) have been assessed in IUCN Red List as of
this publication, making Arthropods the third most evaluated group of the
island, behind vascular plants (291) and land mollusks (165). The low
evaluation percentage is a common motif for Arthropods, hindered by the lack of
data (Cardoso et al. 2011a, b; Cardoso et al. 2012; Wagner et al. 2021) and
charisma of the Arthropods themselves (Cardoso 2012; Wang et al. 2021), leading
to knowledge shortfalls (see Hortal et al. 2015).

Crete is located between three continents (Europe, Africa, Asia), in a
well established global biodiversity hotspot (Myers et al. 2000) of the
Mediterranean basin. Isolated from the rest of the Aegean and the continental
Greece for more than 5 million years (Fassoulas 2018), with a complex geological
and climatic history and long-term human presence (Rackham and Moody 1996),
Crete has developed a species rich biodiversity with high endemism
(Médail and Quézel 1997; Chatzaki et al. 2015; Sfenthourakis and Schmalfuss 2018; Vardinoyannis et al. 2018).
It is a special biogeographical entity for various taxonomic groups: Buprestidae (Mühle et al. 2000),
Tenebrionidae (Fattorini 2006a, 2008), Cerambycidae (Vitali and Schmitt 2017),
Orthoptera (Willemse et al. 2023), Vascular Plants (Kougioumoutzis et al. 2017)
and snails (Vardinoyannis et al. 2018). Moreover, Crete presents the highest
percentage of threatened species of the IUCN assessed Greek fauna and flora
(12\%) (Spiliopoulou et al. 2021) and is the hottest Mediterranean island for
plant endemism (Médail 2017). The biogeographical and conservational
significance of Crete thus becomes apparent.

Arthropod decline is the result of multiple - synergistically acting - causes (Cardoso et al. 2020; Wagner 2020; Wagner et al. 2021).
Habitat loss (Cardoso et al. 2020; Wagner 2020; Wagner et al. 2021),
agricultural intensification (Habel et al. 2019; Raven and Wagner 2021),
urbanization (Wagner et al. 2021), pollution/pesticides (Brühl and Zaller 2019; Cardoso et al. 2020)
and climate change (Cardoso et al. 2020; Harvey et al. 2022) are the major
drivers of this decline. Crete complies with this global trend.

Habitat loss and degradation occurs throughout Crete as a result of urban,
agricultural and touristic development. This is a major issue since habitat
loss is a major threat in Europe for many Arthropod groups,
e.g. Butterflies (van Swaay et al. 2010) Bees (Nieto et al. 2014),
Orthoptera (Hochkirch et al. 2016) and Saproxylic Beetles (Cálix et al. 2018).
Climate change is predicted to induce scarcer yet more intense precipitation,
increase of drought locally (Koutroulis et al. 2011) and shrinkage as well as
possible shifts to the rainfall period (Koutroulis et al. 2013). Groups
associated with fresh water could be deeply impacted from the locally increased
drought and the increase in need of water for irrigation and domestic use,
e.g. Odonata (Kalkman et al. 2010), which has become harsher due to the
increase of agriculture and land use (Tzanakakis et al. 2020). Stock raising
(sheep and goats) has always been an important aspect of Cretan life and
economy (Rackham and Moody 1996). Overgrazing impacts severely soil erosion,
soil moisture and vegetation (Kairis et al. 2015; Kosmas et al. 2015). All the
above contribute to a worrying trend for Crete, i.e. the higher percentage of
Threatened endemic Arthropods when compared with the respective European
groups (Appendix Figure \ref{fig:arthropods-figS1}).

The largest structure of biodiversity conservation in Crete is the Natura 2000
network (N2K). N2K is the only regional assemblage of protected areas
worldwide (Crofts 2014). Operating throughout European Union (EU) since 1992,
the N2K is the alloy of two EU directives, The Birds Directive (Council Directive 79/409/EEC, 1979)
and the Habitats and Species Directive (HSD) (The Council Directive 92/43/EEC, 1992).
The Arthropods are linked with the HSD. None of the Cretan endemic Arthropods
are listed in the annex of the HSD (driven by taxonomical, geographical and
other biases - Cardoso 2012).

Crete has by far the highest percentage of overlap between threatened species’
ranges (flora and fauna) and N2K in Greece (Spiliopoulou et al. 2021).
Sfenthourakis and Legakis (2001) investigated the N2K overlap in Crete with
land mollusks, Orthoptera, Carabidae, Tenebrionidae and Oniscidea, and found
that four out of five endemicity hotspots in Crete (Dia islet, Lefka Ori,
Psiloritis and Dikti massifs) reside in N2K. Kougioumoutzis et al. (2021a, b)
found a great number of endemicity hotspots and threat-spots of Greek vascular
plants in the Cretan mountains with significant overlap with the N2K.
In contrast Dimitrakopoulos et al. (2004) focusing on vascular plants,
recovered small percentages of overlap between plant endemicity / threat-spots and N2K.
Overall, Crete seems to be under an adequate protection regime (Kougioumoutzis et al. 2021b; Spiliopoulou et al. 2021),
however, the aforementioned studies do not focus on Arthropods, leaving space
for a more close up research for their conservation status.

In this study we aim to a) identify cretan endemicity hotspots (EHs)
b) investigate for candidate Key Biodiversity Areas (KBAs)
c) examine the overlap of EHs, KBAs and threatened taxa with the N2K areas and
d) their relation with the anthropogenic pressures in these sites (Figure \ref{fig:arthropods-fig1}).
To do so, we assembled the accumulated knowledge of the past 200 years of
entomological research in Crete with the collections of NHMC for 11 Arthropod groups:
Araneae, Scorpiones, Chilopoda, Diplopoda, Coleoptera, Heteroptera,
Hymenoptera, Lepidoptera, Odonata, Orthoptera and Trichoptera.
Secondly, we committed to the Findable, Accessible, Interoperable and Reproducible (FAIR) principles (Wilkinson et al., 2016)
for data and code (see \href{https://doi.org/10.5281/zenodo.10635645}{Supplementary Material}). Hence Crete could become Greece’s
spearhead in meta-analyses concerning Arthropods. Thus, we contribute to the
ongoing discussion concerning the global conservation status of Arthropods from
the perspective of a continental island, rich in endemic species.

   \begin{figure}[h]
      \centering
      \includegraphics[width=\textwidth,height=\textheight,keepaspectratio]{figures/arthropods-Fig1-flowchart.png}
      \caption[The study's workflow]{Study’s workflow. A. Curation of the literature and the NHMC samplings for the endemic Arthropods occurrences. B. Preliminary Assessment Conservation Assessments (PACA) of the Arthropods and mapping of their distributions. C. Investigation of candidate KBAs and EHs and D. Overlap with N2K and Land Use Change evaluation for human pressures.}
      \label{fig:arthropods-fig1}
   \end{figure}

% METHODS
\section{Methods}
\label{sec:arthropods-method}
   
    \subsection{Taxa selection criteria}
    \label{subsec:arthropods-taxa-selection}

We aggregated data of Cretan endemic Arthropod groups (NHMC collections and bibliography). The taxa included should satisfy the following criteria:

There should be at least one authoritative work on the group for Crete which
should make clear remarks about the group’s taxonomic dynamics, so that future
taxonomical or systematic works on the taxon would not severely affect our inferences.
In essence, we selected groups whose biodiversity is well studied and we do not
expect significant changes in their number of species for Crete. 

The geographic information about the endemic species distribution should be in
a form convertible to coordinates (i.e., either in coordinates or with a
precise locality). The included coordinates are precise, while the converted
ones do not exceed a radius of certainty more than 2 km.

The group should not be dominated by cavernicolous species. We opted to exclude
cave dwelling fauna, since it is a special system, governed by different
biogeographical and ecological processes. Thus, groups like terrestrial Isopods
were excluded for their high percentage of cavernicolous species (Schamlfuss et al. 2004; Sfenthourakis and Schmalfuss 2018).

Based on the above criteria, the selected groups are:
Araneae, Chilopoda, Coleoptera, Diplopoda, Heteroptera, Hymenoptera (Chrysididae, Formicidae, Symphyta), Lepidoptera (Geometridae), Odonata, Orthoptera, Scorpiones and Trichoptera.

    \subsection{Data assemblage}
    \label{subsec:arthropods-data-assemblage}

We curated the bibliography and the NHMC collection to assemble taxa
occurrences (Appendix Figure \ref{fig:arthropods-figS2}). The bibliography used
contains both historical and contemporary published material. Bibliographic
records include: 1) Author of the article 2) Species name and 3) Locality coordinates.
For the records without coordinates, coordinates approximating the site were
given based on the locality and description given. NHMC specimens (over 2
million Arthropods) have been primarily collected by pitfall trapping as part
of MSc, PhD studies and environmental monitoring programs, over the last 40
years (details on trapping protocols are discussed more extensively in Salata et al. 2020b and Willemse et al., 2023).
NHMC data include: 1) NHMC Field Code, 2) Species name and 3) Locality.
The coordinate reference system we used for all location data is WGS84 - EPSG:4326.

We opted for an integrated approach including as many Arthropod taxa as
possible. Thus, our dataset is inhomogeneous. Different Orders and/or Families
require different sampling methods, while there has been an inconsistent
historical interest for various groups. For example Staphylinidae (Coleoptera)
are systematically studied in the last 30 years in Crete, while the research in
Carabidae (Coleoptera) dates back to the 19th century. Sampling methods vary
even within groups. Orthoptera have been classically sampled by net or hand
from the beginning of the 20th century, while in the last 30 years, when NHMC
started studying the Cretan biodiversity, there are numerous specimens that
have been captured with pitfall traps (Willemse et al. 2023).

Different subspecies of the same species were treated separately as in Fattorini (2006b); Dimitrakopoulos et al. (2004); Fattorini and Baselga (2012),
because the distinction between species and subspecies is usually arbitrary and
unstable, hence excluding subspecies could lead to the neglection of important
conservational or evolutionary units. From here on, we refer to both species
and subspecies as “taxa”.
    

    \subsection{Taxa assessments}
    \label{subsec:arthropods-taxa-assessments}
For the taxa assessment we used the Preliminary Automated Conservation Assessment
(PACA) pipeline (Stévart et al. 2019). PACA is an approximation of the IUCN
assessment based on Criterion B, i.e. on the Extent of Occurrence (EOO) and
Area of Occupancy (AOO) and cannot be used as a replacement of a full IUCN
assessment. Some important differences between PACA and a full IUCN assessment
are that PACA always assumes a continuous decline of the species’ habitat
quality and automatizes some processes that require the assessors' engagement
(e.g. defining locations). PACA is a useful tool to obtain a preliminary image
regarding a taxa assemblage of an area, in the absence of a thorough IUCN assessment.
Moreover PACA can be really useful for datasets that have incorporated subspecies,
which enjoy less attention from IUCN mainly for taxonomic reasons, i.e., the
lack of consensus on the subspecies as a biological entity. Criterion B is the
one most widely used for Arthropods (Cardoso et al. 2011a, b; Carpaneto et al. 2015),
since most Arthropod groups lack the data for the other criteria (A, C, D and E),
i.e., mainly population size and trend information or quantitative analyses.
Criterion B could overestimate the danger of Arthropods (Cardoso et al. 2011a),
which should always be taken into consideration.

In order to identify locations (as defined by PACA), we used the European
Environment Agency (EEA) reference grid with a 10 x 10 km grid cell to assign
occurrences to locations. All the occurrences of a taxon that reside in a
10 x 10 km cell constitute one location. The PACA are estimated as shown in Table \ref{table:arthropods-paca}.

Subsequently, we converted the PACA categories to the respective IUCN ones (Stévart et al. 2019).

\begin{table}
\centering
\caption{The PACA and potential IUCN categories based on the number of locations and EOO or AOO.}
\begin{tabular}{p{0.2\linewidth} | p{0.2\linewidth} | p{0.2\linewidth} | p{0.2\linewidth}}
\textbf{Potential IUCN categories}     & \textbf{PACA categories}         & \textbf{\# locations} & \textbf{EOO (km2) OR AOO (km2)}  \\
Potentially Vulnerable (VU)            & Potentially Threatened (PT)      & =10                   & 20000 OR 2000                    \\
Potentially Endangered (EN)            & Likely Threatened (LT)           & =5                    & 5000 OR 500                      \\
Potentially Critically Endangered (CR) & Likely Threatened (LT)           & 1                     & 100 OR 10                        \\
Other                                  & Potentially Not Threatened (PNT) & rest                  & rest                            
\end{tabular}
\label{table:arthropods-paca}
\end{table}
    
    \subsection{Endemicity Hotspots (EHs), and Key Biodiversity Areas (KBAs)}
    \label{subsec:arthropods-ehs-kbas}

Hotspot definitions vary from quantitative methods to experts opinions and curation.
In quantitative methods, grid size and shape influences the determination of
the areas of interest such as hotspots and key biodiversity areas (Hurlbert and Jetz, 2007, Nhancale and Smith, 2011).
Choosing the size of the grid is not trivial (Mo et al., 2019) and is dependent
on the conservation goals (Margules and Pressey, 2000). In the past decade,
there have been major advances for conservation standards, guidelines,
frameworks and tools available to be put into action (IPBES 2019).

We defined the EHs as the 10\% of the grid cells with the highest number of
endemic taxa. In order to avoid biases concerning the grid cell size, the same
pipeline was tested with cells of different size (4 x 4, 8 x 8 and 10 x 10 km).
For the subsequent analyses we opted for the 10 x 10 km grid (see section \ref{subsec:arthropods-grids})
which is also the EEA reference grid, the standard for the reporting format
(Groups of Experts, 2017) of the Resolution No. 8 (2012) of the Standing Committee
to the Bern Convention on the Emerald Network of Areas of Special Conservation Interest (ASCI).
Moreover, the EHs of the various cell sizes are aggregated in the same areas (Figure \ref{fig:arthropods-fig3}).
We made the same treatment for each of the selected groups separately (Appendix Figure \ref{fig:arthropods-figS2}).
We redefined EHs as the 10\% of the grid cells with max overlap of the orders
to check for biases towards more speciose orders (e.g. Coleoptera) (Appendix Figure \ref{fig:arthropods-figS3}).

For the investigation of KBAs (IUCN 2016) we used the WEGE index (Farooq et al. 2020)
on the same grid as hotspots based on the PACA assessments and the distributions
of the taxa. WEGE can be used to indicate candidate KBAs or prioritize already
existing KBAs (given the limited resources available for conservation) but does
not replace a throughout KBAs assessment (Farooq et al. 2020).

    \subsection{Spatial overlaps}
    \label{subsec:arthropods-spatial}
We compared and evaluated the overlap of EHs and KBAs with protected areas and
land use categories. The N2K data were downloaded from the European Environment
Agency portal and filtered for the Habitats Directive and Crete spatial extent.
We retrieved land use categories from the CORINE Land Cover, CLC 2018 version
v.2020-20u1 (Copernicus Land Monitoring Service, 2023).
To evaluate the current land use of yielded EHs we used the CORINE Land Cover
and to examine the human pressure (change in land use, agriculture), we used the
Historic Land Dynamics Assessment (HILDA+) dataset (Winkler et al. 2021) to
estimate the change of land use the from 1998 to 2018. Furthermore, we examined
the overlap of the AOO of each taxon with the N2K.


    \subsection{Tools and scripts}
    \label{subsec:arthropods-tools}
We performed the analyses using the R Statistical Software (v4.3.2; R Core Team 2023),
the visualization using the ggplot2 R package (Wickham, 2016). The figures
created are colored using the colorblind-friendly 'Okabe-Ito' palette (Ichihara et al., 2009).
We calculated EOO and AOO using the ConR R package (Dauby et al. 2017) and PACA
using custom scripts. For the spatial data handling, transformations and
geometry we used the sf v1.0-14 (Pebesma 2018) and terra v1.7-55 R packages (Hijmans 2023).
WEGE index is calculated with the WEGE R package (Farooq et al. 2020).
Adaptive grid is created using the quadtree R package (Friend, 2023).
Jaccard similarity was calculated with the vegan 2.6-4 r package (Oksanen et al. 2022).
All scripts are reproducible by design and available in this 
\href{https://github.com/savvas-paragkamian/arthropoda_assessment_crete}{GitHub repository}.

% RESULTS
\section{Results}
\label{sec:arthropods-results}

Using over 100 publications (as of 2020) and 733 NHMC sampling events
(\href{https://doi.org/10.5281/zenodo.10635645}{Supplementary Material}), we assembled a dataset of 343 taxa (species and subspecies),
with 4,924 records across 1,569 distinct sites of Crete \ref{fig:arthropods-fig2}. The taxa
are distributed to eleven orders, with Coleoptera having the most taxa (206)
and Chilopoda and Scorpiones the least (two) (Table \ref{table:arthropods-results}).

   \begin{figure}[h]
      \centering
      \includegraphics[width=\textwidth,height=\textheight,keepaspectratio]{figures/arthropods-Fig2.png}
      \caption[The common endemicity hotspots of all orders]{A. A high-altitude shrubland on Lefka Ori (top) and the endemic \textit{Orchamus raulinii} (Cretan Stone Grasshopper, Pamphagidae, Acridoidea) (bottom) (photos by A. Trichas). B. The seminal work of E.v. Oertzen on the Coleoptera of Greece and Crete, published on 1886 (available at \href{https://www.biodiversitylibrary.org/page/32058852}{Biodiversity Heritage Library}. C. The island of Crete with its major mountains, the N2K areas, the sampling localities of the NHMC and the ones compiled from the literature.}
      \label{fig:arthropods-fig2}
   \end{figure}


    \subsection{Grid cell size}
    \label{subsec:arthropods-grids}
The grid cell size 10 x 10 km is the most suitable for our study since our
dataset - being compiled from numerous different sources and sampling efforts -
is rather coarse and inhomogenous for a smaller cell size (Figure \ref{fig:arthropods-fig3}).
The unique taxa of the EHs of each grid is distributed as follows: 10 km=283,
8 km=278, 4 km=293, adaptive cells=267, with the 4km grid covering most endemic
species. The 4 km grid mostly highlighted areas known for their tourist/recreational activities,
indicating that it is more sensitive to sampling intensity (Figure \ref{fig:arthropods-fig3}).
Focusing on sampling we applied the adaptive grid size with quadtrees resulting
in 157 grids with 8 km length, 38 with 4 km and 74 with 2 km (Appendix Figure \ref{fig:arthropods-figS5}).
This indicates the preference of larger cells for the majority of our dataset
even though a small percent of regions has higher density of sampling.
The highest overlap among all grids is between the 10 km and 8 km reaching
57\% (Appendix Table \ref{table:arthropods-tableS2}). Finally, the 10 km grid has more taxa
per cell (Appendix Figure \ref{fig:arthropods-figS6}) and is a reference grid system.
Based on our analysis and interoperability and reproducibility aims we choose
the 10 km EEA reference grid for the EHs and candidate KBAs inference.
Nevertheless, we also performed the WEGE analysis for KBAs using the adaptive
grid, yielding practically the same areas as the 10km grid minus Zakros \ref{fig:arthropods-fig5}.
The same pipeline can not be done with EHs for they require a fixed cell size.

   \begin{figure}[h]
      \centering
      \includegraphics[width=\textwidth,height=\textheight,keepaspectratio]{figures/arthropods-Fig3.png}
      \caption[Grids, hotspots and potential KBAs of Cretan endemic arthropods]{A. EHs with different grids, 4, 8 and 10 km2. The top 10\% of cells with most species are considered as hotspots. B. EHs and their overlap with Natura 2000 HSD sites. C. Top 10\% WEGE index grids to approximate KBAs.}
      \label{fig:arthropods-fig3}
   \end{figure}


    \subsection{EHs}
    \label{subsec:EHs}

    The EHs cover 17\% of Crete (Table \ref{table:arthropods-overlaps}). They are aggregated in Lefka Ori, Dikti,
Psiloritis, Thrypti and Selino at the southwest of Chania (Figure \ref{fig:arthropods-fig3}).
Four of the five areas are mountainous (the main massifs of Crete). Lefka Ori
and Dikti host the highest number of EHs. Psiloritis hosts one, the only EH in
central Crete. No satellite island of Crete is yielded as an EH, in spite of
their faunas being in essence a subset of the Cretan biodiversity, and Gavdos
islet having some single island endemic Arthropods. The reason behind this is a
purely numeric one. Compared to the yielded EHs, they have less endemic and local
endemic taxa. With finer grids the number of EHs increases (Figure \ref{fig:arthropods-fig3}). Gavdos
islet is yielded as EH only in the 4 x 4 km grid. The finer grid is less strict
and appropriate for an inhomogeneous dataset as ours, although it does indeed
unveil areas that would be otherwise neglected. From now on we discuss our
results grounded in the 10 x 10 km grid, but refer to Figure \ref{fig:arthropods-fig3} for the other
cell sizes.

The different orders display a variation in their respective EHs (Figure \ref{fig:arthropods-fig2}).
Almost unanimously, they exhibit hotspots in one or more massifs,
with Trichoptera and Odonata being exceptions, driven from their need of inland waters.
When aggregated, the EHs of the different orders generally agree with EHs of
Arthropods as a whole (Figures \ref{fig:arthropods-fig4}, \ref{fig:arthropods-fig5}). Only the
latter approach (Arthropods as a whole) is treated onwards.

\begin{table}[]
\caption{Overlap of Arthropod EHs, WEGE KBAs with N2K HSD, Wildlife Refuges and CORINE Land Cover (LEVEL1) areas of Crete. Areas are measured in km\textsuperscript{2}}
\begin{tabular}{lllll}
Type                          & Area       & \% of Crete & Overlap EHs (\%)                          & Overlap KBAs (\%)                  \\
Crete                         & 8347       & -           & -                                      & -                               \\
Endemic hotspots (EHs)        & 1400       & 17\%        & -                                      & 1200 (86\%)                     \\
WEGE KBAs                     & 1400       & 17\%        & 1200 (86\%)                            & -                               \\
Natura2000 SAC                & 2371       & 28\%        & 858 (61\%)                             & 736 (53\%)                      \\
Wildlife refuges              & 610        & 7\%         & 143 (10\%)                             & 136 (10\%)                      \\
Agricultural areas            & 3618       & 46\%        & 275 (20\%)                             & 301 (22\%)                      \\
Artificial surfaces           & 181        & 2\%         & 5 (0.4\%)                              & 4.56 (0.3\%)                    \\
Forest and semi natural areas & 4508       & 54\%        & 1092 (78\%)                            & 1059 (76\%)                     \\
Water bodies                  & 7          & 0.08\%      & 1 (0.07\%)                             & 0.3 (0.02\%)                   
\end{tabular}
\label{table:arthropods-overlaps}
\end{table}


    \subsection{Species Assessment}
    \label{subsec:arthropods-species-assessment}
    
According to the PACA analysis, 75\% of the taxa are Likely / Potentially
Threatened (from here on referred to as Threatened) and 25\% are assessed as
Near Threatened/Least Concern (Table \ref{table:arthropods-results}). These percentages vary between the groups,
nevertheless there are some concrete patterns. For example, threatened
categories dominate most of the orders except Odonata and Orthoptera (Table \ref{table:arthropods-results}).
Chilopoda and Scorpiones have no threatened taxa at all. Both of these orders
display a low endemic diversity (two species each). On the contrary,
Heteroptera have only threatened taxa, followed by Coleoptera (82\%),
Diplopoda (71\%), Hymenoptera (68\%), Geometridae (67\%) and Aranae (65\%).
Of course, the bias of Criterion B towards a more severe categorization
(Cardoso et al. 2011a) and the fact that we are using a preliminary assessment
advocate a conservative interpretation of our results which are explorative and not concrete assessments.

% \usepackage{color}
% \usepackage{array}
% \usepackage{longtable}
% \usepackage{array}

\begin{sidewaystable}
\caption{Number of taxa included in the dataset in total and per order. In addition, the mean AOO km\textsuperscript{2} for each taxon and its coverage by N2K are given (Standard Deviation in parentheses), as well as the percentages for PACA and IUCN categories. Categories for PACA: LT - Likely Threatened, PT - Potentially Threatened and LNT - Likely Not Threatened. Categories for IUCN: Threatened (sum of Critically Endangered, Endangered and Vulnerable) and NT/LC - Near Threatened/Least Concern.}
\begin{tabular}{lllllllllll}
Order                        & taxa & sites & occurrences & Mean AOO (sd)     & Mean AOO in N2K (sd)        & LT         & PT        & LNT       & Threatened & NT/LC     \\
All taxa                     & 343  & 1539  & 4924        & 58 (109)          & 27 (51)                  & 195 (57\%) & 62 (18\%) & 86 (25\%) & 257 (75\%) & 86 (25\%) \\
Araneae                      & 40   & 253   & 523         & 55 (63)           & 21 (19)                  & 16 (40\%)  & 10 (25\%) & 14 (35\%) & 26 (65\%)  & 14 (35\%) \\
Chilopoda                    & 2    & 234   & 269         & 552 (560)         & 223 (251)                & 0          & 0         & 2 (100\%) & 0          & 2 (100\%) \\
Coleoptera                   & 206  & 925   & 2584        & 50 (90)           & 26 (45)                  & 132 (64\%) & 36 (17\%) & 38 (18\%) & 168 (82\%) & 38 (18\%) \\
Diplopoda                    & 7    & 74    & 101         & 61 (88)           & 27 (43)                  & 4 (57\%)   & 1 (14\%)  & 2 (29\%)  & 5 (71\%)   & 2 (29\%)  \\
Heteroptera                  & 17   & 46    & 58          & 14 (13)           & 4 (3)                    & 14 (82\%)  & 3 (18\%)  & NA        & 17 (100\%) & NA        \\
Hymenoptera                  & 25   & 173   & 285         & 48 (70)           & 20 (30)                  & 14 (56\%)  & 3 (12\%)  & 8 (32\%)  & 17 (68\%)  & 8 (32\%)  \\
Lepidoptera                  & 9    & 40    & 65          & 30 (24)           & 13 (10)                  & 5 (56\%)   & 1 (11\%)  & 3 (33\%)  & 6 (67\%)   & 3 (33\%)  \\
Odonata                      & 3    & 30    & 49          & 63 (26)           & 12 (6)                   & 0          & 1 (33\%)  & 2 (67\%)  & 1 (33\%)   & 2 (67\%)  \\
Orthoptera                   & 20   & 363   & 579         & 122 (133)         & 61 (64)                  & 7 (35\%)   & 3 (15\%)  & 10 (50\%) & 10 (50\%)  & 10 (50\%) \\
Scorpiones                   & 2    & 240   & 254         & 518 (619)         & 240 (279)                & 0          & 0         & 2 (100\%) & 0          & 2 (100\%) \\
Trichoptera                  & 12   & 66    & 157         & 54 (38)           & 16 (10)                  & 3 (25\%)   & 4 (33\%)  & 5 (42\%)  & 7 (58\%)   & 5 (42\%) 
\end{tabular}
\label{table:arthropods-results}
\end{sidewaystable}



\subsection{Potential KBAs}
    \label{subsec:potential-kbas}
    
    The proposed KBAs recovered with WEGE cover 17\% of Crete (Table \ref{table:arthropods-overlaps}). They are
aggregated in the Cretan mountains (as EHs) in the far west (Selino) and far
east (Zakros) Crete (Figure \ref{fig:arthropods-fig3}). WEGE utilizes the threat status and the
distributions of the taxa to rank potential KBAs (Farooq et al. 2020), thus it
is expected for yield areas with high congruence with EHs since many threatened
species are concentrated there. PACA assessment was carried out based on
Criterion B, which primarily uses the range of the species to estimate the
Threat category, thus it is prone to assess geographically restricted species
(e.g., in a mountain plateau) as Threatened (Figure \ref{fig:arthropods-fig3}).

    \subsection{N2K overlap}
    \label{subsec:n2k-overlap}

EHs display 61\% overlap with the N2K (Table \ref{table:arthropods-results}, Figure \ref{fig:arthropods-fig3}), mostly of which is
in the mountains (Figure \ref{fig:arthropods-fig2}, \ref{fig:arthropods-fig3}). The greatest overlap occurs in Psiloritis and
Dikti (Figure \ref{fig:arthropods-fig3}) In contrast, the EHs outside the Cretan mountains
(Selino, Kritsa - near Dikti) display lower overlap with the respective
protected areas near them (Figure \ref{fig:arthropods-fig3}).The areas of agreement of KBAs with N2K
are aggregated in the Cretan mountains, while Selino, Kritsa and Zakros are the
areas with the smallest overlap (Figure \ref{fig:arthropods-fig3}). The average overlap of taxa AOO
with N2K is 52\% and increases to 55\% when only the threatened taxa are
considered (Table \ref{table:arthropods-results}). Orthoptera have the highest average \% overlap (62.38\%),
while Odonata have the lowest (20.39\%) (Figure \ref{fig:arthropods-fig4}).
    

   \begin{figure}[h]
      \centering
      \includegraphics[width=\textwidth,height=\textheight,keepaspectratio]{figures/arthropods-Fig4.png}
      \caption[AOO, EOO and N2K overlaps per order]{A. Locations, EOO and AOO of all orders. Each dot represents one taxon and boxplots show the mean value and the first and third quantiles. The y axis is in log10 scale. B. Proportion of overlap of AOO with Natura 2000 areas per Order. Each dot is a taxon with its respective proportion of AOO overlap. The horizontal line of the boxplot shows the average, and the box shows the 1st and 3rd quantiles of the values.}
      \label{fig:arthropods-fig4}
   \end{figure}

    \subsection{Human Intervention}
    \label{subsec:arthropods-human-intervention}


In order to evaluate the human impact in the yielded EHs and candidate KBAs we
used CORINE layers (Figure \ref{fig:arthropods-fig5}). At LEVEL 1 of the classification, the dominant
habitat is “Forest and semi-natural areas”, covering ~ 76-78\% of the EHs/KBAs,
while agricultural areas also display coverage of 20-22\%. This exhibits the
presence of human activity in the EHs/KBAs (Table \ref{table:arthropods-overlaps}). Using the LEVEL 2 CORINE
layer, we acquired a more detailed image of the coverage. The dominant habitat
seems to consist of scrub and/or herbaceous vegetation  (56\% coverage),
forests (12\%), permanent crops (10\%) and open spaces with little or
no vegetation (10\%) (Appendix Table \ref{table:arthropods-tableS3}). With HILDA+ we
estimated some negative and some positive transitions within the EHs/KBAs in
the previous two decades (Appendix Table \ref{table:arthropods-tableS4}). Around 10-12\% of
forest area has been transformed to cropland. Likewise 22-25\% of cropland has
been transformed to pastureland. On the other hand 15-26\% of cropland area has
been transformed to forest. Worryingly urban areas have increased for about 16.8\%,
although outside EHs and KBAs mainly at the expense of croplands and
pasturelands (Appendix Table \ref{table:arthropods-tableS4}). Finally, water areas remain
stable, albeit more research is needed to assess potential decreases in the
quality of this habitat, especially given the aggressive urban and touristic expansion.


   \begin{figure}[h]
      \centering
      \includegraphics[width=\textwidth,height=\textheight,keepaspectratio]{figures/arthropods-Fig5.png}
      \caption[Land Use of Crete and the changes]{A. CORINE Land Cover, LEVEL 2 of the Copernicus system. B. HILDA+ Land use change for the years 1998-2018.}
      \label{fig:arthropods-fig5}
   \end{figure}


% DISCUSSION
\section{Discussion}
\label{sec:arthropods-discussion}

    \subsection{Endemicity Hotspots}
    \label{subsec:arthropods-Endemicity-Hotspots}

Mountains host a great amount of Earth’s biodiversity, being a main driver for
the birth of species (Antonelli et al. 2018; Noroozi et al. 2018; Rahbek et al. 2019a, b)
and a crucial frontier for their fate (Steinbauer et al. 2018; Urban 2018).
Crete is not an exception to this trend (Trigas et al. 2013; Kougioumoutzis et al. 2020).
Our results conform to that, since the EHs are gathered primarily in the major
Cretan mountains (Figure \ref{fig:arthropods-fig3}). Lefka Ori and Dikti are the sites with the most
EHs, in agreement with studies focused on vascular plants (Dimitrakopoulos et al. 2004; Kougioumoutzis et al. 2020).
Sfenthourakis and Legakis (2001), employing invertebrate groups, also recovered these mountains as EHs.

Only one EH was recovered for Psiloritis in this study. This could result from
Psiloritis’ position in the center of the island, with Lefka Ori and Dikti
filtering taxa moving from west and east, from its relatively smaller volume
(when compared with Lefka Ori), the intense human intervention, and the less
intense topography and relief compared to the other Cretan mountains (for the
importance of topography and relief in speciation - biodiversity see: Stuessy et al. 2006; Muellner-Riehl 2019; Igea and Tanentzap 2021).
Thrypti as EH is consistent with the aforementioned literature. Isolated in the
far east part of the island, Thrypti could be of a major conservational importance for Crete.
A novelty of our study is the relative importance (participating with more grid cells)
of Dikti when compared with the aforementioned studies, even though it is always obtained as an EH site. 

Dia islet, although obtained as an EH for invertebrate fauna in Sfenthourakis
and Legakis (2001), is not recovered as a hotspot for Arthropods in our study.
The island of Dia has indeed some importance for Arthropod taxa, such Isopods,
hosting some single island endemics (Schmalfuss et al. 2004), but its endemic
diversity is mostly driven by snails (Vardinoyannis pers. communication), which
are not treated here.

The accumulation of more EHs in the West (Lefka Ori and west of Lefka Ori) and
East (Dikti and Thrypti) Crete can be explained by their isolation today and in
the past, when Crete was divided in palaeo-islands during the Pliocene
(see Poulakakis et al. 2014 and Fassoulas 2018 for a review). Moreover, the
west and east parts of Crete function as “sinks'' for Balkan and Eastern
species respectively. The footprint of the Balkans and the Middle East in the
Cretan fauna is discussed in various studies (Vardinoyannis 1994; Trichas 1996; Chatzaki 2003; Trichas et al. 2020).
The “redness” of West and East Crete as endemic centers is also obtained in
other studies (e.g. Assing 2019; Kougioumoutzis et al. 2020). 

Islands are biodiversity sanctuaries (Whitaker and Fernández-Palacios 2007), and
so are mountains (Rhabek et al. 2019b). Our work advocates for approaches that
treat islands and mountains under a holistic perspective. The combination of
the two provides a complex biogeographical interplay governing the forces of
speciation, preservation and extinction of biodiversity (Steinbauer et al. 2016).
This synergistic effect of mountains-islands has also been recovered in other
areas such as the Balearic islands (Guardiola and Sáez 2023).

    \subsection{Species assessment}
    \label{subsec:arthropods-species-assessment-disc}

The species assessment from PACA showed that 75\% (Table \ref{table:arthropods-results}) of the
taxa assessed are Potentially/Likely Threatened (hereafter referred as Threatened).
The variation between the different orders is not substantial (most of them
score above 50\% in Threatened taxa) (Table \ref{table:arthropods-results}). Local endemic or restricted taxa
increase an order’s Threatened percentage. Chilopoda and Scorpiones, have zero
Threatened taxa. For Odonata we also recovered a low Threatened percentage
(33.3\% - 2 taxa) compared to the 100\% (2 taxa) of IUCN
\ref{fig:arthropods-figS1}. That is an artifact of the PACA
assessment, not taking into consideration population data and population
fragmentation. This factor, albeit an important aspect of criterion B for the
IUCN assessments, is excluded from PACA for it requires special treatment for
each taxon (Dauby et al. 2019). There are multiple reasons for the disagreement
between the two assessments. An IUCN assessment is an exhaustive, overall
assessment, performed by experts and focusing on each species separately.
A PACA assessment is a rather automated pipeline that allows researchers to
have a preliminary approach on understudied taxa and areas, but in no way an
alternative of a thorough IUCN assessment.

Arthropods with wider ranges that are not assessed as Threatened under
criterion B, are not necessarily Least Concern and should not be neglected.
Arthropod communities can be affected by the reduction of the abundance of
common and abundant species that offer important functions to the biocommunity.
Wide range does not guarantee high abundance (even though this is true for many
taxa) and even common species can be threatened (Habel and Schmitt 2018; Klink et al. 2023).

With 75\% Threatened taxa, Cretan Arthropods appear to be in better fate than
the Cretan vascular plants, assessed as Threatened in their totality
(Kougioumoutzis et al. 2020). This is most likely a result of the combined use
of Criteria A and B in the vascular plant assessment (Kougioumoutzis et al. 2020) -
something impossible for the Arthropods since their data are too coarse for the
utilization of criterion A. This dominant trend of Crete is also true for land
mollusks with 41.7\% of the Cretan endemics being Threatened (IUCN) compared to
the 20.5\% of Threatened endemics for Europe (Neubert et al. 2019). This is
particularly worrying given Crete’s significance as a biodiversity
hotspot (Myers et al. 2000; Médail, 2017) and the fact that it refers to single
island endemics. Cretan taxa display a worse trend not only compared to Europe \ref{fig:arthropods-figS1},
but also when compared to Greece. For example, 46.1\% of the Greek endemic
vascular flora is recovered as Threatened according to Kougioumoutzis et al. (2021b),
compared to the 100\% of the Cretan endemic flora (Kougioumoutzis et al. 2020). 
    
    \subsection{KBAs}
    \label{subsec:arthropods-KBAs}
    
The candidate KBAs yielded by WEGE are gathered in Lefka Ori, Dikti, Thrypti,
Psiloritis, Selino and Zakros. WEGE analysis is stricter in evaluating
potential KBAs in the sense that Crete, having many endemic species, would be
qualified for KBA as a whole, triggering criteria A and B (IUCN 2016).
This is a weakness of KBAs highlighted by Farooq et al. (2023) that WEGE seems
to resolve (Farooq et al. 2020). Furthermore, the use of WEGE overcomes
obstacles in the ranking of areas for conservation such as the lack of robust
phylogenetic information regarding the taxa under focus (Farooq et al. 2020).

Our results are congruent with previous studies that enquire about EHs or
threat-spots in Crete (Dimitrakopoulos et al. 2004; Kougioumoutzis et al. 2020; Kougioumoutzis et al. 2021b).
The KBAs obtained here refer to Arthropods and are not mandatory in any way.
Other areas of Crete could be candidates as well. First of all, when it comes to
Arthropods, areas such as Gavdos islet are yielded as EHs with a different grid (Figure \ref{fig:arthropods-fig3}).
Moreover, other areas may be important for other organisms. For example Asteroussia
are a KBA for Birds (Key Biodiversity Areas Partnership, 2024), while they are
also recovered as a potential climatic refuge for plants (Kougioumoutzis et al. 2020).
The essence is that KBAs should always be under inquiry grounded on the available
resources and will of the stakeholders and political authorities. From the
simple proposal of some KBAs to the implementation of a conservation plan there
are many steps to follow that do not all abide by quantifiable scientific
thresholds. Venter et al. (2018) found that KBAs have been selected in order to
avoid incorporating areas with agricultural activities, while there is a need
for mediation between national and global sites of conservation
interest (Kougioumoutzis et al. 2021b; Lim et al. 2023). In this international
and interdisciplinary questioning, the effective selection of candidate areas
is of great importance (Plumptre et al. 2024). Our work contributes to this
matter by highlighting the significance of island mountains as KBAs.

    \subsection{N2K overlap}
    \label{subsec:arthropods-N2K-overlap}

N2K has been characterized as the only protection structure that “has the
political chance to be implemented in the island” (Dimitrakopoulos et al. 2004).
The overlap of threatened taxa, EHs and KBAs with N2K is thus of major
conservation importance. 

Crete is by far the area of Greece with the highest mean complementary
percentage between threatened species distribution and N2K (Spiliopoulou et al. 2021).
Focusing on vascular endemic plants Kougioumoutzis et al. (2021b) also obtained
high complementarity between the endemicity/threat hotspots
(obtained with various indices) and the N2K. Our work contributes to this
discussion, exhibiting a high overlap between EHs and KBAs with N2K and
obtaining a satisfactory coverage of  EHs/KBAs by N2K (Table \ref{table:arthropods-overlaps}).
Additionally, N2K covers many areas of Crete (peninsulas, gorges, islets and
massifs) which, even though they are not yielded as EHs/KBAs, host a plethora
of endemic Arthropods.

We examined the overlap of each taxon’s AOO with N2K to obtain a more detailed
overview of its conservation status. The mean percentage of coverage was 52\%,
and increased to 55\% for the Threatened taxa. This percentage is close, albeit
lower, to the 62.3\% recovered from Spiliopoulou et al. (2021) for Crete. This
can be attributed to the innate differences of our datasets and methodologies.
We focused strictly on Arthropods, while Spiliopoulou et al. (2021) examined
all the species of Greece (flora and fauna) assessed in a Threatened category.
Moreover, we converted the PACA assessment to the respective IUCN category,
while Spiliopoulou et al. (2021) used the actual IUCN assessments. Despite
these methodological differences, another explanation could be that the
Arthropods are indeed in a worse conservation position than other groups, an
inference which rhymes with the ongoing global discussion around Arthropods’
decline (Chowdhury et al. 2022, 2023).

Orthoptera have the highest average overlap with N2K (62.38\%) (Figure \ref{fig:arthropods-fig4}).
This is mainly caused by the genus Eupholidoptera which is responsible for a
great part of Orthopteran endemism in Crete (Willemse et al. 2023), which
differentiate areas mostly covered by N2K. Odonata and Trichoptera exhibit the
lowest average overlap (Figure \ref{fig:arthropods-fig4}). A closer investigation towards the
freshwater species of Crete, especially those associated with seasonal streams
or ponds, is recommended. The overdrafting of Crete’s natural water reservoirs
and the aggressive urbanization and agricultural intensification could be a
hazard for smaller springs and streams. Kalkman et al. (2010) highlight the
need for a freshwater plan for the conservation of the Cretan dragonflies.
The ill fate of aquatic insects is a global phenomenon (Deacon et al. 2019; Roth et al. 2020; Dia-Silva et al. 2021),
although there are studies that recover more positive trends (Klink et al. 2020, but see also Desquilbet et al. 2020).

In our dataset, 29 (8.4\%) of the taxa have zero overlap with N2K. All of them
are Threatened. Additionally, 44 (17\%) of the Threatened taxa have less than
10\ overlap with N2K. The percentages (25.4\% in aggregate) of disagreement
obtained here are higher than those obtained from Spiliopoulou et al. (2021).
This becomes more acute since only nine Insect species out of 124 (7.2\%) that
were analyzed in Spiliopoulou et al. (2021) are excluded from the protected areas.

The inclusion of Arthropod taxa in protected areas is often insufficient, with
Arthropods experiencing declines inside the protected areas (Borges et al. 2005; Harry et al. 2019; Rada et al. 2019, Chowdhury et al. 2022).
In fact, even when certain Arthropod groups are adequately included in N2K,
there are gaps and omissions (Sánchez-Fernández et al. 2008; Verovnik et al. 2011).
At a global level 75\% of Insects are not sufficiently covered by protected
areas (Chowdhury et al. 2023). Crete stands in an intermediate position,
following the general trend of Greece’s N2K adequacy, being the best covered
area at a national level (Kougioumoutzis et al. 2021b; Spiliopoulou et al. 2021).
However, there are some clear gaps regarding certain taxa, encouraging more
locally focused conservation policies complementary to N2K. For example actions
need to be taken for KBAs that fall outside N2K like Kritsa and Zakros.

Biases towards Arthropods cause their poor coverage by protected
areas (D’Amen et al. 2013; Delso et al. 2021; Chowdhury et al. 2022). These
biases derive from geography, size, color and charisma (Cardoso 2012; Mammola et al. 2020; Wang et al. 2021),
and even from political/economic reasons (Dias-Silva et al. 2021). For example,
the strongest driver for a conservation program funding within the European
Union is the online popularity (Mammola et al. 2020). The unpopularity of
Arthropods has begun to change (Wagner et al. 2021), especially through citizen
science, which is a trend we should build on to properly conserve the Arthropods.

    \subsection{Human Intervention in Arthropods’ EHs}
    \label{subsec:arthropods-human-intervention-ehs}
Human activities account for almost 20\% of the EHs. The primary human activity
in the EHs is agriculture (~19.6\%). Agricultural intensification is one of the
most important drivers of Arthropods’ decline (Habel et al. 2019; Brühl and Zaller 2019; Raven and Wagner 2021).
Moreover, threats associated with agriculture are the number one threat for
Insect species inside protected areas in Europe (Chowdhury et al. 2022).
Nevertheless, regarding change in land use, there is a somewhat equal
transition trend from cropland to forest and vice versa inside EHs and KBAs
(Appendix Table \ref{table:arthropods-tableS4}). This means that while some sites are being
degraded others may recover. More research within EHs and KBAs is essential in
order to quantify the impact (negative or positive) of these transitions to the
endemic Arthropods. A vast amount of cropland has been transformed to pasture
lands (Appendix Table \ref{table:arthropods-tableS4}) which requires further examination,
since grazing has both positive [eg. on Gnaphosidae (Spiders) communities (Kaltsas et al.  2019)]
and negative effects [e.g. Carabidae (Coleoptera) (Kaltsas et al. 2013)].
The reduction of croplands could be interpreted under the general trend of
urbanization (Appendix Table \ref{table:arthropods-tableS4}), which nevertheless occurs
outside EHs and KBAs, but a shift towards montane areas especially under new
forms of tourism could deeply impact the sites of conservation importance.
    
    \subsection{Perspectives and Actions}
    \label{subsec:arthropods-perspectives-actions}

Arthropods are rarely approached as a whole, for biological and practical
reasons. The study of Arthropods is usually limited to a family or even to a
lower taxonomic level and to certain biogeographical areas (e.g. Borges et al. 2017, 2018).
Treatises tackling Arthropod issues in a wider scope are: Azores – Gaspar et al. (2010),
Atacama coast – Pizarro-Araya et al. (2021), Neotropical area - Barahona-Segovia and Zúñiga-Alonso (2021),
or meta-data studies (Klink et al. 2020; Chowdhury et al. 2023). In this study
we compiled a detailed and diverse dataset integrating different Arthropod groups.
Our goal was to obtain a holistic image of Crete’s Arthropods’ conservation
status and place it in the wider frame of the global issues of Arthropod conservation.

Crete follows the global pattern of island biodiversity, with the island biota
being under constant extinction pressure (Triantis et al. 2010; Fernández-Palacios et al. 2021).
All four main culprits for the impoverishing of island biota identified by
Fernández-Palacios et al. (2021) have an intense presence in Crete. The
lowlands of Crete are experiencing significant habitat loss due to urbanization
and transformation to olive tree cultivations. Natural resources are
overexploited - especially water reservoirs - mainly from agriculture and
aggressive touristic development. Invasive species have established populations
(D'Agata et al. 2009; Affre et al. 2010; Christopoulou et al. 2021) and the
impact of climate change is prominent. The aggregation of most of the endemic
Arthropods in the mountains renders them vulnerable not only due to their
insularity but adds extra pressure from mountain related processes. The lack of
space to retreat from climate change and their inability to outcompete with
lowland populations/species moving to higher elevations drives the extinction
of montane populations (Alexander et al. 2015; Steinbauer et al. 2018; Urban 2018; Yadav et al. 2018; Frishkoff et al. 2019).
Thus, the alloy of mountain-island can act not only as a driver for
biodiversity but also as the ground for its loss. Our work highlights the need
for a simultaneous evaluation of mountain and island driven phenomena inside
biodiversity hotspots, as is the Mediterranean basin.

For a better fate for the Cretan Arthropods under the global urgencies for
Arthropods’ conservation, we propose actions that could improve the
conservation status/framework of this special fauna:

1) The conservation situation inside the N2K should be examined to ensure the
correct implementation of the N2K goals and directives, especially given the
studies which have shown a significant decline of Arthropods inside protected
areas (Hallmann et al. 2017; Chwowdhury et al. 2022). This is also true for our
study, which demonstrates contradictory results regarding the human pressures
inside EHs and KBAs. Research efforts focused on the Arthropods species’
populations, abundances and communities will provide empirical data for the
interaction of human activities and the Arthropods of Crete. A multidisciplinary
study, utilizing molecular and geographical tools as well as the local
stakeholders in the spirit of Lehmann et al. (2021), would provide the much
needed research framework regarding the interaction dynamics of human activities
and Arthropods inside N2K in Crete.

2) The discussion for the expansion or optimization of already existing
protected areas like N2K, is imminent in the global bibliography (Chowdhury et al. 2023).
Ignoring important sites outside N2K would lead to neglect some threatened taxa,
and also encourage further human disturbance in unprotected areas (Borges et al. 2005).
Enquiries considering the incorporation of areas outside N2K to the network
and/or communication with local/centralized authorities and stakeholders to
form policies for the management of such areas could optimize the conservation
status of Cretan Arthropods. Despite the admittedly beneficial function of
protected areas in conservation targets such as the reducing of habitat loss (Geldmann et al. 2013),
the need of additional protection actions to tackle certain issues is highlighted (D’Amen et al. 2013; Hochkirch et al. 2013).
In essence, it is important for protected areas to be treated individually for
the achievement of different conservation goals instead of just complying with
a general protection trend. This issue is also brought up for the Greek reality (Dimitrakopoulos et al. 2004; Kougioumoutzis et al. 2020, 2021b).
Our study adds to this conversation, pointing to KBAs for Arthropods.

3) Educational/citizen science programs focused on the awareness of the local
communities towards the specificity and sensitivity of Cretan Arthropods could
build a social dynamic that would lighten Arthropods from the burden of
unpopularity (Wang et al. 2021). Given that Cretan Arthropods suffer from
biases related to their regional geographical position within the EU (Cardoso 2012),
the rise of awareness towards their threats and needs will improve their study
and conservation. Moreover, it would medicate the bias of Habitats Directive
towards central/northern European species and ground a more integrating
conservation approach within the EU.

4) PACA is utilized to map uncharted areas and biota that suffer from reduced
conservational focus. A thorough assessment of as many as possible of the
Cretan Arthropods under IUCN should be carried out and would provide a concrete
image of their threat status. This Herculean task is tackled in the upcoming
Red Data Book of Greece. Besides the value of a detailed IUCN assessment itself,
a Red Data Book will also provide a detailed dataset to test the effectiveness
and the limits of the PACA method, given its common use (e.g., Kougioumoutzis et al. 2021b; Iniesta et al. 2023).

5) Our study is a perfect example of the importance of contemporary research in
faunistics and taxonomy for conservation. Many core elements of our dataset
have been published only in the last five years (e.g., Assing 2019; Salata et al. 2020a).
In fact, 42.5\% (47 species) of the endemic Staphylinidae (Coleoptera) have
been described in 2019 (Assing et al. 2019). Knowledge shortfalls (Hortal et al. 2015)
regarding the Cretan Arthropods create an imperative need for basic taxonomic
and faunistic knowledge, i.e., the discovery of new taxa (tackling Linnean shortfall),
and for the better understanding of the species’ distributions (tackling Wallacean shortfall).
Thus, more funding should be focused on faunistic data assemblage studies.
In contrast to Garnett and Christidis (2017), we believe that taxonomy does not
hinder conservation biology, but instead makes conservation possible,
since when unaware of the existence of a species (whether species are
considered as real entities or not - see Raposo et al. 2017), it is impossible
to protect it. Therefore, trailing the voices of those who advocate for a
better incorporation of taxonomy in conservation (Dubois 2003; de Carvalho et al. 2007; Boero 2010; Andreone et al. 2022)
while acknowledging its innate value (Engel et al. 2021), we passionately call
for an extensive taxonomic and faunistic scrutiny of Cretan Arthropod biodiversity.

\section{Conclusions}
\label{sec:arthropods-conclusions}

The high percentage of potentially/likely Threatened taxa recovered (75\%),
points to immediate need for conservation actions and policies concerning Crete,
as well as a robust assessment of their threat status. These results are
worrying under the light of the “Insect Apocalypse”. Even though none of the
Cretan Arthropods was considered when the N2K was designed for Crete, N2K
appears to be an adequate conservation network for Cretan Arthropods. The EHs
and KBAs recovered here are the “usual suspects” also obtained in other studies
with different datasets. Lefka Ori, Psiloritis, Dikti, Thrypti, Selino and
Zakros are identified as EHs and KBAs for the Cretan Arthropods.  A point of
contradiction recovered here is the double role of an island-mountain system to
the birth and loss of biodiversity. Another contradiction is the one regarding
human activity and N2K coverage of the EHs/KBAs. Therefore, we suggest
multidisciplinary research efforts and policies that are not restricted to
scientific practice but welcome the participation of local communities to
achieve a better perspective for Cretan Arthropods.

\section*{Acknowledgements}
We would like to thank Katerina Vardinoyannis (Curator of Invertebrates - excluding Arthropods, in NHMC) and
Manolis Nikolakakis for their help in construction and design of the databases
used here. Moreover, we are also deeply obliged to professor Moysis Mylonas,
for his crucial theoretical remarks and advice and Leonidas Maroulis (PhD candidate - University of Cete)
for his insights regarding our methodology. Also, we would like to thank all
researchers and students of the NHMC and University of Crete who methodically
collected, sorted and identified Arthropod specimens during their studies in
various NHMC projects. Finally, we are deeply indebted to the comments of the
two anonymous reviewers, which have greatly improved our work.

\section*{Dedication}
We dedicate this work to Volker Assing (1956 - 2022). His research in Cretan
Staphylinidae has been remarkable in quantity and quality. Over the course of
numerous articles, he managed to exhibit and highlight the taxonomical and
biogeographical importance of Cretan rove beetles, by describing numerous new
species and unraveling interesting distributional patterns. His work is of most
significance for the conservation of this special fauna.

\section*{Data availability}
Data, scripts and results of the analysis are available and documented \href{https://github.com/savvas-paragkamian/arthropods_assessment_crete}{here}
Supplementary Material are available at the Zenodo repository \href{https://doi.org/10.5281/zenodo.10635645}{here}
Supplementary Material 1 is the Supplementary-material-1.xlsx file which
contains the occurrences as compiled from the literature and the specimens of the
NHMC. In addition all references of the literature are included in a separate sheet.
Supplementary Material 2 is the Supplementary-material-2.docx which
contains five (5) supplementary figures and four (4) supplementary tables.
Supplementary Material 3 is the
Supplementary-material-3-hilda-crete-1998-2018.mp4 timelapse video of yearly
changes of Land Use based on HILDA+ dataset 

\section*{Author contributions}
Conceptualization: AT, GB;
Data curation: AT, GB, LK, MC, NK;
Formal Analysis: SP; Methodology: GB, SP; Software: SP; Supervision: AT;
Validation: AT, GB, NK; Visualization: SP, AT; Writing – original draft: GB;
Writing – review - editing: All authors reviewed the manuscript.

\section*{Funding}

SP was funded by the 3rd H.F.R.I.
(Hellenic Foundation for Research and Innovation) Scholarships for
PHD Candidates (no. 5726). GB was funded by ELKE.uoc scholarship (no.11526).


% --------------------------------------------------
% 
% This chapter is for Crete system ecology
% 
% --------------------------------------------------


\chapter{Island Sampling Day and Crete soil microbial interactome}
\label{cha:crete-soil}

%\textbf{Citation:} \\ 


% ISD ABSTRACT
\section{Abstract}
    Microbes are known for their versatility, abundance
    and influence on soil ecosystem functioning.
    A synthesized knowledge base of microbial biodiversity, in terms of
    ecological and remote-sensing data remains a major challenge.
    Many worldwide studies have been published regarding soil
    microbiome ecosystems, though there are still many blind spots.
    Islands can be important case studies for this integration for more resolute and dense samplings.
    Here, we utilize the Island Sampling Day Crete 2016 microbial 16S rRNA gene
    amplicon data, integrated with soil and remote
    sensing data, to decipher the drivers of ecosystem function of the island.
    The Island Sampling Day Crete 2016 project has collected 144 topsoil samples
    from 72 sites, capturing a lot of this diversity, accompanied by FAIR
    (Findable, Accessible, Interoperable and Reproducible) data by design. 
    Cretan macroecology has been studied for centuries for its diverse  and endemic
    fauna and flora.
    In addition, Crete has been considered as a miniature continent with high contrasts in
    vegetation cover, elevation, climatic conditions. 
    We show that, higher altitudes in Crete found to
    be inhabited by a more diverse number of microorganisms, a pattern commonly
    seen in several faunistic groups, such as arthropods.
    The integration of the spatial data with state of the art methods enabled warning signals
    in pristine and grazing ecosystems.
    These results along with the
    climatic and desertification index influences on the soil microbiome of Crete,
    provide the basis to identify major drivers of biodiversity, to evaluate hotspots
    and contribute to foreknowledge of threatened ecosystems.

\section{Introduction}\label{intro}

Soil ecosystems are the cornerstone of terrestrial habitats, biodiversity and henceforth human activities.
Soils are characterised by multiple properties; chemical, physical and biological that 
form complex interdependant interactions. Biodiversity of soils covers
all forms of life, fauna, flora, bacteria, archaea, fungi, viruses. 
Bacteria and archaea are considered major drivers for the functionality of soil.
They infuence and are infuenced by their environment and their community structure 
defies their macroscopic functionality \parencite{Bahram2018}.
Global soil microbiome studies have been employed to decipher soil microbiome
compositions \parencite{thompson2017a-communal, Delgado-Baquerizo-atlas, Labouyrie2023},
functions \parencite{Bahram2018} and biogeography \parencite{Martiny2006, guerra2020Blind}.
These results showed the remarkable diversity in soils yet there are blind spots \parencite{guerra2020Blind}
and these sampling are sparce when considering samples per area density. One of most resolute
study is by \parencite{Karimi2020} which exemplified the 
vast complexity of soil bacterial communuties and the requirement of
dense samplings and isolated systems \parencite{Dini-Andreote2021}.

Islands are nature's labs \parencite{Whittaker2017} because of their isolation and smaller scale.
Borrowing this paradigm, soil microbes \parencite{Li2020} and mycorrhizal fungi \parencite{Delavaux2021} studies
in islands show the benefits of using islands as models. Ultimately islands can
set the ground to represent ecosystems with data and complex interactions \parencite{Davies2016}.
This was also the case for the Island Sampling Day (ISD) project \parencite{holm2024}
of the Genome Standards Consortium \parencite{Field2011}
during the 18th workshop in June 2026 in Crete island, Greece. The goal was to "put standards into action"
in a soil microbiome survey with a dense sampling, 0.017 samples per km\textsuperscript{2}.
Hence, ISD is a large scale study in the confined space of the island of Crete which 
is considered a miniature continent \parencite{Vogiatzakis2008_crete}.

Crete is a continental forarc island \parencite{ali2016}, fifth largest island of the mediterranean (8350 km\textsuperscript{2}),
and a mediterranean biodiversity hotspot \parencite{myers2000biodiversity}.
The island of Crete has been studied since the classical times for its'
fauna \parencite{Sidiropoulos_Polymeni_Legakis_2017,Anastasiou2018Tenebrionid}, flora \parencite{Krimbas_2005} and ecosystems \parencite{Grove1993}.
Crete is home to the only endemic mammal of Greece, the Cretan shrew (\textit{Crocidura zimmermanni}),
more than 350 endemic arthropods \parencite{bolanakis2024} and 183 endemic plants \parencite{Kougioumoutzis2020}
among them a tree \textit{Zelkova abelicea}. Multifacet factors have shaped the
biodiversity of the island, for example the sharp elevation gradient \parencite{trigas2013elevational, FAZAN2017},
the complex evolutionary history \parencite{POULAKAKIS2002} and the human - nature
interactions over thousands of years \parencite{Vogiatzakis2008_med, Sfenthourakis2017}.
The major threats of human activities are becoming apperend in the island's ecosystems,
like desertification \parencite{KARAMESOUTI2018266}, intesive grasing \parencite{JouffroyBapicot2016},
climate change \parencite{Kougioumoutzis2020,Vogiatzakis2016} and habitat loss \parencite{ISPIKOUDIS1993259}.
Yet the topsoil microbial diversity of Crete has been unexplored.

Worldwide projects of microbiome studies have collected one or two topsoil
samples from Crete \parencite{Vasar2022, Labouyrie2023, Bahram2018, Orgiazzi2018}.
Some have focused on soil fungi \parencite{Mikryukov2023, Davison2021, Tedersoo2021}
and other soil eukaryotes \parencite{Aslani2022}.
The only thorough microbiome study of a soil ecosystem in Crete, to our knowledge,
is in the north west part of the island, the Koiliaris Critical Zone Observatory \parencite{tsiknia2014}.
Apart from the soil microbiome, soil physical and chemical properties has been
investigaded by globan and european projects like the Forum of European Geological Surveys
(FOREGS) \parencite{nerc19017}, the Geochemical Mapping of Agricultural and Grazing Land
Soil in Europe (GEMAS) \parencite{REIMANN2018302} and the Soil Profile Analytical
Database for Europe (SPADE) \parencite{Hiederer2006}.

Data and metadata of these samplings are stored in different databases, yet 
great effort from these distinct communities have led to establishing standards
to enable FAIR data \parencite{wilkinson2016the-fair}. For amplicon sequences the Genome Standards
Consortium \parencite{Field2011} has established the MIMARKS \parencite{yilmaz2011minimum}
standards among others, and has been a advocate for open and unrestricted data \parencite{Amann2019}.
Examples of rich metadata and platforms of hosting open data and digital soil maps are the 
European Soil Data Centre \parencite{Panagos2022} and the World Soil Information
Service (WoSIS) of the ISRIC \parencite{Batjes2024}. Apart from samplings,
spatial data are openly available terrestrial ecosystems.
Climatic data, land cover, desertification risk, aridity, soil type, normalized
vegetation index, bedrock geological formations. From the bacterial point of view, 
there curated databeses that classify in some baseline functionality. 

Disentagling the soil ecosystem functioning requires an holistic and 
multidisciplinary approach \parencite{vogel2022}. The integration of the afformentioned
data is needed to understand the biogeochemical cycles along with biodiversity interactomes 
that have been characterised as a driver of community composition soil
functioning \parencite{GUSEVA2022108604}.
Regarding the latter there are still big challenges to infer actual microbial
interactions remain \parencite{Faust2021}. All this work is needed in order to meet
UN and EU soil goals for the 2030 and 2050 for healthy soils \parencite{LAL2021e00398}.

In this study we ask: what the differences in the microbiome communities in different land use types?
Are there any climatic, geological, elevational, aridity or functional factors that affected the differences?
How the interactome changes over different land use types and climate?
Are there any distictions between arid regions of Crete?
To address these questions, we integrate multiple types of data and methods to decipher hidden 
signals of the Island Sampling Day soil microbiome data. In total, 144 samples from
72 sites (2 samples per site) of Crete are used with their metadata.
Warning signals are also identified for various types of ecosystems.


\section{Materials and Methods}\label{methods}

\begin{figure}[t] 
    \centering\includegraphics[width=\columnwidth]{isd_crete_soil_microbiome}
    \caption{Workflow of this study. Data integration of ISD data with multiple types of spatial data. Then, a threefold analysis of function annotations, network analysis and differential abundance. All these data and methods are used to focus on specific taxa, ecosystems and threats.}
    \label{fig:workflow}
\end{figure}

\subsection{Island Sampling Day: Crete}\label{isd_data}

Ten teams followed a pre-defined sampling route of sampling locations. Samples
were collected in a single day in order to control for variations in environmental
factors that may impact the abundance and diversity of microbes such as season,
temperature and humidity. Soil samples were collected by 26 participants in 10 teams.
For each sampling location, teams selected two specific sampling sites that were
at least 10 feet from the edge of the road and 2 feet (0.6 m) from the base of
the identified tree or plant. Flora located at the sampling site were
identified and photographed. Soil was collected from the two subsite locations (5-10 feet (1.5-3 m) apart).
At each subsite, soil was sampled (3 replicates, collected one inch apart). 

Two replicate sets of the 144 subsite samples were shipped, on dry ice, to the
University of Maryland Soil Lab (Dr. Stephanie Yarwood) for RNA and DNA extraction
and soil chemistry analysis (USDA PERMIT NUMBER: P330-16-00090).
Prior to DNA extraction, the moisture, total organic Carbon and Nitrogen (CN) weights
and CN analysis were determined. The CN analysis was conducted on 10/4/2016.
Dr. Yarwood’s group submitted 144 combustion capsules of soil for C and N analyses using LECO CN628 

DNA was extracted from 144 soil samples following the Earth Microbiome Project (EMP)
standard protocols, utilzing the MoBio RNA extraction kit and the  MoBio DNA elution kit to go with RNA kit (Qiagen).
Quantification was performed using Qbit (Thermofisher) to confirm extraction, Qbit dsDNA
quantification. Three of the samples, collected in sand, had no detectable DNA
following extractions. Additionally, 17 samples had a DNA amount too low for metagenomics: 1 from ISD1, ISD4, and ISD35, 6 from ISD7, 4 from ISD8, and 3 from ISD10. A second attempt to re-extract with MoBio powermax kit (2g soil). 

Amplicon libraries were prepared and sequenced (16S ribosomal (rRNA) amplicon
sequenced (16S rRNA V3, V4 region)) on an Illumina HiSeq2500 with processing of
the reads conducted to determine the relative abundances and taxonomic diversity of bacterial taxa in soil.
Sequencing analysis was conducted at the Institute for Genome Sciences, Genomics
Resource Center (Maryland Genomics) at the University of Maryland School of Medicine.
The V3-V4 region of 16S rRNA gene from each sample was sequenced, using a
PCR-based protocol that targets the V3-V4 region of the 16S rRNA gene (V3 V4 primers: Forward: 5'-ACTCCTACGGGAGGCAGCAG-3'; Reverse: 5'-GGACTACHVGGGTWTCTAAT-3').

This study uses the ISD Crete data that have been deposited
in the European Nucleotide Archive (ENA) at EMBL-EBI under accession number PRJEB21776.
The raw sequences (fastq files) and metadata (xml files) were downloaded with custom scripts using the ENA API \parencite{Yuan2023}.
Specific information regarding the ISD sampling, DNA extractions, PCR and sequencing
protocols are discussed here \parencite{holm2024}. The bioinformatics workflow is 
reproducible as described in Appendix Figure \ref{fig:isd_workflow_taxonomy}.

Amplicon Sequence Variants were inferred using DADA2 \parencite{Callahan2016} for 
filtering, denoising and chimeric reads removal. Normalisation of the reads
across samples was implemented with the SRS R package \parencite{Beule2020}. Samples
with less than 10000 reads were removed. ASVs were assigned to taxonomy using 
DADA2 with Silva 138 \parencite{quast_silva_2013}.

\begin{figure}[t] 
    \centering\includegraphics[width=\columnwidth]{isd_map_fig1-small}
    \caption{Crete ISD sampling. A. the routes of the single day event sampling. B. Beta diversity differences are represented with color.}
    \label{fig:isd_crete_sampling}
\end{figure}

\subsection{Crete data cube}\label{data}

The complilation of Crete data cube has multiple spatial data layers of global,
european, greek or cretan scale. 
The Copernicus CORINE Land Cover has 3 layers resolution that classify the land
use and cover in shapefile format \parencite{CLC2023}. 
WorldClim 2.0 contains global climatic data for 12 variables, e.g annual mean
temperature and annual precipitation \parencite{Fick2017}.
We also utilised teh Environmentally Sensitive Areas Index to desertification (ESAI) 
of Greece dataset \parencite{KARAMESOUTI2018266}. Additionally we included the 
Global Aridity Index and Potential Evapotranspiration Database \parencite{zomer2022version}.
Geological formations shapefiles were downloaded from the geoportal of
Decentralized Administration of Crete, which were developed by
Crinno-Emeric Group project\footnote{\url{https://geoportal.apdkritis.gov.gr/gis/apps/storymaps/stories/19690f65abbe4e8ab0141b2fe7261a8c}}.
Handling and analysis of these data was done with the sf and terra R packages \parencite{Pebesma2023}.
%Harmonized World Soil Database

\subsection{Integrative Analysis and Annotations}\label{int_analysis}
The network inferrence was facilitated with FlashWeave 0.19.2 \parencite{Tackmann2019}.
To use FlashWeave we reduced 
the abandunce table of the ASVs to keep the ones at the genus or species level.
%In addition, we filtered these taxa that appeared in SOSO samples and had more than 
%SOSO mean relative abundance.
The subsequent network analyses were carried out
with the igraph R package \parencite{Csardi2006}.

For taxa function annotation we used the manually curated FAPROTAX database and python script \parencite{loucaDecouplingFunctionTaxonomy2016}.
Whereas for the differential abundance analysis with ANCOM-BC2 R package \parencite{Lin2023}.
Numerical ecology analyses,e.g diversity indices, NMDS, PERMANOVA we calculated
with the vegan R package \parencite{oksanen2024vegan}.
For PCoA ordination we used ape R packege \parencite{Paradis2004} and UMAP python library\parencite{mcinnes2018umap-software}.

\subsection{Tools}\label{Coding environment}
PEMA for OTU inferrence \parencite{zafeiropoulos2020pema}
U-CIE R package for coloring 3 dimentional data \parencite{Koutrouli2022}

Visualisation was implemented with ggplot2 \parencite{wickham_ggplot2_2016} and pheatmap \parencite{Kolde2019}.
The environment we worked had Python 3.11.4, R version 4.3.2 \parencite{rcoreteam}
and Julia language version 1.9.3 \parencite{Julia-2017}in Julia language version 1.9.3 \parencite{Julia-2017}.
Finally, computations were performed on HPC infrastrure of HCMR \parencite{zafeiropoulos_0s_2021}.

\section{Results}\label{results}

\subsection{Soil microbiome}\label{soil_microbiome}

The Illumina HiSeq2500 libraries yielded 51 million sequenced reads (16-20 GB) with
an average of 355,326 reads/sample (range 2,437-525144). Sequences and the
metadata thereof were submitted to the ENA under the study accession number PRJEB21776.
Sequencing produced 51 million reads, with an average of 250-500 K reads per
sample (15-20 GB of data), with an average of 355,326 average reads per sample
(range: 2,437-525,144) and less than 15,000 reads for reads of either less than
445 bp or over 515 bp, Appendix Figure \ref{fig:isd_srs-curve_samples}.

In summary, DADA2 resulted in 216,360 ASVs (2,704 unique
taxa; 1123 ASVs at species level and 1059 at genus level) and
PEMA (VSEARCH) in 13,285 OTUs.
The representative Phyla ( > 5\% presence in all samples) are:
Actinobacteriota, Proteobacteria, Chloroflexi, Acidobacteriota,
Bacterodota and Planctomycetota, Appendix Figure \ref{fig:isd_top_phyla_samples}.
There are 25 specialist (samples < 10 and mean relative
abundance > 0.003) and 146 generalist taxa (samples > 120), Figure \ref{fig:isd_fig2_taxonomy}.

\begin{figure}[t] 
    \centering\includegraphics[width=\columnwidth]{isd_fig2_taxonomy}
    \caption{Taxonomic prevalence and representative phyla of the soil bacteria of Crete. 
    A. Distribution of ASVs samples proportion. B. Distribution of taxa samples
proportion and categorisation in generalists and specialists. C. Phyla samples proportion.
D. Phyla relative abundance box plots, each dot represents one sample that the phylum occurs.}
    \label{fig:isd_fig2_taxonomy}
\end{figure}

Alpha-diversity (Shannon Index) is significantly differed by physical and chemical features.
Shannon diversity was negatively correlated with elevation (p=0.02), and
positively correlated with both soil moisture and organic carbon (p < 0.001).

\begin{table}[]
    \caption{Taxonomic depth of ASVs and the unique number of taxa of each level.\label{table:asv_taxonomy}}%
\begin{tabular}{@{}lll@{}}
classification depth & Total ASV    & Total taxa \\
Kingdom              & 1974         & 2          \\
Phylum               & 4034         & 33         \\
Order                & 38517        & 193        \\
Class                & 24157        & 83         \\
Family               & 71355        & 287        \\
Genus                & 90137        & 1166       \\
Species              & 9120         & 1338       \\
Total                & $\sim$239000 & 3102      
\end{tabular}
\label{table:asv_taxonomy}
\end{table}


\begin{sidewaystable*}
    \caption{Summary of the different spatial layers in Crete in terms of total area, number of samples and microbial diversity.\label{table:data_cube_summary}}
\begin{tabular*}{\textwidth}{@{\extracolsep{\fill}}llllllll@{\extracolsep{\fill}}}
\tabcolsep=0pt%
class                                           & area & category             & samples & taxa richness & asv richness & mean shannon & sd shannon \\
Arable land                                     & 88   & CLC LABEL2           & 4       & 1518           & 6178          & 4.73          & 0.17        \\
Artificial, non-agricultural vegetated areas    & 21   & CLC LABEL2           & NA      & NA             & NA            & NA            & NA          \\
Forests                                         & 300  & CLC LABEL2           & 4       & 1803           & 8663          & 4.91          & 0.18        \\
Heterogeneous agricultural areas                & 1103 & CLC LABEL2           & 31      & 13701          & 59581         & 4.83          & 0.24        \\
Industrial, commercial and transport units      & 40   & CLC LABEL2           & 4       & 1760           & 6517          & 4.79          & 0.32        \\
Inland waters                                   & 7    & CLC LABEL2           & NA      & NA             & NA            & NA            & NA          \\
Mine, dump and construction sites               & 10   & CLC LABEL2           & NA      & NA             & NA            & NA            & NA          \\
Open spaces with little or no vegetation        & 411  & CLC LABEL2           & 7       & 3132           & 14568         & 4.85          & 0.21        \\
Pastures                                        & 59   & CLC LABEL2           & 4       & 1609           & 7535          & 4.85          & 0.14        \\
Permanent crops                                 & 2368 & CLC LABEL2           & 22      & 9999           & 43833         & 4.88          & 0.18        \\
Scrub and/or herbaceous vegetation associations & 3798 & CLC LABEL2           & 58      & 24394          & 110127        & 4.76          & 0.29        \\
Urban fabric                                    & 111  & CLC LABEL2           & 4       & 1765           & 9828          & 4.78          & 0.19        \\
-                                               & 1    & Geology              & NA      & NA             & NA            & NA            & NA          \\
J-E                                             & 1347 & Geology              & 22      & 8995           & 41730         & 4.75          & 0.25        \\
K-E                                             & 248  & Geology              & NA      & NA             & NA            & NA            & NA          \\
K.k                                             & 1253 & Geology              & 27      & 11277          & 51737         & 4.72          & 0.25        \\
K.m                                             & 13   & Geology              & NA      & NA             & NA            & NA            & NA          \\
Mk                                              & 812  & Geology              & 13      & 5618           & 24633         & 4.79          & 0.2         \\
Mm.I                                            & 1614 & Geology              & 12      & 5259           & 22854         & 4.86          & 0.15        \\
Ph-T                                            & 1012 & Geology              & 14      & 6720           & 29742         & 4.92          & 0.19        \\
Q.al                                            & 911  & Geology              & 34      & 15580          & 64851         & 4.91          & 0.27        \\
T.br                                            & 300  & Geology              & 2       & 625            & 2839          & 4.4           & 0.24        \\
f                                               & 118  & Geology              & 2       & 720            & 4753          & 4.5           & 0.01        \\
fo                                              & 318  & Geology              & 2       & 825            & 4606          & 4.65          & 0.04        \\
ft                                              & 276  & Geology              & 10      & 4062           & 19085         & 4.79          & 0.17        \\
o                                               & 94   & Geology              & NA      & NA             & NA            & NA            & NA          \\
N                                               & 163  & Desertification Risk & NA      & NA             & NA            & NA            & NA          \\
F2                                              & 1945 & Desertification Risk & 53      & 23404          & 103956        & 4.82          & 0.26        \\
F1                                              & 1518 & Desertification Risk & 22      & 8817           & 41020         & 4.7           & 0.23        \\
P                                               & 1593 & Desertification Risk & 36      & 16256          & 73256         & 4.86          & 0.19        \\
C2                                              & 638  & Desertification Risk & 6       & 2113           & 9105          & 4.59          & 0.22        \\
Other areas                                     & 290  & Desertification Risk & NA      & NA             & NA            & NA            & NA          \\
F3                                              & 1144 & Desertification Risk & 14      & 5900           & 25362         & 4.82          & 0.28        \\
C1                                              & 788  & Desertification Risk & 7       & 3191           & 14131         & 4.93          & 0.25        \\
C3                                              & 238  & Desertification Risk & NA      & NA             & NA            & NA            & NA          \\
Semi-Arid                                       & 6336 & Aridity class        & 98      & 43064          & 184930        & 4.83          & 0.25        \\
Dry sub-humid                                   & 1343 & Aridity class        & 32      & 13649          & 65134         & 4.79          & 0.21        \\
Humid                                           & 545  & Aridity class        & 8       & 2968           & 16766         & 4.55          & 0.19       
\end{tabular*}
\end{sidewaystable*}


\subsection{Communities}\label{communities}
Microbial beta diversity differently associated with physical and chemical
features of ISD. PCoA 1 (describing 14\% of the variance) was largely driven by
elevation (F=23, p < 0.001). PCoA 2, which explained 12\% of the variance,
was significantly associated with soil moisture (F=90, p < 0.001).
PCoA 2 also positively correlated with both organic carbon and nitrogen.

Mean annual temperature, elevation and total nitrogen are statistically significant variables for
community dissimilarity across samples (PERMANOVA).

Network of associations (FlashWeave, sensitive) after filtering prevalent ASV (mean normalised relative
abundance>0.001, samples > 2) led to 7,455 ASVs and
29,282 associations (787 are negative).

\subsubsection{Functions}\label{functions}
The functional annotation was performed with FAPROTAX. Potential human pathogens 
appear concetrated in Richtis gorge. Most plant pathogens occur in an agricultural 
field in south Rethymnon. 

\begin{figure}[t] 
    \centering\includegraphics[width=\columnwidth]{isd_functions_faprotax}
\caption{Heatmap of the FAPROTAX functions relative abundances per sample.}
    \label{fig:isd_functions_faprotax}
\end{figure}

\subsubsection{Significant taxa}\label{sig_taxa}
Differential abundance (ANCOM-BC2) showed that there are 294 taxa that are significantly different
with Annual mean temperature and elevation. The geological rock type doesn't distinguish taxa
across the island. \textit{Rhodococcus equi} (pathogen of foal or similar) is significantly abundant in
pastures (Corine Land Cover), and has been found in 16 samples.

\section{Discussion}\label{discussion}

Multiple works have emerged the past 5 years about enumeratic all
biodivarsity \parencite{Anthony2023} and deciphering the biogeochemical 
processes and interactions of fauna, flora and microbes in global
studies \parencite{Fry2019, Crowther2019,GRANDY201640,Delgado-Baquerizo2020} and
in mountain peaks soil microbiomes \parencite{Adamczyk2019}. One of our profound
results is that soil bacterial biodiversity is very complex, even sites a few meters apart can differ
significantly in their community composition, Figure \ref{fig:isd_site_locations}.
This is a fact that is sometimes neglegted in worldwide studies and the island biogeography
paradigm can assist to remove clutter.

Apart from the profound diversity in soils, there is also high speciasation and uniqueness. 
As shown in Figure \ref{fig:isd_fig2_taxonomy}, most ASVs occur in 1 or 2 or 3 samples.
This is different when compering with the ocean. When using the deepest taxonomic
level of ASVs is possible to identify the specialists and generalists \parencite{Barberan2012}. 
In addition, focusing on the phyla, we see a pattern looking like a phase transition, from 
rare phyla to phyla that dominate all samples, Figure \ref{fig:isd_fig2_taxonomy} C. Lastly, in Figure \ref{fig:isd_fig2_taxonomy} D,
and \ref{fig:isd_top_phyla_samples}, we found some distinct top phyla profiles of samples.

Elevation gradients of biodiversity are known since Humboldt's work \parencite{Rahbek2019} 
yet these patterns remain elusive regarding the soil microbiome \parencite{Looby2020, Siles2023}.
Mostly because it's difficult to isolate other cofounding effects \parencite{Nottingham2018}.
In our results elevation showed important distinction of taxa and of diversity. Yet more work is 
needed to explore the Asterousia mountain transect in isolation to avoid indirect influences of 
other variables.

Crete's ecosystems are mostly semi-arid, whereas in the mountain ranges there 
are areas classified as dry sub-humid and humid. Parent material and infuences
on soil functions and bacterial communities have been documented. Here, as shown in Table \ref{table:data_cube_summary},
Quaternary-alluvial sediments (Q.al) and Phyllite-Quartzite series (Ph-T) hold the most diversity 
and richness. Maybe because they are mostly found in riverbeds. Critical areas for 
desertification (C2) hold the most diverse samples. Regarding the land cover the most
rich sample is near the HCMR building, a high touristic area close to the beach. Yet, forests
hold the highest shannon diversity index.

Richtis gorge (highly popular) has alarmingly high values for human pathogens, sulfur respiration and
nitrate reduction. This gorge has water all year long, rare in the gorges of Crete,
and is the highest touristic attraction of eastern Crete. It's name means "throw" and 
there is a local rumor that people threw unwanted stuff throughout the centuries. 
Nevertheless, it is an important freshwater ecosystem which is not included in 
any protection regimen or legislation.
The potential functions of the bacteria in Richtis gorge are an alarming signal which needs to be further investigated.
Another important finding is the statistically significant presence of \textit{Rhodococcus equi} (pathogen of foal or similar)
in pastures. It is one of the most common causes of pneumonia in foals which
become infected by inhaling dust or soil particles contaminated with the bacterium.

\section{Conclusion}

Deciphering and validating the results presented here requires future work.
Even though amplicon studies in soil should be interpeted with caution \parencite{alteio2021} they 
can act as early warning signals towards public health conserns \parencite{banerjee2023Soil}.
In addition, the immediate release and availability of these data is crutial for 
taking action.
The pillar of data integration is the unrestricted open data across disciplines and 
the open source software.
"A holistic perspective on soil architecture is needed as a key to soil functions" \parencite{philippot2024the-interplay}, is 
an important statement for future soil projects.
Shotgun metagenomics and metatranscriptomics can uleash the functional potential of
topsoil along with other advancements like long reads sequencing. Higher resolution
samplings using grid system will enhance the resolution and also the resampling of
ISD sites in different time points will provide additional insights to the complex soil 
functions and expand the positive and negative associations in soils \parencite{Liu2024}.
Lastly, global hotspots \parencite{Guerra2022} and soil ecosystem conservation is needed as 
a whole and expanding current protection of specific species \parencite{guerra2021tracking}.
This along with policy \parencite{KONINGER2022} across countries \parencite{Putten2023}
without borders and the implementation of legislation in Greece \parencite{SCHISMENOS2022100035} is 
imperative.


\section{Data and Code}
The documentation and scripts developed for this study are available in
\href{https://github.com/savvas-paragkamian/crete_soil_microbiome/}{Crete soil microbiome github repository}.
This repository contains all the necessary scripts for the data retrieval,
filtering and ASV inferrence, taxonomy assignment, data integration of spatial data, 
functional annotation and the subsequent analyses and visualisation.
Code is structured to be reproducible and interoperable.

\section{Competing interests}
No competing interest is declared.

\section{Author contributions statement}
Conceptualization: LS, EP, GK, AM;
Data curation: JH, LS, EP, SY, SP;
Formal Analysis: SP, JH;
Funding acquisition: LS, AM, SP;
Investigation: LS, SY, SP, JH;
Methodology: SP, GK, EP, LS, SY, CC, CP, HZ, MS;
Project administration: LS, EP, GK;
Resources: EP, LS, SY, AM;
Software: SP, JH, HZ;
Supervision: LS, EP, GK, PS;
Validation: JH, CP, CC, HZ, PS, DT, MS, LS;
Visualization: SP, JH;
Writing – original draft: SP;
Writing – review and editing: All authors.

\section{Acknowledgments}
The authors thank the anonymous reviewers for their valuable suggestions.

\section{Funding}
SP was supported by the 3rd H.F.R.I. Scholarships for PHD Candidates (no. 5726) for his work.

% --------------------------------------------------
% 
% This chapter is for Crete system ecology
% 
% --------------------------------------------------


\chapter{Towards a Cretan soil biodiversity data model}
\label{cha:crete-idea}

%\textbf{Citation:} \\ 


% ISD ABSTRACT
%\section{Abstract}
%    Microbes are known for their versatility, abundance
%    and influence on soil ecosystem functioning.
%    A synthesized knowledge base of microbial biodiversity, in terms of
%    ecological and remote-sensing data remains a major challenge.
%    Many worldwide studies have been published regarding soil
%    microbiome ecosystems, though there are still many blind spots.
%    Islands can be important case studies for this integration for more resolute and dense samplings.
%    Here, we utilize the Island Sampling Day Crete 2016 microbial 16S rRNA gene
%    amplicon data, integrated with soil and remote
%    sensing data, to decipher the drivers of ecosystem function of the island.
%    The Island Sampling Day Crete 2016 project has collected 144 topsoil samples
%    from 72 sites, capturing a lot of this diversity, accompanied by FAIR
%    (Findable, Accessible, Interoperable and Reproducible) data by design. 
%    Cretan macroecology has been studied for centuries for its diverse  and endemic
%    fauna and flora.
%    In addition, Crete has been considered as a miniature continent with high contrasts in
%    vegetation cover, elevation, climatic conditions. 
%    We show that, higher altitudes in Crete found to
%    be inhabited by a more diverse number of microorganisms, a pattern commonly
%    seen in several faunistic groups, such as arthropods.
%    The integration of the spatial data with state of the art methods enabled warning signals
%    in pristine and grazing ecosystems.
%    These results along with the
%    climatic and desertification index influences on the soil microbiome of Crete,
%    provide the basis to identify major drivers of biodiversity, to evaluate hotspots
%    and contribute to foreknowledge of threatened ecosystems.
%
\section{Introduction}\label{intro_idea}

Examples of rich metadata and platforms of hosting open data and digital soil maps are the 
European Soil Data Centre \parencite{Panagos2022} and the World Soil Information
Service (WoSIS) of the ISRIC \parencite{Batjes2024}. Apart from samplings,
spatial data are openly available terrestrial ecosystems.
Climatic data, land cover, desertification risk, aridity, soil type, normalized
vegetation index, bedrock geological formations. From the bacterial point of view, 
there curated databases that classify in some baseline functionality. 




\section{Materials and Methods}\label{integration_methods}

\subsection{Literature}\label{crete-literature}

Pubmed was searched using the DIG tool \parencite{fanini2021coupling}.

\subsection{Samplings}\label{crete_samplings}

\subsection{Spatial data}\label{crete_spatial}


\subsection{Tools}\label{Coding environment}
Visualisation was implemented with ggplot2 \parencite{wickham_ggplot2_2016} and pheatmap \parencite{Kolde2019}.
The environment we worked had Python 3.11.4, R version 4.3.2 \parencite{rcoreteam}.
Finally, computations were performed on HPC infrastructure of HCMR \parencite{zafeiropoulos_0s_2021}.

\subsection{Data and Code}
Scripts about data integration are available in
\href{https://github.com/savvas-paragkamian/crete-data-integration}{Crete data integration}.
Code is structured to be reproducible and interoperable.

\section{Results}\label{crete_idea_results}

\subsection{Historical and Contemporary literature}


\subsection{Biodiversity and samplings}

Worldwide projects of microbiome studies have collected one or two topsoil
samples from Crete \parencite{Vasar2022, Labouyrie2023, Bahram2018, Orgiazzi2018}.
Some have focused on soil fungi \parencite{Mikryukov2023, Davison2021, Tedersoo2021}
and other soil eukaryotes \parencite{Aslani2022}.
The only thorough microbiome study of a soil ecosystem in Crete, to our knowledge,
is in the north west part of the island, the Koiliaris Critical Zone Observatory \parencite{tsiknia2014}.
Apart from the soil microbiome, soil physical and chemical properties has been
investigated by global and European projects like the Forum of European Geological Surveys
(FOREGS) \parencite{nerc19017}, the Geochemical Mapping of Agricultural and Grazing Land
Soil in Europe (GEMAS) \parencite{REIMANN2018302} and the Soil Profile Analytical
Database for Europe (SPADE) \parencite{Hiederer2006}.
LTER map

\begin{figure}[h] 
    \centering\includegraphics[width=\columnwidth]{crete_integration_ena_terrestrial.png}
    \caption{Available terrestrial metagenomic samples from ENA in Crete.}
    \label{fig:isd_crete_ena}
\end{figure}


\begin{figure}[h] 
    \centering\includegraphics[width=\columnwidth]{crete_integration_map_wosi_soil.png}
    \caption{Soil samples from Crete that are uploaded in WoSIS.}
    \label{fig:isd_crete_wosis}
\end{figure}

GBIF map 


\subsection{Maps}

Multiple maps for Crete.


\section{Discussion}\label{crete_idea_discussion}

\section{Conclusion}




% --------------------------------------------------
% 
% This chapter is for Crete system ecology
% 
% --------------------------------------------------


%\chapter{Towards a Cretan soil biodiversity data model}
\chapter{Exploring the soil microbiome coupling with Crete island data cube}
\label{cha:crete-soil}

%\textbf{Citation:} \\ 


% ISD ABSTRACT
%\section{Abstract}
%    Microbes are known for their versatility, abundance
%    and influence on soil ecosystem functioning.
%    A synthesized knowledge base of microbial biodiversity, in terms of
%    ecological and remote-sensing data remains a major challenge.
%    Many worldwide studies have been published regarding soil
%    microbiome ecosystems, though there are still many blind spots.
%    Islands can be important case studies for this integration for more resolute and dense samplings.
%    Here, we utilize the Island Sampling Day Crete 2016 microbial 16S rRNA gene
%    amplicon data, integrated with soil and remote
%    sensing data, to decipher the drivers of ecosystem function of the island.
%    The Island Sampling Day Crete 2016 project has collected 144 topsoil samples
%    from 72 sites, capturing a lot of this diversity, accompanied by FAIR
%    (Findable, Accessible, Interoperable and Reproducible) data by design. 
%    Cretan macroecology has been studied for centuries for its diverse  and endemic
%    fauna and flora.
%    In addition, Crete has been considered as a miniature continent with high contrasts in
%    vegetation cover, elevation, climatic conditions. 
%    We show that, higher altitudes in Crete found to
%    be inhabited by a more diverse number of microorganisms, a pattern commonly
%    seen in several faunistic groups, such as arthropods.
%    The integration of the spatial data with state of the art methods enabled warning signals
%    in pristine and grazing ecosystems.
%    These results along with the
%    climatic and desertification index influences on the soil microbiome of Crete,
%    provide the basis to identify major drivers of biodiversity, to evaluate hotspots
%    and contribute to foreknowledge of threatened ecosystems.
%
\section{Introduction}\label{intro_integration}

Soil ecosystems are the cornerstone of terrestrial habitats, biodiversity and henceforth human activities.
Soils are characterised by multiple properties; chemical, physical and biological that 
form complex interdependent interactions. Biodiversity of soils covers
all forms of life, fauna, flora, bacteria, archaea, fungi, viruses. 
Bacteria and archaea are considered major drivers for the functionality of soil.
They influence and are influenced by their environment and their community structure 
defies their macroscopic functionality \parencite{Bahram2018}.
Global soil microbiome studies have been employed to decipher soil microbiome
compositions \parencite{thompson2017a-communal, Delgado-Baquerizo-atlas, Labouyrie2023},
functions \parencite{Bahram2018} and biogeography \parencite{Martiny2006, guerra2020Blind}.
These results showed the remarkable diversity in soils yet there are blind spots \parencite{guerra2020Blind}
and these sampling are sparse when considering samples per area density. One of most resolute
study is by \parencite{Karimi2020} which exemplified the 
vast complexity of soil bacterial communities and the requirement of
dense samplings and isolated systems \parencite{Dini-Andreote2021}.

Data and metadata of these samplings are stored in different databases, yet 
great effort from these distinct communities have led to establishing standards
to enable FAIR data \parencite{wilkinson2016the-fair}. For amplicon sequences the Genome Standards
Consortium \parencite{Field2011} has established the MIMARKS \parencite{yilmaz2011minimum}
standards among others, and has been a advocate for open and unrestricted data \parencite{Amann2019}.
Examples of rich metadata and platforms of hosting open data and digital soil maps are the 
European Soil Data Centre \parencite{Panagos2022} and the World Soil Information
Service (WoSIS) of the ISRIC \parencite{Batjes2024}. Apart from samplings,
spatial data are openly available terrestrial ecosystems.
Climatic data, land cover, desertification risk, aridity, soil type, normalized
vegetation index, bedrock geological formations. From the bacterial point of view, 
there curated databases that classify in some baseline functionality. 

Crete is a continental forarc island \parencite{ali2016}, fifth largest island of the Mediterranean (8350 km\textsuperscript{2}),
and a Mediterranean biodiversity hotspot \parencite{myers2000biodiversity}.
The island of Crete has been studied since the classical times for its'
fauna \parencite{Sidiropoulos_Polymeni_Legakis_2017,Anastasiou2018Tenebrionid}, flora \parencite{Krimbas_2005} and ecosystems \parencite{Grove1993}.
Crete is home to the only endemic mammal of Greece, the Cretan shrew (\textit{Crocidura zimmermanni}),
more than 350 endemic arthropods \parencite{bolanakis2024} and 183 endemic plants \parencite{Kougioumoutzis2020}
among them a tree \textit{Zelkova abelicea}. Multifaceted factors have shaped the
biodiversity of the island, for example the sharp elevation gradient \parencite{trigas2013elevational, FAZAN2017},
the complex evolutionary history \parencite{POULAKAKIS2002} and the human - nature
interactions over thousands of years \parencite{Vogiatzakis2008_med, Sfenthourakis2017}.
The major threats of human activities are becoming apparent in the island's ecosystems,
like desertification \parencite{KARAMESOUTI2018266}, intensive grazing \parencite{JouffroyBapicot2016},
climate change \parencite{Kougioumoutzis2020,Vogiatzakis2016} and habitat loss \parencite{ISPIKOUDIS1993259}.
Yet the topsoil microbial diversity of Crete has been unexplored.

Disentangling the soil ecosystem functioning requires an holistic and 
multidisciplinary approach \parencite{vogel2022}. The integration of the aforementioned
data is needed to understand the biogeochemical cycles along with biodiversity interactomes 
that have been characterised as a driver of community composition soil
functioning \parencite{GUSEVA2022108604}.
Regarding the latter there are still big challenges to infer actual microbial
interactions remain \parencite{Faust2021}. All this work is needed in order to meet
UN and EU soil goals for the 2030 and 2050 for healthy soils \parencite{LAL2021e00398}.

In this study we ask: what the differences in the microbiome communities in different land use types?
Are there any climatic, geological, elevational, aridity or functional factors that affected the differences?
How the interactome changes over different land use types and climate?
Are there any distinctions between arid regions of Crete?
To address these questions, we integrate multiple types of data and methods to decipher hidden 
signals of the Island Sampling Day soil microbiome data. In total, 144 samples from
72 sites (2 samples per site) of Crete are used with their metadata.
Warning signals are also identified for various types of ecosystems.


\section{Materials and Methods}\label{integration_methods}

\begin{figure}[t] 
    \centering\includegraphics[width=\columnwidth]{crete_integration_soil_microbiome}
    \caption{Workflow of this study. Data integration of ISD data with multiple types of spatial data. Then, a threefold analysis of function annotations, network analysis and differential abundance. All these data and methods are used to focus on specific taxa, ecosystems and threats.}
    \label{fig:workflow}
\end{figure}

\subsection{Island Sampling Day: Crete}\label{isd_data}

\subsection{Crete data cube}\label{spatial_data}

The compilation of Crete data cube has multiple spatial data layers of global,
European, Greek or Cretan scale. 
The Copernicus CORINE Land Cover has 3 layers resolution that classify the land
use and cover in shapefile format \parencite{CLC2023}. 
WorldClim 2.0 contains global climatic data for 12 variables, e.g annual mean
temperature and annual precipitation \parencite{Fick2017}.
The Environmentally Sensitive Areas Index to desertification (ESAI) 
of Greece dataset \parencite{KARAMESOUTI2018266} were utilised. Additionally the 
Global Aridity Index and Potential Evapotranspiration Database \parencite{zomer2022version} was include.
Geological formations shapefiles were downloaded from the geoportal of
Decentralized Administration of Crete, which were developed by
Crinno-Emeric Group project\footnote{\url{https://geoportal.apdkritis.gov.gr/gis/apps/storymaps/stories/19690f65abbe4e8ab0141b2fe7261a8c}}.
The Harmonised World Soil database v2 was incorporated for the soil mapping units and 
the soil taxonomic classification \parencite{fao2023}.
Handling and analysis of these data was done with the sf and terra R packages \parencite{Pebesma2023}.

\subsection{Integrative Analysis and Annotations}\label{int_analysis}
The network inference was facilitated with FlashWeave 0.19.2 \parencite{Tackmann2019}.
To use FlashWeave we reduced 
the abundance table of the ASVs to keep the ones at the genus or species level.
%In addition, we filtered these taxa that appeared in SOSO samples and had more than 
%SOSO mean relative abundance.
The subsequent network analyses were carried out
with the igraph R package \parencite{Csardi2006}.

For taxa function annotation we used the manually curated FAPROTAX database and python script \parencite{loucaDecouplingFunctionTaxonomy2016}.
Whereas for the differential abundance analysis with ANCOM-BC2 R package \parencite{Lin2023}.
Numerical ecology analyses,e.g diversity indices, NMDS, PERMANOVA we calculated
with the vegan R package \parencite{oksanen2024vegan}.
For PCoA ordination we used ape R packege \parencite{Paradis2004} and UMAP python library\parencite{mcinnes2018umap-software}.

\subsection{Tools}\label{Coding environment}
PEMA for OTU inference \parencite{zafeiropoulos2020pema}
U-CIE R package for coloring 3 dimensional data \parencite{Koutrouli2022}

Visualisation was implemented with ggplot2 \parencite{wickham_ggplot2_2016} and pheatmap \parencite{Kolde2019}.
The environment we worked had Python 3.11.4, R version 4.3.2 \parencite{rcoreteam}
and Julia language version 1.9.3 \parencite{Julia-2017}in Julia language version 1.9.3 \parencite{Julia-2017}.
Finally, computations were performed on HPC infrastructure of HCMR \parencite{zafeiropoulos_0s_2021}.

\subsection{Data and Code}
The documentation and scripts developed for this study are available in
\href{https://github.com/savvas-paragkamian/crete_soil_microbiome/}{Crete soil microbiome github repository}.
This repository contains all the necessary scripts for the data retrieval,
filtering and ASV inference, taxonomy assignment, data integration of spatial data, 
functional annotation and the subsequent analyses and visualisation.
Additional scripts about data integration are available in
\href{https://github.com/savvas-paragkamian/crete-data-integration}{Crete data integration}.
Code is structured to be reproducible and interoperable.

\section{Results}\label{integration_results}


\subsection{Samplings}



\subsection{Data cube}

Crete data cube.


\subsection{Soil microbiome}\label{soil_microbiome}

\begin{sidewaystable*}
    \caption{Summary of the different spatial layers in Crete in terms of total area, number of samples and microbial diversity.\label{table:data_cube_summary}}
\begin{tabular*}{\textwidth}{@{\extracolsep{\fill}}llllllll@{\extracolsep{\fill}}}
\tabcolsep=0pt%
class                                           & area & category             & samples & taxa richness & asv richness & mean shannon & sd shannon \\
Arable land                                     & 88   & CLC LABEL2           & 4       & 1518           & 6178          & 4.73          & 0.17        \\
Artificial, non-agricultural vegetated areas    & 21   & CLC LABEL2           & NA      & NA             & NA            & NA            & NA          \\
Forests                                         & 300  & CLC LABEL2           & 4       & 1803           & 8663          & 4.91          & 0.18        \\
Heterogeneous agricultural areas                & 1103 & CLC LABEL2           & 31      & 13701          & 59581         & 4.83          & 0.24        \\
Industrial, commercial and transport units      & 40   & CLC LABEL2           & 4       & 1760           & 6517          & 4.79          & 0.32        \\
Inland waters                                   & 7    & CLC LABEL2           & NA      & NA             & NA            & NA            & NA          \\
Mine, dump and construction sites               & 10   & CLC LABEL2           & NA      & NA             & NA            & NA            & NA          \\
Open spaces with little or no vegetation        & 411  & CLC LABEL2           & 7       & 3132           & 14568         & 4.85          & 0.21        \\
Pastures                                        & 59   & CLC LABEL2           & 4       & 1609           & 7535          & 4.85          & 0.14        \\
Permanent crops                                 & 2368 & CLC LABEL2           & 22      & 9999           & 43833         & 4.88          & 0.18        \\
Scrub and/or herbaceous vegetation associations & 3798 & CLC LABEL2           & 58      & 24394          & 110127        & 4.76          & 0.29        \\
Urban fabric                                    & 111  & CLC LABEL2           & 4       & 1765           & 9828          & 4.78          & 0.19        \\
-                                               & 1    & Geology              & NA      & NA             & NA            & NA            & NA          \\
J-E                                             & 1347 & Geology              & 22      & 8995           & 41730         & 4.75          & 0.25        \\
K-E                                             & 248  & Geology              & NA      & NA             & NA            & NA            & NA          \\
K.k                                             & 1253 & Geology              & 27      & 11277          & 51737         & 4.72          & 0.25        \\
K.m                                             & 13   & Geology              & NA      & NA             & NA            & NA            & NA          \\
Mk                                              & 812  & Geology              & 13      & 5618           & 24633         & 4.79          & 0.2         \\
Mm.I                                            & 1614 & Geology              & 12      & 5259           & 22854         & 4.86          & 0.15        \\
Ph-T                                            & 1012 & Geology              & 14      & 6720           & 29742         & 4.92          & 0.19        \\
Q.al                                            & 911  & Geology              & 34      & 15580          & 64851         & 4.91          & 0.27        \\
T.br                                            & 300  & Geology              & 2       & 625            & 2839          & 4.4           & 0.24        \\
f                                               & 118  & Geology              & 2       & 720            & 4753          & 4.5           & 0.01        \\
fo                                              & 318  & Geology              & 2       & 825            & 4606          & 4.65          & 0.04        \\
ft                                              & 276  & Geology              & 10      & 4062           & 19085         & 4.79          & 0.17        \\
o                                               & 94   & Geology              & NA      & NA             & NA            & NA            & NA          \\
N                                               & 163  & Desertification Risk & NA      & NA             & NA            & NA            & NA          \\
F2                                              & 1945 & Desertification Risk & 53      & 23404          & 103956        & 4.82          & 0.26        \\
F1                                              & 1518 & Desertification Risk & 22      & 8817           & 41020         & 4.7           & 0.23        \\
P                                               & 1593 & Desertification Risk & 36      & 16256          & 73256         & 4.86          & 0.19        \\
C2                                              & 638  & Desertification Risk & 6       & 2113           & 9105          & 4.59          & 0.22        \\
Other areas                                     & 290  & Desertification Risk & NA      & NA             & NA            & NA            & NA          \\
F3                                              & 1144 & Desertification Risk & 14      & 5900           & 25362         & 4.82          & 0.28        \\
C1                                              & 788  & Desertification Risk & 7       & 3191           & 14131         & 4.93          & 0.25        \\
C3                                              & 238  & Desertification Risk & NA      & NA             & NA            & NA            & NA          \\
Semi-Arid                                       & 6336 & Aridity class        & 98      & 43064          & 184930        & 4.83          & 0.25        \\
Dry sub-humid                                   & 1343 & Aridity class        & 32      & 13649          & 65134         & 4.79          & 0.21        \\
Humid                                           & 545  & Aridity class        & 8       & 2968           & 16766         & 4.55          & 0.19       
\end{tabular*}
\end{sidewaystable*}


\subsection{Communities}\label{communities}
Microbial beta diversity differently associated with physical and chemical
features of ISD. PCoA 1 (describing 14\% of the variance) was largely driven by
elevation (F=23, p < 0.001). PCoA 2, which explained 12\% of the variance,
was significantly associated with soil moisture (F=90, p < 0.001).
PCoA 2 also positively correlated with both organic carbon and nitrogen.

Mean annual temperature, elevation and total nitrogen are statistically significant variables for
community dissimilarity across samples (PERMANOVA).

Network of associations (FlashWeave, sensitive) after filtering prevalent ASV (mean normalised relative
abundance>0.001, samples > 2) led to 7,455 ASVs and
29,282 associations (787 are negative).

\subsubsection{Functions}\label{functions}
PREGO


The functional annotation was performed with FAPROTAX. Potential human pathogens 
appear concentrated in Richtis gorge. Most plant pathogens occur in an agricultural 
field in south Rethymnon. 

\begin{figure}[t] 
    \centering\includegraphics[width=\columnwidth]{crete_integration_functions_faprotax}
\caption{Heatmap of the FAPROTAX functions relative abundances per sample.}
    \label{fig:isd_functions_faprotax}
\end{figure}

\subsubsection{Significant taxa}\label{sig_taxa}
Differential abundance (ANCOM-BC2) showed that there are 294 taxa that are significantly different
with Annual mean temperature and elevation. The geological rock type doesn't distinguish taxa
across the island. \textit{Rhodococcus equi} (pathogen of foal or similar) is significantly abundant in
pastures (Corine Land Cover), and has been found in 16 samples.

\section{Discussion}\label{integration_discussion}

Multiple works have emerged the past 5 years about enumerating all
biodiversity \parencite{Anthony2023} and deciphering the biogeochemical 
processes and interactions of fauna, flora and microbes in global
studies \parencite{Fry2019, Crowther2019,GRANDY201640,Delgado-Baquerizo2020} and
in mountain peaks soil microbiomes \parencite{Adamczyk2019}. One of our profound
results is that soil bacterial biodiversity is very complex, even sites a few meters apart can differ
significantly in their community composition, Figure \ref{fig:isd_site_locations}.
This is a fact that is sometimes neglected in worldwide studies and the island biogeography
paradigm can assist to remove clutter.

Apart from the profound diversity in soils, there is also high speciation and uniqueness. 
As shown in Figure \ref{fig:isd_fig2_taxonomy}, most ASVs occur in 1 or 2 or 3 samples.
This is different when compering with the ocean. When using the deepest taxonomic
level of ASVs is possible to identify the specialists and generalists \parencite{Barberan2012}. 
In addition, focusing on the phyla, we see a pattern looking like a phase transition, from 
rare phyla to phyla that dominate all samples, Figure \ref{fig:isd_fig2_taxonomy} C. Lastly, in Figure \ref{fig:isd_fig2_taxonomy} D,
and \ref{fig:isd_top_phyla_samples}, we found some distinct top phyla profiles of samples.

Elevation gradients of biodiversity are known since Humboldt's work \parencite{Rahbek2019} 
yet these patterns remain elusive regarding the soil microbiome \parencite{Looby2020, Siles2023}.
Mostly because it's difficult to isolate other co-founding effects \parencite{Nottingham2018}.
In our results elevation showed important distinction of taxa and of diversity. Yet more work is 
needed to explore the Asterousia mountain transect in isolation to avoid indirect influences of 
other variables.

Crete's ecosystems are mostly semi-arid, whereas in the mountain ranges there 
are areas classified as dry sub-humid and humid. Parent material and influences
on soil functions and bacterial communities have been documented. Here, as shown in Table \ref{table:data_cube_summary},
Quaternary-alluvial sediments (Q.al) and Phyllite-Quartzite series (Ph-T) hold the most diversity 
and richness. Maybe because they are mostly found in riverbeds. Critical areas for 
desertification (C2) hold the most diverse samples. Regarding the land cover the most
rich sample is near the HCMR building, a high touristic area close to the beach. Yet, forests
hold the highest Shannon diversity index.

Richtis gorge (highly popular) has alarmingly high values for human pathogens, sulfur respiration and
nitrate reduction. This gorge has water all year long, rare in the gorges of Crete,
and is the highest touristic attraction of eastern Crete. It's name means "throw" and 
there is a local rumor that people threw unwanted stuff throughout the centuries. 
Nevertheless, it is an important freshwater ecosystem which is not included in 
any protection regimen or legislation.
The potential functions of the bacteria in Richtis gorge are an alarming signal which needs to be further investigated.
Another important finding is the statistically significant presence of \textit{Rhodococcus equi} (pathogen of foal or similar)
in pastures. It is one of the most common causes of pneumonia in foals which
become infected by inhaling dust or soil particles contaminated with the bacterium.

\section{Conclusion}

Deciphering and validating the results presented here requires future work.
Even though amplicon studies in soil should be interpreted with caution \parencite{alteio2021} they 
can act as early warning signals towards public health concerns \parencite{banerjee2023Soil}.
In addition, the immediate release and availability of these data is crucial for 
taking action.
The pillar of data integration is the unrestricted open data across disciplines and 
the open source software.
"A holistic perspective on soil architecture is needed as a key to soil functions" \parencite{philippot2024the-interplay}, is 
an important statement for future soil projects.
Shotgun metagenomics and metatranscriptomics can unleash the functional potential of
topsoil along with other advancements like long reads sequencing. Higher resolution
samplings using grid system will enhance the resolution and also the resampling of
ISD sites in different time points will provide additional insights to the complex soil 
functions and expand the positive and negative associations in soils \parencite{Liu2024}.
Lastly, global hotspots \parencite{Guerra2022} and soil ecosystem conservation is needed as 
a whole and expanding current protection of specific species \parencite{guerra2021tracking}.
This along with policy \parencite{KONINGER2022} across countries \parencite{Putten2023}
without borders and the implementation of legislation in Greece \parencite{SCHISMENOS2022100035} is 
imperative.



% --------------------------------------------------
% 
% This chapter is for general conclusions
% 
% --------------------------------------------------

\chapter{Conclusions}
\label{cha:conclusions}

The work presented here has combined different approaches of contemporary 
ecological questions. Regarding microbial diversity, on the global scale the available 
knowledge was explored using literature and data mining and on the local scale
the soil microbial diversity of Crete was deciphered. Literature and data mining 
methodologies are also very useful to rescue historical biodiversity data which are
indispensable. This was demonstrated through the DECO workflow, a collection
of tools and standards aiming to assist the curators' process. From the comparison
of such tools it became clear that human curation is a transversal undertaking in all steps.
Expert curation was applied for the compilation of historical and contemporary
literature along with specimens from NHMC for the endemic Cretan arthropods occurrences.
After the compilation of the dataset it became clear that a conservation analysis
was a priority because the majority of species was predicted as threatened.
Nevertheless, the investigation of the soil dwelling arthropods with the soil microbial diversity
and plants is crucial for soil functioning. 

While analysing the ISD Crete 2016 we also exercised the replicability of the sampling in July of 2022.
Using the same protocols and locations, 29 people from HCMR, NHMC, UOC Biology
Department and citizen scientists where split in 10 teams and went sampling. 
The goal of this sampling was to collect a second time point of the same locations
to decipher the metagenomic content of soil. This was a voluntary work supported 
by the SUPP GEN project of HCMR. The DNA extraction and shipment was carried out 
by HCMR and sequencing by the Joint Genome Initiative. DNA extracted by the 72 locations 
is going to be sequenced using deep shotgun sequencing. This is one of the few large scale metagenomic soil projects in
Europe \parencite{nayfach2021a-genomic, ma2023a-genomic}. Currently, 
the ambitious \href{https://www.embl.org/about/info/trec/}{TREC project} is
ongoing aiming to fill this gap with.

Currently there is a wealth of available data and tools as demonstrated in 
multiple chapters in this PhD. Yet the basic conceptual challenges remain. 
Some of these can be formulated as : What are the causes of ecosystem collapse?
What is needed for a sustainable future?
How will climate warming change life on Earth?
These clear questions require what is called scientific transculturalism,
the process of integration of the three cultures—variance, coarse-graining, and exactitude \parencite{Enquist_2024}.
These cultures can be vaguely described as natural history, numerical ecology and complex systems ecology, respectively.
An important step to bring these cultures together is communication and openness across scientists.
These gaps must be eliminated soon to reach predictive ecology goals \parencite{mouquet_review_2015}.


%----------------------------------------------------------------------------------------
%	THESIS CONTENT - APPENDICES
%----------------------------------------------------------------------------------------
\appendix % Cue to tell LaTeX that the following "chapters" are Appendices

% Include the appendices of the thesis as separate files from the Appendices folder
% Uncomment the lines as you write the Appendices

\chapter{PREGO Appendix A}

\section{Mappings}
\label{app:A}


PREGO produces entity identifiers either by Named Entity Recognition (NER) with the EXTRACT tagger or by mapping retrieved identifiers to the selected ones. 
PREGO adopted NCBI taxonomy identifiers for taxa, Environmental Ontology for environments and Gene Ontology as a structure knowledge scheme for Processes (GObp) and Molecular Functions (GOmfs). 
The latter was for reasons that are two-fold, first Gene Ontology has a Creative Commons Attribution 4.0 License and second there are many resources that have mapped their identifiers to Gene Ontology.
MG-RAST metagenomes and JGI/IMG isolates annotations come with KEGG orthology (KO) terms; 
Struo-oriented genome annotations, on the other hand, have Uniprot50 ids. 
The mapping from KO to GOmf and Uniprot50 to GOmf is implemented via UniProtKB mapping files of their FTP server (see \texttt{idmapping.dat} and \texttt{idmapping\_selected.tab} files). 
By using the 3-column mapping file, the initial annotations were mapped to GOmf. As a complement, a list of metabolism-oriented KEGG ORTHOLOGY (KO) terms has been built (see \textit{prego\_mappings} in the Availability of Supporting Source Codes section).
Finally, as STRUO annotations refer to GTDB genomes, \href{http://ftp.tue.mpg.de/ebio/projects/struo/GTDB_release89/metadata/}{publicly available mappings} (accessed on 24 December 2021) were used to link the genomes used with their corresponding NCBI Taxonomy entries.



\section{Daemons}
\label{app:B}

An important component PREGO approach (Figure A1) is the regular updates which keep PREGO in line with the literature and microbiology data advances. 
The updates are implemented with custom scripts called daemons that are executed regularly spanning from once a month up to six-month cycles. 
This variation occurs because of the API requirements of each web resource as well as the computational intensity of the association extraction from the retrieved data.

\begin{figure}[ht]
   \centering
   \includegraphics[width=95mm]{figures/figure_A1_PREGO_daemons.png}
   \caption[PREGO DevOps]{Software daemons perform all steps of the PREGO methodology in a continuous manner similar to the Continuous Development and Continuous Integration method.}
   \label{fig:devops}
\end{figure}


Each Daemon is attached to a resource because its data retrieval methods (API, FTP) and following steps, shown in Figure A1, require special handling and multiple scripts (see \textit{prego\_daemons} in the Availability of Supporting Source Codes section).

\section{Scoring}
\label{app:C}


Scoring in PREGO is used to answer the questions:
\begin{itemize}
   \item Which associations are more thrustworthy?
   \item Which associations are more relevant to the user's query?
\end{itemize}

Relevant, informative, and probable associations are presented to the user through the three channels that were discussed previously. 
Each channel has its own scoring scheme for the associations it contains and all of them are fit in the interval $(0,5]$ to maintain consistency. 
The values of the score are visually shown as stars. 
The Genome Annotation and Isolates channel has fixed values of scores depending on the resource because Genome Annotation is straightforward, and the microbe id is known a priori. 
On the other hand, Environmental Samples channel data are based on samples, which contain metagenomes and OTU tables. 
Thus, it has two levels of organization, microbes with metadata, and sample identifiers. Each association of two entities is scored based on the number of samples they co-occur. 
A Literature channel scoring scheme is based on the co-mention of a pair of entities in each document, paragraph, and sentence. The differences in the nature of data require different scoring schemes in these channels.
The contingency table (Table~\ref{table:pregoA1}) of two random variables, $X$ and $Y$ are the starting point for the calculation of scores. The term $X = 1$ might be a specific NCBI id and $Y = 1$ a ENVO term. 
The $c_{1,1}$ is the number of instances that two terms of $X = 1$ and $Y = 1$ are co-occurring, i.e., the joint frequency. 
The marginals are the $c_{1,.}$ and $c{.,1}$ for $x$ and $y$, respectively, which are the backgrounds for each entity type. 
Different handling of these frequencies leads to different measures. 
There is not a perfect scoring scheme, just the one that works best on a particular instance. 
Consequently, scoring attributes require testing different measures and their parameters.



\begin{table}[ht]
   \centering
   \begin{tabular}{c|llll}
    & \multicolumn{4}{l}{Y = y} \\ \cline{2-5} 
   \multirow{4}{*}{X = x} &  & Yes & No & Total \\ \cline{3-5} 
    & \multicolumn{1}{l|}{Yes} & $c_{x,y}$ & $c_{x,0}$ & $c_{x,.}$ \\
    & \multicolumn{1}{l|}{No} & $c_{0,y}$ & $c_{0,0}$ & $c_{0,.}$ \\
    & \multicolumn{1}{l|}{Total} & $c_{.,y}$ & $c_{.,0}$ & $c_{.,.}$
   \end{tabular}
   \caption[PREGO contingency table between two terms]{Contingency table of co-occurrences between entities $X = x$ and $Y = y$. 
   This is the basic structure for all scoring schemes. $c_{x,y}$ is the count of the co-occurrence of these entities. $c_{x,.}$ is the count of the $x$ with all the entities of $Y$ type (e.g., Molecular function). Conversely, $c_{.,y}$ is the count of $y$ with all the entities of $X$ type (e.g., taxonomy}
   \label{table:pregoA1}
\end{table}


\section*{Literature Channel}

Scoring in the Literature channel is implemented as in STRING 9.1 \citep{franceschini2012string} and COMPARTMENTS \citep{binder2014compartments}, where the text mining method uses a three-step scoring scheme. 
First, for each co-mention/co-occurrence between entities (e.g., Methanosarcina mazei with Sulfur carrier activity), a weighted count is calculated because of the complexity of the text.  


\begin{equation}
   c_{x,y} = \sum_{k=1}^{n}{w_d \delta_{dk}(x,y) +w_p \delta_{p,k}(x,y) + w_s \delta_{sk}(x,y)}
   \label{eq:prego-score-1}
\end{equation}



Different weights are used for each part of the document ($k$) for which both entities have been co-mentioned, $w_d = 1$ for the weight for the whole document level, $w_p = 2$ for the weight of the paragraph level, and $w_s = 0.2$ for the same sentence weight. 
Additionally, the delta functions are one (Equation~\ref{eq:prego-score-1}) in cases the co-mention exists, zero otherwise. Thus, the weighted count becomes higher as the entities are mentioned in the same paragraph and even higher when in the same sentence.
Subsequently, the co-occurrence score is calculated as follows:

\begin{equation}
   score_{x,y} = c_{x,y}^a (\frac{c_{x,y} c_{.,.}}{c_{x, .}c_{.,y}})^{1-a}
   \label{eq:prego-score-2}
\end{equation}
   


where $a = 0.6$ is a weighting factor, and the $c_{x,.}$, $c_{.,.}$, 
$c_{.,y}$ are the weighted counts as shown in Table~\ref{table:pregoA1} estimated using the same Equation~\ref{eq:prego-score-2}. 
This value of the weighting factor has been chosen because it has been optimized and benchmarked in various 
applications of text mining~\citep{franceschini2012string, binder2014compartments, pletscher2015diseases}. 
The value of Equation~\ref{eq:prego-score-2} is sensitive to the increasing size of the number of documents (MEDLINE PubMed—PMC OA).
Therefore, to obtain a more robust measure, the value of the score is transformed to $z$-score. 
This transformation is elaborated in detail in the COMPARTMENTS resource \citep{binder2014compartments}. 
Finally, the confidence score is the $z$-score divided by two. Cases in which the scores exceed the (0,4] interval are capped to a maximum of 4 to reflect the uncertainty of the text mining pipeline.

\section*{Environmental Samples Channel}

Data from environmental samples are OTU tables and metagenomes. 
Thus, for each entity $x$, the number of samples is calculated as the background 
and a number of samples of the associated entity (metadata background) $c.,y$ (see Table~\ref{table:pregoA1}). 
Each association between entities $x$,$y$ has a number of samples, $c_{x,y}$ that they co-occur. 
Note that each resource is independent and the scoring scheme is applied to its entities. 
This means that the same association can appear in multiple resources with different scores. 
The score is calculated with the following formula:

\begin{equation}
   score_{x,y} = 2.0*{\frac{\sqrt{c_{x,y}}}{c_{.,y}^{0.1}}}
\end{equation}


This score is asymmetric because the denominator is the marginal of the associated entity. 
Thus, the score decreases as the marginal of $y$ is increasing, i.e., the number of samples that $y$ is found. 
On the other hand, it promotes associations in which the number of samples of 
the association are similar to the marginal of $y$. 
The exponents on the numerator and denominator equal to $0.5$ and 
to $0.1$, respectively, in order to reduce the rapid increase of score.
Lastly, the value of the score is capped in the range $(0,4]$.


\section{Bulk download}
\label{app:D}

   Users can also download programmatically all associations per channel through the links that are shown in Table~\ref{table:prego-appD-1}. 
   The data are compressed to reduce the download size and md5sum files are provided as well for a sanity check of each download.

   % PREGO BULK DOWNLOAD TABLE 
   \begin{table}[ht]
      
      \begin{adjustwidth}{-2cm}{}

      \begin{tabular}{llll}
      \toprule
      Channel & Link & md5sum & Size (in GB) \\ \midrule

      Literature & \href{https://prego.hcmr.gr/download/literature.tar.gz}{literature.tar.gz} & \href{https://prego.hcmr.gr/download/literature.tar.gz.md5}{literature.tar.gz.md5} & 5.4 \\

      \begin{tabular}[c]{@{}l@{}}Environmental \\ Samples\end{tabular} &
      \href{https://prego.hcmr.gr/download/environmental\_samples.tar.gz}{environmental\_samples.tar.gz} & 
      \href{https://prego.hcmr.gr/download/environmental\_samples.tar.gz.md5}{environmental\_samples.tar.gz.md5}
      & 0.69 \\

      \begin{tabular}[c]{@{}l@{}}Annotated \\ genomes and \\ isolates\end{tabular} & 
      \href{https://prego.hcmr.gr/download/annotated\_genomes\_isolates.tar.gz}{annotated\_genomes\_isolates.tar.gz} &
      \href{https://prego.hcmr.gr/download/annotated\_genomes\_isolates.tar.gz.md5}{annotated\_genomes\_isolates.tar.gz.md5} & 0.26 \\ \bottomrule
      \end{tabular}
      \end{adjustwidth}
      \caption[PREGO Bulk download links and md5sum files.]{Bulk download links and md5sum files.}
      \label{table:prego-appD-1}
   \end{table}


\chapter{Curation of historical literature Appendix} % Main appendix title

\label{AppendixB} 

   \begin{figure}[ht]
      \centering
      \includegraphics[width=\textwidth,height=\textheight,keepaspectratio]{figures/deco-figure-S1.jpg}
      \caption[GNRD taxon names identification]{Screenshot of the web application GNRD identifying taxon names.}
      \label{fig:gnrd-screenshot}
   \end{figure}

   \begin{figure}[ht]
      \centering
      \includegraphics[width=\textwidth,height=\textheight,keepaspectratio]{figures/deco-figure-S2.jpg}
      \caption[BOM performing NER]{Screenshot of the web application BOM performing NER. It provides taxon names, text snippets and term co-occurrences.}
      \label{fig:bom-screenshot}
   \end{figure}
   
   \begin{figure}[ht]
      \centering
      \includegraphics[width=\textwidth,height=\textheight,keepaspectratio]{figures/deco-figure-S3.jpg}
      \caption[Pensoft Annotator performing NER]{Screenshot of the web application Pensoft Annotator performing NER.}
      \label{fig:pensoft-annotator-screenshot}
   \end{figure}

%
\chapter{Crete Microbiome Appendix} % Main appendix title

\label{AppendixC} 

\section{Amplicon 16s rRNA is soil}

\section{Errors in Amplicon Microbial Ecology}

Microbial ecology based on amplicon 16s rRNA sequences has flourished since the
2010s. The endeavor to understand the microbial world faces multiple challenges
across the scientific workflow, from sampling to ecological analyses \cite{Lee2012}.

Errors propagate starting with the sampling. There are contaminations from the
people in the field, in the lab for the DNA extraction. 

Instrument errors from PCR amplification and errors from sequencing.
Approximation and computation errors from algoritms that cluster, measure similarities between
sequences.

Semantic errors because of reductionist approaches and/or oversimplification
of the microbial communities

\section{OTU vs ASV}

Amplicon rRNA sequencing provides a collection of sequence reads per sample. 
The ecological interpretation of the reads requires their transformation to
taxonomic information. To do so there are two approaches currently in use, 
the clustering method and the denoising method. With clustering reads are 
grouped together and a best representing sequence is produced for each 
cluster, i.e. the Operetional Taxonomic Unit. This approach makes the OTUs 
from different runs, i.e executions of the algorithm and/or different studies
incoperable and irreproducible.

Currently, many studies propose the use of Amplicon Sequence Variants \cite{Callahan2017}. 
ASVs are real biological sequences and can be used for comparison.

The influence of the different methods to subsequent ecological analyses has
little impact \cite{Glassman2018}.
%   \begin{figure}[h]
%      \centering
%      \includegraphics[width=\textwidth,height=\textheight,keepaspectratio]{figures/deco-figure-S1.jpg}
%      \caption[GNRD taxon names identification]{Screenshot of the web application GNRD identifying taxon names.}
%      \label{fig:gnrd-screenshot}
%   \end{figure}



%----------------------------------------------------------------------------------------
%	BIBLIOGRAPHY
%----------------------------------------------------------------------------------------

% Add ref 
\printbibliography[heading=bibintoc]

%----------------------------------------------------------------------------------------

\end{document}  
