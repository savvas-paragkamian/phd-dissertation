%%%%%%%%%%%%%%%%%%%%%%%%%%%%%%%%%%%%%%%%%
% Doctoral Thesis 
% LaTeX Template
% Version 2.5 (27/8/17)
%
% This template was downloaded from:
% http://www.LaTeXTemplates.com
%
% Version 2.x major modifications by:
% Vel (vel@latextemplates.com)
%
% This template is based on a template by:
% Steve Gunn (http://users.ecs.soton.ac.uk/srg/softwaretools/document/templates/)
% Sunil Patel (http://www.sunilpatel.co.uk/thesis-template/)
%
% Template license:
% CC BY-NC-SA 3.0 (http://creativecommons.org/licenses/by-nc-sa/3.0/)
%
%%%%%%%%%%%%%%%%%%%%%%%%%%%%%%%%%%%%%%%%%

%----------------------------------------------------------------------------------------
%	PACKAGES AND OTHER DOCUMENT CONFIGURATIONS
%----------------------------------------------------------------------------------------
%\PassOptionsToPackage{english,greek}{babel}
\documentclass[
11pt, % The default document font size, options: 10pt, 11pt, 12pt
%oneside, % Two side (alternating margins) for binding by default, uncomment to switch to one side
english, % ngerman for German
singlespacing, % Single line spacing, alternatives: onehalfspacing or doublespacing
%draft, % Uncomment to enable draft mode (no pictures, no links, overfull hboxes indicated)
%nolistspacing, % If the document is onehalfspacing or doublespacing, uncomment this to set spacing in lists to single
liststotoc, % Uncomment to add the list of figures/tables/etc to the table of contents
toctotoc, % Uncomment to add the main table of contents to the table of contents
%parskip, % Uncomment to add space between paragraphs
%nohyperref, % Uncomment to not load the hyperref package
headsepline, % Uncomment to get a line under the header
%chapterinoneline, % Uncomment to place the chapter title next to the number on one line
%consistentlayout, % Uncomment to change the layout of the declaration, abstract and acknowledgements pages to match the default layout
]{MastersDoctoralThesis} % The class file specifying the document structure

\usepackage[utf8]{inputenc} % Required for inputting international characters
\usepackage[T1]{fontenc} % Output font encoding for international characters
%\usepackage{textgreek}
%\usepackage[main=english,greek]{babel}

\usepackage{mathpazo} % Use the Palatino font by default

\usepackage[
backend=biber,
style=authoryear,
sorting=nyt]{biblatex} % Use the bibtex backend with the authoryear citation style (which resembles APA)

\addbibresource{references.bib} % The filename of the bibliography
% try to mention my works
\DeclareSourcemap{
  \maps[datatype=bibtex]{
    \map{
      \step[fieldsource=author,
            match=Paragkamian,
            final]
      \step[fieldset=keywords, fieldvalue=own]
    }
  }
}


\usepackage[autostyle=true]{csquotes} % Required to generate language-dependent quotes in the bibliography

% Here you can load other packages or provide your own definitions
% ----------------------------------------------------------------
% Finally, hyperref is used for PDF files.
\usepackage[pdfusetitle, colorlinks, citecolor, plainpages=false]{hyperref}
\usepackage{xcolor}
%----------------------------------------------------------------------------------------
% Colors
%----------------------------------------------------------------------------------------
\hypersetup{
linkcolor=teal
,citecolor=Green
}

% tables
\usepackage{graphicx}
% To have multirows in tables
\usepackage{multirow}


% To have rotated tables
\usepackage{rotating} 
\usepackage{float}

\usepackage{changepage}
% To have check boxes
\usepackage{pifont}
\usepackage{amsmath}
\usepackage{multirow}
% For lemas
\usepackage{amsthm}
\usepackage{amssymb}

% To have todo notes
%\usepackage{todonotes}

% To have part A, B etc in a figure
\usepackage{subcaption}
\usepackage{caption}

% to have multiline equations
\usepackage{mathtools}

%----------------------------------------------------------------------------------------
%	MARGIN SETTINGS
%----------------------------------------------------------------------------------------

\geometry{
	paper=a4paper, % Change to letterpaper for US letter
	inner=2.5cm, % Inner margin
	outer=3.8cm, % Outer margin
	bindingoffset=.5cm, % Binding offset
	top=1.5cm, % Top margin
	bottom=1.5cm, % Bottom margin
	%showframe, % Uncomment to show how the type block is set on the page
}

%----------------------------------------------------------------------------------------
%	THESIS INFORMATION
%----------------------------------------------------------------------------------------

\thesistitle{Deciphering the relation of the microbiome interactome with ecosystem function and biogeochemical processes} % Your thesis title, this is used in the title and abstract, print it elsewhere with \ttitle
\supervisor{Prof. Panagiotis F. \textsc{Sarris}} % Your supervisor's name, this is used in the title page, print it elsewhere with \supname
\examiner{} % Your examiner's name, this is not currently used anywhere in the template, print it elsewhere with \examname
\degree{Doctor of Philosophy} % Your degree name, this is used in the title page and abstract, print it elsewhere with \degreename
\author{Savvas \textsc{Paragkamian}} % Your name, this is used in the title page and abstract, print it elsewhere with \authorname
\addresses{} % Your address, this is not currently used anywhere in the template, print it elsewhere with \addressname

\subject{Microbial Ecology} % Your subject area, this is not currently used anywhere in the template, print it elsewhere with \subjectname
\keywords{ecosystem functioning, data integration, microbial ecology, text mining, spatial analysis, networks, conservation} % Keywords for your thesis, this is not currently used anywhere in the template, print it elsewhere with \keywordnames
\university{\href{https://www.uoc.gr}{University of Crete}} % Your university's name and URL, this is used in the title page and abstract, print it elsewhere with \univname
\department{\href{https://www.biology.uoc.gr}{University of Crete, Department of Biology}} % Your department's name and URL, this is used in the title page and abstract, print it elsewhere with \deptname
\group{\href{https://imbbc.hcmr.gr}{Institute of Marine Biology, Biotechnology and Aquaculture (IMBBC) - HCMR}} % Your research group's name and URL, this is used in the title page, print it elsewhere with \groupname
\faculty{\href{https://www.biology.uoc.gr/en/content/faculty-members}{Faculty, Department of Biology}} % Your faculty's name and URL, this is used in the title page and abstract, print it elsewhere with \facname

\AtBeginDocument{
\hypersetup{pdftitle=\ttitle} % Set the PDF's title to your title
\hypersetup{pdfauthor=\authorname} % Set the PDF's author to your name
\hypersetup{pdfkeywords=\keywordnames} % Set the PDF's keywords to your keywords
}

\begin{document}

\frontmatter % Use roman page numbering style (i, ii, iii, iv...) for the pre-content pages

\pagestyle{plain} % Default to the plain heading style until the thesis style is called for the body content
\hypersetup{linkcolor=teal}
%----------------------------------------------------------------------------------------
%	TITLE PAGE
%----------------------------------------------------------------------------------------

\begin{titlepage}
\begin{center}
\begin{minipage}{4cm}
\begin{flushleft}
    \raggedleft
\includegraphics{figures/uoc-logo-1.jpg} % University/department logo - uncomment to place it
\end{flushleft}
\end{minipage}
\begin{minipage}{6cm}
\begin{flushright}
%\vspace{1cm}
\LARGE \univname
%\vspace{1.0cm} % University name
\end{flushright}
\end{minipage}

\vspace{1cm}
\textsc{\Large Doctoral Thesis}\\[0.5cm] % Thesis type

\HRule \\[0.4cm] % Horizontal line
{\huge \bfseries \ttitle\par}\vspace{0.4cm} % Thesis title
\HRule \\[1.0cm] % Horizontal line
 
\begin{minipage}[t]{0.4\textwidth}
\begin{flushleft} \large
\emph{Author:}\\
\href{http://www.johnsmith.com}{\authorname} % Author name - remove the \href bracket to remove the link
\end{flushleft}
\end{minipage}
\begin{minipage}[t]{0.5\textwidth}
\begin{flushright} \large
\emph{Doctoral Advisory Committee:} \\
\href{https://www.imbb.forth.gr/imbb-people/en/sarris-members/item/2695-dr-panagiotis-f-sarris}{Prof. Panagiotis F. \textsc{Sarris}}\\ % Supervisor name - remove the \href bracket to remove the link  
\href{http://lab42open.hcmr.gr/people/evangelospafilis/}{Dr Evangelos \textsc{Pafilis}}\\ % Supervisor name - remove the \href bracket to remove the link  
\href{https://users.auth.gr/~iantonio/MEMBERSAntoniou.html}{Prof. Ioannis \textsc{Antoniou}}\\ % Supervisor name - remove the \href bracket to remove the link  
\emph{Examination Committee:} \\
\href{https://dinalika.weebly.com}{Prof. Dina \textsc{Lika}}\\ % Supervisor name - remove the \href bracket to remove the link  
\href{https://www.biology.uoc.gr/index.php/en/personnel/faculty-members?view=article&id=246:ladoukakis-emmanuel&catid=27:dep-en-gb}{Prof. Emmanuel \textsc{Ladoukakis}}\\ % Supervisor name - remove the \href bracket to remove the link  
\href{http://pop-gen.eu/wordpress/people}{Prof. Pavlos \textsc{Pavlidis}}\\ % Supervisor name - remove the \href bracket to remove the link  
\href{https://www.medschool.umaryland.edu/profiles/holm-johanna/}{Prof. Johanna \textsc{Holm}}\\ % Supervisor name - remove the \href bracket to remove the link  
\end{flushright}
\end{minipage}\\[1cm]
 

\large \textit{A thesis submitted in fulfillment of the requirements\\ for the degree of \degreename}\\[0.3cm] % University requirement text
\textit{in the}\\[0.4cm]
\deptname\\[0.4cm] 

\textit{and the}\\[0.4cm]
\groupname\\[0.5cm] % Research group name and department name
 
%%
\vspace{1cm}
{\large \today}\\[4cm] % Date
 
\end{center}
\end{titlepage}

%----------------------------------------------------------------------------------------
%	DECLARATION PAGE
%----------------------------------------------------------------------------------------

\begin{declaration}
\addchaptertocentry{\authorshipname} % Add the declaration to the table of contents
\noindent I, \authorname, declare that this thesis titled, \enquote{\ttitle} and the work presented in it are my own. The work of three chapters is 
published in peer reviewed journals with impact factor and I am the first author with shared first authorship, see \hyperref[app:publications]{Publications}. I confirm that:

\begin{itemize} 
    \item This work was done wholly while in candidature for a research degree at this University.
    \item No part of this thesis has previously been submitted for a degree or any other qualification at this University or any other institution.
    \item Where I have consulted the published work of others, this is always clearly attributed.
    \item Where I have quoted from the work of others, the source is always given. With the exception of such quotations, this thesis is entirely my own work.
    \item I have acknowledged all main sources of help.
    \item Where the thesis is based on work done by myself jointly with others, I have made clear exactly what was done by others and what I have contributed myself. The same is clearly indicated in each \hyperref[app:publications]{Publication} with the Contributor Roles Taxonomy (CRediT) system.
    \item This thesis was processed with Turnitin software using the account of University of Crete. The similarity report indicated a score below 10\% using the options settings "exclude bibliography, exclude quoted material".
\end{itemize}
% 
\noindent Signed:\\
\rule[0.5em]{25em}{0.5pt} % This prints a line for the signature
 
\noindent Date:\\
\rule[0.5em]{25em}{0.5pt} % This prints a line to write the date
\end{declaration}
%
%\cleardoublepage

%----------------------------------------------------------------------------------------
%	QUOTATION PAGE
%----------------------------------------------------------------------------------------

%\vspace*{0.2\textheight}

%\noindent\enquote{\itshape Thanks to my solid academic training, today I can write hundreds of words on virtually any topic without possessing a shred of information, which is how I got a good job in journalism.}\bigbreak

%\hfill Dave Barry

%----------------------------------------------------------------------------------------
%	ABSTRACT PAGE
%----------------------------------------------------------------------------------------

\begin{abstract}
\addchaptertocentry{\abstractname} % Add the abstract to the table of contents

To comprehend ecosystem functioning, it's imperative to discern
the processes occurring in various environments (where) and the organisms
responsible for them (who). Ecosystems form complex associations of taxa,
abiotic parameters and physical characteristics of the materials. For example,
the soil ecosystem holds 10 billion cells in each gram. The texture and the 
chemistry of the soil affects and is affected by the existing taxa across the
tree of life, plants, arthropods, bacteria, fungi etc.

Apart from plants and fauna, microbes significantly influence soil ecosystem
functioning, yet synthesizing microbial biodiversity data remains challenging.
Island biogeography presents a unique opportunity, to advance the understanding
of microbiome diversity in the soil environment.
Crete, as a continental island, has a distinct natural and evolutionary history
with extreme contrasts in vegetation cover, climatic conditions and geology. 
It has been studied for centuries and because of its location in southeastern
Mediterranean is important for climate change studies.
To this end, the first island-wide soil microbiome study conducted on the
island of Crete in 2016. Together, a team of researchers and citizen scientists collected 435 soil samples
from 72 sites across four distinct ecozones in a single day.
Assessing the microbiome diversity drivers revealed that
Crete microbial diversity is driven by total nitrogen and soil moisture along elevation
gradient. But some patterns and functions of the soil microbiome of Crete weren't explained 
by chemical characteristics. 

Regarding microbes and
metagenomics there is a wealth of data in open omics databases and in the 
literature. Yet, all this information is inhomogeneous and spread in separate 
resources. PREGO, a comprehensive knowledge base, amalgamates
text mining and data integration techniques to extract these what-where-who
associations from scattered scientific literature and omics repositories. It
identifies microorganisms, biological processes, and environmental types,
mapping them to ontology terms. Through text and metagenomics data analysis,
PREGO extracts associations, assigning confidence levels via a scoring scheme.
With 364,508 microbial taxa, 1090 environmental types, 15,091 biological processes,
and 7,971 molecular functions, PREGO aims to aid researchers in experiment
design and interpretation.
Additionally, it facilitates exploration of
environment-process-microbe associations.
For example, regarding soil ecosystems, 
49 soil environments are linked with high score associations to 6276 taxa, 169 biological processes, and
3017 molecular functions. The vast majority of 10,929 associations involved
environments and organisms, followed by 3139 associations between environments
and molecular functions, and 274 between environments and biological processes.

Ecosystem data integration is incorporating the available knowledge at the 
system level. This type of integration is also needed to decipher ecosystem 
function.
Data about Crete island was categorised in literature, samplings and maps. The
literature analysis showed the huge catalogue of historical datasets are not still 
incorporated in open databases. In addition, Crete has been sampled for soil and
biodiversity in many different ecosystems.
Regarding maps, there are different aspects of the soil ecosystem, climate,
geology, elevation and slope, soil type, land cover type, land management
are among the major aspects of the Crete Soil System.
In addition, Crete has been 
assessed for climate change and desertification in many studies adding 
more layers of information to create a Crete Soil System. 
All these data are further establishing Crete as a great soil model island.

Furthermore, to understand the current status of ecosystems is important to 
know their past conditions. Historical biodiversity documents are crucial for
long-term data cycles but pose challenges in data curation due to their historical
context. The data rescue process involves document digitization, transcription, information extraction
using text mining tools, and publication to standardized formats. Information
Extraction (IE) tools, evolved over the years, recognize entities in text, aiding in curation.
To this end, this work expanded on IE from a marine historical biodiversity perspective,
orchestrating tools to provide a unified methodology. The classification of
tools enables curators to choose based on their needs. A new tool, DECO, is
introduced, aiming to enhance the data rescue process in biodiversity research.

Studying ecosystems past and current conditions has made apparent the need for 
conservation actions because there are under multiple threats.
An example of ecosystem collapse is the arthropod decline, a globally documented
trend that remains insufficiently handled by the society. In
Crete, a biodiversity hotspot, research on its arthropod fauna dates back centuries.
Using the latest compiled data from the Natural History Museum of Crete,
the endemicity distributions and hotspots were identified for the arthropods of 
Crete. 
Tackling on the hotspot definition, different grids were tested and the 10km\textsuperscript{2}
was the most robust for these data. Using this grid, which is also the reference 
grid of European Environmental Agency, showed that
Araneae, Chilopoda, Coleoptera, Diplopoda, Heteroptera, Hymenoptera (Chrysididae, Formicidae, Symphyta),
Lepidoptera (Geometridae), Odonata, Orthoptera, Scorpiones and Trichoptera hotspots
are mostly distributed in Cretan mountains. This finding complies with 
similar studies on plants. These hotspots are mostly overlapping with 
protected areas, like the Natura2000 network.
However, human activities negatively impact these areas, raising 
concerns of the status of most endemic arthropods especially those that lack
sufficient protection.

Bringing it all together, the Island Sampling Day Crete data were further
enriched for microbial taxa with platforms like PREGO and
soil ecosystems with spatial and remote sensing data from the Crete Soil System.
Microbe and ecosystem integration are both necessary to decipher ecosystem
function drivers on the island.
High-altitude areas harbor 
lower diversity of microorganisms, reversing the patterns seen in other faunal groups like arthropods.
Spatial data integration identified warning signals in pristine and grazing
ecosystems, aiding in identifying biodiversity drivers and evaluating ecosystem threats.

In summary, studying ecosystems ecology with contemporary tools and data poses unique 
opportunities. Cumulative work across the centuries is becoming integrated to the 
digital and holistic representation of ecosystems. Multiple steps are still required 
to this realisation but there is a ground opportunity to unlock novel insights 
through the integration. Conservation and basic research are occurring simultaneously 
because the window of action in the face of ecosystem services collapse is narrow.
Major presupposition to accomplish this is open data, open standards and open 
source code. All these were implemented and promoted in all chapters of this work
to make it transparent, interoperable and reproducible.

%\textgreek{
%Για να κατανοήσουμε τη λειτουργία των οικοσυστημάτων, είναι ανάγκη να διακρίνουμε
%τις διεργασίες που συμβαίνουν σε διάφορα περιβάλλοντα και τους οργανισμούς
%που τις επιτελούν. Τα οικοσυστήματα αποτελούνται από πολύπλοκες αλληλεπιδράσεις τάξων,
%αβιοτικών παραμέτρων και τα φυσικά χαρακτηριστικά των υλικών. Για παράδειγμα,
%το οικοσύστημα του εδάφους περιέχει 10 δισεκατομμύρια κύτταρα σε κάθε γραμμάριο. Η υφή και η
%η χημεία του εδάφους επηρεάζει και επηρεάζεται από τα υπάρχοντα τάξα από όλο το
%δέντρο της ζωής, φυτά, αρθρόποδα, βακτήρια, μύκητες κλπ. Όσον αφορά τα μικρόβια
%υπάρχει πληθώρα δεδομένων σε ανοιχτές βάσεις δεδομένων και στη
%βιβλιογραφία. Ωστόσο, όλες αυτές οι πληροφορίες είναι ανομοιογενείς και αποθηκευμένες σε 
%βάσεις δεδομένων ως σιλό. Το PREGO, αποτελεί μια συγκεντρωτική βάση γνώσης. Μέσω
%τεχνικών εξόρυξης κειμένου και ενσωμάτωσης δεδομένων συγχωνεύει 
%συσχετίσεις οργανισμών, περιβάλλοντος και λειτουργιών από
%από την επιστημονική βιβλιογραφία και βάσεις δεδομένων. 
%Οι μικροοργανισμοί, βιολογικές διεργασίες και περιβαλλοντικοί τύποι αντιστοιχούνται
%με όρους οντολογίας και κάθε συσχέτιση κατηγοριοποιείται με ένα σύστημα αξιολόγησης 
%βαθμών εμπιστοσύνης.
%Με 364.508 μικροβιακά τάξα, 1090 περιβαλλοντικούς τύπους, 15.091 βιολογικές διεργασίες,
%και 7.971 μοριακές λειτουργίες, το PREGO στοχεύει να βοηθήσει τους ερευνητές στο 
%σχεδιασμό και ερμηνεία πειραμάτων αλλά και την διατύπωση νέων υποθέσων εργασίας. Επιπλέον, διευκολύνει την εξερεύνηση
%συσχετίσεις περιβάλλοντος-διαδικασιών-μικροβίων.
%
%Είναι σημαντικό να κατανοήθεί η τρέχουσα κατάσταση των οικοσυστημάτων. Σε αυτό είναι
%απαραίτητη η ιστορική γνώση που υπάρχει από προηγούμενες μελέτες.
%Τα ιστορικά έγγραφα βιοποικιλότητας είναι ζωτικής σημασίας για
%μακροπρόθεσμες αναλύσεις. Όμως η διάσωση και η επιμέλεια δεδομένων ιστορικών δεδομένων είναι δύσκολη λόγω του ιστορικού τους
%πλαισίου. Η διαδικασία διάσωσης δεδομένων περιλαμβάνει ψηφιοποίηση εγγράφων, μεταγραφή, εξαγωγή πληροφοριών
%χρησιμοποιώντας εργαλεία εξόρυξης κειμένου και τη μετέπειτα δημοσίευση σε βάσεις δεδομένων.
%Τα εργαλεία εξόρυξης γνώσης, που έχουν εξελιχθεί ραγδαία τα τελευταία χρόνια,
%αναγνωρίζουν οντότητες στο κείμενο, βοηθώντας στην επιμέλεια.
%Για το σκοπό αυτό, αυτή η εργασία εστίασε στην διάσωση γνώσης θαλάσσιας ιστορικής βιοποικιλότητας και την
%αξιολόγηση των εργαλείων για τον σχεδιασμό μιας ενοποιημένης μεθοδολογίας επιμέλειας. 
%Πηγαίνοντας ένα βήμα παραπέρα, ένα νέο εργαλείο, το DECO, αναπτύχθηκε
%με στόχο την ενίσχυση της διαδικασίας διάσωσης δεδομένων στην έρευνα για τη βιοποικιλότητα.
%
%Η μελέτη των οικοσυστημάτων του παρελθόντος και των σημερινών συνθηκών έχει κάνει εμφανή την ανάγκη για
%δράσεις διατήρησης επειδή βρίσκονται υπό πολλαπλές απειλές.
%Ένα παράδειγμα είναι η κατάρρευση των αρθροπόδων, μια παγκόσμια τεκμηριωμένη
%τάση που έχει αντιμετωπιστεί ανεπαρκώς από την κοινωνία.
%Η Κρήτη είναι ένα hotspot βιοποικιλότητας και η έρευνα για την πανίδα των αρθροπόδων της χρονολογείται από τον 19ο αιώνα.
%Στην παρούσα εργασία αξιολογήθηκε η κατάσταση της διατήρησης των ενδημικών αρθρόποδων της Κρήτης χρησιμοποιώντας
%μια αυτοματοποιημένη αξιολόγηση (PACA) και η επικάλυψη των κατανομών των αρθρόποδων
%στις προστατευόμενες περιοχές Natura 2000. Επιπλέον 
%να διερευνήθηκαν τα hotspot ενδημικότητάς τους και προτάθηκαν υποψήφιες Βασικές Περιοχές Βιοποικιλότητας.
%Για αυτές τις αναλύσεις, συγκεντρώθηκαν παρουσίες του ενδημικών αρθροπόδων της Κρήτης
%που βρίσκονται στις συλλογές του Μουσείου Φυσικής Ιστορίας της
%Κρήτη μαζί με στοιχεία από την εκτενή βιβλιογραφία. Η αξιολόγηση της κατάστασης διατήρησης με την μέθοδο PACA
%κατέταξε το 75\% των ενδημικών αρθροπόδων ως πιθανά απειλούμενα.
%Τα ενδημικά hotspots και οι υποψήφιες βασικές περιοχές βιοποικιλότητας βρίσκονται κυρίως σε
%ορεινές περιοχές, που συχνά επικαλύπτονται με προστατευόμενες περιοχές Natura 2000.
%Ωστόσο, οι ανθρώπινες δραστηριότητες επηρεάζουν αυτές τις περιοχές και ορισμένα αρθρόποδα δεν 
%προστατεύονται επαρκώς.
%
%Εκτός από την χλωρίδα και την πανίδα, τα μικρόβια επηρεάζουν σημαντικά το οικοσύστημα του εδάφους.
%Όμως η αποσαφήνιση των λειτουργιών της μικροβιακής βιοποικιλότητας παραμένει πρόκληση.
%Η νησιωτική βιογεωγραφία προσφέρει μια μοναδική ευκαιρία, για να προωθηθεί η κατανόηση
%της ποικιλότητας των μικροβιωμάτων στο εδαφικό περιβάλλον.
%Η Κρήτη, ως ηπειρωτικό νησί, έχει μια ξεχωριστή φυσική και εξελικτική ιστορία
%με ακραίες αντιθέσεις στη βλάστηση, τις κλιματικές συνθήκες και τη γεωλογία.
%Έχει μελετηθεί για αιώνες και λόγω της θέσης του στα νοτιοανατολικά
%της Μεσογείου είναι σημαντική για τις μελέτες για την κλιματική αλλαγή.
%Για το σκοπό αυτό, η πρώτη ολοκληρωμένη μελέτη μικροβιώματος εδάφους σε νησί διεξήχθη στο
%νησί της Κρήτης το 2016. Αξιολογήθηκαν οι δείκτες ποικιλότητας του μικροβιώματος και ως αποτέλεσμα
%φαίνεται να είναι ότι η μικροβιακή ποικιλότητα της Κρήτης επηρεάζεται από το pH και
%την υγρασία του εδάφους, σε σχέση και με το υψόμετρο. Αυτή η έρευνα συντονίστηκε από μια ομάδα ερευνητών η οποία συνέλεξε 435 δείγματα εδάφους
%από 72 τοποθεσίες σε τέσσερις διακριτές οικολογικές ζώνες της Κρήτης σε μια μέρα. The Island Sampling Day Κρήτη 2016
%ενσωματώνει επίσης, μικροβιακά δεδομένα με δεδομένα εδάφους, χωρικά και τηλεπισκόπησης.
%Βασικός στόχος είναι η ψηφιακή αναπαράσταση του εδάφους της Κρήτης για να αποκρυπτογραφήθούν
%οι λειτουργίες των ποικίλων οικοσυστημάτων του νησιού. Οι περιοχές με μεγάλο υψόμετρο φιλοξενούν μεγάλη βιοποικιλότητα
%μικροοργανισμών, αντικατοπτρίζοντας μοτίβα που παρατηρούνται σε άλλες ομάδες πανίδας όπως τα αρθρόποδα.
%Η ενοποίηση χωρικών δεδομένων ανέδειξε πιθανά σήματα αρνητικής ανθρωπογενούς επίδρασης σε παρθένα οικοσυστήματα αλλά και 
%βοσκοτόπια βοηθώντας στον εντοπισμό των παραγόντων της βιοποικιλότητας και στην αξιολόγηση των απειλών.
%
%Συνοψίζοντας, η μελέτη της οικολογίας των οικοσυστημάτων με σύγχρονα εργαλεία και δεδομένα
%έχει σημαντικές προοπτικές. Η συσσώρευση δεδομένων καταστεί εφικτή την
%ψηφιακή και ολιστική αναπαράσταση των οικοσυστημάτων. Απαιτούνται ακόμη πολλά βήματα
%για την υλοποίηση αυτή όμως υπάρχουν ευκαιρίες για να νέα ερωτήματα και νέες ιδέες
%μέσω της ενσωμάτωσης. Η προστασία, η διατήρηση και η βασική έρευνα πλεόν χρειάζεται να 
%γίνονται ταυτόχρονα επειδή τα περιθώρια δράσης ενόψει της κατάρρευσης των
%οικοσυστημάτων είναι στενά.
%}

\end{abstract}

\textbf{Keywords}

\keywordnames

%----------------------------------------------------------------------------------------
%	ACKNOWLEDGEMENTS
%----------------------------------------------------------------------------------------

\begin{acknowledgements}
\addchaptertocentry{\acknowledgementname} % Add the acknowledgements to the table of contents

First, I am truly thankful to Dr Kotoulas for introducing me to his HCMR colleagues and
made the connection to begin my PhD.
His true interest in deep scientific and philosophical questions is uplifting. He is constantly
joining people together around the globe while valuing most the creative working environment,
the new generations of scientists. Working with openness and without borders is an
important lesson he taught me that made possible this PhD.

I want to thank Dr Pafilis for having me in the PREGO project team, the Island Sampling
Day and teaching me how to make the most out
of previous work and implementing big projects in small steps, one step at a time. You are the only one that 
thoroughly read my work and came with questions. This level of interest was a great lesson for me. Also thank you 
for these amazing hackathons, a remarkable way to collaborate and setting up projects.
Many thanks go to Prof Panos Sarris for sharing his academic values and excellence and helping me in making
decisions. His deep knowledge in molecular microbiology and thirst for debate inspire for
a collaborating and creative environment. 
Prof Ioannis Antoniou thank you for being a concise and sincere mentor. Your amazing ability to focus
on the important questions and hold on to information that's relevant, removing the clutter, 
is remarkable. Really looking forward to being able to collaborating in ecological 
projects through your thermodynamics, network and dynamics expertise!
Thank you all as a committee, a great assembly, that guided me and simultaneously 
allowed me to develop my interests and creativity as a scientist.

\begin{figure}[hbt!] 
    \centering\includegraphics[width=\columnwidth]{passions.png}
    \caption{Four and a half years of beautiful friendships and expeditions}
    \label{fig:passions}
\end{figure}

I thank the super EMODnet team of HCMR, Dr Gerobasileiou, Dr Arvanitidis, Ms Mavraki and
Ms Georgia Sarafidou. What a great project and nice collaboration about Historical 
biodiversity literature.

Special thanks go to the Arthropods Lab of the Natural History Museum, to
Dr Apostolos Trichas and Ms Giannis Bolanakis for the long hours working and discussing together.
Thank you for contacting and sharing your 
precious data and exchanging movies, music and great scientific texts! Working with you is a joy!

My gratitude and respect go to all the people of Island Sampling Day project and especially
Lynn Schriml and Johanna Holm. I have learned and continue to learn so much from 
you. Working together in a hobby project and giving our care means a lot to me and your
generosity is so humbling.

Haris Zafeiropoulos, my friend, I look forward to many collaborations in the future. Seeing you code and work
is inspiring! Most of all, feeling your positive spirit in people has made me more
open that I thought I could be, life is more fun this way!

Special thanks go to the Heinz Troll photography crew! Thank you Heinz and Stef for teaching me 
how to travel the world, how to be a professional and also for having me in your photoshoots and 
supporting me. You are my unofficial sponsor!

During this PhD I realised many things, the most important being trying to be around the people I love no matter the job.
Giota, Maria, Kaloust thank you so much for the unconditional love and acceptance of people as they are.
My friends Dimitri, Kosta, Stefane thank you for the amazing gatherings in the mountains and discussions while 
hanging in the void. 

To CAT, my Christina, you are the music, the dance, the laughter, the truth, thank you showing me 
how to enrich life and love. Looking forward for our explorations together and sharing our passions.

\end{acknowledgements}

%----------------------------------------------------------------------------------------
%	PREFACE
%----------------------------------------------------------------------------------------

\addcontentsline{toc}{chapter}{Preface}
\chapter*{Preface}

Being an bioinformatician means spending a most of my time in the office. This is a 
stark contrast in what I do in other aspects of my life. Canyoning, Climbing, Caving 
outdoor activities in heights using ropes is truly what brings me peace. Yet the 
amazing questions that come from ecology and the origin of life require theoretical 
knowledge, mathematics and programming. One important realisation
during my PhD was that I felt sheer joy when combining these two extremes. This was
of course during the microbial sampling of Crete and of the deepest cave of Greece, Gourgouthakas
and in capturing raptor birds in the islands of the Aegean sea (Figure \ref{fig:passions}). Bringing data from 
extreme environments and then, after the incredible bench work of colleagues, analyse 
these data made my heart sing.

Another aspect of being a bioinformatician is working on different projects with 
multiple people. This realisation was a revelation to me and this PhD is the outcome 
of deep collaboration with four different groups of people and many new friendships. 
The chapters of this PhD cover many disciplines, similar to most bioinformatic PhDs,
yet they all converge to integrating multiple types of data to investigate the multifaceted
and complex nature and functioning of ecosystems. The main focus is the soil ecosystems and 
their microbiome, with the island of Crete being the field of application.

Academic writing and publishing is important, albeit difficult, and I have learned a
lot during the PhD. Three chapters are published in journals as
mentioned in each chapter. In these papers I share first authorship, meaning that 
both first authors were indispensable for the work. For the complete list of my published work see \hyperref[cv:my_refs]{here}.
The first and last chapters will be 2 back to back submissions. I am proud for this level of collaboration
with so diverse teams because it helped me develop teamwork thinking and trust. These skills are important 
for interdisciplinary work, like this PhD, and in modern science in general.

The chapters are :

\begin{enumerate}
    \item Island Sampling Day Crete project in Chapter \ref{cha:isd-crete-soil}
    \item PREGO project in Chapter \ref{cha:prego}
    \item Gathering soil data to establish a Cretan earth system in Chapter \ref{cha:crete-idea}
    \item DECO workflow in Chapter \ref{cha:deco}
    \item Endemic arthropods of Crete island in Chapter \ref{cha:arthropods}
    \item The data integration and the soil microbiome diversity in \ref{cha:crete-soil}
\end{enumerate}

The Chapter \ref{cha:isd-crete-soil} is about the Island Sampling Day (ISD), a large scale study for the microbial diversity of the
topsoil of Crete. It the first large scale microbiome sampling in Crete and I analysed
all this wealth of data to describe the hidden biodiversity. ISD is a 16s rRNA amplicon study designed to put standards into action. 
ISD is a citizen science standardisation effort, voluntarily designed and
operated by the participants of the 18th Genome Standards Consortium
Workshop and the Hellenic Centre for Marine Research that took place in June 2016.
This project adds a new chapter to the study of biodiversity in Crete,
the soil microbial diversity. But these data and metadata alone are 
insufficient to understand ecosystem function and even the drivers 
of microbial community structures. How to enrich these valuable data 
with existing available information? 

Regarding the microbes there a wealth of information hidden in text and metagenomic resources.
To address this issue, I co-developed PREGO, a global data and text mining metagenomic resourse, designed to assist microbiologists
to enrich their knowledge about microbes, the processes they perform and the environments 
they appear. We developed PREGO's knowledge base with both a web interface and an Application Programming Interface.
PREGO is a meta platform, which is based upon open data and standards of metagenomics. 
It brings the wealth of taxonomic, environmental and biological functions of 
microbial data and literature all in one place. Multiple platforms are harvested and the data 
and metadata are harmonised using text mining. In addition, academic texts contain 
valuable information that is missing from databases and text mining solutions unearth 
this information. Hence, the PREGO approach allowed to bring the heterogeneous
data together in a single knowledge base. 

How to bring all this information from literature, biodiversity and maps in one place? Focusing on Crete
provided a unique opportunity to use it as a model for the digital representation of ecosystems. 
As an island, it is confined in space, and also has been studied for thousands of years.
Being able to collect all the these data in one place allows to perform analyses on the soil ecosystems of Crete.
Hence, the chapter \ref{cha:crete-idea} is about the total environment of Crete.
In this chapter I address the questions of 

\begin{itemize}
    \item How many documents are containing valuable information?
    \item What kind of samplings have been carried out?
    \item Which spatial data about Crete are available? 
\end{itemize}

Bringing all this information together poses unique challenges of different types of data, standards and code. 
But with the establishment of this digital description of the open knowledge about Crete allows the generation of novel questions and hypotheses.
Additionally, a literature and biodiversity data summary is 
provided to set the basis for future integrative studies and for the documentation of the current status of 
research regarding the island.

Yet, another biodiversity pillar of the Cretan ecosystems is arthropods which 
interact with microbiomes. Based on a novel and dataset
that contained historical data and specimens of the Natural History Museum of Crete
we assessed the conservation status of the endemic arthropods of Crete. Doing a assessment of 
conservation of the endemic arthropods was an obvious priority. Because to have the ability to 
study them, first they have to be conserved from their documented collapse. Hence, this work contained 
work from the 19th century along with the 4 decades long samplings of the NHMC. 
The overlap of Cretan Arthropod distributions with the Natura 2000 protected areas
were investigated along with their endemicity hotspots and proposed candidate Key Biodiversity Areas.
This work is serves as an example to further highlight the importance of historical data along with contemporary.

Upon realising the struggles and the benefits of rescuing historical biodiversity data, I
developed a workflow to accelerate the process. Because, apart from the contemporary data and metagenomics, important biodiversity data is 
included in historical biodiversity documents. This information from 
previous centuries reports, books and expeditions notes is an important glimpse into
the past. These data due to their legacy format and other complexities, for example 
multiple languages, different taxa names and approximate coordinates, require 
enormous effort to rescue in databases. Curators are indispensable in this process 
yet there are multiple tools that assist them. Using text mining and other tools 
for Object Character Recognition we demonstrated a workflow to rescue such documents.
This workflow is named DECO and is a software container that can handle multiple 
documents and also combines multiple tools for the steps of the data rescue process.

The soil microbiome is core of the soil ecosystems, and using the aforementioned data I build the Crete Soil System.
In the final chapter, \ref{cha:crete-soil} of this PhD, I integrate the soil biodiversity, it's functional potential and 
the Crete Soil System to decipher the soil ecosystems of concern, vulnerable soils and one health related taxa. The results 
of this analysis provide a new perspective of some well known ecosystems of Crete.
But also sets an example of the importance of integration of prior knowledge and data to address current challenges.

There is so much production of papers and journals that is difficult to decide where to publish. 
So except from the common flagship journals, I discovered the work that shaped my 
PhD in themed issues of Philosophical Transactions of the Royal Society B.
A journal that really stands out and it was established on 1887! This was an
important realisation during the PhD; there are journals launched only 10 years
ago with incredible high impact factor but there are journals in ecology
with rich history and groundbreaking articles.

Most notably:

\begin{itemize}
    \item Conceptual challenges in microbial community ecology \href{https://doi.org/10.1098/rstb.2019.0241}{Volume 375, Issue 1798}
    \item Ecological complexity and the biosphere: the next 30 years \href{https://doi.org/10.1098/rstb.2021.0376}{Volume 377, Issue 1857}
    \item Integrative research perspectives on marine conservation \href{https://doi.org/10.1098/rstb.2019.0444}{Volume 375, Issue 1814}
\end{itemize}

Having these issues as guides covered all the different fields that I am interested 
in working on. Microbial ecology is a great field, here I focused on soil mostly, that
combines all aspects of numerical ecology. There are many things that scientists 
haven't deciphered yet, like how the above ground biodiversity influences the below
ground biodiversity. Ecological complexity is what inspired me to continue to a 
master's degree. Still, I am looking forward to apply and learn the amazing 
theoretical mathematical models. And lastly is the integration of multiple methods 
and data to describe an ecosystem. This is what I worked mostly on my PhD and 
harvesting different types of data, i.e sequences, spatial data, occurrence data, 
text etc, has it's unique challenges. Each type of data requires different tools and has 
different objects to handle them. Furthermore, types of data come from different 
fields with different scientific communities and standards. Incorporating and even 
discovering these data is time consuming and mind bending. Yet, utilising diverse 
published data to analyse an ecosystem can lead to novel insights. 

The projects that are presented in the following chapters are about 
bringing together contrasts. Starting with biodiversity, we investigated taxa that 
have been studied for centuries, e.g coleoptera, as well as new undiscovered ones, the soil microbiome of 
Crete. These works span from basic research to conservation analyses. 
Spatial analysis brought together the micro scale of microbes with the macro
scale of the island of Crete.
Microbes also unlocked some processes and functions they facilitate in their environments, thereby assisting in
categorising them and finding possible one health concerns.
Text mining techniques were used to incorporate historical texts and contemporary literature.
Lastly, in the Island Sampling Day project we covered
the whole spectrum of the analysis from large scale field sampling to informatics.

All this work was made possible because of the adoption of open data and standards
and open source software. Following these values, all the data and software produced
in this work are documented and deposited in platforms with open source licences. 
Working also with the Genome Standards Consortium was a great experience to witness
great scientists devoting a lot of time to design and promote standards across the 
community. This openness is both inspiring and humbling and I hope to contribute and follow 
their lead.

\addcontentsline{toc}{chapter}{Funding}
\chapter*{Funding}

The first two years of this PhD were about the PREGO project, discussed in Chapter \ref{cha:prego}. My work in this 
project was supported by the Hellenic Foundation for Research and
Innovation (HFRI) and the General Secretariat for Research and Innovation (GSRI),
under Grant No. 241, PREGO project.

The last two years were supported by the 3rd HFRI. Scholarships for PHD
Candidates (no. 5726) leading to the Chapters \ref{cha:isd-crete-soil}, \ref{cha:arthropods} and \ref{cha:crete-soil}.

   \begin{figure}[h]
      \centering
      \includegraphics[width=0.30\columnwidth]{figures/ELIDEK_Logo_EN_and_GSRI_logo_EN.png}
      \caption[HFRI and GSRI funding]{
          This PhD was supported by HFRI - GSRI grands. 
      }
      \label{fig:hfri_logo}
   \end{figure}


%----------------------------------------------------------------------------------------
%	LIST OF CONTENTS/FIGURES/TABLES PAGES
%----------------------------------------------------------------------------------------

\tableofcontents % Prints the main table of contents

\listoffigures % Prints the list of figures

\listoftables % Prints the list of tables

%----------------------------------------------------------------------------------------
%	ABBREVIATIONS
%----------------------------------------------------------------------------------------

\begin{abbreviations}{ll} % Include a list of abbreviations (a table of two columns)

\textbf{API} & \textbf{A}pplication \textbf{P}rogramming \textbf{I}nterface\\
\textbf{ASV} & \textbf{A}mplicon \textbf{S}equencing \textbf{V}ariants\\
\textbf{BHL} & \textbf{B}iodiversity \textbf{H}eritage \textbf{L}ibrary\\
\textbf{BOM} & \textbf{B}iodiversity \textbf{O}bservations \textbf{M}iner\\
\textbf{BODC} & \textbf{B}ritish \textbf{O}ceanographic \textbf{D}ata \textbf{C}entre\\
\textbf{CLI} & \textbf{C}ommand \textbf{L}ine \textbf{I}nterface\\
\textbf{CTI} & \textbf{C}ommunity \textbf{T}emperature \textbf{I}ndex\\
\textbf{DECO} & Bio\textbf{D}iv\textbf{E}rsity Data \textbf{C}uration w\textbf{O}rkflow\\
\textbf{DPI} & \textbf{D}ots \textbf{P}er \textbf{I}nch\\
\textbf{DwC-A} & \textbf{D}arwin \textbf{C}ore \textbf{A}rchive\\
\textbf{eDNA} & \textbf{e}nvironmental \textbf{DNA}\\
\textbf{EEA} & \textbf{E}uropean \textbf{E}nvironment \textbf{A}gency\\
\textbf{EM} & \textbf{E}ntity \textbf{M}apping\\
\textbf{EMODnet} & \textbf{E}uropean \textbf{M}arine \textbf{O}bservation and \textbf{D}ata \textbf{Net}work\\
\textbf{ENA} & \textbf{E}uropean \textbf{N}ucleotide \textbf{A}rchive\\
\textbf{ENVO} & \textbf{Env}ironment \textbf{O}ntology\\
\textbf{EOL} & \textbf{E}ncyclopedia \textbf{O}f \textbf{L}ife\\
\textbf{FAIR} & \textbf{F}indable \textbf{A}ccessible \textbf{I}nteroperable \textbf{R}eusable\\
\textbf{GBIF} & \textbf{G}lobal \textbf{B}iodiversity \textbf{I}nformation \textbf{F}acility\\
\textbf{GLOBI} & \textbf{GLO}bal \textbf{B}iotic \textbf{I}nteractions\\
\textbf{GNA} & \textbf{G}lobal \textbf{N}ames \textbf{A}rchitecture\\
\textbf{GNRD} & \textbf{G}lobal \textbf{N}ames \textbf{R}ecognition and \textbf{D}iscovery\\
\textbf{GSC} & \textbf{G}enome \textbf{S}tandards \textbf{C}onsortium\\
\textbf{GUI} & \textbf{G}raphical \textbf{U}ser \textbf{I}nterface\\
\textbf{JGI} & \textbf{J}oint \textbf{G}enome \textbf{I}nstitute\\
\textbf{HPC} & \textbf{H}igh \textbf{P}erformance \textbf{C}omputing\\
\textbf{HTML} & \textbf{H}yper\textbf{T}ext \textbf{M}arkup \textbf{L}anguage\\
\textbf{HTTP} & \textbf{H}yper\textbf{T}ext \textbf{T}ransfer \textbf{P}rotocol\\
\textbf{ICZN} & \textbf{I}nternational \textbf{C}ommission on \textbf{Z}oological \textbf{N}omenclature\\
\textbf{ID} & \textbf{I}dentifier\\
\textbf{IE} & \textbf{I}nformation \textbf{E}xtraction\\
\textbf{IPT} & \textbf{I}ntegrated \textbf{P}ublishing \textbf{T}oolkit\\
\textbf{IR} & \textbf{I}nformation \textbf{R}etrieval\\
\textbf{ISD} & \textbf{I}sland \textbf{S}ampling \textbf{D}ay\\
\textbf{IUCN} & \textbf{I}nternational \textbf{U}nion for \textbf{C}onservation of \textbf{N}ature\\
\textbf{JSON} & \textbf{J}ava\textbf{S}cript \textbf{O}bject \textbf{N}otation\\
\textbf{LTER} & \textbf{L}ong \textbf{Term} \textbf{E}cological \textbf{R}esearch\\
\textbf{LSID} & \textbf{L}ife \textbf{S}cience \textbf{I}dentifier\\
\textbf{MedOBIS} & \textbf{Med}iterranean node of the \textbf{O}cean \textbf{B}iodiversity \textbf{I}nformation \textbf{S}ystem\\
\textbf{NCBI} & \textbf{N}ational \textbf{C}enter for \textbf{B}iotechnology \textbf{I}nformation\\
\textbf{NER} & \textbf{N}amed \textbf{E}ntity \textbf{R}ecognition\\
\textbf{NERC} & \textbf{N}atural \textbf{E}nvironment \textbf{R}esearch \textbf{C}ouncil\\
\textbf{NLP} & \textbf{N}atural \textbf{L}anguage \textbf{P}rocess\\
\textbf{OBIS} & \textbf{O}cean \textbf{B}iodiversity \textbf{I}nformation \textbf{S}ystem\\
\textbf{OCR} & \textbf{O}ptical \textbf{C}haracter \textbf{R}ecognition\\
\textbf{OS} & \textbf{O}perating \textbf{S}ystem\\
\textbf{OTU} & \textbf{O}perational \textbf{T}axonomic \textbf{U}nit\\
\textbf{PEMA} & \textbf{P}ipeline for \textbf{E}nvironmental DNA \textbf{M}etabarcoding \textbf{A}nalysis\\
\textbf{PDF} & \textbf{P}ortable \textbf{D}ocument \textbf{F}ormat\\
\textbf{PNG} & \textbf{P}ortable \textbf{N}etwork \textbf{G}raphics\\
\textbf{REST} & \textbf{RE}presentational \textbf{S}tate \textbf{T}ransfer\\
\textbf{SI} & \textbf{I}nternational \textbf{S}ystem of \textbf{U}nits\\
\textbf{URL} & \textbf{U}niform \textbf{R}esource \textbf{L}ocator\\
\textbf{WoSIS} & \textbf{Wo}rld \textbf{S}oil \textbf{I}nformation \textbf{S}ervice\\
\textbf{WoRMS} & \textbf{Wo}rld \textbf{R}egister of \textbf{M}arine \textbf{S}pecies\\
\end{abbreviations}

%----------------------------------------------------------------------------------------
%	PHYSICAL CONSTANTS/OTHER DEFINITIONS
%----------------------------------------------------------------------------------------

%\begin{constants}{lr@{${}={}$}l} % The list of physical constants is a three column table
%
%% The \SI{}{} command is provided by the siunitx package, see its documentation for instructions on how to use it
%
%Speed of Light & $c_{0}$ & \SI{2.99792458e8}{\meter\per\second} (exact)\\
%%Constant Name & $Symbol$ & $Constant Value$ with units\\
%
%\end{constants}
%
%%----------------------------------------------------------------------------------------
%	SYMBOLS
%----------------------------------------------------------------------------------------

%\begin{symbols}{lll} % Include a list of Symbols (a three column table)
%
%$a$ & distance & \si{\meter} \\
%$P$ & power & \si{\watt} (\si{\joule\per\second}) \\
%%Symbol & Name & Unit \\
%
%\addlinespace % Gap to separate the Roman symbols from the Greek
%
%$\omega$ & angular frequency & \si{\radian} \\
%
%\end{symbols}

%----------------------------------------------------------------------------------------
%	DEDICATION
%----------------------------------------------------------------------------------------

%\dedicatory{For/Dedicated to/To my\ldots} 

%----------------------------------------------------------------------------------------
%	THESIS CONTENT - CHAPTERS
%----------------------------------------------------------------------------------------

\mainmatter % Begin numeric (1,2,3...) page numbering

\pagestyle{thesis} % Return the page headers back to the "thesis" style

% Include the chapters of the thesis as separate files from the Chapters folder
% Uncomment the lines as you write the chapters

% --------------------------------------------------
% 
% This chapter is the introduction
% 
% --------------------------------------------------

\chapter{Introduction}
\label{cha:intro}

% Integration of the general approach of the PhD
\section{Ecosystem ecology}
\label{sec:intro-ecosystem}

Comprehension of ecosystem function is one of the pinnacles of ecology and
requires quantitative and conceptual advances \parencite{Chapin_Matson_Vitousek_2011}.
Ecosystem function refers to the processes that occur in an ecosystem, their
interconnectance, the agents that carry them out
and their relationship with the environment \parencite{Chapin_Matson_Vitousek_2011}. Biodiversity
\parencite{hooperEFFECTSBIODIVERSITYECOSYSTEM2005, loreau2001Biodiversity}
and community structure and dynamics \parencite{gonze2018Microbial,morris2020linking}
are known parameters of ecosystem function. This holistic approach must be taken
into account when tackling complex systems such as ecosystems.

Studying and
combining these parameters covers multiple fields of ecological science;
metabolic ecology, biogeochemistry, community and population ecology and
environmental ecology. Adding to this complexity, ecosystem behaviour cannot be
explained or predicted with just vast amounts of data because interconnectance
and interdependence lead to emergent phenomena in spatial and temporal scales.
Currently, systems ecology is experiencing a shift in understanding ecosystem function
because the accumulated knowledge from ecological theories is quantified and
supported with experimental data \parencite{mouquet_review_2015}.

Soil ecosystems are the cornerstone of terrestrial functioning.
Biodiversity interactions in soil are between all domains of life which form
multilevel associations. Bacteria \parencite{Delgado-Baquerizo-atlas}, archaea,
unicellular eukaryotes, nematodes \parencite{vandenHoogen2019},
springtails \parencite{potapov2023Globally}, earthworms \parencite{Phillips2021},
arthropods \parencite{milo-arthropods}, molluscs, plants, mammals; all occur in unison and 
influence the ecosystems they inhabit with their abundance, biomass \parencite{bar2018biomass} and metabolism.
The plant-insect-soil ecosystem is starting to be studied as a whole to discover
important associations with practical implications such as plant resistance 
to insect attack \parencite{plant-insect-soil2023}.
All of these life forms occur side by side and influence on another. This has been shown in the 
top of the mountains \parencite{winkler2018side}, in plant traits and
soil microbiome interaction \parencite{beugnon2022Abiotic} and others. 

Microbes are known for their versatility, abundance and dominating influence on ecosystem
functions \parencite{falkowski2008microbial}. 
Viruses as well because they
change the metabolic processes of microbes and are highly important in
ecosystem functions \parencite{hurwitz2016Viral}.
Microbes in soils store the vast majority of soi organic carbon \parencite{Crowther2019}.
Plants on the other hand are responsible for carbon fixation \parencite{thompson2012Food}.
Nitrogen fixation is an example of a mutualist relationship of plants with 
microbes in the rhizosphere since plants cannot fix nitrogen from the atmosphere.
In addition, arthropods seem to be the missing link of carbon biogeochemical 
cycles \parencite{GRANDY201640} through processes like decomposition.

Organisms associations are very useful and can
provide important insights. Still, more rigorous and explanatory methodologies
regarding bioenergetics \parencite{kempes2012Growth}, population dynamics
\parencite{gonze2018Microbial}, ecosystem stability \parencite{berdugo2020Global} and
metabolic ecology \parencite{brown2004METABOLIC} should be incorporated as well to
advance our understanding of ecosystem function.

Many worldwide studies regarding soil ecosystems are being implemented, yet
there are many gaps to cover the complexity of functioning and biodiversity
\parencite{guerra2020Blind}. Knowledge is lacking for specific taxonomic groups and
ecosystems for example the Sahara desert and below ground fauna.
These gaps are crucial to be studied in order to
improve our understanding of soil ecosystems \parencite{cameron2018Global}. 
Soil provides ecosystem services that sustain human civilisation 
through agriculture, water quality, human health and climate regulation \parencite{lehmann2020concept, lal2021the-role}.
Yet human activities are the ones that decrease them \parencite{rillig2023Increasing}. 

The need for a holistic approach to soil has become apparent and so the term \textit{"soilscapes"}
was coined \parencite{LAGACHERIE2001105}.
The landscape approach that soilscapes predicates means that soil is 
influenced and influences by physical, chemical, biological and 
human factors all at once \parencite{vogel2022}.
In addition, the deciphering the complexity and heterogeneity of soil requires 
model systems like estuaries \parencite{tsiknia2014} to be studied in long 
term time series.
Islands can be important case study models for this integration
through the islandscapes notion \parencite{Vogiatzakis_land_2017, Davies2016}

\section{Microbial ecology}
\label{sec:microbial_ecology}

Microbial ecology, a field that investigates the interplay of microorganisms with
their environment and with each other, has produced insights into
the complex dynamics of various ecosystems.
Traditional methods for understanding microbial diversity have relied primarily
on culturing techniques, but these approaches only reveal a fraction of the
existing microbial populations \parencite{prosser2020Conceptual}.
The rise of metagenomics and the discovery of
Environmental DNA (eDNA) have revolutionized microbial ecology by enabling the
analysis of microbial communities directly from their natural habitats,
providing a more comprehensive understanding of their diversity and
functioning \parencite{raes2008Molecular}. They allowed to 
study bacteria, archaea, eukaryotic microbes and viruses in different biomes \parencite{nayfach2021a-genomic}.

Large scale -omics experiments have introduced a vast quantity of data from the
environment regarding sequences, proteins and small molecules \parencite{shaffer2022Standardized}.
System biology has been established as a means to study community ecology interactions spanning
from the molecular level to the community level \parencite{raes2008Molecular}.
One area that has emerged from omics analyses is microbial ecology and
consequently the metabolic processes of ecosystems \parencite{perez_garcia2016Metabolic}
which unify all living scales, their diversity and complex dynamics \parencite{smith2016Origin}.
Several large scale sampling projects have been launched like the Earth
Microbiome Project \parencite{thompson2017a-communal}, Tara Oceans \parencite{sunagawa2020tara},
Global Ocean Sampling Expedition (GOS) \parencite{Shibu2007}, and
many smaller ones. These studies discovered the significant role of microbes as the major drivers of ecosystem metabolic
processes of all kinds \parencite{falkowski2008microbial,hall2018understanding} and
their percentage in total biomass is enormous, comparable to plants \parencite{bar2018biomass}.
These discoveries make microbes and microbiomes perfect candidates when
studying ecosystem function \parencite{klitgord2011Ecosystems,widder2016Challenges}.

Biomass and biodiversity distribution of microbial communities in soil and ocean
ecosystems are subjects of substantial significance within microbial ecology \parencite{bar2018biomass}.
Soil microbial biomass, representing a key indicator of soil health, varies
widely depending on factors such as soil type, moisture content, and organic
matter availability \parencite{Crowther2019}. Similarly, the ocean's
microbial biomass, which comprises a vast array of bacteria, archaea, and
microeukaryotes, is distributed heterogeneously, with variations influenced by
nutrient availability, water temperature, and depth \parencite{loucaDecouplingFunctionTaxonomy2016}.

In soil ecosystems, microbial biomass is typically concentrated in the upper
horizons, where organic matter and nutrient availability are the highest \parencite{Anthony2023}.
The rhizosphere, the soil region surrounding plant roots, exhibits particularly
high microbial biomass due to the release of root exudates, which serve as an
energy source for microbial communities \parencite{beugnon2022Abiotic}. The complex
interplay between soil microbes, plants, and environmental factors shapes the
spatial patterns of soil microbial biomass and diversity, with implications for
ecosystem functioning and stability \parencite{philippot2024the-interplay}.

In contrast, the ocean's microbial biomass is distributed across various
microhabitats, including the surface mixed layer, the deep ocean, and hydrothermal
vent systems. The surface ocean, receiving ample sunlight and nutrient inputs
from terrestrial sources, supports high bacterial and phytoplankton biomass,
which forms the base of the marine food web \parencite{Sunagawa2015}. Deep ocean
microbial communities, however, are adapted to low-nutrient and high-pressure
environments, relying on chemosynthesis and recycling of organic matter for their survival \parencite{christakis2018microbial}.

Microbes are the cornerstones of soil ecosystems, exerting a profound influence
to biogeochemical cycles and energy flow \parencite{graham2016Microbes}. The nutrient cycles they influence are
including carbon, nitrogen, sulfur, and phosphorus cycles.
The play pivotal roles in various processes such as decomposition,
and plant growth regulation.
Their metabolic processes are responsible for essential transformations, such as
nitrogen fixation, nitrification, denitrification, and mineralization, that
maintain the balance and flux of these elements in ecosystems \parencite{martiny2023Investigating}.
As primary decomposers, soil microbes facilitate the breakdown of complex
organic materials, liberating essential nutrients and thus ensuring the
sustenance of higher trophic levels \parencite{GRANDY201640}.

Microbial ecology brought together all scales of ecosystem functioning, from
molecules to biogeochemical cycles \parencite{hall2018understanding,kempes2012Growth,raes2011molecular}.
Using the paradigm of natural history and apply to microbiomes \parencite{Fierer2006} is placing microbes
in the spatial scale and enables adding the microbiome dimension in ecology and ecological interactions \parencite{Martiny2006}.
Investigating these
interactions will shed light on the complex relationship between microbial
ecology, soil health, and ecosystem functioning, ultimately aiding in the
development of sustainable land management practices and informing strategies to mitigate climate change impacts \parencite{cavicchioli2019scientists}.


\section{Data integration}
\label{sec:dataintegration}

% add a general paragraph about approaching integration

Data, statistics and mathematics are the indispensable tools of ecology.
Data in particular, didn't receive the attention of the other two fields in the early 20th 
century. Still, John Tuckey in his article \textit{"The future of data analysis"} 
argued that data analysis is a field on it's own \parencite{tukey1962}.
Nowadays, this field is data science and it is indispensable in most scientific endeavors \parencite{peng2022perspective}. 
Handling data means cleaning, dealing with multiple formats, linking different types 
of data, retrieving identifiers, quality filtering, visualising and performing statistics.
Each of these steps, especially in large scale projects, requires programming and 
in some cases multiple programming languages. 

Currently, there is an enormous wealth of data being generated and 
accumulated in databases \parencite{thessen2021from}. 
The sheer size and multifaceted nature of these data 
enable novel insights and pose new challenges \parencite{michener_ecological_2015}.
The integration of biodiversity knowledge in one place is a longstanding
goal in ecological research \parencite{Walter_2012}. The synthesis of multiple
data types and datasets across the globe has enabled 
holistic approaches to crucial scientific and societal questions \parencite{heberling_j_mason_data_2021}.


% add a fair data and standards

The explosion of metagenomic and metabarcoding data was followed by the creation
of many repositories and databases, some project specific and some that
collected many different sources. The metadata richness of metagenomic data
varies depending on the sampling and it may include location data, habitat type
and environmental parameters like pH, humidity and nutrients. Metadata
deficiency of data is a limiting step for the ecosystem function analysis
which, in the short-term can be resolved with data integration solutions from
alternative sources (for example satellite data for the sampling region). In
the long-run, initiatives like data FAIRification will improve their
interoperability \parencite{wilkinson2016the-fair}.

Yet there remain some conceptual challenges in microbial community ecology \parencite{prosser2020Conceptual}.
There multiple roads taken in order to decipher the microbiomes from a macroecological \parencite{Mascarenhas2020}, 
eco-evolutionary \parencite{martiny2023Investigating, loreau2023Opportunities}
and synthetic biology \parencite{Leggieri2021} approaches.
Nevertheless, microbiome data can be integrated with microbe
specific data of metabolism and pathways using ontologies like KEGG, Gene
Ontology and REACTOME and with habitat description from the environment
ontology (EnvO) \parencite{buttigieg2016environment}. Bringing together all the above requires web technologies and
data enrichment methodologies which are only recently starting to be
implemented in microbial ecology \parencite{jiang2016Microbiome}. This integration
will create maps of microbes, fauna, flora, genetic diversity, metabolism and environments. 

%text
However, this mapping derived from well structured data forms like databases
and ontologies will cover only a small part of the current knowledge. Most
scientific information is scattered in journal publications which consist of
text that is unstructured. Hence, Natural Language Processing (NLP) is needed to synthesize
information from publications \parencite{jensen2006Literature,}.
NLP techniques are used in multiple steps to structure information in a machine readable 
format \parencite{10.5555/1199003}. This process is called text mining.
Microbiome and
human diseases research as stated in \parencite{badal2019Challenges} have been
greatly advanced by text mining. This is still in its infancy in microbial
ecology and ecosystem ecology. Bringing together text mining associations and maps derived from
databases will provide a huge knowledge base consisting of microbes, their
metabolism and their habitat. The creation of the knowledge base is the first
step towards relation deciphering between these data types.

Current biodiversity knowledge is in multiple digital forms. 
Taxa names, traits and metadata are mainly in matrix data
structures. Taxa spatial distributions are in spatial data
forms like polygons or points. In additions taxa information
is available in sequences in fasta format. Abiotic data like 
temperature, pH, nutrients are also stored in matrix
and spatial formats. Audiovisual files contain information
of species and ecosystems in images, audio and video formats.
In addition, mathematical models contain mechanistic information
about many different aspects of ecology and algorithms, in the 
form of code, as well. All the above are communicated and disseminated through the
literature in the form of text.

% spatial
In addition to biodiversity data, the spatial dimension is shaping the biodiversity and with the remote sensing revolution now
it's possible to use these data to monitor resilience \parencite{Lenton2022resilience}. 
The concept of Digital Earth, first coined in Al Gore’s book entitled 
“Earth in the Balance” (Gore 1992), was further developed in a speech
written for Gore at the opening of the California Science Center in 1998. Other words with
similar meaning are Digital Twin, Island Digital Ecosystem Avatars (IDEA), Earth system
and Global Earth Observation System of Systems (GEOSS).

\section{Integrative approaches}
\label{sec:approachintegration}


% databases
Currently there are many global databases with distinct and overlapping 
scope like GBIF, OBIS, iDigBio, EOL etc. \parencite{feng2022Review}. 
The contents of these databases covers taxonomy, spatial extend, traits, 
connections to specimens, species literature, distributions,
IUCN status. These portals in order to remain relevant and updated require 
continuous funding and development. Thus, Research Infrastructures (RI) are 
the most suitable organisational models to sustain these portals. Examples 
of such RIs are :

\begin{itemize}

    \item eLTER: European long-term ecosystem research infrastructure 
    \item LifeWatch ERIC
    \item European Marine Biological Resource Centre

\end{itemize}

Yet, a synthesised knowledge base of biodiversity, in terms of ecological and
remote-sensing data remains a major challenge \parencite{feng2022Review}. The challenge
is mainly on the different data standards used but also in the implementation of 
data exchange models and synchronisation.
A user friendly interface web portal can provide all this wealth of data
regarding the multifaceted ecosystem variables. There is a need for policy regarding earth system for the global
sustainability \parencite{reid2010earth}. Communities like GEO BON (Biodiversity Observation Network) and Soil BON
have pushed towards these goals in the past decade.

%integrative methods
Apart from data and their hosting, ecosystem holistic understanding requires
integrative methods. Macrosystems approach uses interactions of local parameters
such as biological, geophysical and sociocultural and explores their influence in
the macro scale \parencite{heffernan2014}. Combining multiple omics data and community ecology and causality has
been an important method to integrate \parencite{jurburg2022community}. Useful are also the network inference methods 
and their analysis for biogeochemical cycling \parencite{jameson2023Network}.
Network analysis based on the knowledge base is assisting in the quantification
of the relations between microbes, their metabolism and ecosystem
function \parencite{graham2016Microbes,muller2018Using, perez_garcia2016Metabolic}.
Microbial interactions can be inferred from the aforementioned data as shown
in \parencite{machado2021Polarization} based on their habitat and metabolism.
These community dynamics can be used to build ecosystem level metabolic
networks \parencite{perez_garcia2016Metabolic} that influence nutrient cycling in
ecosystems \parencite{bauer2018Network}. Another approach is by using species as a
container of pathways and then examining the functions accumulated in the
ecosystem \parencite{loucaDecouplingFunctionTaxonomy2016}. Both approaches have
been applied to specific ecosystems with applicable results but relying on a
global knowledge base will expand ecosystem function analysis into many
different ecosystems that have been sampled.
Going further, multilayer networks \parencite{marine-multilayers}
are the way to include multiple omics data in specific environmental communities. 

Metabolism is a scale independent biological process spanning from chemical reactions
to biogeochemical cycles \parencite{hall2018understanding}. Bioenergetic models on food 
webs can provide information about energy exchanges in the food webs \parencite{valdovinos2023bioenergetic}.
Last but not least, information theory is a conceptual framework to bring together agents,
their interactions and the flow of information through these interactions \parencite{oconnor-information-ecology}.

% add about Planetary biology and islands

% add the transculturalism 

\section{Towards ecological research integrity}
\label{sec:integrity}

Bringing all the aforementioned data and methods together will allow for deciphering ecosystem functioning.

How to assist society?

How to move forward the field and assist fellow researchers 

How to make all results transparent and reproducible 

Data rescue 
\parencite{michener_nongeospatial_1997}

This will be an integral part of the sciences of climate change
and conservation \parencite{cavicchioli2019scientists}. These fields undertake the
sustainability of ecosystems such as coral reefs and tropical forests, when
faced with habitat loss, biodiversity loss, pollution and generally any change
in the natural environment conditions. Monitoring these ecosystems is crucial
and environmental data analysed using network theory will facilitate real time
inspection and inspire immediate action \parencite{derocles2018Biomonitoring}.
Ecosystem services, on the other hand, aim at utilising ecosystem resources
and functions for all sorts of human activities ranging from goods supplied
from agriculture and farming \parencite{alvarez-silva2017Compartmentalized} to
recreational and educational purposes. In addition, the one health concept has
shown that microbiomes are the for human health and prosperity
\parencite{banerjee2023Soil, lehmann2020concept}. Yet, the pressure poised from human activities
is reducing the ecosystem services \parencite{rillig2023Increasing}.

% arthropods devastating decline

Last but not least, terraformation
projects explore ecosystem function with a bottom-up approach for engineering
systems to become habitable for humans, like planet Mars, or transforming
Earth's collapsed ecosystems to new habitable states
\parencite{conde-pueyo2020Synthetic}. Using the data and methods described in this
article will enhance our understanding in each of these areas. Knowledge
integration from data, metadata and text sources combined with ecological and
metabolic theory will expand our knowledge and eventually lead to new
knowledge regarding the aforementioned fields. A soil monitoring framework 
is overdue to track and inform the society about the health of 
soil ecosystems \parencite{guerra2021tracking}.

\section{Aim of this research}
\label{sec:aim}

The research presented here touches on multiple scientific disciplines and
combines contrasts as shown in Figure \ref{fig:phd-one-slide}. 

% data science in general
% data integration challenges 
% homogenisation
% data availability and open data
% bioinformatics and open source
% reproducibility 
% focus on a confined place
% bringing together contrasts
% publish new data 
% conservation actions
% set the basis for future research

biodiversity, microbes and integration. 
First, regarding biodiversity as species occurrences, two projects were implemented.
Using historical biodiversity literature, the methods of data rescue were 
investigated using text mining tools.
In addition, a novel assessment of the endemic arthropods of the island of Crete
was carried combining both historical data along with contemporary data from the field;
an approach that beyond its findings demonstrated also the value of legacy
literature and the rescue thereof.
Second, regarding the microbial biodiversity, the island sampling day project 
topsoil cores of the island of Crete were analysed. This adds a new chapter to 
the soil biodiversity of the island with interesting implications to one health.
And third, the integration of the existing knowledge regarding the metagenomic 
data and literature as well as the available spatial datasets about Crete. 
An additional aim of this work is to demonstrate that the wealth of available open data
and open source tools can inspire novel projects and integrative approaches that lead
to new knowledge.

   \begin{figure}[ht]
      \centering
      \includegraphics[width=\textwidth,height=\textheight,keepaspectratio]{figures/2024_phd_one_slide_en.png}
      \caption[Graphical abstract of this PhD]{The different sections of this PhD. There are 3 disciplines, Biodiversity, Microbiology and Data Integration. Each one contains different data sources, field applications and bioinformatic methodologies.}
      \label{fig:phd-one-slide}
   \end{figure}



% --------------------------------------------------
% 
% This chapter is for Island Samplind Day, Crete
% 
% --------------------------------------------------


\chapter{Island Sampling Day, the case of Crete}
\label{cha:isd-crete-soil}

%\textbf{Citation:} \\ 

\section{Introduction}\label{intro_isd}

Microbes influence global ecosystem functions \parencite{falkowski2008microbial}
and are ubiquitous \parencite{delgado2016microbial}. The terrestrial biomass of bacteria is
70 Gt C, second to plants with sixfold difference \parencite{bar2018biomass}. In subsurface systems, bacteria and
archaea are the most abundant, having 90\% of the biomass. These findings of microbial
ecology are based on amplicon 16s rRNA studies that have flourished since the
2010s. The endeavor to understand the microbial world faces multiple challenges
across the scientific workflow, from sampling to ecological analyses \parencite{Lee2012}.

Amplicon rRNA sequencing provides a collection of sequence reads per sample. 
The ecological interpretation of the reads requires their transformation to
taxonomic information. To do so there are two approaches currently in use, 
the clustering method and the denoising method. With clustering reads are 
grouped together and a best representing sequence is produced for each 
cluster, i.e. the Operetional Taxonomic Unit. This approach makes the OTUs 
from different runs, i.e executions of the algorithm and/or different studies
incoperable and irreproducible. Currently, many studies propose the use of Amplicon Sequence Variants \parencite{Callahan2017}. 
ASVs are real biological sequences and can be used for comparison.
The influence of the different methods to subsequent ecological analyses has
little impact \parencite{Glassman2018}.

Errors propagate starting with the sampling. There are contaminations from the
people in the field, in the lab for the DNA extraction \parencite{EISENHOFER2019105}. 
Errors from PCR amplification and errors from sequencing \parencite{Schloss2011, Schimer2015}.
Approximation and computation errors from algoritms that cluster, measure similarities between
sequences. Semantic errors occur because of reductionist approaches and/or oversimplification
of the microbial communities. Furthermore, microbiome matrix data are compositional, 
meaning that the abundances of the reads are not the real representations of the natural populations \parencite{Gloor2017}.
These limitations are important to be acknowledged in order to avoid overarching conclusions,
especially in complex systems such as soil.

Bacteria and archaea are considered major drivers for the functionality of soil.
In soil ecosystems, the pillars of human activities and ecosystem services, 
there is a positive correlation between soil microbial diversity and their multifunctionality \parencite{Delgado-Baquerizo2020}.
The community structure defies their macroscopic functionality \parencite{Bahram2018}.
Soils are complex environments that are characterised chemical, physical and biological properties. 
These form interdependent associations with feedback loops by their environment. 
Global soil microbiome studies have been employed to decipher soil microbiome
compositions \parencite{thompson2017a-communal, Delgado-Baquerizo-atlas, Labouyrie2023},
functions \parencite{Bahram2018} and biogeography \parencite{Martiny2006, guerra2020Blind}.
The profound diversity of soils is clearly stated in these works, yet there are
significant blind spots \parencite{guerra2020Blind}.
When considering the samples per area density, there global studies are rather sparse.
National wide studies like the one by \parencite{Karimi2020} show the
complexity of soil bacteria and the need to design 
denser samplings and use isolated systems \parencite{Dini-Andreote2021}.

Isolated systems are important to avoid co-founding effects and to reduce complexity.
Islands are considered nature's labs and are used in biogeography extensively due to their smaller scale and isolation \parencite{Whittaker2017}. 
Island microbiomes from soil \parencite{Li2020} and mycorrhizal fungi \parencite{Delavaux2021} studies
show the benefits of using islands as models.
This was the case for the Island Sampling Day (ISD) Crete project
of the Genome Standards Consortium \parencite{Field2011}
during the 18th workshop in June 2016 in Crete island, Greece. The goal of this sampling was to "put standards into action"
in a soil microbiome survey with a dense sampling, 0.017 samples per km\textsuperscript{2}.
Hence, ISD Crete is a large scale study in the confined space of the island of Crete which 
is considered a miniature continent \parencite{Vogiatzakis2008_crete}.

The Island Sampling Day draws inspiration from the Ocean Sampling Day \parencite{kopf_2015} that is a
consortium that facilitates yearly samplings in stations from all around the world.
This has great applications in monitoring and comprehending the complexity of the ocean microbiome and it’s drivers.
The Island Sampling Day has been organised in two islands, Mo'orea in the French Polynisia and Crete in Greece.
In the ISD Crete project, 26 people divided in ten teams went across the island in a single day to sample as much diverse 
ecosystems of Crete. 
Samples were collected in a single day in order to limit the variations of environmental
factors that may influence the abundance and diversity of microbes such as season,
temperature and humidity.
For each sampling site, teams selected two specific sub-sites that were
at least 3 meters from the edge of the road and 0.6 m from the base of
the identified tree or plant. Flora located at the sampling site were
identified and photographed.
At each sub site, soil was sampled (3 replicates, collected one 3cm apart). 

The University of Maryland Soil Lab (Dr. Stephanie Yarwood) received 288 soil cores
144 for DNA extraction and 144 for physicochemical measurments. The shipment was on dry ice,
under the USDA permit and a Material Transfer Agreement with HCMR.
The water content, Total Organic Carbon and Nitrogen weights were measured using LECO CN628 analysis platform.
DNA extraction of 144 soil samples used the protocols of the \href{https://www.protocols.io/view/emp-16s-illumina-amplicon-protocol-cpisvkee}{Earth Microbiome Project} (EMP)
with the the MoBio RNA extraction kit.Three of the samples, collected in sand, had no detectable DNA
following extractions.
The PCR-based protocol used is designed for the V3-V4 region of the 16S rRNA gene (V3 V4 primers: Forward: 5'-ACTCCTACGGGAGGCAGCAG-3'; Reverse: 5'-GGACTACHVGGGTWTCTAAT-3').
The sequencing of 16s rRNA amplicons were carried out on an Illumina HiSeq2500 platform.
at the Institute for Genome Sciences, Genomics Resource Center (Maryland Genomics) at the University of Maryland School of Medicine.

The Genome Standards Consortium uploaded the metadata and the sequences to ENA
database under the project \href{https://www.ebi.ac.uk/ena/browser/view/PRJEB21776}{PRJEB21776} immidiately upon realise. 
This is an act of unrestricted use of genomic data which exemplifies the importance of 
having public data even prior publications \parencite{kopf_2015}.

In this study, the ISD data and metadata were downloaded from ENA database analysed to decipher drivers of 
biodiversity and community structure of the top soil microbiome of Crete. Diversity, 
ordination and correlations of metadata were carried out. Following the examples of 
transparency of the GSC, all code is reproducible and documented as mentioned in FAIR principles \parencite{wilkinson2016the-fair}.
This is the first analysis 
of this scale of the Cretan soil microbiome, henceforth adding a new chapter of 
biodiversity in Crete, the microbiome.

\section{Materials and Methods}\label{isd_methods}

The work purely computational; scripts of the analysis cover the following tasks:
\begin{itemize}
    \item Search ENA for samples in the island of Crete
    \item Get ISD metadata and sequences
    \item HPC jobs and parameter files
    \item Filtering, clustering/denoising and taxonomic assigninments
    \item Biodiversity and ecological analysis 
    \item Visualisation
\end{itemize}

The bioinformatic workflow can be summarized to : GET, INFER and ANALYZE as shown in Figure \ref{fig:isd_workflow_taxonomy}.

\begin{figure}[h]
      \centering
      \includegraphics[width=\textwidth,height=\textheight,keepaspectratio]{figures/isd_crete_flowchart.png}
      \caption[Reproducible workflow of ISD analysis]{Three steps of the ISD bioinformatics workflow, reproducible by design}
      \label{fig:isd_workflow_taxonomy}
   \end{figure}
   
\subsection{Data retrieval}\label{isd_get}

This study uses the ISD Crete data that have been deposited
in the European Nucleotide Archive (ENA) at EMBL-EBI under accession number PRJEB21776.
The raw sequences (fastq files) and metadata (xml files) were downloaded with custom scripts using the ENA API \parencite{Yuan2023}.

\begin{figure}[h] 
    \centering\includegraphics[width=\columnwidth]{isd_map_fig1-small}
    \caption{Crete ISD sampling. A. The routes of the single day event sampling. B. Beta diversity differences are represented with color.}
    \label{fig:isd_crete_sampling}
\end{figure}

\subsection{Taxonomic inference}\label{tax_inference}
Amplicon Sequence Variants were inferred using DADA2 \parencite{Callahan2016} for 
filtering, denoising and chimeric reads removal. Normalisation of the reads
across samples was implemented with the SRS R package \parencite{Beule2020}. Samples
with less than 10000 reads were removed. ASVs were assigned to taxonomy using 
DADA2 with Silva 138 \parencite{quast_silva_2013}.
PEMA was used for OTU inference based on VSEARCH \parencite{zafeiropoulos2020pema}.

\subsection{Biodiversity analyses}\label{biodiversity}

Numerical ecology analyses,e.g diversity indices, NMDS, PERMANOVA we calculated
with the vegan R package \parencite{oksanen2024vegan}.
For PCoA ordination we used ape R packege \parencite{Paradis2004} and UMAP python library\parencite{mcinnes2018umap-software}.
U-CIE R package was used for coloring 3 dimensional data \parencite{Koutrouli2022} to 
visualise the $\beta$ diversity differences of samples on the map.
Visualisation was implemented with ggplot2 \parencite{wickham_ggplot2_2016} and pheatmap \parencite{Kolde2019}.

\subsection{Environment and Code}

Computations were performed on HPC infrastructure of HCMR \parencite{zafeiropoulos_0s_2021}.
The programming environment was Debian 4.19 and conda environments were utilised
with Python 3.11.4, R version 4.3.2 \parencite{rcoreteam}.
Additionally GNU bash 5.0.3 and GNU Awk 4.2.1 were used for the streamlined workflow 
and for the reads statistics, respectively.

The documentation and scripts developed for this study are available in
\href{https://github.com/GenomicsStandardsConsortium/ISD}{ISD Crete soil microbiome github repository}.
This repository contains all the necessary scripts for the data retrieval,
filtering and ASV inference, taxonomy assignment, data integration of spatial data, 
functional annotation and the subsequent analyses and visualisation.
Code is structured to be reproducible and interoperable as described in Figure \ref{fig:isd_workflow_taxonomy}.


\section{Results}\label{isd_results}

\subsection{Inference and taxonomy}\label{inference_taxonomy}
The Illumina HiSeq2500 generated 51 million reads (approximately 16-20 GB),
with an average of 355,326 reads/sample (ranging from 2,437 to 525,144).
This data and its associated metadata were submitted to the ENA under study
accession number PRJEB21776. The sequencing process yielded 51 million reads,
averaging 250-500 K reads/sample (equating to roughly 15-20 GB of data).
On average, each sample contained about 355,326 reads, though this varied from
2,437 to 525,144. Reads shorter than 445 bp or longer than 515 bp were fewer than
15,000, see Figure \ref{fig:isd_srs-curve_samples}.

From the 144 samples, two samples were 
neglected because they had fewer than 10000 reads. Also 3 samples didn't yield any DNA and 1 wasn't uploaded to ENA due 
to some errors. Thus, 138 samples in total were included in the subsequent analysis.
   
   \begin{figure}[hbt!]
      \centering
      \includegraphics[width=\textwidth,height=\textheight,keepaspectratio]{figures/isd_crete_srs_curve.jpeg}
      \caption[SRS curve]{SRS curve of the samples. }
      \label{fig:isd_srs-curve_samples}
   \end{figure}

\begin{figure}[hbt!] 
    \centering\includegraphics[width=\columnwidth]{figures/isd_community_site_locations_dif.png}
\caption{Sample dissimilarity between samples in the same site (2 subsites) and the rest.}
    \label{fig:isd_site_locations}
\end{figure}

In summary, DADA2 resulted in 216,360 ASVs (2,704 unique
taxa; 1123 ASVs at species level and 1059 at genus level) and
PEMA (VSEARCH) in 13,285 OTUs.
The representative Phyla ( > 5\% presence in all samples) are:
Actinobacteriota, Proteobacteria, Chloroflexi, Acidobacteriota,
Bacterodota and Planctomycetota, Figure \ref{fig:isd_top_phyla_samples}.
There are 25 specialist (samples < 10 and mean relative
abundance > 0.003) and 146 generalist taxa (samples > 120), Figure \ref{fig:isd_fig2_taxonomy}.

\begin{figure}[hbt!] 
    \centering\includegraphics[width=\columnwidth]{isd_fig2_taxonomy}
    \caption{Taxonomic prevalence and representative phyla of the soil bacteria of Crete. 
    A. Distribution of ASVs samples proportion. B. Distribution of taxa samples
proportion and categorisation in generalists and specialists. C. Phyla samples proportion.
D. Phyla relative abundance box plots, each dot represents one sample that the phylum occurs.}
    \label{fig:isd_fig2_taxonomy}
\end{figure}

\begin{table}[]
    \caption{Taxonomic depth of ASVs and the unique number of taxa of each level.}%
\begin{tabular}{@{}lllll@{}}
classification depth & Total ASV    & Total (ASV) taxa & Total OTUs & Total (OTU) taxa\\
Kingdom              & 1974         & 2                & 284        & 2               \\
Phylum               & 4034         & 33               & 121        & 15              \\
Order                & 38517        & 193              & 1224       & 135             \\
Class                & 24157        & 83               & 978        & 62              \\
Family               & 71355        & 287              & 2319       & 218             \\
Genus                & 90137        & 1166             & 1920       & 582             \\
Species              & 9120         & 1338             & 44         & 43              \\
Total                & $\sim$239000 & 3102             & 6890       & 1057            
\end{tabular}
\label{table:asv_taxonomy}
\end{table}

\begin{figure}[hbt!]
      \centering
      \includegraphics[width=\textwidth,height=\textheight,keepaspectratio]{figures/isd_asv_taxonomy_ratios_top_phyla_samples.png}
      \caption[Top phyla of each samples]{The top phyla for all samples with their relative abundance}
      \label{fig:isd_top_phyla_samples}
\end{figure}
   

\subsection{Drivers and Communities}\label{isd_communities}
Highest values of sample metadata:
ERR3697708, ERR3697732 (isd 4 site 8 loc 1, isd 6 site 2 loc 1) have the highest total nitrogen values (12.3, 7.9).
ERR3697703, ERR3697702 (isd 4 site 5 loc 2, isd 4 site 5 loc 1) have the highest water content values (141, 102).
ERR3697655, ERR3697675 (isd 1 site 2 loc 2, isd 2 site 5 loc 2) have the total organic carbon values (238, 177).


Regarding the OTUs of the Island Sampling Day:
The highest shannon diversity have the samples ERR3697693 (isd 3 site 4 loc 2) and
ERR3697675 (isd 2 site 5 loc 2) with 4.57 and 4.56, respectively.

ERR3697765 isd 7 site 11 loc 2 have the highest number of OTUs, 1075.
ERR3697765 isd 7 site 11 loc 2 has the second highest number of OTUs, 1051.

ERR3697703 isd 4 site 5 loc 2 has the most taxa summing at 871.
The ERR3697702 (isd 4 site 5 loc 1) has 869 taxa.

The ASV richness of the samples is correlated (0.25, Pearson correlation) with total organic carbon (p=0.003).
Taxa richness is negatively correlated (-0.30, Pearson correlation) with elevation (p=0.0003). 
Taxa richness is positevely correlated (0.26, Pearson correlation) with water content (p=0.002).

$\alpha$-diversity (Shannon Index) is not correlated significantly with any
by physical and chemical features.


\begin{figure}[hbt!]
      \centering
      \includegraphics[width=\textwidth,height=\textheight,keepaspectratio]{figures/isd_abiotic_metadata_elevation_bin_boxplot.png}
      \caption[Elevation and metadata distributions]{The distributions of the available metadata across the elevation of samples}
      \label{fig:isd_elevation_metadata}
\end{figure}

\begin{figure}[hbt!]
      \centering
      \includegraphics[width=\textwidth,height=\textheight,keepaspectratio]{figures/isd_diversity_elevation_bin_boxplot.png}
      \caption[Elevation and diversity indices]{The distributions of asv, taxa and multiple $\alpha$ diversity indices across the elevation of samples}
      \label{fig:isd_elevation_taxa}
\end{figure}

Total nitrogen and water content increases with elevation \ref{fig:isd_elevation_metadata}. 

Microbial $\beta$ diversity is associated in different ways with physical and chemical
measurements. UMAP 1 ordination was largely driven by
elevation (F=23, p < 0.001). 
UMAP 2,
was significantly associated with soil moisture (F=90, p < 0.001).
PCoA 2 also positively correlated with both organic carbon and nitrogen.

\begin{figure}[hbt!]
      \centering
      \includegraphics[width=0.7\textwidth,keepaspectratio]{figures/isd_asv_ordination_UMAP1_elevation_bin_boxplot.png}
      \caption[Elevation and UMAP1]{UMAP major axis and elevation bins}
      \label{fig:isd_elevation_umap1}
\end{figure}

In addition, the dissimilarity of samples was clusted using the complete linkage method, Figure \ref{fig:isd_samples_dendro}. 
A chi square test was performed to investigate whether the cluster membership of the samples
is statisticaly associated with metadata. Elevation bin, p=0.07646, didn't show any 
significance. Vegetation zone (p=0.0004998) on the other hand was significant.

\begin{figure}[hbt!]
      \centering
      \includegraphics[width=\textwidth,keepaspectratio]{figures/isd_asv_clustering_bray_hclust_samples.png}
      \caption[Samples dendrogram]{Dendrogram of clustering of samples based on the Bray-Curtis dissimilarity}
      \label{fig:isd_samples_dendro}
\end{figure}


\section{Discussion}\label{isd_discussion}

Soil bacterial biodiversity is very complex, even sites a few meters apart can differ
significantly in their community composition, Figure \ref{fig:isd_site_locations}.
This is a fact that is sometimes neglected in worldwide studies and the island biogeography
paradigm can assist to remove clutter.

Apart from the profound diversity in soils, there is also high speciation and uniqueness. 
As shown in Figure \ref{fig:isd_fig2_taxonomy}, most ASVs occur in 1 or 2 or 3 samples.
This is different when compering with the ocean. When using the deepest taxonomic
level of ASVs is possible to identify the specialists and generalists \parencite{Barberan2012}. 
In addition, focusing on the phyla, we see a pattern looking like a phase transition, from 
rare phyla to phyla that dominate all samples, Figure \ref{fig:isd_fig2_taxonomy} C. Lastly, in Figure \ref{fig:isd_fig2_taxonomy} D,
and \ref{fig:isd_top_phyla_samples}, we found some distinct top phyla profiles of samples.

Elevation gradients of biodiversity are known since Humboldt's work \parencite{Rahbek2019} 
yet these patterns remain elusive regarding the soil microbiome \parencite{Looby2020, Siles2023}.
Mostly because it's difficult to isolate other co-founding effects \parencite{Nottingham2018}.
In our results elevation showed important distinction of taxa and of diversity. Yet more work is 
needed to explore the Asterousia mountain transect in isolation to avoid indirect influences of 
other variables.

Deciphering and validating the results presented here requires future work.
There are missing links of the drivers and community compositions. These 
gaps can be filled by sampling more data and/or by data integration methods.
The former create new knowledge and are important to continue but they 
require a lot of resources. The latter uses already available knowledge 
to enrich the microbial information (e.g traits) and samples metadata
(e.g. climatic, land use, other taxa occurring in the same area).

"A holistic perspective on soil architecture is needed as a key to soil functions" \parencite{philippot2024the-interplay}, is 
an important statement for future soil projects.
The pillar of data integration is the unrestricted open data across disciplines and 
the open source software. 



% --------------------------------------------------
% 
% This chapter is for PREGO
% 
% --------------------------------------------------


\chapter{Harmonising the literature and metagenomic resources to infer microbial, environmental and functional relations}
\label{cha:prego}


\section{Introduction}
\label{sec:prego-intro}

The rapid advancements in genomic technology have significantly impacted the
all fields of biology, creating a need for the establishment
of genomic standards to ensure the reliability and reproducibility of research
findings \parencite{Field2011}. As genomic datasets grow exponentially, the importance of these standards
in genomic research cannot be overstated. Standards are essential for enabling
the comparison of results across different studies, facilitating data sharing and
collaboration, and ultimately driving the progression of genomic medicine \parencite{vangay2021microbiome}.
However, the potential of genomic data is often limited by the lack of accompanying
metadata.

Data devoid of metadata are essentially a black box, with no contextual
information to interpret the results or to assess their relevance \parencite{michener_nongeospatial_1997}.
This issue is
prevalent in genomics, where the complexity and diversity of the data
necessitate comprehensive metadata for accurate analysis \parencite{vangay2021microbiome}.
Without proper metadata, it becomes challenging to assess data quality,
reproducibility, and to ensure that research conclusions are robust.
The emergence of metagenomic resources has added a new dimension to genomic
research, with a myriad of databases providing a wealth of data on microbial
communities from various environments.
Examples of public resources are MGnify \parencite{mitchell2020mgnify},
JGI/IMG \parencite{chen2021img} and MG-RAST \parencite{wilke2015restful}). Each one 
has it's one pipeline to analyse user submitted environmental sequences. MGnify
directly imports data from ENA sequence database.
These resources are invaluable for
understanding the genetic diversity and functional potential of microbial
ecosystems. Adding to the complexity, the sheer volume of data and the varying formats in which
they are presented can make comparing and combining these resources a daunting task. 

Metagenomic mining approaches are required to leverage vast databases and analyze
microorganisms from diverse environments \parencite{delmont2011metagenomic}.
Mining allows scientists to extract and interpret genetic material 
from multiple resources thus providing a comprehensive view of the microbiome.
Metadata associated with each sample in these databases, such as source information and collection conditions,
contribute significantly to the contextual understanding and statistical interpretation of the data.
Enable researchers to make meaningful comparisons across different studies.
Furthermore, the application of metagenomic mining to these rich databases can help
uncover novel genes, metabolic pathways, and potential therapeutic targets \parencite{ma2023a-genomic},
thus driving advancements in health, agriculture, and environmental sustainability.
To accomplish this, named entity recognition techniques have become essential for
efficiently harmonising relevant information from metadata.

Named entity recognition is a term identification process. It is based upon
vocabularies and ontologies that provided multiple versions of terms and 
hierarchy relationships. Environment Ontology (ENVO) \parencite{buttigieg2016environment} 
is an important ontology that has classified environments and metagenomic 
resources suggest it's usage in the metadata in fields like biome and 
environmental feature. Molecular functions and biological processes are two separate 
ontologies, part of Gene Ontology \parencite{ashburner2000gene, gene2021gene} that
are widely used in bioinformatics and are open source. 
Regarding functions and pathways, the KEGG database \parencite{kanehisa2000kegg} is the most used resource
in microbial ecology, yet it requires a licence to implement it in a new resource.
Taxa names are also extracted from free text based on different taxonomies. In microbiology,
NCBI taxonomy \parencite{schoch2020ncbi} is convenient to use and implement because every taxon 
is represented by sequences. Nevertheless, LPSN (List of Prokaryotic names with Standing in Nomenclature) \parencite{parte2020list}
is a curated nomenclature that has high accuracy. Based on some of the aforementioned 
systems it is possible to extract entity names from metadata fields and free text in general 
in order to homogenise metadata and data fields.

Apart from metagenomic data, scientific literature contains invaluable information
about microbes, their functions and habitats. This information is usually hidden 
or captured by experts on the fields. 
Text mining methods have been long assisted in extracting knowledge from 
the scientific literature \parencite{jensen2006Literature}. 
By leveraging text mining, researchers can more readily identify and use the rich information that
is often hidden within the text, thereby accelerating discoveries in this
rapidly evolving field \parencite{badal2019Challenges}. Text mining is a multiple step
process \parencite{10.5555/1199003}. It begins with establishing a corpus with 
documents specifically formatted for the tools used. PubMed \parencite{roberts2001pubmed}
holds more that 40 million abstracts through MEDLINE. These data are accessible 
and open access, therefore useful for text mining applications. A downside is the 
domain of publications because it doesn't cover most of ecological and environmental 
journals. Then these documents are analysed with Named Entity Recognition and a 
score of co-occurrences between entities is produced. This lies in the principle 
that the highest the co-mention of terms the higher the probability there is 
a biological meaning of this association \parencite{jensen2006Literature}.
Text mining approaches have been very successful in unearthing novel 
associations of proteins and deceases \parencite{pletscher2015diseases}.

PREGO, the resource developed and presented in \textcite{microorganisms10020293}
as part of this PhD (see \ref{fig:prego_paper}), is a hypothesis generation platform designed to integrate the 
available knowledge about microbes and their processes and environments. 
It is a global platform, first of it's kind that combines metagenomic data with literature
using an established text mining methodology. 
In this chapter, my contributions in the development of PREGO are presented along with the analysis of global 
soil microbiome knowledge.

\section{Methods}
\label{sec:prego-methods}

   \subsection{Pipeline and structure}
   \label{subsec:prego-pipeline}

PREGO's pipeline can be summarised in 6 steps as presented in Figure \ref{fig:prego-pipeline}.
In steps 1 and 2, the web resources are harvested from 3 different types of data,
literature, environmental samples and genome annotations. 
In step 3, the Named Entity Recognition system of the Jensen Lab Tagger is employed \parencite{jensen2016one}
to identify taxa, environments and processes. In addition, in genome annotations
the processes entities are mapped to Gene Ontology terms with custom scripts. 
During the step 4, the co-occurrences of terms is calculated and scored,
each type of resource has it's own scoring scheme. 
This creates a global association network with taxa, processes and environments as 
nodes, co-occurrences as edges and score values as edge weights. 
Lastly, this knowledge graph is uploaded and provided in a web interface 
and through Application Programming Interface.

   \begin{figure}[hbt!]
      \centering
      \includegraphics[width=0.85\columnwidth]{figures/prego_analysis.png}
      \caption[PREGO analysis methodology]{
         PREGO methodology: retrieval of 3 types of open access data resources, literature, environmental DNA samples, and genomic annotations. 
         Named Entity Recognition and co-occurrences bring together associations from organisms, environments and processes. 
         These associations has a score and are provided is a Web interface and an API. Figure modified version of \parencite{microorganisms10020293}.
      }
      \label{fig:prego-pipeline}
   \end{figure}

The knowledge graph in Figure \ref{fig:prego-pipeline} step 5, contains
nodes categorised in three entity types: \textit{Process}, \textit{Environment}, and \textit{Organism}. 
Organisms, i.e taxa, are the microbial NCBI Taxonomy Ids (Bacteria, Archaea, and unicellular eukaryotes).
All environments are mapped to terms of Environmental Ontology. 
Each process corresponds to either a Biological process (GObp) or a Molecular function (GOmf) identifier of Gene Ontology. 

PREGO's knowledge base is structured into three channels, depended on the category of the data type.
\textit{Literature} channel is about the associations extracted by abstracts and full open access text of scientific articles.
The \textit{Annotated Genomes and Isolates} channel contains genome annotations and their harmonised metadata.
The \textit{Environmental Samples} channel integrates metagenomic analyses from amplicon and shotgun sequencing studies. 
Hence, in this channel, the taxonomic and functional profiles and their metadata are harmonised and the associations are extracted.

\subsection{Dictionary of entities}
\label{subsec:prego-dict}

PREGO's entities are 4 types and each term corresponds to specific identifier. 
NCBI taxonomy identifiers for taxa, Environmental Ontology for environments and
Gene Ontology for Biological Processes (GObp) and Molecular Functions (GOmfs).
Gene Ontology was selected because it has a Creative Commons Attribution 4.0 License
and because it has been mapped to many other resource identifiers.
These terms, their names and synonyms comprise the PREGO dictionary. In addition,
some terms have been removed because of their uniqueness and/or ubiquitousness which reduce the precision of text mining systems.
These terms comprise the blocklist, a separate file in the dictionary.
The ORGANISMS \parencite{pafilis2013species} and ENVIRONMENTS \parencite{pafilis2015environments} are evaluated dictionaries
whereas Gene Ontology Biological Process and Molecular Function are in experimental stage.
The dictionary is 
available for download in the Jensen Lab text mining \href{https://jensenlab.org/resources/textmining/#dictionaries}{resources}. 
NCBI contains all taxonomy, therefore it was filtered for bacteria, archaea and
for the unicellular eukaryotes. Due to unicellular complex taxonomy a manually curated
list was populated with these taxa.

Some resources use KEGG orthology terms or Uniprot50 ids to indicate functions. 
Hence, a mapping KEGG orthology to GOmf and Uniprot50 to GOmf was carried out via UniProtKB mapping files (see \texttt{idmapping.dat} and \texttt{idmapping\_selected.tab} files). 

\subsection{Scoring associations}
\label{scoring}

In the vast knowledge base of PREGO, scoring is crucial to sort 
the results. Ranking the association based on their trustworthiness and
relevance to the user's query is what makes platforms useful to users.
Hence, scoring is an indispensable component of data platforms 
that respond to queries.

PREGO responses with relevant and probable associations are the user query
through the three channels of information. 
Each channel has a tailored scoring scheme depending on the data structure to
rank the associations of terms.
All associations are scored and normalised in the interval $(0,5]$ to maintain consistency. 
The scoring starts with the contingency table (Table~\ref{table:pregoA1}) of two random variables, $X$ and $Y$.
For example, $X = 1$ is a NCBI id and $Y = 1$ represents an GObp term id. 
The $c_{1,1}$ is the co-mention count of these two terms, $X = 1$ and $Y = 1$, i.e., the joint frequency. 
The marginals, $c_{1,.}$ and $c{.,1}$ for $x$ and $y$ respectively, are the counts for entity type id. 
These frequencies are used differently depending on the scoring indices.
Each scoring scheme, must be evaluated and benchmarked because there is not a 
single perfect measure. 

\begin{table}[ht]
   \centering
   \begin{tabular}{c|llll}
    & \multicolumn{4}{l}{Y = y} \\ \cline{2-5} 
   \multirow{4}{*}{X = x} &  & Yes & No & Total \\ \cline{3-5} 
    & \multicolumn{1}{l|}{Yes} & $c_{x,y}$ & $c_{x,0}$ & $c_{x,.}$ \\
    & \multicolumn{1}{l|}{No} & $c_{0,y}$ & $c_{0,0}$ & $c_{0,.}$ \\
    & \multicolumn{1}{l|}{Total} & $c_{.,y}$ & $c_{.,0}$ & $c_{.,.}$
   \end{tabular}
   \caption[PREGO contingency table between two terms]{Contingency table of co-occurrences between entities $X = x$ and $Y = y$. 
   $c_{x,y}$ is the count of the
co-occurrence of these entities. $c_{x,.}$ is the count of the $x$ with all the
entities of $Y$ typy. Conversely, $c_{.,y}$ is the count of $y$ with all the entities of $X$ type (e.g., taxonomy}
   \label{table:pregoA1}
\end{table}

The Literature channel scores associations of each pair of entities based on their co-occurrence in each document, paragraph, and sentence.
Environmental Samples channel scores associations based on abundance tables and metadata. Hence, the number of samples that entities co-occur. 
Thus, it has two depth levels, microbes with metadata, and sample identifiers.
The Genome Annotation and Isolates channel has predetermined values of scores
based on the resource and the level of manual curation which contain the most trustworthy associations. 

   \subsection{Literature}
   \label{subsec:prego-tm}

   \subsubsection{Corpus}
The \textit{Literature} channel of PREGO contains the associations between 
entities as extracted by PubMeD®. Abstracts and full text articles were retrieved
from MEDLINE® and PubMed Central® Open Access Subset (PMC OA Subset) \parencite{sayers2021database}, respectively. 
NCBI, which hosts these data, has a powerful File Transfer Protocol (FTP) service 
that is used to periodically retrieve new published documents. The download time 
is around 5 hours.
The retrieved files are XML files, compressed. These raw files are the are converted
to tabular format while maintaining the following information:

\begin{itemize}
    \item PMID DOI
    \item Authors
    \item Journal.volume:pages
    \item year
    \item title
    \item Abstract
\end{itemize}

Each file contains ~30000 lines which, sum to 33 million abstracts. Some
abstracts are duplicated across files, so in order to avoid bias in tagging,
a script keeps only the latest occurrence of duplicated abstracts.
The PMC Open Access Subset has two directories: oa comm and oa noncomm.
They are separated because they contain different licences. For commercial usage,
articles in the oa comm directory are allowed because they have CC BY and CC0 licenses.
For non-commercial usage, oa noncomm can be used which has full text documents
licensed under all Creative Commons license types with the exclusion of CC BY and CC0)
and oa comm directories. 

   \subsubsection{Tagger}

Text mining is on the corpus using the dictionary through the EXTRACT tagger \parencite{pafilis2016extract, jensen2016one}.
The tagger is implemented in C++ and first loads all entities of the dictionary
and then it recognises the co-occurrences in the corpus and returns them with a score. 
The tagger output files, the pubmed tsv files and dictionary files are used by
the 6 perl scripts, Figure \ref{fig:prego-textmining-structure}. More specifically, 
the create documents script reads all the tsv files of pubmed and stores all abstracts
in a single file while simultaneously selecting the latest abstract of the duplicated ones. 
The total duration is 211 minutes. The maximum amount of memory used is 6gb and the tagger
is running using 8 threads.
   
\begin{figure}[hbt!]
      \centering
      \includegraphics[width=0.85\columnwidth]{figures/prego_textmining-structure.png}
      \caption[PREGO analysis methodology]{
         Structure of the Literature channel with multiple scripts working together.
         The initial data are the Dictionary and Pubmed. Each one has preprocessing steps. 
         The Text mining steps are the last steps to create the database pairs file 
         with all associations and scores.
      }
      \label{fig:prego-textmining-structure}
\end{figure}

\subsubsection{Score}

Literature channel adopts the scoring scheme of STRING 9.1 \parencite{franceschini2012string}
and COMPARTMENTS \parencite{binder2014compartments}. This scoring has three steps. 
First, for each co-occurrence of a pair of entities, a weighted count is
calculated because of the complexity of the text.

\begin{equation}
   c_{x,y} = \sum_{k=1}^{n}{w_d \delta_{dk}(x,y) +w_p \delta_{p,k}(x,y) + w_s \delta_{sk}(x,y)}
   \label{eq:prego-score-1}
\end{equation}

Different weights are used for each part of the document ($k$) for which both
entities have been co-mentioned, $w_d = 1$ for the weight for the whole document
level, $w_p = 2$ for the weight of the paragraph level, and $w_s = 0.2$ for the
same sentence weight. 
The delta functions equal to one (Equation~\ref{eq:prego-score-1}) if the co-occurrence
exists and is zero if it doesn't. Therefore, the weighted count becomes higher as
the entities are mentioned in the same paragraph and even higher when in the same sentence.
Subsequently, the score is calculated with the following equation:

\begin{equation}
   score_{x,y} = c_{x,y}^a (\frac{c_{x,y} c_{.,.}}{c_{x, .}c_{.,y}})^{1-a}
   \label{eq:prego-score-2}
\end{equation}

where $a = 0.6$ is a weighting factor, and the $c_{x,.}$, $c_{.,.}$, 
$c_{.,y}$ are the weighted counts as shown in Table~\ref{table:pregoA1} estimated using the same Equation~\ref{eq:prego-score-2}. 
This weighting factor has been optimized and benchmarked in multiple 
applications of text mining~\parencite{franceschini2012string, binder2014compartments, pletscher2015diseases}. 
The value of Equation~\ref{eq:prego-score-2} has been noted that is sensitive to
the increasing size of the number of documents (MEDLINE PubMed—PMC OA).
Therefore, to obtain a more robust measure, the value of the score is transformed to $z$-score. 
This transformation is presented in detail in the COMPARTMENTS resource \parencite{binder2014compartments}. 
Finally, the confidence score is the $z$-score divided by two. Cases in which
the scores exceed the (0,4] interval are capped to a maximum of 4 out of 5 to
reflect the uncertainty of the text mining pipeline.

   \subsection{Environmental Samples}
   \label{subsec:prego-envsamples}

Environmental Samples channel contains associations extracted by environmental DNA studies, 
namely amplicon and shotgun metagenomic analyses. These data were downloaded with custom API clients
of MGnify \parencite{mitchell2020mgnify} and MG-RAST \parencite{wilke2015restful} repositories.
From these repositories the taxonomic and functional profiles were downloaded along with their 
studies metadata. Both amplicon and shotgun metagenomic have taxonomic profiles, while 
only the latter include also functional profiles.
KEGG Orthology terms and/or Uniref identifiers of functional profiles are mapped to GOmf terms 
prior to the analysis with the custom mapping scripts, see \ref{subsec:prego-dict}.

PREGO processes these data per sample, so per sample the profiles are associated with the metadata. 
To harmonise the data and metadata names, organisms, environments, processes, functions, the EXTRACT tagger and the dictionary is used, as described previously. 
This process creates a entity pairs co-occurrence file for all samples. 
These associations are subsequently scored. The score is based on the count of
samples the entity of interest co-occurs with specific sample metadata or profile (function or taxonomic). 

More specifically, for every term $x$, the number of samples it is present is calculated as the background 
and the count of samples of the co-occurrent term (metadata background) $c.,y$ (see Table~\ref{table:pregoA1}). 
Each association between terms $x$,$y$ is found in $c_{x,y}$ samples. 
These calculations are performed separately for each resource.
The equation of the score is:

\begin{equation}
   score_{x,y} = 2.0*{\frac{\sqrt{c_{x,y}}}{c_{.,y}^{0.1}}}
\end{equation}

This is asymmetric because the denominator is the marginal of the associated entity. 
Thus, the value is reduced as the marginal of $y$ is increasing, i.e., the number of samples that $y$ is found. 
Conversely, it promotes pairs of associations in which the number of samples of 
the association are similar to the marginal of $y$. 
The exponents on the numerator and denominator equal to $0.5$ and 
to $0.1$, respectively, in order to reduce the rapid increase of score.
Lastly, the value of the score is capped in the range $(0,4]$.

   \subsection{Annotated Genomes and Isolates}
   \label{subsec:prego-isolates}

Annotated genomes and isolates are the manually curated data of PREGO's knowledge base.
This fact makes this channel the most trustworthy because every single species/strain have manually curated metadata. 
JGI-IMG \parencite{chen2021img, mukherjee2021genomes} has millions of genes that 
correspond to isolated genomes, SAGs and MAGs. These annotations and their corresponding metadata
were collected using web scrapping because even though the data are open there isn't an API client for these data.
The metadata about environments were harmonised using the EXTRACT tagger leading to organisms—environments co-occurrences.
The KEGG ids from the annotations were mapped to GOmf terms and were used for the organisms—processes associations.
   
The Struo pipeline \parencite{de2020struo}, which is based on the Genome Taxonomy DataBase (GTDB) (v.03-RS86) \parencite{parks2020complete} was also used.
This pipeline contains organisms—processes associations. 
UniRef50 annotations were changed to GOmf terms with custom mapping. 
GTDB genome taxonomy was mapped to the corresponding NCBI taxa. 
Struo's associations were assigned a confidence level of four out of five. 
This score reflects the high-quality of these data and metadata.
   
Lastly, BioProject data were used in PREGO from the NCBI FTP/e-utils services \parencite{sayers2021database}. 
The BioProject ids that has been assigned a PubMed abstract, a unicellular taxon, and Genome sequencing as data type were filtered.
The text mining pipeline described above, was used to extract associations of the
assigned taxon with the other entities in the abstracts. 
The BioProject derived associations were assigned the score of three (out of five)
because of the implementation of the text mining pipeline.


\subsection{Soil use case}
\label{soil-prego-m}

PREGO's knowledge base was searched for it's contents about soil environments. 
The dictionary of PREGO was first searched for the ENVO term of soil, ENVO:00001998,
and then all the child terms were also retrieved. Using this list of 
ENVO terms about soil, all channels of information were filtered, summarised and 
analysed for patterns as a use case. ARENA3D Web \parencite{karatzas2021arena3dweb} and
igraph \parencite{Csardi2006} were used for network analysis.

\subsection{Behind the scenes}
\label{deamons}

PREGO's pipeline and hosting is running on a 64 GB RAM DELL R540, 20 core, Debian server.
The server has 2 terabyte SSD storage for high responsiveness. 
All code is versioned with the git versioning tool to keep history and 
resolve bugs. The development cycles of PREGO are based on DevOps practices that
combine development and operations and enable the efficient deployment of software services.
The Continuous Development and Continuous Integration (CD/CI) method proved to be
important aspect during the developed of PREGO in terms of multiple testing,
resolving bugs and effective collaborative coding. 

The pipeline of PREGO is streamlined and integrated to be executed regularly,
spanning from once a month to six-month cycles, ensuring that changes of resources are
continuously integrated into PREGO. These cycles are called daemons in PREGO and 
each channel has a dedicated script that executes all the code while 
maintaining backups, Figure \ref{fig:devops}.


\begin{figure}[hbt!]
   \centering
   \includegraphics[width=95mm]{figures/figure_A1_PREGO_daemons.png}
   \caption[PREGO DevOps]{Scripts called daemons are executing the PREGO methodology in cycles depending on the updates of the databases used. Figure adopted by \parencite{microorganisms10020293}.}
   \label{fig:devops}
\end{figure}


   \subsection{Code} 
Code produced for PREGO are available under BSD 2-Clause “Simplified” License.
Scripts, where third party libraries have been used, are subject to their individual licenses.
API calls were implemented in Python 3. Daemons and FTP downloads were written in Bash and 
mappings and statistics in GNU AWK.
   
   \begin{itemize}
      \item prego\_gathering\_data 
      \href{https://github.com/lab42open-team/prego_gathering_data}{github.com/lab42open-team/prego\_gathering\_data}
      \item prego\_daemons \href{https://github.com/lab42open-team/prego_daemons}{github.com/lab42open-team/prego\_daemons}
      \item prego\_mappings \href{https://github.com/lab42open-team/prego_mappings}{github.com/lab42open-team/prego\_mappings} 
      \item prego\_statistics \href{https://github.com/lab42open-team/prego_statistics}{github.com/lab42open-team/prego\_statistics}
      \item prego\_use\_case \href{https://github.com/lab42open-team/prego_use_case}{github.com/lab42open-team/prego\_use\_case}
   \end{itemize}

Additional software and curated lists along with their individual license are:
   \begin{itemize}
      \item tagger:	\href{https://github.com/larsjuhljensen/tagger}{https://github.com/larsjuhljensen/tagger}, BSD 2-Clause "Simplified" License
      \item mamba: \href{https://github.com/larsjuhljensen/mamba}{https://github.com/larsjuhljensen/mamba}, BSD 2-Clause "Simplified" License 
      \item tagger dictionary:  \href{https://download.jensenlab.org/}{https://download.jensenlab.org/} and there in: \\
      \href{https://download.jensenlab.org/prego_dictionary.tar.gz}{https://download.jensenlab.org/prego\_dictionary.tar.gz}, CC-BY 4.0 license
   \end{itemize}

\section{Results}
\label{sec:prego-results}

   \subsection{PREGO Contents}
   \label{subsec:prego-contents}

The data sources of PREGO can be categorized into six types: abstracts, full articles,
isolates, annotated genomes, markergene samples, and metagenomic samples, Figure \ref{fig:prego-summary}.
In terms of metadata availability, JGI IMG, Struo, BioProject, and MG-RAST provide
metadata for their respective data types.
BioProject provides metadata for annotated genomes with abstracts, which could be useful for researchers seeking comprehensive information.
The number of items varies significantly across sources, with MEDLINE and PubMed
having the largest collection of items (33 million) and PubMed Central OA Subset
having a significant collection of full articles (2.7 million).
JGI IMG, Struo, BioProject, and MG-RAST have a smaller but still substantial number of isolates, annotated genomes, and markergene samples.
In terms of licenses, the majority of the sources have open licenses,
such as CC0, CC BY-SA 4.0, CC-BY, and NLM Copyright, which allows for the free
use and sharing of the data. This is an essential aspect of data sharing and
collaboration in the scientific community.

   \begin{figure}[hbt!]
      \centering
      \includegraphics[width=\columnwidth]{figures/PREGO_summary_channels.png}
      \caption[PREGO contents summary]{
         Resources, entities and associations summary in each channel of information of PREGO.}
      \label{fig:prego-summary}
   \end{figure}

Regarding the entity types, PREGO's knowledge base contains 364,000 taxa
(NCBI Taxonomy has 620,000 Bacteria, Archaea, and microbial eukaryotes). 
About 258,000 taxa are at the species level. 
All environment Ontology terms are found at about 1000 terms. Regarding Gene ontology, 15,000 biological process terms and 7.900 molecular function term are present.
PREGO's knowledge base of entities and associations are a multipartite network
with entities as nodes and co-occurrences as links with score as weight.

\subsection{Bulk download}
\label{bulk-download}

All the knowledge base of PREGO is available for download programmatically.
In Table~\ref{table:prego-appD-1} the hyperlinks are provided, one per channel, along with md5sum files for sanity checks.
Data are compressed to tar.gz format so expect an order of magnitude higher file size when decompressed.

   \begin{table}[ht]
      

      \begin{tabular}{lll}
      \toprule
      Channel Link & md5sum & Size (in GB) \\ \midrule

      \href{https://prego.hcmr.gr/download/literature.tar.gz}{Literature} & \href{https://prego.hcmr.gr/download/literature.tar.gz.md5}{literature.tar.gz.md5} & 5.4 \\

      \href{https://prego.hcmr.gr/download/environmental\_samples.tar.gz}{Environmental samples} & 
      \href{https://prego.hcmr.gr/download/environmental\_samples.tar.gz.md5}{environmental\_samples.tar.gz.md5}
      & 0.69 \\

      \href{https://prego.hcmr.gr/download/annotated\_genomes\_isolates.tar.gz}{Annotated genomes} &
      \href{https://prego.hcmr.gr/download/annotated\_genomes\_isolates.tar.gz.md5}{annotated\_genomes\_isolates.tar.gz.md5} & 0.26 \\ \bottomrule
      \end{tabular}
      \caption[PREGO Bulk download.]{Bulk download links and md5sum files.}
      \label{table:prego-appD-1}
   \end{table}

%%%%%%%%%%%%%%%%%%%%%%%%%%%%%%%% soil %%%%%%%%%%%%%%%%%%%%%%%%%%%%
   \subsection{PREGO about soil}
   \label{subsec:prego-soil}

Searching for an environment in PREGO's knoweldge base is possible through
it's web interface, it's API and bulk data. This analysis used PREGO's bulk data. 
In total, the dictionary contain 115 ENVO terms related to soil. These terms were
used to filter associations of environments with organisms, processes and functions 
from all channels. This search resulted in 207939 PREGO associations. 
To keep the most trustful associations and to be comparable with the Chapter \ref{cha:isd-crete-soil}
filters were applied. In Literature channel, only associations with score 4.8 or higher were kept. 
In Environmental Samples and Annotated Genomes associations with score of 3 or higher were kept. 
In addition, only taxa from species and strain taxonomic level were selected.
This filtering led to 14,379 associations, 599 from Literature, 7003 from Environmental
Samples and 6777 from Annotated Genomes.

In total, 49 ENVO terms were associated with 6276 taxa, 169 biological processes
and 3017 molecular functions. The network was further reduced to keep the 
biggest connected component. 
The majority of associations, 10,929, were with environments and organisms.
Environments with molecular functions have 3139 associations, followed by 
environments with biological processes with 274 associations, Figure \ref{fig:prego-soil-network} and \ref{fig:prego-soil-network_k}.

The soil (ENVO:00001998) has 11694 associations
loam (ENVO:00002258) has 1222 associations,
rhizosphere (ENVO:00005801) 393 associations,
forest soil (ENVO:00002261) has 300 associations and
meadow soil (ENVO:00005761) has 226 associations.

   \begin{figure}[hbt!]
      \centering
      \includegraphics[width=\columnwidth]{figures/prego_soil_arena.png}
      \caption[PREGO soil network]{
         The four layers of associations, -27 is environments, -2 is organisms, -21 is biological processes, -23 is molecular functions. The nodes that are labeled are the with the most associations. With gray are displayed the Literature derived associations while with pale yellow the Environmental samples. }
      \label{fig:prego-soil-network}
   \end{figure}
   

   \begin{figure}[hbt!]
      \centering
      \includegraphics[width=\columnwidth]{figures/prego_soil_arena_knowledge.png}
      \caption[PREGO soil network]{
         Associations of Annotated Genomes in PREGO, -27 is environments, -2 is organisms.}
      \label{fig:prego-soil-network_k}
   \end{figure}

Regarding taxonomy, archaea have 122 taxa and 7 phyla, bacteria have 5985 taxa
in 119 phyla and eukaryotes have 224 in 5 phyla \ref{fig:prego-soil-phyla}

   \begin{figure}[hbt!]
      \centering
      \includegraphics[width=\columnwidth]{figures/prego_soil_phyla_summary.png}
      \caption[PREGO soil taxonomy summary]{
         Phyla associated with soil in the PREGO knowledge base.}
      \label{fig:prego-soil-phyla}
   \end{figure}


\section{Discussion}
\label{sec:prego-discussion}

   \subsection{One stop shop}
   \label{subsec:prego-contents-disc}

PREGO approach is based upon four principles. First, open access data and
metadata integration along with literature are the pillar of the current
available knowledge to microbiologists. Second, the associations of microorganisms,
environments and processes are key for the deciphering of ecosystem function.
Third, the modular structure and regular cycles of updates are fundamental to
keep up with the rapid advances of microbial ecology. And last but not least,

The harmonisation of information from multiple types of resources and of different
entities is a major issue in science today. PREGO approached this problem from 
the Natural Language Processing perspective offering a global integration 
of microbial knowledge. To accomplish this some strategic choices were made, 
namely using the same dictionary of terms for all resources and used 
the EXTRACT tagger to identify the terms in free text and metadata.
PREGO manages to a great proportion of microbial taxa, even unicellular eukaryotes, in the knowledge base.
Because different channels, PREGO extracts associations in the species and higher taxonomic levels.
Higher level taxonomic associations with environments and processes are present
which are useful for classification processes and versatility of taxa hypotheses.

One important caveat of this approach is the difficulty to 
identify false positives, i.e associations that are apparent but without 
biological meaning. This is because, in the current version, the text mining
pipeline is based on Named Entity Recognition without taking into account 
the context. This is an important limitation that Large Language Models 
are promising.

PREGO provides all this knowledge base in multiple 
ways and is implemented to be easy to use and openly accessible to all with it’s graphical and programming user interface.
A dedicated web interface is useful to users for single queries
and exploration of literature and/or samples. The API is useful and 
robust for multiple calls and, not for the faint of heart, PREGO is 
available for bulk download. 

   \subsection{Platform comparison}
   \label{subsec:prego-similar-platforms}


Based on the table \ref{table:prego4}, BacDive seems to be the most versatile
platform among the four, offering a wide range of features. It has a high level
of manual curation, with environment-taxa associations, process/function-taxa
associations, and phenotypic data available. Additionally, BacDive provides
spatial coordinates, an application programming interface but limited bulk download of data.
It has a high score for manual curation, indicating that it is well-maintained and regularly updated with accurate information.
As for the presence of various types of data, BacDive appears to be the platform
with the most comprehensive dataset, offering original and integrated data from
various sources. Its high scores for environment-taxa associations,
process/function-taxa associations, and phenotypic data suggest that it contains a broad range of microbial data.

PREGO appears to have some notable benefits. Specifically, it has a high score
for literature integration, which suggests that it is able to integrate microbial
data from various sources and publications. Additionally, PREGO is capable of
processing and storing data on environment-process/function associations, which
could be useful for researchers studying microbial processes and functions.
Another benefit of PREGO is its ability to facilitate bulk downloads of data,
making it easier for researchers to access and analyze large datasets.
This feature can be particularly useful for big data analysis and machine learning applications.
It is worth noting that PREGO's scores for other features, such as manual
curation and environment-taxa associations, are relatively low compared to
BacDive and Web of Microbes. However, its strengths in literature integration
and bulk data download may make it a useful platform for researchers with specific needs.


   \begin{table}[ht]

      \begin{adjustwidth}{-0.75cm}{}

      \begin{tabular}{@{}lllll@{}}
      
      \toprule
      Functionality & BacDive & Web of Microbes & NMDC & PREGO \\ \midrule
      manual curation & high & high & intermediate & low \\ 

      literature integration & limited & no & no & yes \\

      environment—taxa associations & yes & yes & yes & yes \\

      \begin{tabular}[c]{@{}l@{}}environment—process/\\ function associations\end{tabular} & no & no & no & yes \\

      process/function—taxa associations & yes & yes & yes & yes \\
      phenotypic data & yes & no & no & no \\

      data origin & \begin{tabular}[c]{@{}l@{}} original \\integration \end{tabular} & original & \begin{tabular}[c]{@{}l@{}} original \\integration \end{tabular} & integration \\

      spatial coordinates & yes & no & yes & no \\

      application programming interface & yes & no & yes & yes \\

      bulk download & limited & yes & yes & yes \\ \bottomrule

      \end{tabular}
      \end{adjustwidth}
      \caption[Feature comparison between PREGO and other similar platforms]{Feature comparison among platforms that facilitate knowledge discovery and integration of microbial data.}
      \label{table:prego4}
   
   \end{table}      


   \subsection{Soil microbiome}
   \label{subsec:prego-soil}

Soil is the most prominent environment in terrestrial ecosystems. Yet 
biological information about the microbes and their processes remains 
scattered. PREGO's knowledge base and interface provides a unique 
approach towards the harmonisation of this information. As shown 
in Figure \ref{fig:prego-soil-phyla} the information is scattered in multiple ways.
The source of information, i.e the channels of PREGO, is a major barrier of unification and
secondly a important limitation is the taxonomic barrier, i.e
different super-kingdoms of microbes, because different kingdoms have different experts.
Euryarchaeota and Crenarchaeota are phyla from
Archaea and 
small eukaryotes are mostly represented by the fungi phylum Ascomycota. Proteobacteria
and Actinobacteriae are dominating soil environments (Figure \ref{fig:prego-soil-phyla}),
global environments in general as shown in \textcite{microorganisms10020293} 
and Crete island soils (Figure \ref{fig:isd_fig2_taxonomy} of Chapter \ref{cha:isd-crete-soil}).
The abundance of taxa of Ascomycota indicates the importance of their interactions 
with bacteria and archaea which only recently has been explored \parencite{Labouyrie2023} in scale. 
There is an absence of Environmental samples about Ascomycota, but the Literature and Annotated genomes 
channel is evidence of their importance.
The soil case study of PREGO shows the importance of data integration techniques
and metadata in order to identify gaps in knowledge and discover what is known.



% --------------------------------------------------
% 
% This chapter is for Crete system ecology
% 
% --------------------------------------------------


\chapter{Towards a Cretan soil biodiversity data system}
\label{cha:crete-idea}

\section{Introduction}\label{intro_idea}

Ecosystem ecology studies the interactions of organisms with their surroundings
as a whole \parencite{van-dyne1966ecosystems}. These interactions are complex and hard to interpret and quantify.
Macroscopically they represent the flows of energy and materials in the total environment.
The implications of this approach to ecosystems are that multiple types of disciplines 
are intersected. For example, soils are defined by their physical characteristics,
i.e grain size, clay content, their chemistry, i.e pH, elements compositions, and 
their biodiversity, i.e plants, arthropods and microbes \parencite{vogel2022}. Humans have influenced 
and/or exploited almost all the environments of earth with their activities.
Hence, understanding ecosystems requires a synthesis which in the digital era 
is becoming more and more plausible. 

There are multiple terms that describe the digital representation of ecosystems.
One of the earliest is "digital earth" which mentioned in the late 90s \parencite{Goodchild_2012}.
This notion of the virtual earth was initially realised with satellite imaging,
on-land sensors and the relevant software. Space, an area on Earth, is
the starting point of ecosystems functioning which is influenced by geography,
climate, geology, biodiversity and human management across scales \parencite{Zarnetske2019_towards}. The coupling of all these
fields, and the data they provide, with spatial information leads to the
digital twin of ecosystems. All these data can be considered as different layers
of an Earth system data cube \parencite{mahecha2020earth}. The data cube representation
allows for multivariate modeling and ecosystem level predictions \parencite{mahecha2020earth}.

Studying Earth as a whole and thus generating data at the global scale, all at once,
is part of a new field called planetary biology. Smaller scales of such integrating systems exist like
the Island Digital Ecosystem Avatar \parencite{Davies2016}. Islandscapes 
is also an interesting approach which defines the holistic ecosystem of the island
and the interface of land and sea \parencite{Vogiatzakis_land_2017}. Also islands
due to their smaller scale and confined nature are useful virtual ecological labs. They harbour 20\% of 
biodiversity even though they cover less than 7\% of land surface area. Also they are under major threats 
that diminish their biodiversity \parencite{fernandez-palacios2021scientists}.
Further focusing on soil, could assist bringing the natural and human activities 
together in the so-called soilscape \parencite{LAGACHERIE2001105}.

There are multiple tangible outcomes of integrating all these data gather together about a
specific place and biome. First, it allows for sampling biases and gaps identification. Second, 
it enables testing new hypotheses about unforeseen associations of variables. Third, 
it forms a foundation of the current state of knowledge to assist decision making 
processes about conservation, ecosystem services management and comparisons for 
future states. Building such systems is also a cultural treasure for society because
they will present the uniqueness of each place.

Biodiversity is represented in ecosystems as point occurrences of each taxon. The
DarwinCore standards have assisted a lot the homogeneity of the data. GBIF is a 
global aggregator that contains such information about all ecosystems from 
different resources \parencite{noauthor_gbif_nodate}. It is crucial that each sampling point has rich metadata. 
Sequences from eDNA studies are also part of biodiversity and are stored in 
databases like ENA, JGI IMG and GenBank.
From the bacterial point of view, metagenomic samples are abundant in these 
databases and there are also curated databases that classify them in some baseline functionality. 
There is also a wealth of knowledge about biodiversity in the literature. 
Databases like PubMed, Goolge Scholar, Dimensions and Biodiversity Heritage Library 
are world class providers with different content.

Examples of platforms that host open data soil data are the 
European Soil Data Centre \parencite{Panagos2022} and the World Soil Information
Service (WoSIS) of the ISRIC \parencite{Batjes2024}. Regarding the latter, a unified
platforms called SoilGrids provides open access to many predicted maps \parencite{poggio-soil-7-217-2021}.
Apart from samplings and literature,
spatial data are openly available terrestrial ecosystems.
To name a few, there openly available climatic, land cover, desertification risk, aridity, soil type, normalized
vegetation index, bedrock geological formations data.

In this chapter, the focus is system's approach of the soil ecosystems of the island of Crete.
First, the literature about Crete is aggregated and summarised. Second, biodiversity data
from GBIF and IUCN \parencite{iucn2024} are collected and third a spatial data cube is created. 
Having all these data joined about Crete shows the culmination of intensive 
research of the Cretan environments over the centuries and provokes for new
hypotheses and conservation priorities.


\section{Materials and Methods}\label{integration_methods}

\subsection{Literature}\label{crete-literature}

Crete is an unambiguous name, this fact allows for simple keyword searches to discover 
the relevant literature in multiple platforms. Given various possible references to Crete,i.e \textit{Crete, Kriti, Kreta, and Cretan},
it is essential to account for these variations during the keyword search process.

The corpus of PubMed was downloaded via the ftp protocol and the E-utilities API, details of which can be found on the PubMed Download page.
The ftp method generates multiple XML files, which were later converted into
files with tab-separated values. These files were then compressed. Each .tsv.gz file
features six distinct fields in this format: PMID, DOI, Authors, Journal, Volume:Pages, Year, Title, Abstract.
Pubmed was searched with the aforementioned keywords using the DIG tool \parencite{fanini2021coupling}. To further 
analyse the results, the MeSH terms were retrieved using the \href{https://www.ncbi.nlm.nih.gov/books/NBK25497/}{E-utilities API}.

The Dimensions platform has a user interface specifically tailored for academic literature searches.
It is the open source equivalent of Web of Science.
With this interface, a query for \textit{Crete, Kriti, Kreta, and Cretan} within
the title and abstract fields was requested yielded the articles, each annotated with the respective Field of Research.

Google scholar was also searched through the \textit{serpapi} API using
a python script. The same keywords were used as in the previous searches.

Historical literature related to the biodiversity of Crete has been explored
using the BHL data model and schema, conducting searches on titles, items, and subjects.
Items represent the digitised documents, while subjects are assigned to each title.
The schema includes a page table with page-level information.
A comprehensive archive of OCR texts was downloaded, totaling approximately
40GB compressed and 300GB uncompressed, that includes 62 million OCR pages from 292,000 unique documents.
To ensure a thorough analysis, search was performed through all pages since
relevant information about Crete may be embedded within the documents
themselves, and not necessarily in the title or summaries.

\subsection{Samplings}\label{crete_samplings}

The samplings are distinguished as sequences, biodiversity and soil, using ENA, GBIF and WoSIS respectively.
The ENA API was used with a POST request containing the bounding box of Crete as a query \parencite{Yuan2023}. This 
query returned a list with the sample identifiers. Using this list and a custom python script the API
was invoked with GET requests to retrieve all the available metadata for each sample.
GBIF has the in the online user interface the functionality of selecting the area of interest with a bounding box and requesting
the data \parencite{noauthor_gbif_nodate}. The request was then approved and was downloaded. All species 
from IUCN that are in Crete were downloaded from the website \parencite{iucn2024}. IUCN provides the 
interactive maps selection functionality. 
WoSIS has a dedicated Web Feature Service (WFS) that is specialised for spatial data \parencite{Batjes2024}. 
Using the available tutorial and a adjusted R script the samplings and metadata of 
soil in Crete were retrieved.
Edaphobase was also searched for sampling information from Crete. Edaphobase is
a curated database of museum records that aims to include all aspects of soil biodiversity \parencite{BURKHARDT20143}.

In addition, the JGI IMG GOLD database contains many samples which were downloaded from \href{https://gold.jgi.doe.gov/downloads}{the webservice}.
This path of Public Studies / Biosamples / SPs / APs / Organisms was selected in the excel file format option. 
GOLD didn't have any unique samples when compared with ENA.

\subsection{Spatial data}\label{crete_spatial}

Maps were downloaded and cropped from multiple resources.
In terms of managing spatial data, geometry, and transformations, the following
R packages were used:
sf package version 1.0-14 \parencite{Pebesma2023}, for polygons, and the terra
package version 1.7-55 \parencite{hijmans2024terra}, for rasters.

The land use categories were obtained from the CORINE Land Cover (CLC) 2018
version v.2020.20u1 \parencite{CLC2023}.
The Historic Land Dynamics Assessment (HILDA+) dataset \parencite{winkler2021global}
facilitated the exploration of human pressures, such as changes in land usage and agriculture,
allowing the calculation of land use alterations from 1998 to 2018.

The protected areas of Crete were included as well from the wdpar package \parencite{Hanson2022}.
In WorldClim 2.0, twelve global climatic variables, such as the annual mean
temperature and annual precipitation, were included \parencite{Fick2017}.
The dataset on the Environmentally Sensitive Areas Index to desertification (ESAI)
in Greece \parencite{KARAMESOUTI2018266} was also employed, alongside the
Global Aridity Index Database \parencite{zomer2022version}.
The geological formations of Crete were derived from the geoportal of the
Decentralized Administration of Crete. These maps were created by the Crinno-Emeric Group
project \url{https://geoportal.apdkritis.gov.gr/gis/apps/storymaps/stories/19690f65abbe4e8ab0141b2fe7261a8c}.
Furthermore, the Harmonised World Soil Database v2 was integrated for soil mapping units and taxonomic soil classification \parencite{fao2023}.

\subsection{Tools and code}\label{Coding environment}
Visualisation was implemented with ggplot2 \parencite{wickham_ggplot2_2016} and pheatmap \parencite{Kolde2019}.
The environment we worked had Python 3.11.4, R version 4.3.2 \parencite{rcoreteam}.
Finally, computations were performed on HPC infrastructure of HCMR \parencite{zafeiropoulos_0s_2021}.

Scripts about data integration are available in
\href{https://github.com/savvas-paragkamian/crete-data-integration}{Crete data integration repository}.
Code is structured to be reproducible and interoperable.

\section{Results}\label{crete_idea_results}

\subsection{Contemporary literature}

Regarding PubMed, the search process yielded 1556 unique abstracts about Crete.
These were retrieved with only a handful of false positives.
The majority of these abstracts pertain to biomedical topics.
Out of those, 170, 11\%, originate from environmental sciences.

Google scholar 

Dimensions 

\subsection{Historical literature}
Regarding the Biodiversity Heritage Library and historical literature, it was
observed that 25,812 documents contained fewer than five mentions of keywords related to Crete.
Conversely, a smaller set of 3,696 documents exhibited five or more mentions of such keywords.
Items were filtered if they had less than five keyword hits related to Crete.
Consequently, 35,130 pages were extracted representing the most commonly referenced Crete-related content in the document corpus.
Of the filtered 35,130 pages, a content analysis was carried out to identify the presence of taxon names.
BHL employs the gnfinder tool on each OCRed page \parencite{mozzherin_gnamesgnfinder_2022}.
The results showed that 23,391 pages (66.61\%) contained taxon names within 3,000 distinct items.
From these, 890, contained more that 50 different taxon names.
Additionally, 75,405 unique taxon names were detected across these pages, with a total of 214,215 occurrences.
The items with taxon names were further analyzed based on the titles mentioned
in the Biodiversity Heritage Library (BHL). It was found that 1,879 items had titles in various languages, with the following distribution:

Czech (CZE): 1
Danish (DAN): 1
Hungarian (HUN): 1
Japanese (JPN): 1
Portuguese (POR): 1
Russian (RUS): 1
Spanish (SPA): 1
Norwegian (NOR): 2
Swedish (SWE): 2
Romanian (RUM): 3
Undetermined (UND): 11
Dutch (DUT): 12
Latvian (LAT): 16
Italian (ITA): 31
Chinese (CHI): 34
Multilingual (MUL): 37
French (FRE): 105
German (GER): 268
English (ENG): 1,350

The oldest document was published in 1554 and there are 1482 items published before 1960.

\subsection{Biodiversity and samplings}

Worldwide projects of microbiome studies have collected one or two topsoil
samples from Crete \parencite{Vasar2022, Labouyrie2023, Bahram2018, Orgiazzi2018}.
Some have focused on soil fungi \parencite{Mikryukov2023, Davison2021, Tedersoo2021}
and other soil eukaryotes \parencite{Aslani2022}.
The only thorough microbiome study of a soil ecosystem in Crete, to our knowledge,
is in the north west part of the island, the Koiliaris Critical Zone Observatory \parencite{tsiknia2014}.
Apart from the soil microbiome, soil physical and chemical properties has been
investigated by global and European projects like the Forum of European Geological Surveys
(FOREGS) \parencite{nerc19017}, the Geochemical Mapping of Agricultural and Grazing Land
Soil in Europe (GEMAS) \parencite{REIMANN2018302} and the Soil Profile Analytical
Database for Europe (SPADE) \parencite{Hiederer2006}.
LTER map

\begin{figure}[hbt!] 
    \centering\includegraphics[width=\columnwidth]{crete_integration_ena_terrestrial.png}
    \caption{Available terrestrial metagenomic samples from ENA in Crete.}
    \label{fig:isd_crete_ena}
\end{figure}


\begin{figure}[hbt!] 
    \centering\includegraphics[width=\columnwidth]{crete_integration_map_wosi_soil.png}
    \caption{Soil samples from Crete that are uploaded in WoSIS.}
    \label{fig:isd_crete_wosis}
\end{figure}

GBIF map 

Edaphobase has in total, 26 resources with 17 different sampling points covering 120 distinct taxa.



\subsection{Maps}

For the description of Crete terrestrial environment 10 layers were used as shown in Figure \ref{fig:crete_data_cube_map}.

\begin{figure}[hbt!] 
    \centering\includegraphics[width=0.93\columnwidth]{crete_data_cube_maps.png}
    \caption[Crete data cube]{The ten layers of the terrestrial environments of Crete.
    A. The digital elevation map. B. The Corine Land Cover EEA, Level 2.
    C. The geology of Crete. D. The Harmonised World Soil dataset, version 2.
    E. The Historic Land Dynamics Assessment dataset. F. The different protection areas of the island.
    G. The mean surface temperature of WorldClim2. H. The mean precipitation of  WorldClim2.
    I. The aridity index of soil, from the Global Aridity Index v3. J. The desertification risk assessment}
    \label{fig:crete_data_cube_map}
\end{figure}


The area per category is presented in Table \ref{table:crete_data_cube_area}.

\begin{longtable}{llll}
    \caption{Area cover, in km\textsuperscript{2}, of the different categories per spatial data layer of Crete. \label{table:crete_data_cube_area}} \\
\textbf{hilda+ transition 1978-2018}               & \textbf{area} & \textbf{Aridity index class}               & \textbf{area}    \\
\endfirsthead
%
\endhead
%
urban (stable)                                     & 250.23        & Semi-Arid                                  & 6244.44          \\
urban to cropland                                  & 5.06          & Dry sub-humid                              & 1323.28          \\
urban to pasture/rangeland                         & 4.03          & Humid                                      & 537.20           \\
urban to unmanaged grass/shrubland                 & 1.01          &                                            &                  \\
cropland to urban                                  & 56.57         & \textbf{Soil (HWSD2)}                      & \textbf{area}    \\
cropland (stable)                                  & 1,193.61      & Calcaric Fluvisols                         & 598.40           \\
cropland to pasture/rangeland                      & 648.72        & Calcaric Regosols                          & 2250.80          \\
cropland to forest                                 & 163.42        & Chromic Luvisols                           & 767.04           \\
cropland to unmanaged grass/shrubland              & 5.05          & Eutric Cambisols                           & 97.92            \\
pasture/rangeland to urban                         & 64.47         & Lithic Leptosols                           & 4777.00          \\
pasture/rangeland to cropland                      & 145.34        & NA                                         & 4.08             \\
pasture/rangeland (stable)                         & 4,694.29      &                                            &                  \\
pasture/rangeland to forest                        & 205.83        & \textbf{elevation}                         & \textbf{area}    \\
pasture/rangeland to unmanaged grass/shrubland     & 127.28        & (0,200{]}                                  & 2227.85          \\
pasture/rangeland to sparse/no vegetation          & 1.01          & (200,400{]}                                & 2192.21          \\
forest to cropland                                 & 1.01          & (400,600{]}                                & 1584.84          \\
forest to pasture/rangeland                        & 23.22         & (600,800{]}                                & 815.99           \\
forest (stable)                                    & 162.33        & (800,1000{]}                               & 484.21           \\
unmanaged grass/shrubland to urban                & 10.08         & (1000,1200{]}                              & 355.34           \\
unmanaged grass/shrubland to cropland             & 41.40         & (1200,1400{]}                              & 249.44           \\
unmanaged grass/shrubland to pasture/rangeland    & 271.59        & (1400,1600{]}                              & 150.83           \\
unmanaged grass/shrubland to forest               & 20.20         & (1600,1800{]}                              & 91.55            \\
unmanaged grass/shrubland (stable)                & 10.09         & (1800,2000{]}                              & 75.50            \\
sparse/no vegetation to cropland                   & 9.11          & (2000,2200{]}                              & 39.69            \\
sparse/no vegetation to pasture/rangeland          & 84.87         & (2200,2400{]}                              & 12.23            \\
sparse/no vegetation to forest                     & 1.01          & (2400,2600{]}                              & 0.21             \\
sparse/no vegetation (stable)                      & 6.05          &                                            &                  \\
water                                              & 164.57        &                                            &                  \\
\textbf{CLC 2}                                     &               & \textbf{geology}                           &                  \\
Arable land                                        & 88.29         & -                                          & 0.57             \\
Artificial, non-agricultural vegetated areas       & 20.68         & J-E                                        & 1346.89          \\
Forests                                            & 299.93        & K-E                                        & 247.97           \\
Heterogeneous agricultural areas                   & 1102.96       & K.k                                        & 1252.97          \\
Industrial, commercial and transport units         & 39.77         & K.m                                        & 12.95            \\
Inland waters                                      & 6.74          & Mk                                         & 811.96           \\
Mine, dump and construction sites                  & 10.05         & Mm.I                                       & 1614.25          \\
Open spaces with little or no vegetation           & 410.66        & Ph-T                                       & 1012.08          \\
Pastures                                           & 59.25         & Q.al                                       & 910.75           \\
Permanent crops                                    & 2367.70       & T.br                                       & 299.58           \\
Scrub and/or herbaceous vegetation associations    & 3797.83       & f                                          & 117.83           \\
Urban fabric                                       & 110.63        & fo                                         & 317.59           \\
                                                   &               & ft                                         & 275.55           \\
\textbf{Protected areas}                           & \textbf{area} & o                                          & 93.94            \\
Aesthetic Forest                                   & 0.17          &                                            &                  \\
Sites of Community Importance (Habitats Directive) & 0.34          & \textbf{Desertification risk}              & \textbf{area}    \\
Game breeding station                              & 1.02          & N                                          & 159.10           \\
Controlled hunting area                            & 11.59         & F2                                         & 1900.60          \\
Core zone in National (Woodland) Park              & 47.56         & F1                                         & 1483.50          \\
UNESCO-MAB Biosphere Reserve                       & 88.65         & P                                          & 1556.60          \\
Protected Forest                                   & 417.74        & C2                                         & 623.50           \\
Wildlife Refuge                                    & 610.70        & Other areas                                & 283.80           \\
Special Protection Area (Birds Directive)          & 1261.78       & F3                                         & 1118.00          \\
Special Areas of Conservation (Habitats Directive) & 2371.42       & C1                                         & 769.70           \\
total protected (no overlap)                       & 2900.59       & C3                                         & 232.20           \\
total protected                                    & 4810.96       & \textbf{Crete total}                       & \textbf{8345}
\end{longtable}


\section{Discussion}\label{crete_idea_discussion}

Compare with Samothrace \parencite{noll2024insights}.

The wide range of languages and year of publications illustrates the global impact and interdisciplinary nature Cretan historical biodiversity literature.


Focusing in a confined area provides unique opportunities for data integration.
And as complex systems ecologists have stated bringing together information across
scales \parencite{brown2004METABOLIC} and functions leads to emergent properties that weren't possible to 
predict before \parencite{smith2016Origin}. The techniques to integrate are multiple in terms of statistics and modelling. But,
the advent of Large Language Models revolutionised the data interpretation and 
user interaction. In ecology and conservation this is very promising \parencite{doi2024biodiversity}.
In my opinion, the presented knowledge base of Crete would be a suitable 
case study to investigate these significant advancements to tighten the integration and form a 
user application.


% --------------------------------------------------
% 
% This chapter is for DECO
% 
% --------------------------------------------------


\chapter{How to accelerate the rescue of historical biodiversity literature?}
\label{cha:deco}

This is part of the work about automating the rescue of biodiversity historical literature \parencite{Paragkamian2022}.
\textbf{Citation:} \\ 
Paragkamian Savvas, Sarafidou Georgia, Mavraki Dimitra, Pavloudi Christina,
Beja Joana, Eliezer Menashè, Lipizer Marina, Boicenco Laura, Vandepitte Leen,
Perez-Perez Ruben, Zafeiropoulos Haris, Arvanitidis Christos, Pafilis Evangelos, Gerovasileiou Vasilis.
Automating the Curation Process of Historical Literature on Marine Biodiversity Using Text Mining: The DECO Workflow.

Shared co-first authorship.

\textbf{Journal:} Frontiers in Marine Science

DOI: \href{https://www.frontiersin.org/articles/10.3389/fmars.2022.940844}{10.3389/fmars.2022.940844} \footnote{For author contributions and supplementary material please refer 
    to the relevant sections. This is a modified version of the published 
    version, in terms of relevance, coherence and formatting.}


% DECO ABSTRACT
%\section{Abstract}
%
%Historical biodiversity documents comprise an important link to the long-term data 
%life cycle and provide useful insights on several aspects of biodiversity research 
%and management. However, because of their historical context, they present 
%specific challenges, primarily time- and effort-consuming in data curation. 
%The data rescue process requires a multidisciplinary effort involving four tasks: 
%(a) Document digitisation (b) Transcription, which involves text recognition and 
%correction, and (c) Information Extraction, which is performed using text mining 
%tools and involves the entity identification, their normalisation and their 
%co-mentions in text. Finally, the extracted data go through (d) Publication to 
%a data repository in a standardised format. Each of these tasks requires a 
%dedicated multistep methodology with standards and procedures. During the past 
%8 years, Information Extraction (IE) tools have undergone remarkable advances, 
%which created a landscape of various tools with distinct capabilities specific
%to biodiversity data. These tools recognise entities in text such as taxon names, 
%localities, phenotypic traits and thus automate, accelerate and facilitate 
%the curation process. Furthermore, they assist the normalisation and mapping 
%of entities to specific identifiers. This work focuses on the IE step (c) from 
%the marine historical biodiversity data perspective. It orchestrates IE tools 
%and provides the curators with a unified view of the methodology; as a result 
%the documentation of the strengths, limitations and dependencies of several 
%tools was drafted. Additionally, the classification of tools into Graphical 
%User Interface (web and standalone) applications and Command Line Interface 
%ones enables the data curators to select the most suitable tool for their needs, 
%according to their specific features. In addition, the high volume of already 
%digitised marine documents that await curation is amassed and a demonstration 
%of the methodology, with a new scalable, extendable and containerised tool, 
%“DECO” (bioDivErsity data Curation programming wOrkflow) is presented. DECO’s 
%usage will provide a solid basis for future curation initiatives and an 
%augmented degree of reliability towards high value data products that allow 
%for the connection between the past and the present, in marine biodiversity research.

% DECO INTRODUCTION
\section{Introduction}
\label{sec:deco-intro}
Species’ occurrence patterns across spatial and temporal scales are the 
cornerstone of ecological research \parencite{levin_problem_1992}. The compilation of both past 
and present marine data to a unified census is crucial to predict the future of 
ocean life \parencite{ausubel_guest_1999, anderson_does_2006, lo_brutto_historical_2021}. This compilation has 
been attempted by big collaborative projects, like 
Census of Marine Life\footnote{http://www.coml.org/} \parencite{vermeulen_understanding_2013}, 
that follow metadata standards and guidelines \parencite{michener_nongeospatial_1997, wilkinson2016the-fair} 
and modern web technologies \parencite{michener_ecological_2015}. The project has resulted in the incorporation 
of census data from the past, i.e. historical data, to modern data platforms, such 
as the Ocean Biodiversity Information System (OBIS) \parencite{klein_obis_2019}, which feeds 
the Global Biodiversity Information Facility (GBIF) \parencite{noauthor_gbif_nodate}. The transformation of 
historical data to modern standards is necessary for their rescue (data archaeology)
from decay and inevitable loss \parencite{bowker_biodiversity_2000}.

Historical data are usually found in the form of (a) historical literature and 
(b) specimens stored in biodiversity museum collections \parencite{rainbow_marine_2009} (the 
digital transformation process and progress of specimens is reviewed by \parencite{nelson_history_2019}). 
Historical biodiversity documents (also known as legacy, ancient or simply old 
documents) comprise literature from 1000 AD until 1960 and therefore are stored 
in an analogue and/ or obsolete format \parencite{lotze_historical_2009, beja_chapter_2022}. 
These old documents can be found in institutional libraries, publications, books, 
expedition logbooks, project reports, newspapers \parencite{faulwetter_emodnet_2016, 10.3897/BDJ.4.e11054, kwok_historical_2017} 
or other types of legacy formats (e.g. stored in floppy disks, microfilms or CDs).

From the scientific point of view, historical biodiversity data are as relevant 
as modern data \parencite{GRIFFIN2019238,beja_chapter_2022}.
They are valuable for studies on biodiversity loss \parencite{stuart-smith_thermal_2015,goethem_biodiversity_2021},
as forming baseline studies for the design of future samplings \parencite{rivera-quiroz_extracting_2019}
and for predictions of future trends \parencite{mouquet_review_2015}. Furthermore, 
historical data offer the kind of evidence needed for conservation policy and 
marine resource management, allowing for past patterns and processes to be
compared with current ones \parencite{fortibuoni_coding_2010,https://doi.org/10.1111/j.1755-263X.2012.00253.x, costello_biodiversity_2013,engelhard_ices_2016}.
Hundreds of historical marine data held in documents have already been uploaded
to OBIS, yet a Herculean effort is required to curate the thousands of available
documents of the Biodiversity Heritage Library (BHL) \parencite{gwinn_biodiversity_2009}
and other repositories.

Adequate and interoperable metadata are equally necessary and have to be
curated alongside data \parencite{heidorn_shedding_2008,mouquet_review_2015}.
In this context, standards and guidelines have been recently formulated in 
policies as Findable, Accessible, Interoperable and Reusable (FAIR) 
(meta)data \parencite{wilkinson2016the-fair,reiser_fair_2018}. Identifiers and 
semantics are used to accomplish the interoperability and reusability of
biodiversity data as well as the monitoring of their use \parencite{mouquet_review_2015}.
Indispensable to the curation process of marine data have been the standards of
the Biodiversity Information Standards \footnote{https://www.tdwg.org/}, more
specifically Darwin Core \parencite{wieczorek_darwin_2012}
and vocabularies such as those included in the International Commission on
Zoological Nomenclature \footnote{https://www.iczn.org/}, the World Register of
Marine Species \footnote{http://www.marinespecies.org/}
(WoRMS) \parencite{horton_world_2022}, 
the Environment Ontology \footnote{https://sites.google.com/site/environmentontology/home} (ENVO) \parencite{buttigieg2016environment}
and Marine Regions \footnote{https://www.marineregions.org/} \parencite{claus_marine_2014}.
These standards and vocabularies and their adoption by biodiversity initiatives
like GBIF and OBIS align with the goal of marine biodiversity Linked Open Data
and support their interoperability and reusability \parencite{page_towards_2016, penev_openbiodiv_2019,zarate_lobd_2021}.

The rescue process of historical biodiversity documents can be summarised in
four tasks Figure \ref{fig:rescue-workflow}. 
The first task is the digitisation of the document, which involves locating 
and cataloguing the original data sources, imaging/scanning with specific 
equipment and standards and uploading them to digital libraries \parencite{lin_quality_2006, thompson_moving_2013}.
In the second task, the images are analysed with text recognition software,
mainly through Optical Character Recognition (OCR)
(for standards see \parencite{groom_improved_2019} and for reviews
see \parencite{lyal_digitising_2016, 10.3897/rio.6.e58030}.
Text recognition errors are then corrected manually by professionals or
citizen scientists \parencite{herrmann_building_2020}. The third task is named
Information Extraction (IE) as it involves the steps of named entity
recognition, mapping and normalisation of biodiversity information \parencite{hakenberg_applications_2012}.
Here, the curators may compile a species’ occurrence census enriched with
metadata of the study, geolocation, environment, sampling methods and traits
among others \parencite{faulwetter_emodnet_2016}.
Lastly, the fourth task, is the data publishing to online biodiversity
databases/repositories \parencite{costello_biodiversity_2013,penev_strategies_2017}.
Expert manual curation is a cross-cutting action through all the aforementioned
tasks for quality control and stewardship \parencite{vandepitte_fishing_2015}.
This article focuses on the tools and curation procedures encompassed in
the third and fourth tasks described above.
   \begin{figure}[ht]
      \centering
      \includegraphics[width=\textwidth,height=\textheight,keepaspectratio]{figures/deco-figure-1.jpg}
      \caption[Historical document rescue process]{Summarised process of historical document rescue. Four tasks are required to complete the data rescue process of biodiversity documents. Each of these has several steps, methodology, tools and standards. Curation is needed in every task, for tool handling and error correction. The stars represent the 5-star ranking system of Linked Data as introduced by W3C \footnotemark \parencite{heath_linked_2011}. Availability of information from historical data increases as the curation tasks are completed (as exemplified by the fan on the right). Icons used from the Noun Project released under CC BY: book by Oleksandr Panasovskyi, scanning by LAFS, Book info by Xinh Studio, Library by ibrandify, Scanner Text by Wolf Böse, Check form by allex, Whale by Alina Oleynik, Fish by Asmuh, tag code vigorn, pivot layout by paisan, Certificate by P Thanga Vignesh, web service by mynamepong.}
      \label{fig:rescue-workflow}
   \end{figure}

   \footnotetext{\url{https://dvcs.w3.org/hg/gld/raw-file/default/glossary/index.html\#x5-star-linked-open-data}}

   \begin{figure}[ht]
      \centering
      \includegraphics[width=\textwidth,height=\textheight,keepaspectratio]{figures/deco-figure-2.jpg}
      \caption[Common problems encountered in historical data]{Common problems encountered in historical data, such as old ligatures, absence of taxon names, ambiguous symbols, shortened words and descriptive information instead of numerical (page 185 from \textcite{WoRMS:SourceID:40714})}
      \label{fig:historical-data-problems}
   \end{figure}

Several factors may turn the curation of historical documents into a
serious challenge \parencite{faulwetter_emodnet_2016,beja_chapter_2022}.
Errors from the first and second tasks, as presented in Figure 
\ref{fig:rescue-workflow} (i.e. bad quality imaging, mis-recognised characters etc.)
are propagated through the whole process. In terms of georeferencing
constraints, location names or sampling points on an old map may be provided
instead of the actual coordinates. Additionally, taxonomic constraints
(e.g. old, currently unaccepted synonyms, lack of authority associated with the
taxon names) combined with the absence of taxonomic literature or voucher
specimens (e.g. identifier number for samples of natural history/expedition
collections) require the taxonomists’ assistance. Numerical measurement units
often need to be converted to the International System of Units (SI system)
(e.g. fathoms to metres)\parencite{calder_proposal_1982,wieczorek_darwin_2012}.
Old toponyms and political boundaries that have now changed should also be
taken into consideration, as well as coordinates that now fall on land instead
of in the sea, due to the changes in the coastline. Lastly, the use of
languages other than English is quite common in old scientific publications, so
multilingual curators are required. Some of the aforementioned issues are
presented in Figure \ref{fig:historical-data-problems}. Because of these
limitations, the manual curation of data and metadata is mandatory when it
comes to historical data \parencite{faulwetter_emodnet_2016}.

Manual curation, a tedious and multistep process, requires substantial effort
for the correct interpretation of valuable historical information; however,
text mining tools appear to be promising in assisting and accelerating this
part of the curation process \parencite{alex_assisted_2008}. Text mining is the
automatic extraction of information from unstructured data
\parencite{hearst_untangling_1999,10.5555/1199003}. These mining tools build upon
standardised knowledge, vocabularies, dictionaries and perform multistep
Natural Language Processes. Named Entity Recognition (NER) is a key step in
this process for locating terms of interest in text \parencite{perera_named_2020}.
The entities of interest for biodiversity documents include: (1) taxon names,
(2) people’s names \parencite{page_text-mining_2019,groom_people_2020},
(3) environments/ habitats \parencite{pafilis_environments_2015,pafilis_extract_2017},
(4) geolocations/ localities \parencite{alex_adapting_2015,stahlman_geoparsing_2019},
(5) phenotypic traits/morphological characteristics \parencite{thessen_automated_2018}
(6) physico-chemical variables, and (7) quantities, measurement units and/or
values. Subsequent steps include the relation extraction between entities.
Multiple tools have emerged to retrieve a single or a collection of these
entities in the past few years \parencite{batista-navarro_text_2017,10.3897/BDJ.7.e28737,dimitrova_pensoft_2020,le_guillarme_taxonerd_2022}.

The work described in this document has a threefold structure: (a) the
abundance of marine historical literature digitised/available for curation
is attempted to be estimated; (b) bioinformatics tools, focusing on automating
and assisting the curation process for these documents, are compiled/reviewed.
Two categories of such curation software are described: (i) the first one
relies on web and standalone applications with Graphical User Interface (GUI)
and the second (ii) combines Command Line Interface (CLI) programming libraries
and software packages; lastly, (c) a demonstrator biodiversity data curation
workflow, named DECO (bioDivErsity data Curation programming wOrkflow \footnote{https://github.com/lab42open-team/deco}),
developed using programming tools, is presented.

% DECO METHODS
\section{Method}
\label{sec:deco-method}

    \subsection{Historical Literature Discovery}
A search was conducted on BHL to amass the historical literature on BHL
regarding marine biodiversity. Using the keywords “marine”, “ocean”, “fishery”,
“fisheries” and “sea” on the items’ titles and their subjects (the scripts,
results and documentation are available in this
repository \footnote{https://github.com/savvas-paragkamian/historical-marine-literature})
the documents available for information extraction were estimated. Subjects are
categories provided for each title and multiple subjects can be assigned to
each title. The items that were originally published before 1960 were selected,
in order to include only historical documents, according to the definition
included in the Introduction section. Furthermore, the taxon names on each page,
which were identified by BHL using the Global Names parser tool \parencite{mozzherin_gnamesgnfinder_2022},
were summarised for every document. Hence, summaries of the number of
automatically identified taxon names were calculated along with the page number
for each item. Additionally, OBIS’ historical datasets originally published
before 1960 were downloaded and analysed. This analysis provides an
approximation of the size of available marine historical literature compared to
the already rescued documents. All analysis scripts were written in GNU AWK
programming language and the visualisation scripts were written in R using
the ggplot2 library \parencite{wickham_ggplot2_2016}.


    \subsection{Historical Document Rescue Methodology}
    Data curators thoroughly read each page of a document and insert the data
into spreadsheets, mapping them to Darwin Core terms, adding metadata and
creating a standard Darwin Core Archive \footnote{https://manual.obis.org}.
This whole process, which is mostly manual,means reading the information
(e.g. the occurrence of a specific taxon and its locality) and inputting it
through typing to the corresponding cell of the data file. It is, as expected,
a time- and resource-consuming procedure. Taxon names, traits, environments
and localities can be identified as well and the transformation of these
results to database identifiers (IDs), like Life Science Identifier
(LSID) \footnote{http://www.lsid.info/} of
Aphia IDs \footnote{https://www.marinespecies.org/aphia.php?p=webservice},
Encyclopedia of Life \footnote{https://eol.org/} (EOL)
IDs \parencite{parr_encyclopedia_2014}, Marine regions gazetteer IDs,
marine species traits \footnote{https://www.marinespecies.org/traits/} among
others, can be facilitated through web applications and programming software.
The Natural Environment Research Council
\footnote{\url{https://www.bodc.ac.uk/resources/products/web_services/vocab/}}
Vocabulary Server, developed and hosted by the British Oceanographic Data
Centre \footnote{https://www.bodc.ac.uk/} was used for mapping facts and
additional measurements included in documents.

   \begin{figure}[ht]
      \centering
      \includegraphics[width=\textwidth,height=\textheight,keepaspectratio]{figures/deco-figure-3.jpg}
      \caption[curation process of marine historical documents]{The curation process of marine historical biodiversity documents: on the left column are the required steps starting from the scanned document (usually a PDF file) and ending with the data publishing step. Two approaches are presented: in the middle column are the GUI tools whereas on the right column are the CLI and/or the executable programming tools. Note that the list of given examples is non-exhaustive. Icons used from the Noun Project released under CC BY: Whale by Alina Oleynik, Fish by Asmuh, tag code by vigorn, pivot layout by paisan, Certificate by P Thanga Vignesh, web service by mynamepong}
      \label{fig:curation-process}
   \end{figure}

Tools assist curators in this process for the NER, Entity Mapping, data
structure manipulation and finally data upload steps. Curation tools can be
categorised as GUI applications (computer programs and web applications) and
CLI applications (interconnected programming tools, libraries and packages)
(Figure \ref{fig:curation-process}). As an example, multiple page documents can
be searched for taxon names in seconds, with technologies that find synonyms
and fuzzy search for the OCR transformation misspelling. The interconnection
and guidance of these steps still requires human interaction and correction.

GUI applications are standalone applications or web applications, the latter
support document upload and, once they are processed in a server, the results
are delivered back to the user \parencite{lamurias_text_2019}. CLI tools include
programming packages and libraries of any programming language in UNIX (Linux
and Mac operating systems - OS) and Windows OS. Even though programming
packages and libraries are fast and scalable they require familiarity and
expertise in CLI and programming which, on the other hand, takes effort and
time because of its initial learning curve. The CLI tools,
Application Programming Interfaces (APIs) and programming packages chosen during
this study are open-source, are in active development, can process many
documents and can be combined with other tools in some of the considered steps.


    \subsection{Case Study}
The historical document “Report on the Mollusca and Radiata of the Aegean Sea:
and on their Distribution, Considered as Bearing on Geology” by
\cite{WoRMS:SourceID:40714} and its curated dataset were used as a case study
for the tool usage description and evaluation (where applicable). More
specifically, the six page long Appendix No. 1 (pages 180-185) document has
been manually curated and published, thus serving as a golden standard. It was digitised and transcribed on
2009-04-22 by the Internet Archive \footnote{https://archive.org/details/reportofbritisha43cor}
and on 2021-09-30 it was manually curated \cite{mavraki_digitization_2021} and
published in MedOBIS \footnote{\url{ https://www.lifewatchgreece.eu/?q=content/medobis}}
\cite{arvanitidis_medobis_2006}. The rescue process resulted in a Darwin Core
Archive file with 530 occurrence records, 17 unique sampling stations and
260 taxa, covering 217 species. The effort required from the information
extraction task to data publishing was roughly 50 working days (8 hours per day)
by a single data curator.

    \subsection{Tool Usability Evaluation}
The web applications mentioned in this work were tested in November 2020 in 
two web browsers, Mozilla Firefox version 83 and Google Chrome version 87,
both on Microsoft Windows 10 and MacOS 10.14.

    \subsection{Demonstrator}

DECO was developed for the automation of biodiversity historical data
curation. Its workflow combines image processing tools for scanned historical
documents OCR with text mining technologies, Figure \ref{fig:deco-workflow}. It extracts biodiversity entities
such as taxon names, environments as described in ENVO and tissue mentions.
The extracted entities are further enriched with marine data identifiers from
public APIs (e.g. WoRMS) and presented in a structured format as well as in
report format with automated visualisation components. Furthermore, the
workflow was implemented as a Docker container to ease its installation and its
scalable application on large documents. DECO is under the GNU GPLv3 licence
(for 3rd party components separate licences apply) and is available via the
GitHub repository (\url{https://github.com/lab42open-team/deco}).

DECO is available here:\url{https://github.com/lab42open-team/deco}. Historical
marine literature analysis is here:
\href{https://github.com/savvas-paragkamian/historical-marine-literature}{https://github.com/savvas-paragkamian/historical-marine-literature}.
BHL, EMODnet Biology and OBIS data are available for download here
\href{https://about.biodiversitylibrary.org/tools-andservices/developer-and-data-tools/}{https://about.biodiversitylibrary.org/tools-andservices/developer-and-data-tools/}
and \url{https://www.emodnetbiology.eu/toolbox/en/download/occurrence/explore}
and here \url{https://obis.org/manual/access/}, respectively. The digitised
document of the “Report on the Mollusca andRadiata of the Aegean Sea, and on
their distribution, considered as bearing on Geology. 13th Meeting of the
British Association for the Advancement of Science, London, 1844” is available
here: \url{https://www.biodiversitylibrary.org/page/12920789}. The curated
dataset of the case study is available here (version 1.9 and above):
\url{http://ipt.medobis.eu/resource?r=mollusca_forbes}.

   \begin{figure}[ht]
      \centering
      \includegraphics[width=\textwidth,height=\textheight,keepaspectratio]{figures/DECO-workflow.png}
      \caption[DECO workflow]{The steps and tools used in the DECO software container.}
      \label{fig:deco-workflow}
   \end{figure}

% DECO RESULTS
\section{Results}
\label{sec:deco-results}

   \subsection{Historical Literature Discovery}

Marine literature analysis on BHL holdings revealed that there are
1,627 different digital items that contain at least 100 distinct taxa to a
maximum of 10,000 taxa, as identified automatically from the Global Names
GNfinder tool. These items cover the period from 1558 to 1960, contain 648,927
pages, written in 10 different languages, 80\% of which being English. An
absolute estimation of historical marine data is difficult to be made as
several more documents are stored locally in legacy formats.

The rescued historical marine data uploaded on OBIS are 223 datasets, published
from 1753 to 1960. Hence, the manual curated literature is much lower than the
available digitised documents. These rescued biogeographical datasets cover
46,000 species and 38 phyla that contain about 1.5 million occurrences at the
species level.

    \subsection{Bioinformatics Tools Compilation and Review}

This section describes the tools used in the curation workflow
(Figure \ref{fig:curation-process}). In each step, the main up-to-date
programming tools, web services and applications, used for the extraction of
biodiversity data, are presented. These curation tools are listed, accompanied
with features such as extracted information, input format and their interface
in Table \ref{table-tools}.


\begin{table}[]
\LARGE
\resizebox{\textwidth}{!}{%
\renewcommand{\arraystretch}{2}%
\begin{tabular}{lllll}
\hline
    \textbf{Tool} & \textbf{Curation Step} & \textbf{Input} & \textbf{Interface} & \textbf{Reference} \\
    \hline
Global Names Recognition and Discovery & NER - Taxon names & User query, Free text, PDF or image & WA, API, CLI & Pyle (2016) \\
BOM (Biodiversity Observations Miner) & \begin{tabular}[c]{@{}l@{}}OCR\\ NER - Taxon names, Biotic interactions, Traits\end{tabular} & User query, Free text, PDF & WA, API & Muñoz et al. (2019) \\
TextAnnotator & NER - Generic Annotations & User query, Free text & WA & Abrami et al. (2021) \\
Pensoft Annotator & \begin{tabular}[c]{@{}l@{}}NER - Annotation of free text with ontology terms\\ Entity Mapping\end{tabular} & User query, Free text & WA, API & Dimitrova et al. (2020) \\
Taxon Finder & NER - Taxon names & User query, Free text & WA, API &  \\
EXTRACT & NER - Taxon names, Environments and Tissue & Free text & API, CLI & Pafilis et al. (2017) \\
TaxoNerd & NER - Taxon names & Free text, PDF, png & CLI & Le Guillarme and Thuiller (2022) \\
Stanford NER & NER - People, organisation, locality & Free text & CLI & Finkel et al. (2005) \\
Clear Earth & NER - Locality, unit, value, functional traits, taxon names & Free text & CLI & Thessen et al. (2018) \\
BioStor & \begin{tabular}[c]{@{}l@{}}Literature identification, \\ NER - geolocation\end{tabular} & Taxon names and other keywords & WA & Page (2011) \\
Marine Regions Gazetteer & Entity mapping & User input & WA, API & Claus et al. (2014) \\
Edinburgh geoparser & \begin{tabular}[c]{@{}l@{}}NER - geolocation\\ Entity mapping\end{tabular} & Free text & CLI & Alex et al. (2015) \\
Ontobee & Entity mapping & User input & WA & Xiang et al. (2011) \\
WoRMS taxon match & Entity mapping & Taxon list on comma separated / spreadsheet file & WA, CLI, API & WoRMS Editorial Board (2022) \\
worrms R package & \begin{tabular}[c]{@{}l@{}}Entity mapping\\ Data transformations\end{tabular} & Taxon list: comma / tab separated file & CLI, API & Chamberlain (2020) \\
Taxize R package & \begin{tabular}[c]{@{}l@{}}Entity mapping\\ Data transformations\end{tabular} & Taxon list: comma / tab separated file & CLI, API & Chamberlain and Szöcs (2013) \\
GloBI nomer tool & \begin{tabular}[c]{@{}l@{}}Entity mapping\\ Data transformations\end{tabular} & Tab separated file & CLI & Poelen and Salim (2022) \\
 &  &  &  &  \\
OpenRefine & Data transformations, Quality control & Spreadsheet files, Comma / tab separated files, XML, RDF, JSON, SQL database & GUI app & Verborgh and Wilde (2013) \\
LifeWatch Belgium \& EMODnet Biology QC tool & Quality control & IPT or a DwC-A file & WA &  \\
LifeWatch Belgium Data Services & Quality control, Entity mapping & Comma / tab separated file, spreadsheet excel file & WA &  \\
EMODnetBiocheck & Quality Control & IPT, comma / tab separated file & CLI & De Pooter and Perez-Perez (2019) \\
GBIF Data Validator & Quality control & Comma separated, IPT or a DwC-A file & WA, CLI, API &  \\
Obistools R package & Entity Mapping, Data transformations, Quality control & Free text, comma / tab separated file & CLI & Provoost et al. (2019) \\
IPT server nodes & Quality control, Data Upload & Comma / tab separated file & GUI app & Robertson et al. (2014) \\
GoldenGate-Imagine & OCR, NER, Entity mapping & PDF & GUI app & Sautter et al. (2007) \\
DECO & OCR, NER, Entity mapping, & PDF, png, free text & CLI & This work
\end{tabular}%
}
\caption{Functions, interface and curation step of the tools tested in this work.}
\label{table-tools}
\end{table}

   \subsubsection{Named Entity Recognition}
   The Global Names Recognition and Discovery\footnote{https://gnrd.globalnames.org/}
(GNRD) tool, within Global Names Architecture
\footnote{http://globalnames.org/} (GNA), is a web application used for the
recognition of scientific names. It can use files such as PDF, images or
Microsoft Office documents and one can still input URLs or even free-form text.
It supports OCR transformation from PDF files using the tool
Tesseract\footnote{\url{https://github.com/tesseract-ocr/tesseract}} and uses
the GNfinder\footnote{https://github.com/gnames/gnfinder} discovery engine, in
order to provide the list of names. It offers an API and can be installed
locally. GNA is also used by the BHL platform to locate taxonomic names within
the pages of its collections \parencite{richard_improving_2020}.

The test performed on the \parencite{WoRMS:SourceID:40714} six-page PDF template
provided 128 unique scientific names at species level, out of the 218
identified through the manual curation
(Figure \ref{fig:gnrd-screenshot}). WoRMS Aphia IDs
\parencite{vandepitte_fishing_2015,martin_miguez_european_2019} are widely used and
included in GNRD.

The Biodiversity Observation Miner \footnote{\url{https://fgabriel1891.shinyapps.io/biodiversityobservationsminer/}}
(BOM) is a web application based on R Shiny \footnote{https://shiny.rstudio.com/},
also available on GitHub \footnote{https://github.com/fgabriel1891/BiodiversityObservationsMiner},
that allows for the semi-automated discovery of biodiversity observations (e.g.
biotic interactions, functional or behavioural traits and natural history
descriptions) associated with the species scientific names \parencite{10.3897/BDJ.7.e28737}.
It uses the GNfinder discovery engine through the R package
taxize \footnote{https://github.com/ropensci/taxize} \parencite{chamberlain_taxize_2013}.
BOM is still under development (April 2022) and an OCR processed PDF file must
be used as input. The novelty of this tool is the provision of text snippets
(Figure \ref{fig:bom-screenshot}) and the co-occurrence of words, accompanied
with their count, to inform curators for terms that appear together in the
document.

TextAnnotator\footnote{http://www.textannotator.texttechnologylab.org/},
provided by the specialised information service BIOfid\footnote{https://biofid.de/en/},
is focused on information extraction about taxon names of vascular plants,
birds, moths and butterflies, location and time mentioned in German texts
\parencite{driller_workflow_2018,driller_fast_2020}. This could be extended to
other environments, languages and taxonomic groups with the BIOfid Github
page\footnote{\url{https://github.com/FID-Biodiversity/BIOfid/tree/master/BIOfid-Dataset-NER}}
serving as the starting point. The TextAnnotator - in beta version - accepts
web pages or free text. Evidence of recent use of this tool was found in \parencite{driller_fast_2020}.

The Pensoft Annotator\footnote{https://annotator.pensoft.net/} is another beta
web application that works with ontologies \parencite{dimitrova_pensoft_2020}
(Figure \ref{fig:pensoft-annotator-screenshot}). The Pensoft Annotator has Relation
Ontology\footnote{\url{https://github.com/oborel/obo-relations}} (RO) and ENVO
built in but it is extendable to any ontology with curation modifications for
stopwords. The character limitation, however, can be expanded upon
communication with the tool’s administrators.

Taxonfinder\footnote{http://taxonfinder.org/} is a web application for the
extraction of scientific names mentioned in web pages. It features an API that
was used in BHL for large scale annotations of taxonomic names until 2019,
when it was replaced by GNfinder \parencite{richard_improving_2020}.

The most notable NER tool, with CLI, for taxon names is the Global Names Finder
(GNfinder) \parencite{pyle_towards_2016,mozzherin_gnamesgnfinder_2022} which
provides fuzzy search and is the underlying engine of most biodiversity text
mining tools. It is in active development, deeming it a reliable tool for this
work. The main command line tool is gnfinder find which returns two arrays
(metadata and names). The metadata are the language, date of the execution of
the command and total number of words. The data have one entry per identified
string which contains the matched string, the returned name and the positional
boundaries in character sequence.

In order to simultaneously extract taxa, environment and tissue mentions, the
tool EXTRACT\footnote{https://extract.jensenlab.org/} \parencite{pafilis_extract_2017}
implements the JensenLab tagger API \parencite{jensen2016one} with advanced
dictionaries SPECIES-ORGANISMS\footnote{https://species.jensenlab.org} \parencite{pafilis_species_2013},
ENVIRONMENTS\footnote{https://environments.jensenlab.org}
\parencite{pafilis_environments_2015} and TISSUES\footnote{https://tissues.jensenlab.org/About}
\parencite{palasca_tissues_2018}. It returns NCBI Taxonomy IDs \parencite{schoch2020ncbi},
ENVO terms and BRENDA IDs\footnote{https://www.brenda-enzymes.org/},
respectively to a file with 3 columns: tagged text, entity type and term ID.
TaxoNERD \parencite{le_guillarme_taxonerd_2022}, using Deep neural networks, scores
higher than other NER tools on taxon name recognition based on golden standard
corpora.

An important NER system is the Stanford NER\footnote{\url{https://nlp.stanford.edu/software/CRF-NER.html}}
\parencite{finkel_incorporating_2005} which recognises locations, persons and
organisations in text. It has a generic scope but it can also assist in the
curation of biodiversity data. The general tokenisation and normalisation
procedures developed by the NLP Stanford team are the basis of many text mining
tools. Additionally, the ClearEarth \footnote{http://github.com/ClearEarthProject/ClearEarthNLP}
project \parencite{thessen_automated_2018} can tag biotic and abiotic entities,
localities, units and values in text and is built using the ClearTK NLP
toolkit \footnote{http://cleartk.github.io/cleartk/} \parencite{bethard_cleartk_2014}.
Upon installation it downloads multiple dictionaries and takes up to six
gigabytes of space. It relies on Stanford NLP and other dependencies and
provides a Python wrapper and a CLI.

A common constraint in historical documents is the lack of coordinates from the
sampling areas, so the data curator should provide the coordinates using the
toponyms given. There are tools that enable this procedure, such as Marine
Gazetteer. BioStor-Lite map \footnote{https://biostor.org/map.php}, which
contains automated geolocation annotation of BHL documents
\parencite{page_text-mining_2019}, displays the points on the global map providing
the user the ability to search for additional documents with selected points on
the map or by drawing rectangles. The Edinburgh geoparser
\parencite{alex_adapting_2015}, a command line tool, recognises places in text and
is one of very few tools to have this functionality. The Stanford NER system
has been used as well \parencite{stahlman_geoparsing_2019} upon receiving training,
for geolocation recognition.

   \subsubsection{Entity Normalisation and Mapping}
   Mapping the information retrieved from the NER tools to different IDs is
crucial for cross-platform interoperability, ensuring a good output requires
the mapping services to be up to date.

Taxon names can have multiple IDs depending on the platform, taxonomy common
IDs, apart from the Linnaean system, are the LSID, NCBI taxonomy identifiers,
EOL identifiers etc. For marine species LSIDs based on Aphia IDs are the most
widely adopted.

Ontobee \footnote{https://www.ontobee.org/}, a web server that links
ontologies, is useful for the annotation of text to ontology IDs
\parencite{xiang_ontobee_2011}. Curators can provide text snippets to Ontobee in
order to retrieve ontology terms regarding environmental features (e.g. ENVO
IDs), functional traits (e.g. PATO
IDs\footnote{\url{https://github.com/pato-ontology}} \parencite{tan_pato-ontologypato_2022})
or other ontology terms of interest. Currently, the use of entire documents is
not recommended.

The WoRMS Taxon match\footnote{\url{http://www.marinespecies.org/aphia.php?p=match}}
tool matches the taxon list found against the World Register of Marine Species
(WoRMS) taxon LSID. Geographic regions are confirmed with the use of the
georeference tool developed for the Marine Gazetteer, users can enter the
location name in the gazetteer search field of the web interface and the
result’s output includes the region’s boundaries and the corresponding MRGID.

Most vocabulary servers provide APIs that map the different IDs. EMODnet Biology
has adopted LSIDs for marine species based on Aphia IDs from the
WoRMS vocabulary, which provides a dedicated API and an R package worrms
\parencite{chamberlain_worrms_2020}. Additionally, the R package taxize
\parencite{chamberlain_taxize_2013} provides taxon mapping capabilities across many
data sources (i.e. NCBI taxonomy, Integrated Taxonomic Information System,
Encyclopedia of Life, WoRMS). Functions like get\_eolid, get\_nbnid, get\_wormsid
can perform mapping across rows of the taxon name of the case study. In
addition, the GloBI \footnote{https://www.globalbioticinteractions.org/}
(Global Biotic Interactions) nomer tool
\footnote{https://github.com/globalbioticinteractions/nomer}
\parencite{poelen_globalbioticinteractionsnomer_2022} can also be used as it
provides entity mapping functionality via CLI \parencite{poelen_global_2014}.

   \subsubsection{Data Transformations}
   In this step, curators organise data according to the Darwin
Core\footnote{https://dwc.tdwg.org/} standard and extensions, such as extended
Measurement or Fact Extension\footnote{https://manual.obis.org}, resulting in
the creation of a Darwin Core Archive (see guidelines via the
link\footnote{\url{https://www.gbif.org/tool/81282/darwin-core-archive-assistant}})
with detailed sampling descriptors and terms based on controlled vocabularies.

When considering data transformations, curators tend to use GUI spreadsheet
applications like Microsoft Excel, Google Sheets and LibreOffice Calc.
OpenRefine \footnote{\url{https://openrefine.org/}} is a free, open source software that
handles messy data and provides their transformation in various
ways \parencite{ham_openrefine_2013}. The software’s main goal is to provide data
cleaning, fixing and analysing while also enhancing the interconnection between
different datasets \parencite{verborgh_using_2013}.

Automation can be used for this transformation through CLI tools like the R
tidyverse\footnote{https://www.tidyverse.org} package suite, Python
pandas\footnote{https://pandas.pydata.org} library and AWK programming
language\footnote{https://en.wikipedia.org/wiki/AWK}, among others. These tools
support fast and scalable tabular and text data handling, manipulations,
merging and filtering. The choice of tools depends on the users’ familiarity,
expertise and operating system.

   \subsubsection{Quality Control}
Prior to publishing the dataset it is important to perform sanity checks and
quality checks to ensure that the data comply with the Darwin Core
Standards \parencite{vandepitte_fishing_2015}. LifeWatch- EMODnetBiology QC
tool\footnote{https://rshiny.lifewatch.be/BioCheck/} allows the use of the IPT
URL or the dataset’s DwC-A files and provides a list of the quality issues
encountered, according to the EMODnet Biology criteria, as an output. It is
available as a Web Application interface, based on RShiny, and as a R
package\footnote{https://github.com/EMODnet/EMODnetBiocheck} \parencite{de_pooter_emodnetbiocheck_2019}.
LifeWatch Belgium Data Services’\footnote{\url{https://www.lifewatch.be/data-services/}}
has similar functionalities, providing a compilation of data services from
plain text and spreadsheet files as input. The GBIF Data
Validator\footnote{\url{http://gbif.org/tools/data-validator}} combines all the
above mentioned options, in terms of input, and provides a detailed summary of
issues encountered in data and metadata. Open Refine, is equipped with a few
extensions that can also check for taxon names and reconcile them.

The Obistools\footnote{https://github.com/iobis/obistools} R package \parencite{provoost_iobisobistools_2019},
the basis of the LifeWatch-EMODnetBiology QC tool, can be used to check the
coordinate boundaries and calculate centroids in cases where the exact location
is unknown. It also checks for dates’ formats and events. It has comprehensive
documentation and is in active development.

   \subsubsection{Upload to Database}
The last step of the curation process is the publication of the standards’
compliant formatted data, which is facilitated by the Integrated Publishing
Toolkit\footnote{https://www.gbif.org/ipt} (IPT) software platform \parencite{robertson_gbif_2014}.
Curators create an IPT resource entry with the aforementioned data and
associated metadata, which is then uploaded to an IPT instance, e.g. the
MedOBIS\footnote{\url{https://www.lifewatchgreece.eu/?q=content/medobis}}
Repository \parencite{arvanitidis_medobis_2006}. In the case of MedOBIS, the IPT is
subsequently harvested and made available by the central OBIS\footnote{https://manual.obis.org}
system, thus being a strong example and supporter of the ‘collect once, use
many times’ concept.

   \subsubsection{One-Stop-Shop Tools}
The main all-in-one GUI computer program is Golden-GATE-
imagine\footnote{\url{https://github.com/plazi/GoldenGATE-Imagine}}, an updated
version of GoldenGATE editor \parencite{sautter_semi-automated_2007}. This tool
supports OCR, NER and entity mapping, as described in the various steps of the
curator’s workflow by providing annotations on PDF backed up by ontologies. It
was developed by Plazi in 2015 and was last updated in 2016. Several recent
biodiversity data related publications still report the use of it although it
has not been updated since that time
\parencite{10.3897/biss.3.37078,rivera-quiroz_extracting_2019,10.3897/biss.4.59178}.
Due to its open source nature, Golden-Gate-imagine can be further developed by
any interested parties, as exemplified in GNfinder.

\begin{table}[ht]
\large
\resizebox{\textwidth}{!}{%
\begin{tabular}{lllll}
\hline
\textbf{OS} & \textbf{Source code - running time} & \textbf{Container - running time (minutes)} & \textbf{CPU} & \textbf{RAM (GB)} \\
\hline
macOS Catalina 10.15.7 & 28 minutes & Docker - 33’ & Intel(R) Core(TM) i5-4258U CPU @ 2.40GHz & 8 \\
Linux Ubuntu 18.04.5 LTS (Bionic Beaver) & 20 minutes & Docker - 27’ & Intel(R) Pentium(R) Dual-Core CPU T4200 @ 2.00GHz & 4 \\
Linux Debian server 4.9.0-8-amd64 & --- & Singularity - 20’ & Intel(R) Xeon(R) Silver 4114 CPU @ 2.20GHz & 4
\end{tabular}%
}
\caption{The platforms where the CLI workflow was tested.Please note that running time can be affected by internet speed and stability due to API calls. The workflow uses open source tools and software libraries that are distributed across the major platforms; Linux, Mac and Windows.}
\label{table-CLI}
\end{table}

   \subsection{DECO: A Biodiversity Data Curation Programming Workflow}
A CLI workflow named DECO developed to demonstrate the advantages of the CLI
approach, is available via this GitHub repository\footnote{\url{https://github.com/lab42open-team/deco}}.
DECO has connected different tools of the programming curation steps
\ref{fig:curation-process}. The execution is via a single command with a
user-provided PDF file and the output are the taxon names and records from
WoRMS API, taxonomy NCBI IDs and ENVO terms from the Environmental Ontology.
Complementary tools (i.e. Ghostscript\footnote{https://www.ghostscript.com/index.html},
jq\footnote{https://stedolan.github.io/jq/} and ImageMagick\footnote{https://imagemagick.org/index.php})
and UNIX commands are also called in a single Bash script which unifies
the workflow. In order to simplify the setup procedure of the workflow a Docker
container and a Singularity container were developed that include all the
dependencies and the code. The code and both containers have been tested on
Ubuntu, Mac and Debian server (Table \ref{table-CLI}). For a larger corpus of biodiversity
historical data the recommendation is to use the Singularity container in a
remote server or a High Performance Computing (HPC) cluster.

% DECO DISCUSSION
\section{Discussion}
\label{sec:deco-discussion}

   \subsection{Data Rescue Landscape}
   The huge difference between rescued historical marine datasets uploaded on
OBIS and the available digital items on BHL holdings reflects the challenges
faced by curators and the minimal attention paid by the wider community,
when compared to other data rescue activities (e.g. specimens, oceanographic
data, etc.). Many publications lack basic metadata such as location, date,
purpose or method of sampling. Tracing this information is limited as the data
providers may (a) have forgotten these details, (b) be retired or (c) be
deceased \parencite{michener_nongeospatial_1997}.

The project ‘Census of Marine Life’ included, among its initial objectives, the
rescue of historical marine data. Since then, there have been ongoing efforts
within the EMODnet Biology project and LifeWatch Research Infrastructure, among
others. Similarly, initiatives like Global Oceanographic Data Archaeology and
Rescue \footnote{\url{https://www.ncei.noaa.gov/products/ocean-climate-laboratory/global-oceanographic-data-archaeology-and-rescue}}
(GODAR), Oceans Past Initiative\footnote{https://oceanspast.org} (OPI) and
RECovery of Logbooks And International Marine data
\footnote{https://icoads.noaa.gov/reclaim/} (RECLAIM) \parencite{wilkinson_recovery_2011}
rescue data from ship logs for oceanographic, climate and biodiversity data.
More effort is however needed, as exemplified by museum specimen collections
and herbaria digitisation \parencite{mora_how_2011,wheeler_mapping_2012}.
The museum specimen collections and herbaria digitisation has multiple projects
and infrastructures like Distributed System of Scientific
Collections\footnote{https://www.dissco.eu} (DiSSCo), Innovation and
consolidation for large scale digitisation of natural
heritage\footnote{https://icedig.eu/} (ICEDIG), Integrated Digitized
Biocollections\footnote{https://www.idigbio.org/} (iDigBio) and Biodiversity
Community Integrated Knowledge Library (BiCIKL) \parencite{penev_biodiversity_2022}.
Similar attention is required to rescue marine biodiversity data from
historical documents that can contribute to a more complete global biodiversity
synthesis \parencite{heberling_j_mason_data_2021}.

In the last few years, an upsurge in web applications development regarding the
enhancement of biodiversity data digitisation has been observed. This is an
indication of the need for such initiatives. Advancements in the field of OCR,
text mining and information technology promise semi-automation and acceleration
of the curator’s work, which could transform the biodiversity curation field
into an -omics like, interdisciplinary field that requires complementary skills
of document handling, web technologies and text mining, to name but a few.

   \subsection{Interface Remarks}
   Web applications provide the advantage of visual aids (e.g. highlights of
NER terms), which improve the evaluation easiness and intuitiveness when using
their graphical interfaces. Emerging web development technologies like R Shiny,
Flask \footnote{https://flask.palletsprojects.com/} and
Django\footnote{https://www.djangoproject.com/} among others, have simplified
the processes of web application development. These applications are powerful
and effective in most cases but are siloed in functionality and extendability,
they also have many software dependencies which increase instability, when not
maintained in the long term.

CLI tools are a powerful way to implement scalable, reproducible and replicable
workflows: scalable because the same code can be applied to multiple files
(e.g. in this case, the various documents); reproducible and replicable because
the code can be executed multiple times and with different types of documents,
respectively. Furthermore, they usually have additional functionalities that
have not been implemented in their web application counterparts. The
difficulties regarding CLI tools’ dependency and portability are being resolved
with the rise of containerised applications which include all system
requirements and are distributed through web repositories like Docker
Hub\footnote{https://hub.docker.com/}, the downside is that without
interactiveness they are cumbersome when assisting the curation process.

   \subsection{Sustainability}
   Tool usability relies on active development and continuous support and
debugging. Sustainability is considered the main issue regarding the tools’
functionality. An example is EnvMine \parencite{tamames_envmine_2010}, a promising
2010 cutting edge tool which is no longer available. One-stop-shop purpose
software applications for domain specific usage, like GoldenGate, are very
helpful but require more effort to stay up to date with the integrated tools.
Other tools are often out of date, as active development and contribution to
reporting issues in open-source repositories, such as Github, is lacking, thus
becoming obsolete and unsupported in only a few years from their first release.

   \subsection{Curation Step-Wise Remarks}
   The curators’ role is invaluable in the data rescue process, as their domain
specific expertise is far from becoming entirely automated. There are plenty of
available digitised historical documents that are not curated in web libraries,
such as BHL, the Belgian Marine
Bibliography\footnote{\url{https://www.vliz.be/en/belgian-marine-bibliography}},
Web of Science \footnote{https://www.webofknowledge.com}, Wiley Online
Library \footnote{https://onlinelibrary.wiley.com} and Taylor \& Francis
Online \footnote{https://www.tandfonline.com/}, among others \parencite{kearney_its_2019}.
BHL provides “OCRed” documents and there are plenty of other tools that can
tackle this process which are reviewed elsewhere \parencite{10.3897/rio.6.e58030},
however OCR is a crucial limiting step in the workflows, in terms of the
information transformed from image to text, because there are many cases that
lead to mispelled or lost text; especially the case with handwritten text and
poor quality images \parencite{lyal_digitising_2016}.

Information extraction can be performed both on a small and a large scale.
Named Entities are what most text mining tools extract. Taxon names recognition
is the main function of the majority of the current tools and has matured
significantly over the past decade, especially through the integration of
multiple platforms with the GNA \parencite{pyle_towards_2016}. Environments and
geolocations have strong background data, Environment Ontology terms (retrieved
with the EXTRACT tool) and GeoNames\footnote{http://www.geonames.org}/
Marineregions gazetteers, respectively. Geolocation mining, in particular, has
not been adapted in biodiversity curation but there are generic tools
(e.g. mordecai\footnote{https://github.com/openeventdata/mordecai} -
\parencite{halterman2017mordecai} that are able to be trained with gazetteers to
extract approximate localities from text. Also extraction of sample location
from maps is possible by first geolocating the historic map in Geographic
Information Systems \parencite{jenny_studying_2011} and then using computer vision
to find the locations’ coordinates \parencite{10.1145/2557423}. Characteristics of
taxa, i.e. phenotypic traits, associated physico-chemical variables, units and
the use of semantics to describe relations, are still under standardisation
\parencite{thessen_transforming_2020} and NER prototypes have been made with
ClearEarth and Pensoft Annotator, for example.

Entity mapping has also seen an important development because there are many
open public APIs for vocabularies like those used in WoRMS, and Marine Regions
and aggregators such as GBIF and OBIS, among others, and in some cases software
packages (mostly in the R programming language). The task for Publication has
its dedicated applications and tools with the CLI tools being able to perform
quality control and deliver a preferred on-the-fly format.

   \subsection{DECO}
   The CLI scientific workflow assembled in this paper, DECO, is a demonstration
of EMODnet Biology’s vision for biodiversity data rescue using programming
tools. To the best of our knowledge, this is the first task-driven CLI that
brings together state-of-the-art image processing, OCR tools, text mining
technologies and Web APIs, in order to assist curators. By using programming
interface and Command Line Tools the workflow is scalable, customisable and
modular, meaning that more tools can be incorporated to, e.g. include the
entities mentioned in the previous section. It is fast, may be used on a
personal computer, and is available as a Docker and a Singularity container.
The containerised versions of the workflow simplify the installation procedure
and increase its stability, scalability and portability because they include
all the necessary dependencies. This CLI scientific workflow promises a faster
and high throughput processing that could be applied to any type of data, not
only historical, thus contributing to the overall digitisation of biodiversity
knowledge.

   \subsection{Future Outlook}
   Progress has been made in the advancement of the historical data rescue
process, from digitisation platforms to standards, services and publication
\parencite{beja_chapter_2022}. To bridge the gap between tools and curators
requires effort on both ends; namely the data curators and the tool developers.
It is recommended that curators are trained in basic programming skills from
which they and the historical data rescue process in general would benefit in
the long term \parencite{10.12688/f1000research.25413.2}. Regarding software
development, important features are highlighted, like the use of multiple
ontologies in Pensoft Annotator. This is a direction which should be further
expanded to all entities of interest. Multidisciplinary cooperation between
scientific communities and partners of tools, ontologies and databases is
needed to accomplish this task \parencite{bowker_biodiversity_2000}. An important
example was set by GNA which advanced scientific names recognition
significantly. In addition, the co-occurrence feature, that was present in
Biodiversity Observation Miner, once expanded to other entities and associated
with a scoring scheme will be a state-of-the-art text mining application that
goes beyond NER to actually infer relations. The rise of deep neural networks
is promising as well in all different tasks of Information Extraction, as seen
in TaxoNERD \parencite{le_guillarme_taxonerd_2022}. Lastly, the community is
pushing to Semantic Publishing, FAIR completeness of new data and new taxonomic
publishing guidelines to eliminate the need of text mining and curation in
current publications \parencite{penev_openbiodiv_2019,fawcett_digital_2022}.

The implementation of crowdsourced curation within citizen science projects for
the historical biodiversity data is encouraged \parencite{clavero_mine_2014,arnaboldi_text_2020,10.12688/f1000research.25413.2}.
Practices like this are already in place in the digitisation of natural history
collections and have been proved fruitful \parencite{ellwood_accelerating_2015}.
EMODnet Biology’s Phase IV will launch such a citizen science project for
historical documents curation during the second half of 2022. Approaches from
other fields of science that handle historical and old data, such as history,
linguistics, archaeology would provide useful insights for the text mining of
historical biodiversity data.


   \subsection{Concluding Remarks}
   Historical marine biodiversity data provide important and significant
snapshots of the past that can help understand the current status of ocean
ecosystems and predict future trends in face of the climate crisis. There is a
wealth of historical documents that have been digitised yet, most of their data
have not been rescued or published in online systems. To accelerate the tedious
data rescue process it is essential that more curators become engaged, and
tools for Information Extraction and Publication get improved to satisfy their
needs. Tools like DECO and GoldenGATE demonstrate possible future directions
for one-stop-shop applications for command line and graphical user interfaces,
respectively. Research Infrastructures can play a pivotal role towards this
goal. Last but not least, the community and funding bodies should prioritise
the data rescue of these invaluable documents before their decay and inevitable
loss.





   \begin{figure}[ht]
      \centering
      \includegraphics[width=\textwidth,height=\textheight,keepaspectratio]{figures/deco-figure-S1.jpg}
      \caption[GNRD taxon names identification]{Screenshot of the web application GNRD identifying taxon names.}
      \label{fig:gnrd-screenshot}
   \end{figure}

   \begin{figure}[ht]
      \centering
      \includegraphics[width=\textwidth,height=\textheight,keepaspectratio]{figures/deco-figure-S2.jpg}
      \caption[BOM performing NER]{Screenshot of the web application BOM performing NER. It provides taxon names, text snippets and term co-occurrences.}
      \label{fig:bom-screenshot}
   \end{figure}
   
   \begin{figure}[ht]
      \centering
      \includegraphics[width=\textwidth,height=\textheight,keepaspectratio]{figures/deco-figure-S3.jpg}
      \caption[Pensoft Annotator performing NER]{Screenshot of the web application Pensoft Annotator performing NER.}
      \label{fig:pensoft-annotator-screenshot}
   \end{figure}

% --------------------------------------------------
% 
% This chapter is for Cretan endemic Arthropods
% 
% --------------------------------------------------


\chapter{Cretan Endemic Arthropods: distribution, threats and hotspots}
\label{cha:arthropods}

% INTRODUCTION
\section{Introduction}
\label{sec:arthropods-intro}

Functioning of arthropods has been investigating since the seventies \parencite{rosswall1997energetical}.

Importance of arthropods in functioning

Crete data since 19th century. 

To being able to study first they have to be conserved.


%% need to modify
Hotspot definitions vary from quantitative methods to experts opinions and curation.
In quantitative methods, grid size and shape influences the determination of
the areas of interest such as hotspots and key biodiversity areas \parencite{hurlbert2007species,nhancale2011the-influence}.
Choosing the size of the grid is not trivial \parencite{mo2019influences} and is dependent
on the conservation goals \parencite{margules2000systematic}. In the past decade,
there have been major advances for conservation standards, guidelines,
frameworks and tools available to be put into action \parencite{bongaarts2019ipbes}.


%%%%% old
Numbering approximately 7,000 species [extrapolated from Fauna Europaea
\parencite{jong2014fauna} and \textcite{legakis2018}], the Arthropods of Crete, Greece,
have been studied for almost two centuries \parencite{Anastasiou2018Tenebrionid}.
Only 135 of these species (1.9\%) have been assessed in IUCN Red List as of
this publication, making Arthropods the third most evaluated group of the
island, behind vascular plants (291) and land mollusks (165).

Habitat loss and degradation occurs throughout Crete as a result of urban,
agricultural and touristic development. This is a major issue since habitat
loss is a major threat in Europe for many Arthropod groups,
e.g. Butterflies \parencite{VanSwaaycommission2010european}, Bees \parencite{nieto2014},
Orthoptera \parencite{hochkirch2016} and Saproxylic Beetles \parencite{Calix_2018}.
Climate change is predicted to induce scarcer yet more intense precipitation,
increase of drought locally \parencite{koutroulis2011spatiotemporal} and shrinkage as well as
possible shifts to the rainfall period \parencite{koutroulis2013impact}. Groups
associated with fresh water could be deeply impacted from the locally increased
drought and the increase in need of water for irrigation and domestic use,
e.g. Odonata \parencite{kalkman2010}, which has become harsher due to the
increase of agriculture and land use \parencite{tzanakakis2020challenges}. Stock raising
(sheep and goats) has always been an important aspect of Cretan life and
economy \parencite{rackham1996the-making}. Overgrazing impacts severely soil erosion,
soil moisture and vegetation \parencite{kosmas2015exploring,orestis2015exploring}. All the
above contribute to a worrying trend for Crete, i.e. the higher percentage of
Threatened endemic Arthropods when compared with the respective European
groups.

The bibliography and the NHMC collection was curated to assemble taxa
occurrences. The bibliography used
contains both historical and contemporary published material.
Using over 100 publications (as of 2020) and 733 NHMC sampling events
(\href{https://doi.org/10.5281/zenodo.10635645}{data online availability}), a dataset of 343 taxa (species and subspecies) was assembled,
with 4,924 records across 1,569 distinct sites of Crete. The taxa
are distributed to eleven orders, with Coleoptera having the most taxa (206)
and Chilopoda and Scorpiones the least (two).
Based on the criteria described in \textcite{bolanakis2024} the data resulted in the
following orders:
Araneae, Chilopoda, Coleoptera, Diplopoda, Heteroptera, Hymenoptera (Chrysididae, Formicidae, Symphyta), Lepidoptera (Geometridae), Odonata, Orthoptera, Scorpiones and Trichoptera.


In this study I aim to a) identify cretan arthropods available information
b) compare with the endemics of Crete
c) their relation with the anthropogenic pressures in these sites.
To do so, I assembled the accumulated knowledge of 
online databases.


% METHODS
\section{Methods}
\label{sec:arthropods-method}
   


    \subsection{Species occurrences}
    \label{subsec:arthropods-data-assemblage}

The coordinate reference system we used for all location data is WGS84 - EPSG:4326.

From here on, we refer to both species
and subspecies as “taxa”.
    
All species 
from IUCN that are in Crete were downloaded from the website \parencite{iucn2024}. IUCN provides the 
interactive maps selection functionality. 

Criterion B, i.e. on the Extent of Occurrence (EOO) and
Area of Occupancy (AOO).

Criterion B is the widely used for Arthropods \parencite{cardoso2011adapting,carpaneto2015a-red-list},
because the majority of Arthropods taxa are missing neccessary information of the other criteria (A, C, D and E).
Criterion B could overestimate the danger of Arthropods \parencite{cardoso2011adapting},
which should always be taken into consideration.

    
    \subsection{Grids and hotspots}
    \label{subsec:arthropods-ehs-kbas}

In order to avoid biases concerning the grid cell size, the same
pipeline was tested with cells of different size (4 x 4, 8 x 8 and 10 x 10 km).
For the subsequent analyses we opted for the 10 x 10 km grid (see section \ref{subsec:arthropods-grids})
which is also the EEA reference grid, the standard for the reporting format
(Groups of Experts, 2017) of the Resolution No. 8 (2012) of the Standing Committee
to the Bern Convention on the Emerald Network of Areas of Special Conservation Interest (ASCI).
Moreover, the EHs of the various cell sizes are aggregated in the same areas.
We made the same treatment for each of the selected groups separately.
We redefined EHs as the 10\% of the grid cells with max overlap of the orders
to check for biases towards more speciose orders (e.g. Coleoptera).

    \subsection{EOO and AOO}
    \label{subsec:arthropods-eoo-aoo}



    \subsection{Spatial overlaps}
    \label{subsec:arthropods-spatial}
To examine the human pressure (change in land use, agriculture), I used the
Historic Land Dynamics Assessment (HILDA+) dataset \parencite{winkler2021global} to
estimate the change of land use the from 1998 to 2018. Furthermore, I examined
the overlap of the AOO.


    \subsection{Code}
    \label{subsec:arthropods-tools}

    We performed the analyses using the R Statistical Software \parencite{rcoreteam},
the visualization using the ggplot2 R package \parencite{wickham_ggplot2_2016}. The figures
created are colored using the colorblind-friendly 'Okabe-Ito' palette \parencite{ichihara2008color}.
We calculated EOO and AOO using the ConR R package \parencite{dauby2017conr:} and PACA
using custom scripts. For the spatial data handling, transformations and
geometry we used the sf v1.0-14 \parencite{pebesma2018simple} and terra v1.7-55 R packages \parencite{hijmans2024terra}.
Adaptive grid is created using the quadtree R package \parencite{friend2023quadtree}.
Jaccard similarity was calculated with the vegan 2.6-4 R package \parencite{oksanen2024vegan}.
All scripts are reproducible by design and available in this 
\href{https://github.com/savvas-paragkamian/arthropoda_assessment_crete}{GitHub repository}.

Data, scripts and results of the analysis are available and documented \href{https://github.com/savvas-paragkamian/arthropods_assessment_crete}{here} 

% RESULTS
\section{Results}
\label{sec:arthropods-results}

   \begin{figure}[htp!]
      \centering
      \includegraphics[width=\textwidth,height=\textheight,keepaspectratio]{figures/arthropods-FigS1.png}
      \caption[Comparisons of Threatened Endemics in Crete, Greece, Europe and the World]{Comparisons of Threatened Endemics in Crete, Greece, Europe and the World for seven Arthropod groups (top to bottom, data aggregated from the IUCN web resource). Values are absolute (proportion in parentheses). Also, note that there are 3 threatened species of Scorpiones in the World Red List column and 5 not threatened species in the Trichoptera column.}
      \label{fig:arthropods-figS1}
   \end{figure}

   \begin{figure}[htp!]
      \centering
      \includegraphics[width=\textwidth,height=\textheight,keepaspectratio]{figures/arthropods-fig_crete_sampling_intensity_order.png}
      \caption[Sampling intensity]{Sampling intensity in Crete (i.e number of samples) in each 10 km grid for all samples and for each arthropod order.}
      \label{fig:arthropods-sampling-intesity}
   \end{figure}

    \subsection{Grid cell size}
    \label{subsec:arthropods-grids}
The grid cell size 10 x 10 km is the most suitable for our study since our
dataset - being compiled from numerous different sources and sampling efforts -
is rather coarse and inhomogenous for a smaller cell size.
The unique taxa of the EHs of each grid is distributed as follows: 10 km=283,
8 km=278, 4 km=293, adaptive cells=267, with the 4km grid covering most endemic
species. The 4 km grid mostly highlighted areas known for their tourist/recreational activities,
indicating that it is more sensitive to sampling intensity.
Focusing on sampling we applied the adaptive grid size with quadtrees resulting
in 157 grids with 8 km length, 38 with 4 km and 74 with 2 km (Figure \ref{fig:arthropods-figS5}).
This indicates the preference of larger cells for the majority of our dataset
even though a small percent of regions has higher density of sampling.
The highest overlap among all grids is between the 10 km and 8 km reaching
57\% (Table \ref{table:arthropods-tableS2}). Finally, the 10 km grid has more taxa
per cell and is a reference grid system.
Based on our analysis and interoperability and reproducibility aims we choose
the 10 km EEA reference grid for the EHs and candidate KBAs inference.
Nevertheless, we also performed the WEGE analysis for KBAs using the adaptive
grid, yielding practically the same areas as the 10km grid minus Zakros (Figure \ref{fig:arthropods-fig5}).
The same pipeline can not be done with EHs for they require a fixed cell size.


   \begin{figure}[htp!]
      \centering
      \includegraphics[width=\textwidth,height=\textheight,keepaspectratio]{figures/arthropods_crete_multiple_grids_hotspots.png}
      \caption[Comparisons of different grid sizes]{Comparisons of different grid sizes to identify hotspots.}
      \label{fig:arthropods-different-hotposts}
   \end{figure}

   \begin{figure}[htp!]
      \centering
      \includegraphics[width=\textwidth,height=\textheight,keepaspectratio]{figures/arthropods-fig_grid_stat.png}
      \caption[Comparisons of proportions of endemics across grid sizes hotspots]{Comparisons of different grid sizes. A. The proportion of endemics. B. The Jaccard similarity across hotspots with the same grid size.}
      \label{fig:arthropods-different-hotposts-stat}
   \end{figure}


   \begin{figure}[htp!]
      \centering
      \includegraphics[width=\textwidth,height=\textheight,keepaspectratio]{figures/arthropods-Fig_quads.png }
      \caption[Adaptive grid size based on sampling using quadtrees]{Adaptive grid size based on sampling using quadtrees.}
      \label{fig:arthropods-figS5}
   \end{figure}

\begin{table}[]
    \caption{Different grid size hotspots overlap. The top 10\% of cells with most species are considered as hotspots. All units are in km\textsuperscript{2}.}
\begin{tabular}{cccccc}
\textbf{Grids}              & \textbf{1 km\textsuperscript{2}} & \textbf{4 km\textsuperscript{2}} & \textbf{8 km\textsuperscript{2}} & \textbf{10 km\textsuperscript{2}} & \textbf{Adaptive cell size} \\
\textbf{1 km\textsuperscript{2}}              & \textbf{126}   & 72             & 55             & 53              & 41                          \\
\textbf{4 km\textsuperscript{2}}              & -              & \textbf{918}   & 503            & 497             & 349                         \\
\textbf{8 km\textsuperscript{2}}              & -              & -              & \textbf{1317}  & 800             & 691                         \\
\textbf{10 km\textsuperscript{2}}             & -              & -              & -              & \textbf{1400}   & 646                         \\
\textbf{Adaptive cell size} & -              & -              & -              & -               & \textbf{1312}              
\end{tabular}
\label{table:arthropods-tableS2}
\end{table}


   \begin{figure}[htp!]
      \centering
      \includegraphics[width=\textwidth,height=\textheight,keepaspectratio]{figures/arthropods_aoo-eoo_order.png}
      \caption[AOO, EOO relationship per order]{The relationship of AOO and EOO of species of different orders. Each dot is a species. }
      \label{fig:arthropods-eoo-aoo}
   \end{figure}


   \begin{figure}[ht]
      \centering
      \includegraphics[width=\textwidth,height=\textheight,keepaspectratio]{figures/arthropods-fig_crete-hotspots_order.png}
      \caption[Hotspots of every order]{Endemicity hotspots across orders.}
      \label{fig:arthropods-hotspots-order}
   \end{figure}

    \subsection{Human Intervention}
    \label{subsec:arthropods-human-intervention}


\begin{table}[]
    \caption{The land use transitions in the 20 year period (1998-2018) from the HILDA+ dataset.}
\resizebox{\textwidth}{!}{%
\begin{tabular}{llll}
HiLDA+ transitions                             & Crete (km\textsuperscript{2}) & Natura2000 (km\textsuperscript{2}) & EHs (km\textsuperscript{2}) \\
urban (stable)                                 & 326         & 46               & 14        \\
urban to pasture/rangeland                     & 3           & NA               & NA        \\
cropland to urban                              & 42          & 1                & NA        \\
cropland (stable)                              & 1314        & 186              & 71        \\
cropland to pasture/rangeland                  & 565         & 74               & 32        \\
cropland to forest                             & 178         & 40               & 20        \\
cropland to unmanaged grass/shrubland          & 3           & 1                & NA        \\
cropland to sparse/no vegetation               & 3           & 3                & 3         \\
pasture/rangeland to urban                     & 13          & NA               & NA        \\
pasture/rangeland to cropland                  & 82          & 15               & 3         \\
pasture/rangeland (stable)                     & 5124        & 2594             & 1198      \\
pasture/rangeland to forest                    & 11          & 6                & 3         \\
pasture/rangeland to unmanaged grass/shrubland & 137         & 73               & 38        \\
pasture/rangeland to sparse/no vegetation      & 2           & 2                & 2         \\
forest to pasture/rangeland                    & 34          & 33               & 25        \\
forest (stable)                                & 364         & 182              & 174       \\
forest to unmanaged grass/shrubland            & 2           & 1                & 2         \\
unmanaged grass/shrubland (stable)             & 1           & NA               & NA        \\
sparse/no vegetation (stable)                  & 2           & 2                & 2         \\
water                                          & 165         & 78               & 5         
\end{tabular}%
}
\label{table:arthropods-tableS4}
\end{table}


% DISCUSSION
\section{Discussion}
\label{sec:arthropods-discussion}

    \subsection{Endemicity Hotspots}
    \label{subsec:arthropods-Endemicity-Hotspots}

Mountains host a great amount of Earth’s biodiversity, being a main driver for
the birth of species \parencite{antonelli2018geological,noroozi2018hotspots,rahbek2019building,Rahbek2019}
and a crucial frontier for their fate \parencite{steinbauer2018accelerated,urban2018escalator}.
Crete is not an exception to this trend \parencite{kougioumoutzis2020plant,trigas2013elevational}.
Our results conform to that, since the EHs are gathered primarily in the major
Cretan mountains. Lefka Ori and Dikti are the sites with the most
EHs, in agreement with studies focused on vascular plants \parencite{dimitrakopoulos2004questioning,kougioumoutzis2020plant}.
\textcite{sfenthourakis2001hotspots}, employing invertebrate groups, also recovered these mountains as EHs.

Islands are biodiversity sanctuaries \parencite{whittaker2007island}, and
so are mountains \parencite{rahbek2019humboldts}. Our work advocates for approaches that
treat islands and mountains under a holistic perspective. The combination of
the two provides a complex biogeographical interplay governing the forces of
speciation, preservation and extinction of biodiversity \parencite{steinbauer2016topography-driven}.
This synergistic effect of mountains-islands has also been recovered in other
areas such as the Balearic islands \parencite{guardiola2023are-mediterranean}.

    \subsection{Species assessment}
    \label{subsec:arthropods-species-assessment-disc}

Arthropods with wider ranges that are not assessed as Threatened under
criterion B, are not necessarily Least Concern and should not be neglected.
Arthropod communities can be affected by the reduction of the abundance of
common and abundant species that offer important functions to the biocommunity.
Wide range does not guarantee high abundance (even though this is true for many
taxa) and even common species can be threatened \parencite{habel2018vanishing,klink2023disproportionate}.

The inclusion of Arthropod taxa in protected areas is often insufficient, with
Arthropods experiencing declines inside the protected areas \parencite{borges2005ranking,chowdhury2023protected,harry2019protected,rada2019protected}.
In fact, even when certain Arthropod groups are adequately included in N2K,
there are gaps and omissions \parencite{sanchez-fernandez2008are-the-endemic,verovnik2011is-the-natura}.
At a global level 75\% of Insects are not sufficiently covered by protected
areas \parencite{chowdhury2023three-quarters}. Crete stands in an intermediate position,
following the general trend of Greece’s N2K adequacy, being the best covered
area at a national level \parencite{kougioumoutzis2021plant,spiliopoulou2021the-natura}.
However, there are some clear gaps regarding certain taxa, encouraging more
locally focused conservation policies complementary to N2K. For example actions
need to be taken for KBAs that fall outside N2K like Kritsa and Zakros.

Biases towards Arthropods cause their poor coverage by protected
areas \parencite{chowdhury2023protected,damen2013protected,delso2021protected}. These
biases derive from geography, size, color and charisma \parencite{cardoso2012habitats,mammola2020towards,wang2021out-of-sight},
and even from political/economic reasons \parencite{dias-silva2021protected}. For example,
the strongest driver for a conservation program funding within the European
Union is the online popularity \parencite{mammola2020towards}. The unpopularity of
Arthropods has begun to change \parencite{wagner2021insect}, especially through citizen
science, which is a trend we should build on to properly conserve the Arthropods.

    \subsection{Human Intervention in Arthropods’ EHs}
    \label{subsec:arthropods-human-intervention-ehs}
Human activities account for almost 20\% of the EHs. The primary human activity
in the EHs is agriculture (~19.6\%). Agricultural intensification is one of the
most important drivers of Arthropods’ decline \parencite{bruhl2019biodiversity,habel2019agricultural,raven2021agricultural}.
Moreover, threats associated with agriculture are the number one threat for
Insect species inside protected areas in Europe \parencite{chowdhury2023protected}.
Nevertheless, regarding change in land use, there is a somewhat equal
transition trend from cropland to forest and vice versa inside EHs and KBAs
(Table \ref{table:arthropods-tableS4}). This means that while some sites are being
degraded others may recover. More research within EHs and KBAs is essential in
order to quantify the impact (negative or positive) of these transitions to the
endemic Arthropods. A vast amount of cropland has been transformed to pasture
lands (Table \ref{table:arthropods-tableS4}) which requires further examination,
since grazing has both positive [eg. on Gnaphosidae (Spiders) communities \parencite{kaltsas2019overgrazed}]
and negative effects [e.g. Carabidae (Coleoptera) \parencite{kaltsas2013ground}].
The reduction of croplands could be interpreted under the general trend of
urbanization (Table \ref{table:arthropods-tableS4}), which nevertheless occurs
outside EHs and KBAs, but a shift towards montane areas especially under new
forms of tourism could deeply impact the sites of conservation importance.



% --------------------------------------------------
% 
% This chapter is for Crete system ecology
% 
% --------------------------------------------------


%\chapter{Towards a Cretan soil biodiversity data model}
\chapter{Exploring the soil microbiome coupling with Crete island data cube}
\label{cha:crete-soil}

%\textbf{Citation:} \\ 


% ISD ABSTRACT
%\section{Abstract}
%    Microbes are known for their versatility, abundance
%    and influence on soil ecosystem functioning.
%    A synthesized knowledge base of microbial biodiversity, in terms of
%    ecological and remote-sensing data remains a major challenge.
%    Many worldwide studies have been published regarding soil
%    microbiome ecosystems, though there are still many blind spots.
%    Islands can be important case studies for this integration for more resolute and dense samplings.
%    Here, we utilize the Island Sampling Day Crete 2016 microbial 16S rRNA gene
%    amplicon data, integrated with soil and remote
%    sensing data, to decipher the drivers of ecosystem function of the island.
%    The Island Sampling Day Crete 2016 project has collected 144 topsoil samples
%    from 72 sites, capturing a lot of this diversity, accompanied by FAIR
%    (Findable, Accessible, Interoperable and Reproducible) data by design. 
%    Cretan macroecology has been studied for centuries for its diverse  and endemic
%    fauna and flora.
%    In addition, Crete has been considered as a miniature continent with high contrasts in
%    vegetation cover, elevation, climatic conditions. 
%    We show that, higher altitudes in Crete found to
%    be inhabited by a more diverse number of microorganisms, a pattern commonly
%    seen in several faunistic groups, such as arthropods.
%    The integration of the spatial data with state of the art methods enabled warning signals
%    in pristine and grazing ecosystems.
%    These results along with the
%    climatic and desertification index influences on the soil microbiome of Crete,
%    provide the basis to identify major drivers of biodiversity, to evaluate hotspots
%    and contribute to foreknowledge of threatened ecosystems.
%
\section{Introduction}\label{intro_integration}

Soil ecosystems are the cornerstone of terrestrial habitats, biodiversity and henceforth human activities.
Soils are characterised by multiple properties; chemical, physical and biological that 
form complex interdependent interactions. Biodiversity of soils covers
all forms of life, fauna, flora, bacteria, archaea, fungi, viruses. 
Bacteria and archaea are considered major drivers for the functionality of soil.
They influence and are influenced by their environment and their community structure 
defies their macroscopic functionality \parencite{Bahram2018}.
Global soil microbiome studies have been employed to decipher soil microbiome
compositions \parencite{thompson2017a-communal, Delgado-Baquerizo-atlas, Labouyrie2023},
functions \parencite{Bahram2018} and biogeography \parencite{Martiny2006, guerra2020Blind}.
These results showed the remarkable diversity in soils yet there are blind spots \parencite{guerra2020Blind}
and these sampling are sparse when considering samples per area density. One of most resolute
study is by \parencite{Karimi2020} which exemplified the 
vast complexity of soil bacterial communities and the requirement of
dense samplings and isolated systems \parencite{Dini-Andreote2021}.

Data and metadata of these samplings are stored in different databases, yet 
great effort from these distinct communities have led to establishing standards
to enable FAIR data \parencite{wilkinson2016the-fair}. For amplicon sequences the Genome Standards
Consortium \parencite{Field2011} has established the MIMARKS \parencite{yilmaz2011minimum}
standards among others, and has been a advocate for open and unrestricted data \parencite{Amann2019}.
Examples of rich metadata and platforms of hosting open data and digital soil maps are the 
European Soil Data Centre \parencite{Panagos2022} and the World Soil Information
Service (WoSIS) of the ISRIC \parencite{Batjes2024}. Apart from samplings,
spatial data are openly available terrestrial ecosystems. Soil compositions in Greece 
have been mapped but there are still some gaps \parencite{yassoglou2017soils}.
Climatic data, land cover, desertification risk, aridity, soil type, normalized
vegetation index, bedrock geological formations. From the bacterial point of view, 
there curated databases that classify in some baseline functionality. 

Crete is a continental forarc island \parencite{ali2016}, fifth largest island of the Mediterranean (8350 km\textsuperscript{2}),
and a Mediterranean biodiversity hotspot \parencite{myers2000biodiversity}.
The island of Crete has been studied since the classical times for its'
fauna \parencite{Sidiropoulos_Polymeni_Legakis_2017,Anastasiou2018Tenebrionid}, flora \parencite{Krimbas_2005} and ecosystems \parencite{Grove1993}.
Crete is home to the only endemic mammal of Greece, the Cretan shrew (\textit{Crocidura zimmermanni}),
more than 350 endemic arthropods \parencite{bolanakis2024} and 183 endemic plants \parencite{Kougioumoutzis2020}
among them a tree \textit{Zelkova abelicea}. Multifaceted factors have shaped the
biodiversity of the island, for example the sharp elevation gradient \parencite{trigas2013elevational, FAZAN2017},
the complex evolutionary history \parencite{POULAKAKIS2002} and the human - nature
interactions over thousands of years \parencite{Vogiatzakis2008_med, Sfenthourakis2017}.
The major threats of human activities are becoming apparent in the island's ecosystems,
like desertification \parencite{KARAMESOUTI2018266}, soil erosion \parencite{PANAGOS2014147, su14052738}, intensive grazing \parencite{JouffroyBapicot2016},
climate change \parencite{Kougioumoutzis2020,Vogiatzakis2016} and habitat loss \parencite{ISPIKOUDIS1993259}.
Yet the topsoil microbial diversity of Crete has been unexplored. Hence, using Crete as model can 
assist define its soilscape \parencite{LAGACHERIE2001105}.

Disentangling the soil ecosystem functioning requires an holistic and 
multidisciplinary approach \parencite{vogel2022}. The integration of the aforementioned
data is needed to understand the biogeochemical cycles along with biodiversity interactomes 
that have been characterised as a driver of community composition soil
functioning \parencite{GUSEVA2022108604}.
Regarding the latter there are still big challenges to infer actual microbial
interactions remain \parencite{Faust2021}. All this work is needed in order to meet
UN and EU soil goals for the 2030 and 2050 for healthy soils \parencite{LAL2021e00398}.

In this study we ask: what the differences in the microbiome communities in different land use types?
Are there any climatic, geological, elevational, aridity or functional factors that affected the differences?
How the interactome changes over different land use types and climate?
Are there any distinctions between arid regions of Crete?
To address these questions, we integrate multiple types of data and methods to decipher hidden 
signals of the Island Sampling Day soil microbiome data. In total, 144 samples from
72 sites (2 samples per site) of Crete are used with their metadata.
Warning signals are also identified for various types of ecosystems.


\section{Materials and Methods}\label{integration_methods}

\begin{figure}[t] 
    \centering\includegraphics[width=\columnwidth]{crete_integration_soil_microbiome}
    \caption{Workflow of this study. Data integration of ISD data with multiple types of spatial data. Then, a threefold analysis of function annotations, network analysis and differential abundance. All these data and methods are used to focus on specific taxa, ecosystems and threats.}
    \label{fig:workflow}
\end{figure}

\subsection{Island Sampling Day: Crete}\label{isd_data}
The sequences and metadata were downloaded and infered as described in Chapter \ref{isd_methods}.
The results from DADA2, i.e ASVs, were used for the subsequent analyses.

\subsection{Crete data cube}\label{spatial_data}

The compilation of Crete data cube has multiple spatial data layers of global,
European, Greek or Cretan scale. 
The Copernicus CORINE Land Cover has 3 layers resolution that classify the land
use and cover in shapefile format \parencite{CLC2023}. 
WorldClim 2.0 contains global climatic data for 12 variables, e.g annual mean
temperature and annual precipitation \parencite{Fick2017}.
The Environmentally Sensitive Areas Index to desertification (ESAI) 
of Greece dataset \parencite{KARAMESOUTI2018266} were utilised. Additionally the 
Global Aridity Index and Potential Evapotranspiration Database \parencite{zomer2022version} was include.
Geological formations shapefiles were downloaded from the geoportal of
Decentralized Administration of Crete, which were developed by
Crinno-Emeric Group project\footnote{\url{https://geoportal.apdkritis.gov.gr/gis/apps/storymaps/stories/19690f65abbe4e8ab0141b2fe7261a8c}}.
The Harmonised World Soil database v2 was incorporated for the soil mapping units and 
the soil taxonomic classification \parencite{fao2023}.
Handling and analysis of these data was done with the sf and terra R packages \parencite{Pebesma2023}.

\subsection{Integrative Analysis and Annotations}\label{int_analysis}
The network inference was facilitated with FlashWeave 0.19.2 \parencite{Tackmann2019}.
To use FlashWeave we reduced 
the abundance table of the ASVs to keep the ones at the genus or species level.
%In addition, we filtered these taxa that appeared in SOSO samples and had more than 
%SOSO mean relative abundance.
The subsequent network analyses were carried out
with the igraph R package \parencite{Csardi2006}.

For taxa function annotation we used the manually curated FAPROTAX database and python script \parencite{loucaDecouplingFunctionTaxonomy2016}.
PREGO \parencite{microorganisms10020293} was used to manually find the environments of specific bacteria.
Whereas for the differential abundance analysis with ANCOM-BC2 R package \parencite{Lin2023}.
Numerical ecology analyses,e.g diversity indices, NMDS, PERMANOVA we calculated
with the vegan R package \parencite{oksanen2024vegan}.
For PCoA ordination we used ape R packege \parencite{Paradis2004} and UMAP python library\parencite{mcinnes2018umap-software}.

\subsection{Tools}\label{Coding environment}
PEMA for OTU inference \parencite{zafeiropoulos2020pema}
U-CIE R package for coloring 3 dimensional data \parencite{Koutrouli2022}

Visualisation was implemented with ggplot2 \parencite{wickham_ggplot2_2016} and pheatmap \parencite{Kolde2019}.
The environment we worked had Python 3.11.4, R version 4.3.2 \parencite{rcoreteam}
and Julia language version 1.9.3 \parencite{Julia-2017}in Julia language version 1.9.3 \parencite{Julia-2017}.
Finally, computations were performed on HPC infrastructure of HCMR \parencite{zafeiropoulos_0s_2021}.

\subsection{Data and Code}
The documentation and scripts developed for this study are available in
\href{https://github.com/savvas-paragkamian/crete_soil_microbiome/}{Crete soil microbiome github repository}.
This repository contains all the necessary scripts for the data retrieval,
filtering and ASV inference, taxonomy assignment, data integration of spatial data, 
functional annotation and the subsequent analyses and visualisation.
Additional scripts about data integration are available in
\href{https://github.com/savvas-paragkamian/crete-data-integration}{Crete data integration}.
Code is structured to be reproducible and interoperable.

\section{Results}\label{integration_results}


\subsection{Samplings}



\subsection{Data cube}

Crete data cube.


\subsection{Soil microbiome}\label{soil_microbiome}

\begin{sidewaystable*}
    \caption{Summary of the different spatial layers in Crete in terms of total area, number of samples and microbial diversity.\label{table:data_cube_summary}}
\begin{tabular*}{\textwidth}{@{\extracolsep{\fill}}llllllll@{\extracolsep{\fill}}}
\tabcolsep=0pt%
class                                           & area & category             & samples & taxa richness & asv richness & mean shannon & sd shannon \\
Arable land                                     & 88   & CLC LABEL2           & 4       & 1518           & 6178          & 4.73          & 0.17        \\
Artificial, non-agricultural vegetated areas    & 21   & CLC LABEL2           & NA      & NA             & NA            & NA            & NA          \\
Forests                                         & 300  & CLC LABEL2           & 4       & 1803           & 8663          & 4.91          & 0.18        \\
Heterogeneous agricultural areas                & 1103 & CLC LABEL2           & 31      & 13701          & 59581         & 4.83          & 0.24        \\
Industrial, commercial and transport units      & 40   & CLC LABEL2           & 4       & 1760           & 6517          & 4.79          & 0.32        \\
Inland waters                                   & 7    & CLC LABEL2           & NA      & NA             & NA            & NA            & NA          \\
Mine, dump and construction sites               & 10   & CLC LABEL2           & NA      & NA             & NA            & NA            & NA          \\
Open spaces with little or no vegetation        & 411  & CLC LABEL2           & 7       & 3132           & 14568         & 4.85          & 0.21        \\
Pastures                                        & 59   & CLC LABEL2           & 4       & 1609           & 7535          & 4.85          & 0.14        \\
Permanent crops                                 & 2368 & CLC LABEL2           & 22      & 9999           & 43833         & 4.88          & 0.18        \\
Scrub and/or herbaceous vegetation associations & 3798 & CLC LABEL2           & 58      & 24394          & 110127        & 4.76          & 0.29        \\
Urban fabric                                    & 111  & CLC LABEL2           & 4       & 1765           & 9828          & 4.78          & 0.19        \\
-                                               & 1    & Geology              & NA      & NA             & NA            & NA            & NA          \\
J-E                                             & 1347 & Geology              & 22      & 8995           & 41730         & 4.75          & 0.25        \\
K-E                                             & 248  & Geology              & NA      & NA             & NA            & NA            & NA          \\
K.k                                             & 1253 & Geology              & 27      & 11277          & 51737         & 4.72          & 0.25        \\
K.m                                             & 13   & Geology              & NA      & NA             & NA            & NA            & NA          \\
Mk                                              & 812  & Geology              & 13      & 5618           & 24633         & 4.79          & 0.2         \\
Mm.I                                            & 1614 & Geology              & 12      & 5259           & 22854         & 4.86          & 0.15        \\
Ph-T                                            & 1012 & Geology              & 14      & 6720           & 29742         & 4.92          & 0.19        \\
Q.al                                            & 911  & Geology              & 34      & 15580          & 64851         & 4.91          & 0.27        \\
T.br                                            & 300  & Geology              & 2       & 625            & 2839          & 4.4           & 0.24        \\
f                                               & 118  & Geology              & 2       & 720            & 4753          & 4.5           & 0.01        \\
fo                                              & 318  & Geology              & 2       & 825            & 4606          & 4.65          & 0.04        \\
ft                                              & 276  & Geology              & 10      & 4062           & 19085         & 4.79          & 0.17        \\
o                                               & 94   & Geology              & NA      & NA             & NA            & NA            & NA          \\
N                                               & 163  & Desertification Risk & NA      & NA             & NA            & NA            & NA          \\
F2                                              & 1945 & Desertification Risk & 53      & 23404          & 103956        & 4.82          & 0.26        \\
F1                                              & 1518 & Desertification Risk & 22      & 8817           & 41020         & 4.7           & 0.23        \\
P                                               & 1593 & Desertification Risk & 36      & 16256          & 73256         & 4.86          & 0.19        \\
C2                                              & 638  & Desertification Risk & 6       & 2113           & 9105          & 4.59          & 0.22        \\
Other areas                                     & 290  & Desertification Risk & NA      & NA             & NA            & NA            & NA          \\
F3                                              & 1144 & Desertification Risk & 14      & 5900           & 25362         & 4.82          & 0.28        \\
C1                                              & 788  & Desertification Risk & 7       & 3191           & 14131         & 4.93          & 0.25        \\
C3                                              & 238  & Desertification Risk & NA      & NA             & NA            & NA            & NA          \\
Semi-Arid                                       & 6336 & Aridity class        & 98      & 43064          & 184930        & 4.83          & 0.25        \\
Dry sub-humid                                   & 1343 & Aridity class        & 32      & 13649          & 65134         & 4.79          & 0.21        \\
Humid                                           & 545  & Aridity class        & 8       & 2968           & 16766         & 4.55          & 0.19       
\end{tabular*}
\end{sidewaystable*}


\subsection{Communities}\label{communities}
Microbial beta diversity differently associated with physical and chemical
features of ISD. PCoA 1 (describing 14\% of the variance) was largely driven by
elevation (F=23, p < 0.001). PCoA 2, which explained 12\% of the variance,
was significantly associated with soil moisture (F=90, p < 0.001).
PCoA 2 also positively correlated with both organic carbon and nitrogen.

Mean annual temperature, elevation and total nitrogen are statistically significant variables for
community dissimilarity across samples (PERMANOVA).

Network of associations (FlashWeave, sensitive) after filtering prevalent ASV (mean normalised relative
abundance>0.001, samples > 2) led to 7,455 ASVs and
29,282 associations (787 are negative).

\subsubsection{Functions}\label{functions}
PREGO


The functional annotation was performed with FAPROTAX. Potential human pathogens 
appear concentrated in Richtis gorge. Most plant pathogens occur in an agricultural 
field in south Rethymnon. 

\begin{figure}[t] 
    \centering\includegraphics[width=\columnwidth]{crete_integration_functions_faprotax}
\caption{Heatmap of the FAPROTAX functions relative abundances per sample.}
    \label{fig:isd_functions_faprotax}
\end{figure}

\subsubsection{Significant taxa}\label{sig_taxa}
Differential abundance (ANCOM-BC2) showed that there are 294 taxa that are significantly different
with Annual mean temperature and elevation. The geological rock type doesn't distinguish taxa
across the island. \textit{Rhodococcus equi} (pathogen of foal or similar) is significantly abundant in
pastures (Corine Land Cover), and has been found in 16 samples.

\section{Discussion}\label{integration_discussion}

Multiple works have emerged the past 5 years about enumerating all
biodiversity \parencite{Anthony2023} and deciphering the biogeochemical 
processes and interactions of fauna, flora and microbes in global
studies \parencite{Fry2019, Crowther2019,GRANDY201640,Delgado-Baquerizo2020} and
in mountain peaks soil microbiomes \parencite{Adamczyk2019}. One of our profound
results is that soil bacterial biodiversity is very complex, even sites a few meters apart can differ
significantly in their community composition, Figure \ref{fig:isd_site_locations}.
This is a fact that is sometimes neglected in worldwide studies and the island biogeography
paradigm can assist to remove clutter.

Apart from the profound diversity in soils, there is also high speciation and uniqueness. 
As shown in Figure \ref{fig:isd_fig2_taxonomy}, most ASVs occur in 1 or 2 or 3 samples.
This is different when compering with the ocean. When using the deepest taxonomic
level of ASVs is possible to identify the specialists and generalists \parencite{Barberan2012}. 
In addition, focusing on the phyla, we see a pattern looking like a phase transition, from 
rare phyla to phyla that dominate all samples, Figure \ref{fig:isd_fig2_taxonomy} C. Lastly, in Figure \ref{fig:isd_fig2_taxonomy} D,
and \ref{fig:isd_top_phyla_samples}, we found some distinct top phyla profiles of samples.

Elevation gradients of biodiversity are known since Humboldt's work \parencite{Rahbek2019} 
yet these patterns remain elusive regarding the soil microbiome \parencite{Looby2020, Siles2023}.
Mostly because it's difficult to isolate other co-founding effects \parencite{Nottingham2018}.
In our results elevation showed important distinction of taxa and of diversity. Yet more work is 
needed to explore the Asterousia mountain transect in isolation to avoid indirect influences of 
other variables.

Crete's ecosystems are mostly semi-arid, whereas in the mountain ranges there 
are areas classified as dry sub-humid and humid. Parent material and influences
on soil functions and bacterial communities have been documented. Here, as shown in Table \ref{table:data_cube_summary},
Quaternary-alluvial sediments (Q.al) and Phyllite-Quartzite series (Ph-T) hold the most diversity 
and richness. Maybe because they are mostly found in riverbeds. Critical areas for 
desertification (C2) hold the most diverse samples. Regarding the land cover the most
rich sample is near the HCMR building, a high touristic area close to the beach. Yet, forests
hold the highest Shannon diversity index.

Richtis gorge (highly popular) has alarmingly high values for human pathogens, sulfur respiration and
nitrate reduction. This gorge has water all year long, rare in the gorges of Crete,
and is the highest touristic attraction of eastern Crete. It's name means "throw" and 
there is a local rumor that people threw unwanted stuff throughout the centuries. 
Nevertheless, it is an important freshwater ecosystem which is not included in 
any protection regimen or legislation.
The potential functions of the bacteria in Richtis gorge are an alarming signal which needs to be further investigated.
Another important finding is the statistically significant presence of \textit{Rhodococcus equi} (pathogen of foal or similar)
in pastures. It is one of the most common causes of pneumonia in foals which
become infected by inhaling dust or soil particles contaminated with the bacterium.

\section{Conclusion}

Deciphering and validating the results presented here requires future work.
Even though amplicon studies in soil should be interpreted with caution \parencite{alteio2021} they 
can act as early warning signals towards public health concerns \parencite{banerjee2023Soil}.
In addition, the immediate release and availability of these data is crucial for 
taking action.
The pillar of data integration is the unrestricted open data across disciplines and 
the open source software.
"A holistic perspective on soil architecture is needed as a key to soil functions" \parencite{philippot2024the-interplay}, is 
an important statement for future soil projects.
Shotgun metagenomics and metatranscriptomics can unleash the functional potential of
topsoil along with other advancements like long reads sequencing. Higher resolution
samplings using grid system will enhance the resolution and also the resampling of
ISD sites in different time points will provide additional insights to the complex soil 
functions and expand the positive and negative associations in soils \parencite{Liu2024}.
In addition, using modeling these data can be used to generate microbiome maps of Crete as 
implemented in Australia \parencite{Peipei2024drivers}.
Lastly, global hotspots \parencite{Guerra2022} and soil ecosystem conservation is needed as 
a whole and expanding current protection of specific species \parencite{guerra2021tracking}.
There are many blind spots in our knowledge about soil functioning and in the conservation
priorities \parencite{farfan2024preliminary}. 
This along with policy \parencite{KONINGER2022} across countries \parencite{Putten2023}
without borders and the implementation of legislation in Greece \parencite{SCHISMENOS2022100035} is 
imperative.



% --------------------------------------------------
% 
% This chapter is for general conclusions
% 
% --------------------------------------------------

\chapter{Conclusions}
\label{cha:conclusions}

The work presented here has combined different approaches of contemporary 
ecological questions.
Regarding microbial diversity, on the global scale the available 
knowledge was harmonised using literature and data mining methods, chapter \ref{cha:prego}.
On the local scale,
in chapter \ref{cha:crete-idea}, the Crete soil model is assembled 
with different types of data, from literature, to biodiversity to spatial data. 
Crete has long been established as a model island for evolution, desertification, biogeographical patterns etc.
This system will assist future analyses of the most studied region of Greece.

A new chapter of the well-studied biodiversity of Crete was initiated, 
the soil microbial diversity.
As expected, the bacterial diversity of the island is very heterogeneous 
with patterns across soil moisture, elevation, land use, mean temperature and bedrock geology.
The large scale of ISD and the confined space of Crete allowed for more resolute 
correlations of diversity. At the same time it removes aspects like latitudinal factors
of the global scale studies which in some cases cannot be explained biologically. 
One important result of the ISD analysis is the difference of microbial 
communities from samples a few meters apart. This is a important result
that often is not explicitly mentioned in soil microbial studies. This heterogeneity 
makes finding patterns extremely difficult, a fact that demonstrates that
there are important processes of microbial functioning and
dispersal in soil that are not deciphered yet.

Arthropods have been neglected from most conservation frameworks, biogeochemical models
and microbial studies. With the work presented here,
based on the Crete soil system (chapter \ref{cha:crete-idea})
and the endemic arthropods assessment (chapter \ref{sec:arthropods-intro}),
the foundations are established to move forward both 
conservation and basic research on ecosystem modeling with 
data from the most diverse phylum of terrestrial fauna.
Expert curation was applied for the compilation of historical and contemporary
literature along with specimens from NHMC for the endemic Cretan arthropods occurrences.
After the compilation of the dataset it became clear that a conservation analysis
was a priority because the majority of species were predicted to be threatened.

Literature and data mining 
methodologies are also very useful to rescue historical biodiversity data which are
indispensable as demonstrated in chapter \ref{sec:arthropods-intro}.
The gap of historical data rescue and curator tools was narrowed
through the DECO workflow. DECO streamlined a collection of 
of tools and standards aiming to assist the curators' process. From the comparison
of such tools it became clear that human curation is a transversal undertaking in all steps.
Tools like DECO, are very useful to curators for combining all tools for their process 
to improve quality control.

There is a wealth of available data and tools as demonstrated in 
multiple chapters in this PhD. 
The previous generations of scientists were the first to 
compile an enormous amount of biodiversity data but a large portion
of this data remains unpublished in legacy formats. Therefore is under risk of destruction and loss.
This PhD has focused on rescue of this data and bringing them to contemporary standards and digital media,
a task that is a challenge of current generations. 
An additional challenge is to bring together and analysing historical data
and contemporary data under the ecosystem approach.

The basic conceptual challenges remain of systems ecology. 
Some of these can be formulated as : What are the causes of ecosystem collapse?
What is needed for a sustainable future?
How will climate warming change life on Earth?
These clear questions require collaboration of across scientific fields. 
Interdisciplinary collaboration requires effort from all sides to 
achieve effective communication. 
The exemplary collaboration of science, enterprise and society was
illustrated during the pandemic of COVID-19 in 2020 \parencite{ioannidis2021the-rapid,lee2021scientific}. 
This level of synergy is needed to tackle the environmental issues
as well, like sustainable agriculture.
Different academic cultures can be integrated to form a scientific transculturalism,
the process of integration of the three cultures, variance, coarse-graining, and exactitude \parencite{Enquist_2024}.
These cultures can be vaguely described as natural history, numerical ecology and complex systems ecology, respectively.
An important step to bring these cultures together is communication and openness across scientists.
These gaps must be eliminated soon to reach predictive ecology goals \parencite{mouquet_review_2015}.

Soil health,
and one health in general, is key to avoid the disastrous projections of
current practices in agriculture and industries \parencite{banerjee2023Soil}.
For example, 19\% of soils of Crete are under desertification risk \parencite{KARAMESOUTI2018266}.
The integrative analysis of ISD Crete revealed some health potential risks in 
natural ecosystems like Richtis gorge. This gorge has the highest abundance of 
potential human pathogens of all soil samples of Crete. This is alarming of 
the human activities that take place in this ecosystem like intensive tourism 
and agriculture. Richtis gorge in not protected by any framework, chapters \ref{cha:crete-idea} and \ref{cha:crete-soil}.
More than 60\% of
soils in Europe are considered unhealthy leading to erosion, degradation, contamination,
and disruption of global nutrient cycles \parencite{commission2020caring}.
Economically, it is estimated that due to soil degradation about 50 billion euro per year
are lost in the European Union. 
Yet, soil, the foundation of 
terrestrial ecosystems, is not protected with legislation across Europe.
Since 2020, after devastating realisations about soil, the European Parliament 
is in the process of structuring a soil monitoring and resilience law to achieve 
the goal of having healthy soils by 2050. As of writing this Thesis, the EU 
parliament has just voted the \href{https://www.europarl.europa.eu/thinktank/en/document/EPRS_BRI(2024)757627}{soil monitoring law}
while Trialoge, Second reading and Adoption steps are still in progress to make it final.

While this PhD is mostly on bioinformatics, field sampling was organised
for the second sampling expedition for ISD Crete.
While analysing the ISD Crete 2016 the sampling replicability was put into action with a 
new sampling in July of 2022.
Using the same protocols and locations, 29 people from HCMR, NHMC, UOC Biology
Department and citizen scientists where split in 10 teams and went sampling. 
The goal of this sampling was to collect a second time point of the same locations
to decipher the metagenomic content of soil. This was a voluntary work supported 
by the SUPP GEN project of HCMR. The DNA extraction and shipment was carried out 
by HCMR and sequencing by the Joint Genome Initiative. DNA extracted by the 72 locations 
is going to be sequenced using deep shotgun sequencing. 
This project is part of the Island Microbiome Encyclopedia (ISME) consortium.
This is one of the few large scale metagenomic soil projects in
Europe \parencite{nayfach2021a-genomic, ma2023a-genomic}.

Findable, Accessible, Interoperable and Reusable (FAIR) data and reproducible analyses
are the only way for science to assist the society and convince stakeholders for action. 
This Thesis, is doing exactly that using Crete as a model of application. Across the chapters it is shown how 
integrating multiple types of data can help identify signals of health concerns and 
pressured ecosystems.

The outcome of this bioinformatic PhD is one software tool (\hyperref[cha:deco]{DECO}),
one knowledge base (\hyperref[cha:crete-idea]{Crete soil system}), one online
database (\hyperref[cha:prego]{PREGO}) and
two novel data analyses (\hyperref[cha:arthropods]{Endemic arthropods} and \hyperref[cha:isd-crete-soil]{ISD Crete}).
The code produced during the aforementioned projects consists of more that 12 thousand lines of code 
and is under open access licences and reproducible by design to allow for full transparency and assist similar research projects. 


%----------------------------------------------------------------------------------------
%	THESIS CONTENT - APPENDICES
%----------------------------------------------------------------------------------------
\appendix % Cue to tell LaTeX that the following "chapters" are Appendices

% Include the appendices of the thesis as separate files from the Appendices folder
% Uncomment the lines as you write the Appendices

\chapter{PREGO Appendix A}

\section{Mappings}
\label{app:A}


PREGO produces entity identifiers either by Named Entity Recognition (NER) with the EXTRACT tagger or by mapping retrieved identifiers to the selected ones. 
PREGO adopted NCBI taxonomy identifiers for taxa, Environmental Ontology for environments and Gene Ontology as a structure knowledge scheme for Processes (GObp) and Molecular Functions (GOmfs). 
The latter was for reasons that are two-fold, first Gene Ontology has a Creative Commons Attribution 4.0 License and second there are many resources that have mapped their identifiers to Gene Ontology.
MG-RAST metagenomes and JGI/IMG isolates annotations come with KEGG orthology (KO) terms; 
Struo-oriented genome annotations, on the other hand, have Uniprot50 ids. 
The mapping from KO to GOmf and Uniprot50 to GOmf is implemented via UniProtKB mapping files of their FTP server (see \texttt{idmapping.dat} and \texttt{idmapping\_selected.tab} files). 
By using the 3-column mapping file, the initial annotations were mapped to GOmf. As a complement, a list of metabolism-oriented KEGG ORTHOLOGY (KO) terms has been built (see \textit{prego\_mappings} in the Availability of Supporting Source Codes section).
Finally, as STRUO annotations refer to GTDB genomes, \href{http://ftp.tue.mpg.de/ebio/projects/struo/GTDB_release89/metadata/}{publicly available mappings} (accessed on 24 December 2021) were used to link the genomes used with their corresponding NCBI Taxonomy entries.



\section{Daemons}
\label{app:B}

An important component PREGO approach (Figure A1) is the regular updates which keep PREGO in line with the literature and microbiology data advances. 
The updates are implemented with custom scripts called daemons that are executed regularly spanning from once a month up to six-month cycles. 
This variation occurs because of the API requirements of each web resource as well as the computational intensity of the association extraction from the retrieved data.

\begin{figure}[ht]
   \centering
   \includegraphics[width=95mm]{figures/figure_A1_PREGO_daemons.png}
   \caption[PREGO DevOps]{Software daemons perform all steps of the PREGO methodology in a continuous manner similar to the Continuous Development and Continuous Integration method.}
   \label{fig:devops}
\end{figure}


Each Daemon is attached to a resource because its data retrieval methods (API, FTP) and following steps, shown in Figure A1, require special handling and multiple scripts (see \textit{prego\_daemons} in the Availability of Supporting Source Codes section).

\section{Scoring}
\label{app:C}


Scoring in PREGO is used to answer the questions:
\begin{itemize}
   \item Which associations are more thrustworthy?
   \item Which associations are more relevant to the user's query?
\end{itemize}

Relevant, informative, and probable associations are presented to the user through the three channels that were discussed previously. 
Each channel has its own scoring scheme for the associations it contains and all of them are fit in the interval $(0,5]$ to maintain consistency. 
The values of the score are visually shown as stars. 
The Genome Annotation and Isolates channel has fixed values of scores depending on the resource because Genome Annotation is straightforward, and the microbe id is known a priori. 
On the other hand, Environmental Samples channel data are based on samples, which contain metagenomes and OTU tables. 
Thus, it has two levels of organization, microbes with metadata, and sample identifiers. Each association of two entities is scored based on the number of samples they co-occur. 
A Literature channel scoring scheme is based on the co-mention of a pair of entities in each document, paragraph, and sentence. The differences in the nature of data require different scoring schemes in these channels.
The contingency table (Table~\ref{table:pregoA1}) of two random variables, $X$ and $Y$ are the starting point for the calculation of scores. The term $X = 1$ might be a specific NCBI id and $Y = 1$ a ENVO term. 
The $c_{1,1}$ is the number of instances that two terms of $X = 1$ and $Y = 1$ are co-occurring, i.e., the joint frequency. 
The marginals are the $c_{1,.}$ and $c{.,1}$ for $x$ and $y$, respectively, which are the backgrounds for each entity type. 
Different handling of these frequencies leads to different measures. 
There is not a perfect scoring scheme, just the one that works best on a particular instance. 
Consequently, scoring attributes require testing different measures and their parameters.



\begin{table}[ht]
   \centering
   \begin{tabular}{c|llll}
    & \multicolumn{4}{l}{Y = y} \\ \cline{2-5} 
   \multirow{4}{*}{X = x} &  & Yes & No & Total \\ \cline{3-5} 
    & \multicolumn{1}{l|}{Yes} & $c_{x,y}$ & $c_{x,0}$ & $c_{x,.}$ \\
    & \multicolumn{1}{l|}{No} & $c_{0,y}$ & $c_{0,0}$ & $c_{0,.}$ \\
    & \multicolumn{1}{l|}{Total} & $c_{.,y}$ & $c_{.,0}$ & $c_{.,.}$
   \end{tabular}
   \caption[PREGO contingency table between two terms]{Contingency table of co-occurrences between entities $X = x$ and $Y = y$. 
   This is the basic structure for all scoring schemes. $c_{x,y}$ is the count of the co-occurrence of these entities. $c_{x,.}$ is the count of the $x$ with all the entities of $Y$ type (e.g., Molecular function). Conversely, $c_{.,y}$ is the count of $y$ with all the entities of $X$ type (e.g., taxonomy}
   \label{table:pregoA1}
\end{table}


\section*{Literature Channel}

Scoring in the Literature channel is implemented as in STRING 9.1 \citep{franceschini2012string} and COMPARTMENTS \citep{binder2014compartments}, where the text mining method uses a three-step scoring scheme. 
First, for each co-mention/co-occurrence between entities (e.g., Methanosarcina mazei with Sulfur carrier activity), a weighted count is calculated because of the complexity of the text.  


\begin{equation}
   c_{x,y} = \sum_{k=1}^{n}{w_d \delta_{dk}(x,y) +w_p \delta_{p,k}(x,y) + w_s \delta_{sk}(x,y)}
   \label{eq:prego-score-1}
\end{equation}



Different weights are used for each part of the document ($k$) for which both entities have been co-mentioned, $w_d = 1$ for the weight for the whole document level, $w_p = 2$ for the weight of the paragraph level, and $w_s = 0.2$ for the same sentence weight. 
Additionally, the delta functions are one (Equation~\ref{eq:prego-score-1}) in cases the co-mention exists, zero otherwise. Thus, the weighted count becomes higher as the entities are mentioned in the same paragraph and even higher when in the same sentence.
Subsequently, the co-occurrence score is calculated as follows:

\begin{equation}
   score_{x,y} = c_{x,y}^a (\frac{c_{x,y} c_{.,.}}{c_{x, .}c_{.,y}})^{1-a}
   \label{eq:prego-score-2}
\end{equation}
   


where $a = 0.6$ is a weighting factor, and the $c_{x,.}$, $c_{.,.}$, 
$c_{.,y}$ are the weighted counts as shown in Table~\ref{table:pregoA1} estimated using the same Equation~\ref{eq:prego-score-2}. 
This value of the weighting factor has been chosen because it has been optimized and benchmarked in various 
applications of text mining~\citep{franceschini2012string, binder2014compartments, pletscher2015diseases}. 
The value of Equation~\ref{eq:prego-score-2} is sensitive to the increasing size of the number of documents (MEDLINE PubMed—PMC OA).
Therefore, to obtain a more robust measure, the value of the score is transformed to $z$-score. 
This transformation is elaborated in detail in the COMPARTMENTS resource \citep{binder2014compartments}. 
Finally, the confidence score is the $z$-score divided by two. Cases in which the scores exceed the (0,4] interval are capped to a maximum of 4 to reflect the uncertainty of the text mining pipeline.

\section*{Environmental Samples Channel}

Data from environmental samples are OTU tables and metagenomes. 
Thus, for each entity $x$, the number of samples is calculated as the background 
and a number of samples of the associated entity (metadata background) $c.,y$ (see Table~\ref{table:pregoA1}). 
Each association between entities $x$,$y$ has a number of samples, $c_{x,y}$ that they co-occur. 
Note that each resource is independent and the scoring scheme is applied to its entities. 
This means that the same association can appear in multiple resources with different scores. 
The score is calculated with the following formula:

\begin{equation}
   score_{x,y} = 2.0*{\frac{\sqrt{c_{x,y}}}{c_{.,y}^{0.1}}}
\end{equation}


This score is asymmetric because the denominator is the marginal of the associated entity. 
Thus, the score decreases as the marginal of $y$ is increasing, i.e., the number of samples that $y$ is found. 
On the other hand, it promotes associations in which the number of samples of 
the association are similar to the marginal of $y$. 
The exponents on the numerator and denominator equal to $0.5$ and 
to $0.1$, respectively, in order to reduce the rapid increase of score.
Lastly, the value of the score is capped in the range $(0,4]$.


\section{Bulk download}
\label{app:D}

   Users can also download programmatically all associations per channel through the links that are shown in Table~\ref{table:prego-appD-1}. 
   The data are compressed to reduce the download size and md5sum files are provided as well for a sanity check of each download.

   % PREGO BULK DOWNLOAD TABLE 
   \begin{table}[ht]
      
      \begin{adjustwidth}{-2cm}{}

      \begin{tabular}{llll}
      \toprule
      Channel & Link & md5sum & Size (in GB) \\ \midrule

      Literature & \href{https://prego.hcmr.gr/download/literature.tar.gz}{literature.tar.gz} & \href{https://prego.hcmr.gr/download/literature.tar.gz.md5}{literature.tar.gz.md5} & 5.4 \\

      \begin{tabular}[c]{@{}l@{}}Environmental \\ Samples\end{tabular} &
      \href{https://prego.hcmr.gr/download/environmental\_samples.tar.gz}{environmental\_samples.tar.gz} & 
      \href{https://prego.hcmr.gr/download/environmental\_samples.tar.gz.md5}{environmental\_samples.tar.gz.md5}
      & 0.69 \\

      \begin{tabular}[c]{@{}l@{}}Annotated \\ genomes and \\ isolates\end{tabular} & 
      \href{https://prego.hcmr.gr/download/annotated\_genomes\_isolates.tar.gz}{annotated\_genomes\_isolates.tar.gz} &
      \href{https://prego.hcmr.gr/download/annotated\_genomes\_isolates.tar.gz.md5}{annotated\_genomes\_isolates.tar.gz.md5} & 0.26 \\ \bottomrule
      \end{tabular}
      \end{adjustwidth}
      \caption[PREGO Bulk download links and md5sum files.]{Bulk download links and md5sum files.}
      \label{table:prego-appD-1}
   \end{table}


\chapter{Curation of historical literature Appendix} % Main appendix title

\label{AppendixB} 

   \begin{figure}[ht]
      \centering
      \includegraphics[width=\textwidth,height=\textheight,keepaspectratio]{figures/deco-figure-S1.jpg}
      \caption[GNRD taxon names identification]{Screenshot of the web application GNRD identifying taxon names.}
      \label{fig:gnrd-screenshot}
   \end{figure}

   \begin{figure}[ht]
      \centering
      \includegraphics[width=\textwidth,height=\textheight,keepaspectratio]{figures/deco-figure-S2.jpg}
      \caption[BOM performing NER]{Screenshot of the web application BOM performing NER. It provides taxon names, text snippets and term co-occurrences.}
      \label{fig:bom-screenshot}
   \end{figure}
   
   \begin{figure}[ht]
      \centering
      \includegraphics[width=\textwidth,height=\textheight,keepaspectratio]{figures/deco-figure-S3.jpg}
      \caption[Pensoft Annotator performing NER]{Screenshot of the web application Pensoft Annotator performing NER.}
      \label{fig:pensoft-annotator-screenshot}
   \end{figure}

%
\chapter{Crete Microbiome Appendix} % Main appendix title

\label{AppendixC} 

\section{Amplicon 16s rRNA is soil}

\section{Errors in Amplicon Microbial Ecology}

Microbial ecology based on amplicon 16s rRNA sequences has flourished since the
2010s. The endeavor to understand the microbial world faces multiple challenges
across the scientific workflow, from sampling to ecological analyses \cite{Lee2012}.

Errors propagate starting with the sampling. There are contaminations from the
people in the field, in the lab for the DNA extraction. 

Instrument errors from PCR amplification and errors from sequencing.
Approximation and computation errors from algoritms that cluster, measure similarities between
sequences.

Semantic errors because of reductionist approaches and/or oversimplification
of the microbial communities

\section{OTU vs ASV}

Amplicon rRNA sequencing provides a collection of sequence reads per sample. 
The ecological interpretation of the reads requires their transformation to
taxonomic information. To do so there are two approaches currently in use, 
the clustering method and the denoising method. With clustering reads are 
grouped together and a best representing sequence is produced for each 
cluster, i.e. the Operetional Taxonomic Unit. This approach makes the OTUs 
from different runs, i.e executions of the algorithm and/or different studies
incoperable and irreproducible.

Currently, many studies propose the use of Amplicon Sequence Variants \cite{Callahan2017}. 
ASVs are real biological sequences and can be used for comparison.

The influence of the different methods to subsequent ecological analyses has
little impact \cite{Glassman2018}.
%   \begin{figure}[h]
%      \centering
%      \includegraphics[width=\textwidth,height=\textheight,keepaspectratio]{figures/deco-figure-S1.jpg}
%      \caption[GNRD taxon names identification]{Screenshot of the web application GNRD identifying taxon names.}
%      \label{fig:gnrd-screenshot}
%   \end{figure}



%----------------------------------------------------------------------------------------
%	BIBLIOGRAPHY
%----------------------------------------------------------------------------------------

% Add ref 
\printbibliography[heading=bibintoc]

%----------------------------------------------------------------------------------------

\end{document}  
