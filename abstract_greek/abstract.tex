\documentclass[11pt]{article}
\usepackage[LGR]{fontenc}
\usepackage[utf8]{inputenc} % Use utf-8 encoding for input
\usepackage[greek]{babel} % Greek and English language support
\usepackage{textgreek}
\usepackage{geometry}
\geometry{
	paper=a4paper, % Change to letterpaper for US letter
	inner=2.5cm, % Inner margin
	outer=3.8cm, % Outer margin
	bindingoffset=.5cm, % Binding offset
	top=1.5cm, % Top margin
	bottom=1.5cm, % Bottom margin
	%showframe, % Uncomment to show how the type block is set on the page
}
\pagestyle{empty}

\begin{document}

\section*{Περίληψη}
{\LARGE Αποσαφηνίζοντας τη σχέση των αλληλεπιδράσεων των μικροβίων με τη λειτουργία των οικοσυστημάτων και των βιογεωχημικών διεργασιών τους.}


\vspace{1cm}

Για να κατανοήσουμε τη λειτουργία των οικοσυστημάτων, είναι απαραίτητο να διακριθούν
οι διεργασίες που συμβαίνουν σε διάφορα περιβάλλοντα και τους οργανισμούς
που τις επιτελούν. Τα οικοσυστήματα αποτελούνται από πολύπλοκες αλληλεπιδράσεις των οργανισμών,
των αβιοτικών παραμέτρων και τα φυσικά χαρακτηριστικά των υλικών. Για παράδειγμα,
το οικοσύστημα του εδάφους περιέχει 100 εκατομμύρια κύτταρα σε κάθε γραμμάριο. Η υφή και η
η χημεία του εδάφους επηρεάζει και επηρεάζεται από τα υπάρχοντα τάξα από όλο το
δέντρο της ζωής, φυτά, αρθρόποδα, βακτήρια, μύκητες κλπ.

Τα μικρόβια έχουν τη μεγαλύτερη επιρροή στην λειτουργία των οικοσυστημάτων του εδάφους. 
Παρόλα αυτά, η αποσαφύνιση της μικροβιακής βιοποικιλότητας παραμένει υπό εξερεύνηση.
Τα νησιά, λόγω της ποικιλομορφίας τους σε μικρότερη κλίμακας αποτελούν μια σημαντική ευκαιρία για 
την βαθύτερη κατανόηση της μικροβιακής ποικιλότητας του εδάφους. 
Η Κρήτη, ένα ηπειρωτικό νησί, έχει ξεχωριστή βιοποικιλότητα με έντονη γεωλογική 
ιστορία και με πολλές αντιθέσεις σχετικά με την βλάστηση, το κλίμα και τα πετρώματα. 
Μελετάται από της αρχαιότητα και η θέση στης στο νοτιοανατολικό άκρο της Μεσογείου 
την καθιστά σημαντική για την μελέτη της κλιματικής αλλαγής.
Λαμβάνοντας αυτά υπόψη, μια από της πρώτες μελέτες νησιωτικού μικροβιώματος 
οργανώθηκε στην Κρήτη από το \textlatin{Genome Standards Consortium} το 2016. 
Στο \textlatin{Island Sampling Day}, όπως ονομάζεται το συγκεκριμένο \textlatin{project},
ερευνητές και εθελοντές σύλλεξαν 432 δείγματα εδάφους από 72 διαφορετικά σημεία της Κρήτης
στη διάρκεια μιας μέρας.
Η ανάλυση του μικροβιώματος της Κρήτης έδειξε ότι η μικροβιακή ποικιλότητα
συσχετίζεται κυρίως από το άζωτο, τον οργανικό άνθρακα και την υγρασία του εδάφους.
Επίσης η μεγάλη ετερογένεια της β-ποικιλότητας μεταξύ των δειγμάτων είναι δείκτης ότι χρειάζονται περισσότερα 
δεδομένα για τα χαρακτηριστικά του εδάφους και των άλλων οργανισμών για την ερμηνεία μοτίβων και λειτουργιών 
των μικροβίων.

Τα τελευταία 15 χρόνια, μελέτες περιβαλλοντικής μικροβιολογίας, όπως το \textlatin{ISD}, 
έχουν οδηγήσει σε μια πληθώρα δεδομένων σε ανοιχτές βάσεις δεδομένων και στη
βιβλιογραφία.
Ωστόσο, όλες αυτές οι πληροφορίες είναι ανομοιογενείς και κατακερματισμένες σε διαφορετικές πλατφόρμες.
Για την ομογενοποίηση όλων αυτών των δεδομένων και μεταδεδομένων 
αναπτύχθηκε το \textlatin{PREGO}, μια συγκεντρωτική βάση γνώσης για τα μικρόβια, τις διεργασίες τους και
τα περιβάλλοντα που βρίσκονται.
Μέσω τεχνικών εξόρυξης κειμένου και ενσωμάτωσης δεδομένων το \textlatin{PREGO} συγχωνεύει 
συσχετίσεις οργανισμών, περιβάλλοντος και λειτουργιών από
από την επιστημονική βιβλιογραφία και βάσεις δεδομένων. 
Οι μικροοργανισμοί, οι βιολογικές διεργασίες και οι περιβαλλοντικοί τύποι αντιστοιχούνται
με όρους οντολογίας και κάθε συσχέτιση κατηγοριοποιείται με ένα σύστημα αξιολόγησης 
βαθμών εμπιστοσύνης.
Με 364.508 μικροβιακά τάξα, 1090 περιβαλλοντικούς τύπους, 15.091 βιολογικές διεργασίες,
και 7.971 μοριακές λειτουργίες, το \textlatin{PREGO} στοχεύει να βοηθήσει τους ερευνητές στο 
σχεδιασμό και ερμηνεία πειραμάτων αλλά και την διατύπωση νέων υποθέσων εργασίας.
Επιπλέον, διευκολύνει την διερεύνηση συσχετίσεων περιβάλλοντος-διαδικασιών-μικροβίων.
Για παράδειγμα, σχετικά με το έδαφος, 49 περιβάλλοντα έχουν συσχέτισεις υψηλής βαθμολογίας με 
6276 τάξα, 169 βιολογικές διεργασίες και 3017 μοριακές λειτουργίες.
Η πλειονότητα των συσχετίσεων, από τις 10.929 στο σύνολο, είναι μεταξύ περιβαλλόντων 
και οργανισμών. Ενώ 3139 συσχετίσεις είναι μεταξύ περιβαλλόντων και μοριακών λειτουργιών, 
και 274 μεταξύ περιβαλλόντων και βιολογικών διεργασιών.

Για τον εμπλουτισμό των δεδομένων για τα οικοσυστήματα της Κρήτης έγινε για την μελέτη
της λειτουργίας οικοσυστημάτων και την αναζήτηση συσχετίσεων με τα 
μικρόβια του εδάφους.
Πιο συγκεκριμένα, χρησιμοποιώντας ανοιχτά δεδομένα πραγματοποιήθηκε ανάλυση 
δεδομένων βιβλιογραφίας, δειγματοληψιών και διαφορετικών χαρτών. Όλα αυτά τα δεδομένα
συγκεντρώθηκαν για να αποτελέσουν ένα ενιαίο εδαλοφολικό σύστημα της Κρήτης. 
Βασικός στόχος είναι η ψηφιακή αναπαράσταση του εδάφους της Κρήτης για να αποκρυπτογραφηθούν
οι λειτουργίες των ποικίλων οικοσυστημάτων του νησιού.
Η ανάλυση της βιβλιογραφίας έδειξε τον μεγάλο πλούτο γνώσης που υπάρχει για την 
βιοποικιλότητα της Κρήτης η οποία όμως ακόμη δεν έχει ενσωματωθεί σε μεγάλο βαθμό σε βάσεις 
δεδομένων. 
Επίσης σχετικά με τις δειγματοληψίες, φαίνεται ότι μεγάλες παγκόσμιες 
μελέτες εδαφολογικού μικροβιώματος έχουν δείγματα από την Κρήτη. 
Υπάρχει ακόμη και μεγάλη πληθώρα χωρικών δεδομένων όπως κλιματικά, 
τύπος εδάφους, γεωλογίας, τύπος βλάστησης κλπ.
Όλα αυτά μαζί με ανοιχτά δεδομένα από μελέτες ερημοποίησης και κλιματικής αλλαγής
συντελούν την Κρήτη ένα σημαντικό μοντέλο για μελέτες χερσαίων οικοσυστημάτων.

Εν μέσω της δραματικής αύξησης των δραστηριοτήτων του ανθρώπου είναι σημαντικό
να κατανοήθεί η παρούσα κατάσταση των οικοσυστημάτων. Σε αυτό είναι
απαραίτητη η ιστορική γνώση που υπάρχει από προηγούμενες μελέτες.
Τα ιστορικά έγγραφα βιοποικιλότητας είναι ζωτικής σημασίας για
μακροπρόθεσμες αναλύσεις. Όμως η διάσωση και η επιμέλεια ιστορικών δεδομένων είναι δύσκολη λόγω του ιστορικού τους
πλαισίου. Η διαδικασία διάσωσης δεδομένων περιλαμβάνει ψηφιοποίηση εγγράφων, μεταγραφή, εξαγωγή πληροφοριών
χρησιμοποιώντας εργαλεία εξόρυξης κειμένου και τη μετέπειτα δημοσίευση σε βάσεις δεδομένων.
Τα εργαλεία εξόρυξης γνώσης, που έχουν εξελιχθεί ραγδαία τα τελευταία χρόνια,
αναγνωρίζουν οντότητες στο κείμενο, βοηθώντας στην επιμέλεια.
Για το σκοπό διάσωσης γνώσης από ιστορικά έγγραφα βιοποικιλότητας
αναπτύχθηκε ένα νέο εργαλείο, το \textlatin{DECO}. 
Το \textlatin{DECO} ενισχύει τη διαδικασία διάσωσης κειμένων βιοποικιλότητας
καθώς ενσωματώνει εργαλεία από την μετατροπή εικόνων σάρωσης σε κείμενο, την 
εύρεση ονομάτων των ειδών και των περιβαλλόντων που αναφέρονται με διαφορετικά εργαλεία καθώς 
και την αντιστοίχισή τους σε αριθμούς μητρώου βάσεων δεδομένων. 
Η ροή εργασιών του \textlatin{DECO} μπορεί να χειριστεί παράλληλα μελέτες
εκατοντάδων σελίδων για την επιτάχυνση της διαδικασίας διάσωσης δεδομένων.

Η μελέτη των οικοσυστημάτων του παρελθόντος και των σημερινών συνθηκών έχει κάνει εμφανή την ανάγκη για
δράσεις διατήρησης επειδή βρίσκονται υπό πολλαπλές απειλές.
Ένα παράδειγμα είναι η κατάρρευση των αρθροπόδων, μια παγκόσμια τεκμηριωμένη
τάση που έχει αντιμετωπιστεί ανεπαρκώς από την κοινωνία.
Η Κρήτη είναι ένα \textlatin{hotspot} βιοποικιλότητας και η έρευνα για την πανίδα των αρθροπόδων της χρονολογείται από τον 19ο αιώνα.
Στην παρούσα εργασία αξιολογήθηκε η κατάσταση της διατήρησης των ενδημικών αρθρόποδων της Κρήτης χρησιμοποιώντας
την επικάλυψη των κατανομών των αρθρόποδων και τον εντοπισμό \textlatin{hotspots}.
Διερευνήθηκαν τα \textlatin{hotspots} ενδημικότητάς τους τα οποία εντοπίστηκαν κυρίων
στους ορεινούς όγκους του νησιού.
Για αυτές τις αναλύσεις, συγκεντρώθηκαν παρουσίες του ενδημικών αρθροπόδων της Κρήτης
που βρίσκονται στις συλλογές του Μουσείου Φυσικής Ιστορίας της
Κρήτη μαζί με στοιχεία από την εκτενή βιβλιογραφία.
Τα ενδημικά \textlatin{hotspots} επίσης αξιολογήθηκαν σχετικά με τις ανθρώπινες 
πιέσεις και την αλλαγή χρήσης γης.

Ο εμπλουτισμός των δεδομένων των οικοσυστημάτων και των μικροβίων του \textlatin{ISD}
με τα παραπάνω τα δεδομένα επέτρεψε την στοχευμένη μελέτη σχετικά με την υγεία και
την ολιστική ψηφιακή αναπαράστασή τους. 
Πιο συγκεκριμένα, ανιχνεύτηκαν δυνητικά παθογόνοι μικροοργανισμοι σε
παρθένα οικοσυστήματα και βοσκοτόπια.
Επιπλέον, οι περιοχές με μεγάλο υψόμετρο φιλοξενούν λιγότερη βιοποικιλότητα
μικροοργανισμών, σε αντίθεση με τα μοτίβα που παρατηρούνται σε άλλες ομάδες πανίδας όπως τα αρθρόποδα.
Απαιτούνται ακόμη πολλά βήματα πλήρη ψηφιοποίηση των οικοσυστημάτων
όμως υπάρχουν ευκαιρίες για να νέα ερωτήματα και νέες ιδέες
μέσω της ενσωμάτωσης.

Η προστασία, η διατήρηση και η βασική έρευνα πλέον χρειάζεται να 
γίνονται ταυτόχρονα επειδή τα περιθώρια δράσης ενόψει της κατάρρευσης των
οικοσυστημάτων είναι στενά.
Σημαντική προϋπόθεση για να επιτευχθούν τα παραπάνω είναι τα ανοιχτά δεδομένα,
ανοιχτά πρότυπα και ανοιχτός κώδικας. Κάθε κεφάλαιο αυτής της εργασίας είχε σαν 
επιπλέον στόχο να τη διαφάνεια, την διαλειτουργικότητα και την επαναληψιμότηα 
των αναλύσεων ώστε να μπορέσουν να χρησιμοποιηθούν, να αξιολογηθούν και να βελτιωθούν στο μέλλον.

\end{document}
