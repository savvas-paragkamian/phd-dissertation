\documentclass[11pt]{article}
\usepackage[LGR]{fontenc}
\usepackage[utf8]{inputenc} % Use utf-8 encoding for input
\usepackage[greek]{babel} % Greek and English language support
\usepackage{textgreek}
\usepackage{geometry}
\geometry{
	paper=a4paper, % Change to letterpaper for US letter
	inner=2.5cm, % Inner margin
	outer=3.8cm, % Outer margin
	bindingoffset=.5cm, % Binding offset
	top=1.5cm, % Top margin
	bottom=1.5cm, % Bottom margin
	%showframe, % Uncomment to show how the type block is set on the page
}
\pagestyle{empty}

\begin{document}

\section*{Περίληψη}

Για να κατανοήσουμε τη λειτουργία των οικοσυστημάτων, είναι ανάγκη να διακρίνουμε
τις διεργασίες που συμβαίνουν σε διάφορα περιβάλλοντα και τους οργανισμούς
που τις επιτελούν. Τα οικοσυστήματα αποτελούνται από πολύπλοκες αλληλεπιδράσεις τάξων,
αβιοτικών παραμέτρων και τα φυσικά χαρακτηριστικά των υλικών. Για παράδειγμα,
το οικοσύστημα του εδάφους περιέχει 10 δισεκατομμύρια κύτταρα σε κάθε γραμμάριο. Η υφή και η
η χημεία του εδάφους επηρεάζει και επηρεάζεται από τα υπάρχοντα τάξα από όλο το
δέντρο της ζωής, φυτά, αρθρόποδα, βακτήρια, μύκητες κλπ. Όσον αφορά τα μικρόβια
υπάρχει πληθώρα δεδομένων σε ανοιχτές βάσεις δεδομένων και στη
βιβλιογραφία. Ωστόσο, όλες αυτές οι πληροφορίες είναι ανομοιογενείς και αποθηκευμένες σε 
βάσεις δεδομένων ως σιλό. Το \textlatin{PREGO}, αποτελεί μια συγκεντρωτική βάση γνώσης. Μέσω
τεχνικών εξόρυξης κειμένου και ενσωμάτωσης δεδομένων συγχωνεύει 
συσχετίσεις οργανισμών, περιβάλλοντος και λειτουργιών από
από την επιστημονική βιβλιογραφία και βάσεις δεδομένων. 
Οι μικροοργανισμοί, βιολογικές διεργασίες και περιβαλλοντικοί τύποι αντιστοιχούνται
με όρους οντολογίας και κάθε συσχέτιση κατηγοριοποιείται με ένα σύστημα αξιολόγησης 
βαθμών εμπιστοσύνης.
Με 364.508 μικροβιακά τάξα, 1090 περιβαλλοντικούς τύπους, 15.091 βιολογικές διεργασίες,
και 7.971 μοριακές λειτουργίες, το \textlatin{PREGO} στοχεύει να βοηθήσει τους ερευνητές στο 
σχεδιασμό και ερμηνεία πειραμάτων αλλά και την διατύπωση νέων υποθέσων εργασίας. Επιπλέον, διευκολύνει την εξερεύνηση
συσχετίσεις περιβάλλοντος-διαδικασιών-μικροβίων.

Είναι σημαντικό να κατανοήθεί η τρέχουσα κατάσταση των οικοσυστημάτων. Σε αυτό είναι
απαραίτητη η ιστορική γνώση που υπάρχει από προηγούμενες μελέτες.
Τα ιστορικά έγγραφα βιοποικιλότητας είναι ζωτικής σημασίας για
μακροπρόθεσμες αναλύσεις. Όμως η διάσωση και η επιμέλεια δεδομένων ιστορικών δεδομένων είναι δύσκολη λόγω του ιστορικού τους
πλαισίου. Η διαδικασία διάσωσης δεδομένων περιλαμβάνει ψηφιοποίηση εγγράφων, μεταγραφή, εξαγωγή πληροφοριών
χρησιμοποιώντας εργαλεία εξόρυξης κειμένου και τη μετέπειτα δημοσίευση σε βάσεις δεδομένων.
Τα εργαλεία εξόρυξης γνώσης, που έχουν εξελιχθεί ραγδαία τα τελευταία χρόνια,
αναγνωρίζουν οντότητες στο κείμενο, βοηθώντας στην επιμέλεια.
Για το σκοπό αυτό, αυτή η εργασία εστίασε στην διάσωση γνώσης θαλάσσιας ιστορικής βιοποικιλότητας και την
αξιολόγηση των εργαλείων για τον σχεδιασμό μιας ενοποιημένης μεθοδολογίας επιμέλειας. 
Πηγαίνοντας ένα βήμα παραπέρα, ένα νέο εργαλείο, το \textlatin{DECO}, αναπτύχθηκε
με στόχο την ενίσχυση της διαδικασίας διάσωσης δεδομένων στην έρευνα για τη βιοποικιλότητα.

Η μελέτη των οικοσυστημάτων του παρελθόντος και των σημερινών συνθηκών έχει κάνει εμφανή την ανάγκη για
δράσεις διατήρησης επειδή βρίσκονται υπό πολλαπλές απειλές.
Ένα παράδειγμα είναι η κατάρρευση των αρθροπόδων, μια παγκόσμια τεκμηριωμένη
τάση που έχει αντιμετωπιστεί ανεπαρκώς από την κοινωνία.
Η Κρήτη είναι ένα hotspot βιοποικιλότητας και η έρευνα για την πανίδα των αρθροπόδων της χρονολογείται από τον 19ο αιώνα.
Στην παρούσα εργασία αξιολογήθηκε η κατάσταση της διατήρησης των ενδημικών αρθρόποδων της Κρήτης χρησιμοποιώντας
μια αυτοματοποιημένη αξιολόγηση (\textlatin{PACA}) και η επικάλυψη των κατανομών των αρθρόποδων
στις προστατευόμενες περιοχές Νατούρα 2000. Επιπλέον 
να διερευνήθηκαν τα \textlatin{hotspot} ενδημικότητάς τους και προτάθηκαν υποψήφιες Βασικές Περιοχές Βιοποικιλότητας.
Για αυτές τις αναλύσεις, συγκεντρώθηκαν παρουσίες του ενδημικών αρθροπόδων της Κρήτης
που βρίσκονται στις συλλογές του Μουσείου Φυσικής Ιστορίας της
Κρήτη μαζί με στοιχεία από την εκτενή βιβλιογραφία. Η αξιολόγηση της κατάστασης διατήρησης με την μέθοδο PACA
κατέταξε το 75\% των ενδημικών αρθροπόδων ως πιθανά απειλούμενα.
Τα ενδημικά \textlatin{hotspots} και οι υποψήφιες βασικές περιοχές βιοποικιλότητας βρίσκονται κυρίως σε
ορεινές περιοχές, που συχνά επικαλύπτονται με προστατευόμενες περιοχές Νατούρα 2000.
Ωστόσο, οι ανθρώπινες δραστηριότητες επηρεάζουν αυτές τις περιοχές και ορισμένα αρθρόποδα δεν 
προστατεύονται επαρκώς.

Εκτός από την χλωρίδα και την πανίδα, τα μικρόβια επηρεάζουν σημαντικά το οικοσύστημα του εδάφους.
Όμως η αποσαφήνιση των λειτουργιών της μικροβιακής βιοποικιλότητας παραμένει πρόκληση.
Η νησιωτική βιογεωγραφία προσφέρει μια μοναδική ευκαιρία, για να προωθηθεί η κατανόηση
της ποικιλότητας των μικροβιωμάτων στο εδαφικό περιβάλλον.
Η Κρήτη, ως ηπειρωτικό νησί, έχει μια ξεχωριστή φυσική και εξελικτική ιστορία
με ακραίες αντιθέσεις στη βλάστηση, τις κλιματικές συνθήκες και τη γεωλογία.
Έχει μελετηθεί για αιώνες και λόγω της θέσης του στα νοτιοανατολικά
της Μεσογείου είναι σημαντική για τις μελέτες για την κλιματική αλλαγή.
Για το σκοπό αυτό, η πρώτη ολοκληρωμένη μελέτη μικροβιώματος εδάφους σε νησί διεξήχθη στο
νησί της Κρήτης το 2016. Αξιολογήθηκαν οι δείκτες ποικιλότητας του μικροβιώματος και ως αποτέλεσμα
φαίνεται να είναι ότι η μικροβιακή ποικιλότητα της Κρήτης επηρεάζεται από το \textlatin{pH} και
την υγρασία του εδάφους, σε σχέση και με το υψόμετρο. Αυτή η έρευνα συντονίστηκε από μια ομάδα ερευνητών η οποία συνέλεξε 435 δείγματα εδάφους
από 72 τοποθεσίες σε τέσσερις διακριτές οικολογικές ζώνες της Κρήτης σε μια μέρα. The \textlatin{Island Sampling Day} Κρήτη 2016
ενσωματώνει επίσης, μικροβιακά δεδομένα με δεδομένα εδάφους, χωρικά και τηλεπισκόπησης.
Βασικός στόχος είναι η ψηφιακή αναπαράσταση του εδάφους της Κρήτης για να αποκρυπτογραφήθούν
οι λειτουργίες των ποικίλων οικοσυστημάτων του νησιού. Οι περιοχές με μεγάλο υψόμετρο φιλοξενούν μεγάλη βιοποικιλότητα
μικροοργανισμών, αντικατοπτρίζοντας μοτίβα που παρατηρούνται σε άλλες ομάδες πανίδας όπως τα αρθρόποδα.
Η ενοποίηση χωρικών δεδομένων ανέδειξε πιθανά σήματα αρνητικής ανθρωπογενούς επίδρασης σε παρθένα οικοσυστήματα αλλά και 
βοσκοτόπια βοηθώντας στον εντοπισμό των παραγόντων της βιοποικιλότητας και στην αξιολόγηση των απειλών.

Συνοψίζοντας, η μελέτη της οικολογίας των οικοσυστημάτων με σύγχρονα εργαλεία και δεδομένα
έχει σημαντικές προοπτικές. Η συσσώρευση δεδομένων καταστεί εφικτή την
ψηφιακή και ολιστική αναπαράσταση των οικοσυστημάτων. Απαιτούνται ακόμη πολλά βήματα
για την υλοποίηση αυτή όμως υπάρχουν ευκαιρίες για να νέα ερωτήματα και νέες ιδέες
μέσω της ενσωμάτωσης. Η προστασία, η διατήρηση και η βασική έρευνα πλέον χρειάζεται να 
γίνονται ταυτόχρονα επειδή τα περιθώρια δράσης ενόψει της κατάρρευσης των
οικοσυστημάτων είναι στενά.
\end{document}
