% --------------------------------------------------
% 
% This chapter is for Crete system ecology
% 
% --------------------------------------------------


\chapter{Towards a Cretan soil biodiversity data model}
\label{cha:crete-idea}

%\textbf{Citation:} \\ 


% ISD ABSTRACT
%\section{Abstract}
%    Microbes are known for their versatility, abundance
%    and influence on soil ecosystem functioning.
%    A synthesized knowledge base of microbial biodiversity, in terms of
%    ecological and remote-sensing data remains a major challenge.
%    Many worldwide studies have been published regarding soil
%    microbiome ecosystems, though there are still many blind spots.
%    Islands can be important case studies for this integration for more resolute and dense samplings.
%    Here, we utilize the Island Sampling Day Crete 2016 microbial 16S rRNA gene
%    amplicon data, integrated with soil and remote
%    sensing data, to decipher the drivers of ecosystem function of the island.
%    The Island Sampling Day Crete 2016 project has collected 144 topsoil samples
%    from 72 sites, capturing a lot of this diversity, accompanied by FAIR
%    (Findable, Accessible, Interoperable and Reproducible) data by design. 
%    Cretan macroecology has been studied for centuries for its diverse  and endemic
%    fauna and flora.
%    In addition, Crete has been considered as a miniature continent with high contrasts in
%    vegetation cover, elevation, climatic conditions. 
%    We show that, higher altitudes in Crete found to
%    be inhabited by a more diverse number of microorganisms, a pattern commonly
%    seen in several faunistic groups, such as arthropods.
%    The integration of the spatial data with state of the art methods enabled warning signals
%    in pristine and grazing ecosystems.
%    These results along with the
%    climatic and desertification index influences on the soil microbiome of Crete,
%    provide the basis to identify major drivers of biodiversity, to evaluate hotspots
%    and contribute to foreknowledge of threatened ecosystems.
%
\section{Introduction}\label{intro_idea}

Examples of rich metadata and platforms of hosting open data and digital soil maps are the 
European Soil Data Centre \parencite{Panagos2022} and the World Soil Information
Service (WoSIS) of the ISRIC \parencite{Batjes2024}. Apart from samplings,
spatial data are openly available terrestrial ecosystems.
Climatic data, land cover, desertification risk, aridity, soil type, normalized
vegetation index, bedrock geological formations. From the bacterial point of view, 
there curated databases that classify in some baseline functionality. 




\section{Materials and Methods}\label{integration_methods}

\subsection{Literature}\label{crete-literature}

\subsection{Samplings}\label{crete_samplings}

\subsection{Spatial data}\label{crete_spatial}


\subsection{Tools}\label{Coding environment}
Visualisation was implemented with ggplot2 \parencite{wickham_ggplot2_2016} and pheatmap \parencite{Kolde2019}.
The environment we worked had Python 3.11.4, R version 4.3.2 \parencite{rcoreteam}.
Finally, computations were performed on HPC infrastructure of HCMR \parencite{zafeiropoulos_0s_2021}.

\subsection{Data and Code}
Scripts about data integration are available in
\href{https://github.com/savvas-paragkamian/crete-data-integration}{Crete data integration}.
Code is structured to be reproducible and interoperable.

\section{Results}\label{crete_idea_results}

\subsection{Historical and Contemporary literature}


\subsection{Biodiversity and samplings}

Worldwide projects of microbiome studies have collected one or two topsoil
samples from Crete \parencite{Vasar2022, Labouyrie2023, Bahram2018, Orgiazzi2018}.
Some have focused on soil fungi \parencite{Mikryukov2023, Davison2021, Tedersoo2021}
and other soil eukaryotes \parencite{Aslani2022}.
The only thorough microbiome study of a soil ecosystem in Crete, to our knowledge,
is in the north west part of the island, the Koiliaris Critical Zone Observatory \parencite{tsiknia2014}.
Apart from the soil microbiome, soil physical and chemical properties has been
investigated by global and European projects like the Forum of European Geological Surveys
(FOREGS) \parencite{nerc19017}, the Geochemical Mapping of Agricultural and Grazing Land
Soil in Europe (GEMAS) \parencite{REIMANN2018302} and the Soil Profile Analytical
Database for Europe (SPADE) \parencite{Hiederer2006}.
LTER map

\begin{figure}[h] 
    \centering\includegraphics[width=\columnwidth]{crete_integration_ena_terrestrial.png}
    \caption{Available terrestrial metagenomic samples from ENA in Crete.}
    \label{fig:isd_crete_ena}
\end{figure}


\begin{figure}[h] 
    \centering\includegraphics[width=\columnwidth]{crete_integration_map_wosi_soil.png}
    \caption{Soil samples from Crete that are uploaded in WoSIS.}
    \label{fig:isd_crete_wosis}
\end{figure}

GBIF map 


\subsection{Maps}

Multiple maps for Crete.


\section{Discussion}\label{crete_idea_discussion}

\section{Conclusion}



