% --------------------------------------------------
% 
% This chapter is for general conclusions
% 
% --------------------------------------------------


\chapter{Conclusions}
\label{cha:conclusions}

The work presented here has combined different approaches of contemporary 
ecological questions. Regarding microbial diversity, on the global scale the available 
knowledge was explored using literature and data mining and on the local scale
the soil microbial diversity of Crete was deciphered. Literature and data mining 
methodologies are also very useful to rescue historical biodiversity data which are
indispensable. This was demostrated through the DECO workflow, a collection
of tools and standards aiming to assist the curators' process. From the comparison
of such tools it became clear that human curation is a transerval undertaking in all steps.
Expert curation was applied for the compilation of historical and contemporary
literature along with speciments from NHMC for the endemic Cretan arthropods occurrences.
After the compilation of the dataset it became clear that a conservation analysis
was a priority because the majority of species was predicted as threatened.
Nevertheless, the investigation of the soil dwelling arthopods with the soil microbial diversity
and plants is crucial for soil functioning. 


\section{Integrative methods}
\label{sec:integration}

\section{ISD Crete 2022}
\label{sec:isd-crete-2022}

The replicability of the ISD Crete sampling was excersised in July of 2022.
Using the same protocols and locations, 29 people from HCMR, NHMC, UOC Biology
Department and citizen scientists where split in 10 teams and went sampling. 
The goal of this sampling was to collect a second time point of the same locations
to decipher the metagenomic content of soil. This was a voluntary work supported 
by the SUPP GEN project of HCMR. The DNA extraction and shipment was carried out 
by HCMR and sequencing by the Joint Genome Initiative. DNA extracted by the 72 locations 
is going to be sequenced using deep shotgun sequencing.

This is one of the few large scale metagenomic soil projects in
Europe \parencite{nayfach2021a-genomic, ma2023a-genomic}. Currently, 
the ambitious \href{https://www.embl.org/about/info/trec/}{TREC project} is
ongoing aiming to fill this gap with.

\section{Conservation}
\label{sec:conservation}

% 
