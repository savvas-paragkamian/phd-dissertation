% --------------------------------------------------
% 
% This chapter is for general conclusions
% 
% --------------------------------------------------

\chapter{Conclusions}
\label{cha:conclusions}

The work presented here has combined different approaches of contemporary 
ecological questions. Regarding microbial diversity, on the global scale the available 
knowledge was explored using literature and data mining and on the local scale
the soil microbial diversity of Crete was deciphered. Literature and data mining 
methodologies are also very useful to rescue historical biodiversity data which are
indispensable. This was demonstrated through the DECO workflow, a collection
of tools and standards aiming to assist the curators' process. From the comparison
of such tools it became clear that human curation is a transversal undertaking in all steps.
Expert curation was applied for the compilation of historical and contemporary
literature along with specimens from NHMC for the endemic Cretan arthropods occurrences.
After the compilation of the dataset it became clear that a conservation analysis
was a priority because the majority of species was predicted as threatened.
Nevertheless, the investigation of the soil dwelling arthropods with the soil microbial diversity
and plants is crucial for soil functioning. 

While analysing the ISD Crete 2016 we also exercised the replicability of the sampling in July of 2022.
Using the same protocols and locations, 29 people from HCMR, NHMC, UOC Biology
Department and citizen scientists where split in 10 teams and went sampling. 
The goal of this sampling was to collect a second time point of the same locations
to decipher the metagenomic content of soil. This was a voluntary work supported 
by the SUPP GEN project of HCMR. The DNA extraction and shipment was carried out 
by HCMR and sequencing by the Joint Genome Initiative. DNA extracted by the 72 locations 
is going to be sequenced using deep shotgun sequencing. This is one of the few large scale metagenomic soil projects in
Europe \parencite{nayfach2021a-genomic, ma2023a-genomic}. Currently, 
the ambitious \href{https://www.embl.org/about/info/trec/}{TREC project} is
ongoing aiming to fill this gap with.

Currently there is a wealth of available data and tools as demonstrated in 
multiple chapters in this PhD. Yet the basic conceptual challenges remain. 
Some of these can be formulated as : What are the causes of ecosystem collapse?
What is needed for a sustainable future?
How will climate warming change life on Earth?
These clear questions require what is called scientific transculturalism,
the process of integration of the three cultures—variance, coarse-graining, and exactitude \parencite{Enquist_2024}.
These cultures can be vaguely described as natural history, numerical ecology and complex systems ecology, respectively.
An important step to bring these cultures together is communication and openness across scientists.
These gaps must be eliminated soon to reach predictive ecology goals \parencite{mouquet_review_2015}.
