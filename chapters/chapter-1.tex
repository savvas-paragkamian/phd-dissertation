% --------------------------------------------------
% 
% This chapter is for Crete system ecology
% 
% --------------------------------------------------


\chapter{Island Sampling Day, the case of Crete}
\label{cha:isd-crete-soil}

%\textbf{Citation:} \\ 

\section{Introduction}\label{intro}

Microbes influence global ecosystem functions \parencite{falkowski2008microbial}
and are ubiquitous \parencite{delgado2016microbial}. The terrestrial biomass of bacteria is
70 Gt C, second to plants with sixfold difference \parencite{bar2018biomass}. In subsurface systems, bacteria and
archaea are the most abundant, having 90\% of the biomass. These findings of microbial
ecology are based on amplicon 16s rRNA studies that have flourished since the
2010s. The endeavor to understand the microbial world faces multiple challenges
across the scientific workflow, from sampling to ecological analyses \parencite{Lee2012}.

Amplicon rRNA sequencing provides a collection of sequence reads per sample. 
The ecological interpretation of the reads requires their transformation to
taxonomic information. To do so there are two approaches currently in use, 
the clustering method and the denoising method. With clustering reads are 
grouped together and a best representing sequence is produced for each 
cluster, i.e. the Operetional Taxonomic Unit. This approach makes the OTUs 
from different runs, i.e executions of the algorithm and/or different studies
incoperable and irreproducible. Currently, many studies propose the use of Amplicon Sequence Variants \parencite{Callahan2017}. 
ASVs are real biological sequences and can be used for comparison.
The influence of the different methods to subsequent ecological analyses has
little impact \parencite{Glassman2018}.

Errors propagate starting with the sampling. There are contaminations from the
people in the field, in the lab for the DNA extraction \parencite{EISENHOFER2019105}. 
Errors from PCR amplification and errors from sequencing \parencite{Schloss2011, Schimer2015}.
Approximation and computation errors from algoritms that cluster, measure similarities between
sequences. Semantic errors occur because of reductionist approaches and/or oversimplification
of the microbial communities. Furthermore, microbiome matrix data are compositional, 
meaning that the abundances of the reads are not the real representations of the natural populations \parencite{Gloor2017}.
These limitations are important to be acknowledged in order to avoid overarching conclusions,
especially in complex systems such as soil.

Bacteria and archaea are considered major drivers for the functionality of soil.
In soil ecosystems, the pillars of human activities and ecosystem services, 
there is a positive correlation between soil microbial diversity and their multifunctionality \parencite{Delgado-Baquerizo2020}.
The community structure defies their macroscopic functionality \parencite{Bahram2018}.
Soils are complex environments that are characterised chemical, physical and biological properties. 
These form interdependent associations with feedback loops by their environment. 
Global soil microbiome studies have been employed to decipher soil microbiome
compositions \parencite{thompson2017a-communal, Delgado-Baquerizo-atlas, Labouyrie2023},
functions \parencite{Bahram2018} and biogeography \parencite{Martiny2006, guerra2020Blind}.
The profound diversity of soils is clearly stated in these works, yet there are
significant blind spots \parencite{guerra2020Blind}.
When considering the samples per area density, there global studies are rather sparse.
National wide studies like the one by \parencite{Karimi2020} show the
complexity of soil bacteria and the need to design 
denser samplings and use isolated systems \parencite{Dini-Andreote2021}.

Isolated systems are important to avoid co-founding effects and to reduce complexity.
Islands are considered nature's labs and are used in biogeography extensively due to their smaller scale and isolation \parencite{Whittaker2017}. 
Island microbiomes from soil \parencite{Li2020} and mycorrhizal fungi \parencite{Delavaux2021} studies
show the benefits of using islands as models.
This was the case for the Island Sampling Day (ISD) Crete project \parencite{holm2024}
of the Genome Standards Consortium \parencite{Field2011}
during the 18th workshop in June 2016 in Crete island, Greece. The goal of this sampling was to "put standards into action"
in a soil microbiome survey with a dense sampling, 0.017 samples per km\textsuperscript{2}.
Hence, ISD Crete is a large scale study in the confined space of the island of Crete which 
is considered a miniature continent \parencite{Vogiatzakis2008_crete}.

The Island Sampling Day draws inspiration from the Ocean Sampling Day \parencite{kopf_2015} that is a
consortium that facilitates yearly samplings in stations from all around the world.
This has great applications in monitoring and comprehending the complexity of the ocean microbiome and it’s drivers.
The Island Sampling Day has been organised in two islands, Mo'orea in the French Polynisia and Crete in Greece.
In the ISD Crete project, 26 people divided in ten teams went across the island in a single day to sample as much diverse 
ecosystems of Crete. 
Samples were collected in a single day in order to limit the variations of environmental
factors that may influence the abundance and diversity of microbes such as season,
temperature and humidity.
For each sampling site, teams selected two specific sub-sites that were
at least 3 meters from the edge of the road and 0.6 m from the base of
the identified tree or plant. Flora located at the sampling site were
identified and photographed.
At each sub site, soil was sampled (3 replicates, collected one 3cm apart). 

The University of Maryland Soil Lab (Dr. Stephanie Yarwood) received 288 soil cores
144 for DNA extraction and 144 for physicochemical measurments. The shipment was on dry ice,
under the USDA PERMIT NUMBER: P330-16-00090 and a Material Transfer Agreement with HCMR.
The water content, Total Organic Carbon and Nitrogen weights were measured using LECO CN628 analysis platform.
DNA extraction of 144 soil samples used the protocols of the \href{https://www.protocols.io/view/emp-16s-illumina-amplicon-protocol-cpisvkee}{Earth Microbiome Project} (EMP)
with the the MoBio RNA extraction kit.Three of the samples, collected in sand, had no detectable DNA
following extractions.
The PCR-based protocol used is designed for the V3-V4 region of the 16S rRNA gene (V3 V4 primers: Forward: 5'-ACTCCTACGGGAGGCAGCAG-3'; Reverse: 5'-GGACTACHVGGGTWTCTAAT-3').
The sequencing of 16s rRNA amplicons were carried out on an Illumina HiSeq2500 platform.
at the Institute for Genome Sciences, Genomics Resource Center (Maryland Genomics) at the University of Maryland School of Medicine.

The Genome Standards Consortium uploaded the metadata and the sequences to ENA
database under the project \href{https://www.ebi.ac.uk/ena/browser/view/PRJEB21776}{PRJEB21776} immidiately upon realise. 
This is an act of unrestricted use of genomic data which exemplifies the importance of 
having public data even prior publications \parencite{kopf_2015}.

In this study, the ISD data and metadata were downloaded from ENA database analysed to decipher drivers of 
biodiversity and community structure of the top soil microbiome of Crete. Diversity, 
ordination and correlations of metadata were carried out. Following the examples of 
transparency of the GSC, all code is reproducible and documented as mentioned in FAIR principles \parencite{wilkinson2016the-fair}.
This is the first analysis 
of this scale of the Cretan soil microbiome, henceforth adding a new chapter of 
biodiversity in Crete, the microbiome.

\section{Materials and Methods}\label{methods}

The work purely computational; scripts of the analysis cover the following tasks:
\begin{itemize}
    \item Search ENA for samples in the island of Crete
    \item Get ISD metadata and sequences
    \item HPC jobs and parameter files
    \item Filtering, clustering/denoising and taxonomic assigninments
    \item Biodiversity and ecological analysis 
    \item Visualisation
\end{itemize}

The bioinformatic workflow can be summarized to : GET, INFER and ANALYZE as shown in Figure \ref{fig:isd_workflow_taxonomy}.

\begin{figure}[h]
      \centering
      \includegraphics[width=\textwidth,height=\textheight,keepaspectratio]{figures/isd_crete_flowchart.png}
      \caption[Reproducible workflow of ISD analysis]{Three steps of the ISD bioinformatics workflow, reproducible by design}
      \label{fig:isd_workflow_taxonomy}
   \end{figure}
   
\subsection{Data retrieval}\label{isd_data}

This study uses the ISD Crete data that have been deposited
in the European Nucleotide Archive (ENA) at EMBL-EBI under accession number PRJEB21776.
The raw sequences (fastq files) and metadata (xml files) were downloaded with custom scripts using the ENA API \parencite{Yuan2023}.

\begin{figure}[h] 
    \centering\includegraphics[width=\columnwidth]{isd_map_fig1-small}
    \caption{Crete ISD sampling. A. The routes of the single day event sampling. B. Beta diversity differences are represented with color.}
    \label{fig:isd_crete_sampling}
\end{figure}

\subsection{Taxonomic inference}\label{inference}
Amplicon Sequence Variants were inferred using DADA2 \parencite{Callahan2016} for 
filtering, denoising and chimeric reads removal. Normalisation of the reads
across samples was implemented with the SRS R package \parencite{Beule2020}. Samples
with less than 10000 reads were removed. ASVs were assigned to taxonomy using 
DADA2 with Silva 138 \parencite{quast_silva_2013}.
PEMA was used for OTU inference based on VSEARCH \parencite{zafeiropoulos2020pema}.

\subsection{Biodiversity analyses}\label{inference}

Numerical ecology analyses,e.g diversity indices, NMDS, PERMANOVA we calculated
with the vegan R package \parencite{oksanen2024vegan}.
For PCoA ordination we used ape R packege \parencite{Paradis2004} and UMAP python library\parencite{mcinnes2018umap-software}.
U-CIE R package was used for coloring 3 dimensional data \parencite{Koutrouli2022} to 
visualise the $\beta$ diversity differences of samples on the map.
Visualisation was implemented with ggplot2 \parencite{wickham_ggplot2_2016} and pheatmap \parencite{Kolde2019}.

\subsection{Environment and Code}

Computations were performed on HPC infrastructure of HCMR \parencite{zafeiropoulos_0s_2021}.
The programming environment was Debian 4.19 and conda environments were utilised
with Python 3.11.4, R version 4.3.2 \parencite{rcoreteam}.
Additionally GNU bash 5.0.3 and GNU Awk 4.2.1 were used for the streamlined workflow 
and for the reads statistics, respectively.

The documentation and scripts developed for this study are available in
\href{https://github.com/GenomicsStandardsConsortium/ISD}{ISD Crete soil microbiome github repository}.
This repository contains all the necessary scripts for the data retrieval,
filtering and ASV inference, taxonomy assignment, data integration of spatial data, 
functional annotation and the subsequent analyses and visualisation.
Code is structured to be reproducible and interoperable as described in Figure \ref{fig:isd_workflow_taxonomy}.


\section{Results}\label{results}

\subsection{Inference and taxonomy}\label{inference_taxonomy}

The Illumina HiSeq2500 libraries yielded 51 million sequenced reads (16-20 GB) with
an average of 355,326 reads/sample (range 2,437-525144). Sequences and the
metadata thereof were submitted to the ENA under the study accession number PRJEB21776.
Sequencing produced 51 million reads, with an average of 250-500 K reads per
sample (15-20 GB of data), with an average of 355,326 average reads per sample
(range: 2,437-525,144) and less than 15,000 reads for reads of either less than
445 bp or over 515 bp, Figure \ref{fig:isd_srs-curve_samples}.
   
   \begin{figure}[h]
      \centering
      \includegraphics[width=\textwidth,height=\textheight,keepaspectratio]{figures/isd_crete_srs_curve.jpeg}
      \caption[SRS curve]{SRS curve of the samples. }
      \label{fig:isd_srs-curve_samples}
   \end{figure}

\begin{figure}[h] 
    \centering\includegraphics[width=\columnwidth]{figures/isd_community_site_locations_dif.png}
\caption{Sample dissimilarity between samples in the same site (2 subsites) and the rest.}
    \label{fig:isd_site_locations}
\end{figure}

In summary, DADA2 resulted in 216,360 ASVs (2,704 unique
taxa; 1123 ASVs at species level and 1059 at genus level) and
PEMA (VSEARCH) in 13,285 OTUs.
The representative Phyla ( > 5\% presence in all samples) are:
Actinobacteriota, Proteobacteria, Chloroflexi, Acidobacteriota,
Bacterodota and Planctomycetota, Figure \ref{fig:isd_top_phyla_samples}.
There are 25 specialist (samples < 10 and mean relative
abundance > 0.003) and 146 generalist taxa (samples > 120), Figure \ref{fig:isd_fig2_taxonomy}.

\begin{figure}[hbt!] 
    \centering\includegraphics[width=\columnwidth]{isd_fig2_taxonomy}
    \caption{Taxonomic prevalence and representative phyla of the soil bacteria of Crete. 
    A. Distribution of ASVs samples proportion. B. Distribution of taxa samples
proportion and categorisation in generalists and specialists. C. Phyla samples proportion.
D. Phyla relative abundance box plots, each dot represents one sample that the phylum occurs.}
    \label{fig:isd_fig2_taxonomy}
\end{figure}

\begin{table}[]
    \caption{Taxonomic depth of ASVs and the unique number of taxa of each level.\label{table:asv_taxonomy}}%
\begin{tabular}{@{}lllll@{}}
classification depth & Total ASV    & Total (ASV) taxa & Total OTUs & Total (OTU) taxa\\
Kingdom              & 1974         & 2                & 284        & 2               \\
Phylum               & 4034         & 33               & 121        & 15              \\
Order                & 38517        & 193              & 1224       & 135             \\
Class                & 24157        & 83               & 978        & 62              \\
Family               & 71355        & 287              & 2319       & 218             \\
Genus                & 90137        & 1166             & 1920       & 582             \\
Species              & 9120         & 1338             & 44         & 43              \\
Total                & $\sim$239000 & 3102             & 6890       & 1057            
\end{tabular}
\label{table:asv_taxonomy}
\end{table}

\begin{figure}[hbt!]
      \centering
      \includegraphics[width=\textwidth,height=\textheight,keepaspectratio]{figures/isd_asv_taxonomy_ratios_top_phyla_samples.png}
      \caption[Top phyla of each samples]{The top phyla for all samples with their relative abundance}
      \label{fig:isd_top_phyla_samples}
\end{figure}
   


\subsection{Drivers and Communities}\label{communities}
Highest values of sample metadata:
ERR3697708, ERR3697732 (isd 4 site 8 loc 1, isd 6 site 2 loc 1) have the highest total nitrogen values (12.3, 7.9).
ERR3697703, ERR3697702 (isd 4 site 5 loc 2, isd 4 site 5 loc 1) have the highest water content values (141, 102).
ERR3697655, ERR3697675 (isd 1 site 2 loc 2, isd 2 site 5 loc 2) have the total organic carbon values (238, 177).


Regarding the OTUs of the Island Sampling Day:
The highest shannon diversity have the samples ERR3697693 (isd 3 site 4 loc 2) and
ERR3697675 (isd 2 site 5 loc 2) with 4.57 and 4.56, respectively.

ERR3697765 isd 7 site 11 loc 2 have the highest number of OTUs, 1075.
ERR3697765 isd 7 site 11 loc 2 has the second highest number of OTUs, 1051.

ERR3697703 isd 4 site 5 loc 2 has the most taxa summing at 871.
The ERR3697702 (isd 4 site 5 loc 1) has 869 taxa.

The ASV richness of the samples is correlated (0.25, Pearson correlation) with total organic carbon (p=0.003).
Taxa richness is negatively correlated (-0.30, Pearson correlation) with elevation (p=0.0003). 
Taxa richness is positevely correlated (0.26, Pearson correlation) with water content (p=0.002).

$\alpha$-diversity (Shannon Index) is not correlated significantly with any
by physical and chemical features.

Regarding the ASVs of the Island Sampling Day:



\begin{figure}[hbt!]
      \centering
      \includegraphics[width=\textwidth,height=\textheight,keepaspectratio]{figures/isd_abiotic_metadata_elevation_bin_boxplot.png}
      \caption[Elevation and metadata distributions]{The distributions of the available metadata across the elevation of samples}
      \label{fig:isd_elevation_metadata}
\end{figure}

\begin{figure}[hbt!]
      \centering
      \includegraphics[width=\textwidth,height=\textheight,keepaspectratio]{figures/isd_diversity_elevation_bin_boxplot.png}
      \caption[Elevation and diversity indices]{The distributions of asv, taxa and multiple $\alpha$ diversity indices across the elevation of samples}
      \label{fig:isd_elevation_metadata}
\end{figure}

Total nitrogen and water content increases with elevation \ref{fig:isd_elevation_metadata}. 

\begin{figure}[hbt!]
      \centering
      \includegraphics[width=\textwidth,height=\textheight,keepaspectratio]{figures/isd_diversity_elevation_bin_boxplot.png}
      \caption[Elevation and diversity indices]{The distributions of asv, taxa and multiple $\alpha$ diversity indices across the elevation of samples}
      \label{fig:isd_elevation_metadata}
\end{figure}


Microbial $\beta$ diversity is associated in different ways with physical and chemical
measurements. UMAP 1 ordination was largely driven by
elevation (F=23, p < 0.001). 
UMAP 2,
was significantly associated with soil moisture (F=90, p < 0.001).
PCoA 2 also positively correlated with both organic carbon and nitrogen.

UMAP , NMDS , PCoA


\begin{figure}[hbt!]
      \centering
      \includegraphics[width=0.7\textwidth,keepaspectratio]{figures/isd_asv_ordination_UMAP1_elevation_bin_boxplot.png}
      \caption[Elevation and UMAP1]{UMAP major axis and elevation bins}
      \label{fig:isd_elevation_umap1}
\end{figure}


\section{Discussion}\label{discussion}

Soil bacterial biodiversity is very complex, even sites a few meters apart can differ
significantly in their community composition, Figure \ref{fig:isd_site_locations}.
This is a fact that is sometimes neglected in worldwide studies and the island biogeography
paradigm can assist to remove clutter.

Apart from the profound diversity in soils, there is also high speciation and uniqueness. 
As shown in Figure \ref{fig:isd_fig2_taxonomy}, most ASVs occur in 1 or 2 or 3 samples.
This is different when compering with the ocean. When using the deepest taxonomic
level of ASVs is possible to identify the specialists and generalists \parencite{Barberan2012}. 
In addition, focusing on the phyla, we see a pattern looking like a phase transition, from 
rare phyla to phyla that dominate all samples, Figure \ref{fig:isd_fig2_taxonomy} C. Lastly, in Figure \ref{fig:isd_fig2_taxonomy} D,
and \ref{fig:isd_top_phyla_samples}, we found some distinct top phyla profiles of samples.

Elevation gradients of biodiversity are known since Humboldt's work \parencite{Rahbek2019} 
yet these patterns remain elusive regarding the soil microbiome \parencite{Looby2020, Siles2023}.
Mostly because it's difficult to isolate other co-founding effects \parencite{Nottingham2018}.
In our results elevation showed important distinction of taxa and of diversity. Yet more work is 
needed to explore the Asterousia mountain transect in isolation to avoid indirect influences of 
other variables.

Deciphering and validating the results presented here requires future work.
There are missing links of the drivers and community compositions. These 
gaps can be filled by sampling more data and/or by data integration methods.
The former create new knowledge and are important to continue but they 
require a lot of resources. The latter uses already available knowledge 
to enrich the microbial information (e.g traits) and samples metadata
(e.g. climatic, land use, other taxa occurring in the same area).

"A holistic perspective on soil architecture is needed as a key to soil functions" \parencite{philippot2024the-interplay}, is 
an important statement for future soil projects.
The pillar of data integration is the unrestricted open data across disciplines and 
the open source software. 


