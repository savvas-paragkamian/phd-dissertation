% --------------------------------------------------
% 
% This chapter is for DECO
% 
% --------------------------------------------------


\chapter{Automating the Curation Process of Historical Literature on Marine Biodiversity Using Text Mining: The DECO Workflow}
\label{cha:deco}


\textbf{Citation:} \\ 
Paragkamian Savvas, Sarafidou Georgia, Mavraki Dimitra, Pavloudi Christina,
Beja Joana, Eliezer Menashè, Lipizer Marina, Boicenco Laura, Vandepitte Leen,
Perez-Perez Ruben, Zafeiropoulos Haris, Arvanitidis Christos, Pafilis Evangelos, Gerovasileiou Vasilis

Shared co-first authorship.

DOI: \href{https://www.frontiersin.org/articles/10.3389/fmars.2022.940844}{10.3389/fmars.2022.940844}

\textit{For author contributions and supplementary material please refer 
    to the relevant sections. This is a modified version of the published 
    version, in terms of relevance, coherence and formatting.}


% DECO ABSTRACT
\section{Abstract}

Historical biodiversity documents comprise an important link to the long-term data 
life cycle and provide useful insights on several aspects of biodiversity research 
and management. However, because of their historical context, they present 
specific challenges, primarily time- and effort-consuming in data curation. 
The data rescue process requires a multidisciplinary effort involving four tasks: 
(a) Document digitisation (b) Transcription, which involves text recognition and 
correction, and (c) Information Extraction, which is performed using text mining 
tools and involves the entity identification, their normalisation and their 
co-mentions in text. Finally, the extracted data go through (d) Publication to 
a data repository in a standardised format. Each of these tasks requires a 
dedicated multistep methodology with standards and procedures. During the past 
8 years, Information Extraction (IE) tools have undergone remarkable advances, 
which created a landscape of various tools with distinct capabilities specific
to biodiversity data. These tools recognise entities in text such as taxon names, 
localities, phenotypic traits and thus automate, accelerate and facilitate 
the curation process. Furthermore, they assist the normalisation and mapping 
of entities to specific identifiers. This work focuses on the IE step (c) from 
the marine historical biodiversity data perspective. It orchestrates IE tools 
and provides the curators with a unified view of the methodology; as a result 
the documentation of the strengths, limitations and dependencies of several 
tools was drafted. Additionally, the classification of tools into Graphical 
User Interface (web and standalone) applications and Command Line Interface 
ones enables the data curators to select the most suitable tool for their needs, 
according to their specific features. In addition, the high volume of already 
digitised marine documents that await curation is amassed and a demonstration 
of the methodology, with a new scalable, extendable and containerised tool, 
“DECO” (bioDivErsity data Curation programming wOrkflow) is presented. DECO’s 
usage will provide a solid basis for future curation initiatives and an 
augmented degree of reliability towards high value data products that allow 
for the connection between the past and the present, in marine biodiversity research.

% DECO INTRODUCTION
\section{Introduction}
\label{sec:deco-intro}
Species’ occurrence patterns across spatial and temporal scales are the 
cornerstone of ecological research \cite{levin_problem_1992}. The compilation of both past 
and present marine data to a unified census is crucial to predict the future of 
ocean life \cite{ausubel_guest_1999, anderson_does_2006, lo_brutto_historical_2021}. This compilation has 
been attempted by big collaborative projects, like 
Census of Marine Life\footnote{http://www.coml.org/} \cite{vermeulen_understanding_2013}, 
that follow metadata standards and guidelines \cite{michener_nongeospatial_1997, wilkinson_fair_2016} 
and modern web technologies \cite{michener_ecological_2015}. The project has resulted in the incorporation 
of census data from the past, i.e. historical data, to modern data platforms, such 
as the Ocean Biodiversity Information System (OBIS) \cite{klein_obis_2019}, which feeds 
the Global Biodiversity Information Facility (GBIF) \cite{noauthor_gbif_nodate}. The transformation of 
historical data to modern standards is necessary for their rescue (data archaeology)
from decay and inevitable loss \cite{bowker_biodiversity_2000}.

Historical data are usually found in the form of (a) historical literature and 
(b) specimens stored in biodiversity museum collections \cite{rainbow_marine_2009} (the 
digital transformation process and progress of specimens is reviewed by \cite{nelson_history_2019}). 
Historical biodiversity documents (also known as legacy, ancient or simply old 
documents) comprise literature from 1000 AD until 1960 and therefore are stored 
in an analogue and/ or obsolete format \cite{lotze_historical_2009, beja_chapter_2022}. 
These old documents can be found in institutional libraries, publications, books, 
expedition logbooks, project reports, newspapers \cite{faulwetter_emodnet_2016, 10.3897/BDJ.4.e11054, kwok_historical_2017} 
or other types of legacy formats (e.g. stored in floppy disks, microfilms or CDs).

From the scientific point of view, historical biodiversity data are as relevant 
as modern data \cite{GRIFFIN2019238,beja_chapter_2022}.
They are valuable for studies on biodiversity loss \cite{stuart-smith_thermal_2015,goethem_biodiversity_2021},
as forming baseline studies for the design of future samplings \cite{rivera-quiroz_extracting_2019}
and for predictions of future trends \cite{mouquet_review_2015}. Furthermore, 
historical data offer the kind of evidence needed for conservation policy and 
marine resource management, allowing for past patterns and processes to be
compared with current ones \cite{fortibuoni_coding_2010,https://doi.org/10.1111/j.1755-263X.2012.00253.x, costello_biodiversity_2013,engelhard_ices_2016}.
Hundreds of historical marine data held in documents have already been uploaded
to OBIS, yet a Herculean effort is required to curate the thousands of available
documents of the Biodiversity Heritage Library (BHL) \cite{gwinn_biodiversity_2009}
and other repositories.

Adequate and interoperable metadata are equally necessary and have to be
curated alongside data \cite{heidorn_shedding_2008,mouquet_review_2015}.
In this context, standards and guidelines have been recently formulated in 
policies as Findable, Accessible, Interoperable and Reusable (FAIR) 
(meta)data \cite{wilkinson_fair_2016,reiser_fair_2018}. Identifiers and 
semantics are used to accomplish the interoperability and reusability of
biodiversity data as well as the monitoring of their use \cite{mouquet_review_2015}.
Indispensable to the curation process of marine data have been the standards of
the Biodiversity Information Standards \footnote{https://www.tdwg.org/}, more
specifically Darwin Core \cite{wieczorek_darwin_2012}
and vocabularies such as those included in the International Commission on
Zoological Nomenclature \footnote{https://www.iczn.org/}, the World Register of
Marine Species \footnote{http://www.marinespecies.org/}
(WoRMS) \cite{horton_world_2022}, 
the Environment Ontology \footnote{https://sites.google.com/site/environmentontology/home} (ENVO) \cite{buttigieg2016environment}
and Marine Regions \footnote{https://www.marineregions.org/} \cite{claus_marine_2014}.
These standards and vocabularies and their adoption by biodiversity initiatives
like GBIF and OBIS align with the goal of marine biodiversity Linked Open Data
and support their interoperability and reusability \cite{page_towards_2016, penev_openbiodiv_2019,zarate_lobd_2021}.

The rescue process of historical biodiversity documents can be summarised in
four tasks Figure \ref{fig:rescue-workflow}. 
The first task is the digitisation of the document, which involves locating 
and cataloguing the original data sources, imaging/scanning with specific 
equipment and standards and uploading them to digital libraries \cite{lin_quality_2006, thompson_moving_2013}.
In the second task, the images are analysed with text recognition software,
mainly through Optical Character Recognition (OCR)
(for standards see \cite{groom_improved_2019} and for reviews
see \cite{lyal_digitising_2016, 10.3897/rio.6.e58030}.
Text recognition errors are then corrected manually by professionals or
citizen scientists \cite{herrmann_building_2020}. The third task is named
Information Extraction (IE) as it involves the steps of named entity
recognition, mapping and normalisation of biodiversity information \cite{hakenberg_applications_2012}.
Here, the curators may compile a species’ occurrence census enriched with
metadata of the study, geolocation, environment, sampling methods and traits
among others \cite{faulwetter_emodnet_2016}.
Lastly, the fourth task, is the data publishing to online biodiversity
databases/repositories \cite{costello_biodiversity_2013,penev_strategies_2017}.
Expert manual curation is a cross-cutting action through all the aforementioned
tasks for quality control and stewardship \cite{vandepitte_fishing_2015}.
This article focuses on the tools and curation procedures encompassed in
the third and fourth tasks described above.
   \begin{figure}[h]
      \centering
      \includegraphics[width=\textwidth,height=\textheight,keepaspectratio]{figures/deco-figure-1.jpg}
      \caption[Historical document rescue process]{Summarised process of historical document rescue. Four tasks are required to complete the data rescue process of biodiversity documents. Each of these has several steps, methodology, tools and standards. Curation is needed in every task, for tool handling and error correction. The stars represent the 5-star ranking system of Linked Data as introduced by W3C \footnotemark \cite{heath_linked_2011}. Availability of information from historical data increases as the curation tasks are completed (as exemplified by the fan on the right). Icons used from the Noun Project released under CC BY: book by Oleksandr Panasovskyi, scanning by LAFS, Book info by Xinh Studio, Library by ibrandify, Scanner Text by Wolf Böse, Check form by allex, Whale by Alina Oleynik, Fish by Asmuh, tag code vigorn, pivot layout by paisan, Certificate by P Thanga Vignesh, web service by mynamepong.}
      \label{fig:rescue-workflow}
   \end{figure}

   \footnotetext{\url{https://dvcs.w3.org/hg/gld/raw-file/default/glossary/index.html\#x5-star-linked-open-data}}

   \begin{figure}[h]
      \centering
      \includegraphics[width=\textwidth,height=\textheight,keepaspectratio]{figures/deco-figure-2.jpg}
      \caption[Common problems encountered in historical data]{Common problems encountered in historical data, such as old ligatures, absence of taxon names, ambiguous symbols, shortened words and descriptive information instead of numerical (page 185 from \cite{WoRMS:SourceID:40714})}
      \label{fig:historical-data-problems}
   \end{figure}

Several factors may turn the curation of historical documents into a
serious challenge \cite{faulwetter_emodnet_2016,beja_chapter_2022}.
Errors from the first and second tasks, as presented in Figure 
\ref{fig:rescue-workflow} (i.e. bad quality imaging, mis-recognised characters etc.)
are propagated through the whole process. In terms of georeferencing
constraints, location names or sampling points on an old map may be provided
instead of the actual coordinates. Additionally, taxonomic constraints
(e.g. old, currently unaccepted synonyms, lack of authority associated with the
taxon names) combined with the absence of taxonomic literature or voucher
specimens (e.g. identifier number for samples of natural history/expedition
collections) require the taxonomists’ assistance. Numerical measurement units
often need to be converted to the International System of Units (SI system)
(e.g. fathoms to metres)\cite{calder_proposal_1982,wieczorek_darwin_2012}.
Old toponyms and political boundaries that have now changed should also be
taken into consideration, as well as coordinates that now fall on land instead
of in the sea, due to the changes in the coastline. Lastly, the use of
languages other than English is quite common in old scientific publications, so
multilingual curators are required. Some of the aforementioned issues are
presented in Figure \ref{fig:historical-data-problems}. Because of these
limitations, the manual curation of data and metadata is mandatory when it
comes to historical data \cite{faulwetter_emodnet_2016}.

Manual curation, a tedious and multistep process, requires substantial effort
for the correct interpretation of valuable historical information; however,
text mining tools appear to be promising in assisting and accelerating this
part of the curation process \cite{alex_assisted_2008}. Text mining is the
automatic extraction of information from unstructured data
\cite{hearst_untangling_1999,10.5555/1199003}. These mining tools build upon
standardised knowledge, vocabularies, dictionaries and perform multistep
Natural Language Processes. Named Entity Recognition (NER) is a key step in
this process for locating terms of interest in text \cite{perera_named_2020}.
The entities of interest for biodiversity documents include: (1) taxon names,
(2) people’s names \cite{page_text-mining_2019,groom_people_2020},
(3) environments/ habitats \cite{pafilis_environments_2015,pafilis_extract_2017},
(4) geolocations/ localities \cite{alex_adapting_2015,stahlman_geoparsing_2019},
(5) phenotypic traits/morphological characteristics \cite{thessen_automated_2018}
(6) physico-chemical variables, and (7) quantities, measurement units and/or
values. Subsequent steps include the relation extraction between entities.
Multiple tools have emerged to retrieve a single or a collection of these
entities in the past few years \cite{batista-navarro_text_2017,10.3897/BDJ.7.e28737,dimitrova_pensoft_2020,le_guillarme_taxonerd_2022}.

The work described in this document has a threefold structure: (a) the
abundance of marine historical literature digitised/available for curation
is attempted to be estimated; (b) bioinformatics tools, focusing on automating
and assisting the curation process for these documents, are compiled/reviewed.
Two categories of such curation software are described: (i) the first one
relies on web and standalone applications with Graphical User Interface (GUI)
and the second (ii) combines Command Line Interface (CLI) programming libraries
and software packages; lastly, (c) a demonstrator biodiversity data curation
workflow, named DECO (bioDivErsity data Curation programming wOrkflow \footnote{https://github.com/lab42open-team/deco}),
developed using programming tools, is presented.

% DECO METHODS
\section{Method}
\label{sec:deco-method}

    \subsection{Historical Literature Discovery}
A search was conducted on BHL to amass the historical literature on BHL
regarding marine biodiversity. Using the keywords “marine”, “ocean”, “fishery”,
“fisheries” and “sea” on the items’ titles and their subjects (the scripts,
results and documentation are available in this
repository \footnote{https://github.com/savvas-paragkamian/historical-marine-literature})
the documents available for information extraction were estimated. Subjects are
categories provided for each title and multiple subjects can be assigned to
each title. The items that were originally published before 1960 were selected,
in order to include only historical documents, according to the definition
included in the Introduction section. Furthermore, the taxon names on each page,
which were identified by BHL using the Global Names parser tool \cite{mozzherin_gnamesgnfinder_2022},
were summarised for every document. Hence, summaries of the number of
automatically identified taxon names were calculated along with the page number
for each item. Additionally, OBIS’ historical datasets originally published
before 1960 were downloaded and analysed. This analysis provides an
approximation of the size of available marine historical literature compared to
the already rescued documents. All analysis scripts were written in GNU AWK
programming language and the visualisation scripts were written in R using
the ggplot2 library \cite{wickham_ggplot2_2016}.


    \subsection{Historical Document Rescue Methodology}
    Data curators thoroughly read each page of a document and insert the data
into spreadsheets, mapping them to Darwin Core terms, adding metadata and
creating a standard Darwin Core Archive \footnote{https://manual.obis.org}.
This whole process, which is mostly manual,means reading the information
(e.g. the occurrence of a specific taxon and its locality) and inputting it
through typing to the corresponding cell of the data file. It is, as expected,
a time- and resource-consuming procedure. Taxon names, traits, environments
and localities can be identified as well and the transformation of these
results to database identifiers (IDs), like Life Science Identifier
(LSID) \footnote{http://www.lsid.info/} of
Aphia IDs \footnote{https://www.marinespecies.org/aphia.php?p=webservice},
Encyclopedia of Life \footnote{https://eol.org/} (EOL)
IDs \cite{parr_encyclopedia_2014}, Marine regions gazetteer IDs,
marine species traits \footnote{https://www.marinespecies.org/traits/} among
others, can be facilitated through web applications and programming software.
The Natural Environment Research Council
\footnote{\url{https://www.bodc.ac.uk/resources/products/web_services/vocab/}}
Vocabulary Server, developed and hosted by the British Oceanographic Data
Centre \footnote{https://www.bodc.ac.uk/} was used for mapping facts and
additional measurements included in documents.

   \begin{figure}[h]
      \centering
      \includegraphics[width=\textwidth,height=\textheight,keepaspectratio]{figures/deco-figure-3.jpg}
      \caption[curation process of marine historical documents]{The curation process of marine historical biodiversity documents: on the left column are the required steps starting from the scanned document (usually a PDF file) and ending with the data publishing step. Two approaches are presented: in the middle column are the GUI tools whereas on the right column are the CLI and/or the executable programming tools. Note that the list of given examples is non-exhaustive. Icons used from the Noun Project released under CC BY: Whale by Alina Oleynik, Fish by Asmuh, tag code by vigorn, pivot layout by paisan, Certificate by P Thanga Vignesh, web service by mynamepong}
      \label{fig:curation-process}
   \end{figure}

Tools assist curators in this process for the NER, Entity Mapping, data
structure manipulation and finally data upload steps. Curation tools can be
categorised as GUI applications (computer programs and web applications) and
CLI applications (interconnected programming tools, libraries and packages)
(Figure \ref{fig:curation-process}). As an example, multiple page documents can
be searched for taxon names in seconds, with technologies that find synonyms
and fuzzy search for the OCR transformation misspelling. The interconnection
and guidance of these steps still requires human interaction and correction.

GUI applications are standalone applications or web applications, the latter
support document upload and, once they are processed in a server, the results
are delivered back to the user \cite{lamurias_text_2019}. CLI tools include
programming packages and libraries of any programming language in UNIX (Linux
and Mac operating systems - OS) and Windows OS. Even though programming
packages and libraries are fast and scalable they require familiarity and
expertise in CLI and programming which, on the other hand, takes effort and
time because of its initial learning curve. The CLI tools,
Application Programming Interfaces (APIs) and programming packages chosen during
this study are open-source, are in active development, can process many
documents and can be combined with other tools in some of the considered steps.

   \begin{figure}[h]
      \centering
      \includegraphics[width=\textwidth,height=\textheight,keepaspectratio]{figures/deco-figure-4.jpg}
      \caption[Scanned page of the dataset used]{ A screenshot of the dataset used where the structure of the data and metadata provided can be seen (page 180 from \cite{WoRMS:SourceID:40714})}
      \label{fig:forbes-screenshot}
   \end{figure}

    \subsection{Case Study}
The historical document “Report on the Mollusca and Radiata of the Aegean Sea:
and on their Distribution, Considered as Bearing on Geology” by
\cite{WoRMS:SourceID:40714} and its curated dataset were used as a case study
for the tool usage description and evaluation (where applicable). More
specifically, the six page long Appendix No. 1 (pages 180-185) document has
been manually curated and published, thus serving as a golden standard
(Figure \ref{fig:forbes-screenshot}). It was digitised and transcribed on
2009-04-22 by the Internet Archive \footnote{https://archive.org/details/reportofbritisha43cor}
and on 2021-09-30 it was manually curated \cite{mavraki_digitization_2021} and
published in MedOBIS \footnote{\url{ https://www.lifewatchgreece.eu/?q=content/medobis}}
\cite{arvanitidis_medobis_2006}. The rescue process resulted in a Darwin Core
Archive file with 530 occurrence records, 17 unique sampling stations and
260 taxa, covering 217 species. The effort required from the information
extraction task to data publishing was roughly 50 working days (8 hours per day)
by a single data curator.

    \subsection{Tool Usability Evaluation}
The web applications mentioned in this work were tested in November 2020 in 
two web browsers, Mozilla Firefox version 83 and Google Chrome version 87,
both on Microsoft Windows 10 and MacOS 10.14.

    \subsection{Demonstrator}

DECO was developed for the automation of biodiversity historical data
curation. Its workflow combines image processing tools for scanned historical
documents OCR with text mining technologies. It extracts biodiversity entities
such as taxon names, environments as described in ENVO and tissue mentions.
The extracted entities are further enriched with marine data identifiers from
public APIs (e.g. WoRMS) and presented in a structured format as well as in
report format with automated visualisation components. Furthermore, the
workflow was implemented as a Docker container to ease its installation and its
scalable application on large documents. DECO is under the GNU GPLv3 licence
(for 3rd party components separate licences apply) and is available via the
GitHub repository (\url{https://github.com/lab42open-team/deco}).


% DECO RESULTS
\section{Results}
\label{sec:deco-results}

   \subsection{Historical Literature Discovery}

Marine literature analysis on BHL holdings revealed that there are
1,627 different digital items that contain at least 100 distinct taxa to a
maximum of 10,000 taxa, as identified automatically from the Global Names
GNfinder tool. These items cover the period from 1558 to 1960, contain 648,927
pages, written in 10 different languages, 80\% of which being English. An
absolute estimation of historical marine data is difficult to be made as
several more documents are stored locally in legacy formats.

The rescued historical marine data uploaded on OBIS are 223 datasets, published
from 1753 to 1960. Hence, the manual curated literature is much lower than the
available digitised documents. These rescued biogeographical datasets cover
46,000 species and 38 phyla that contain about 1.5 million occurrences at the
species level.

    \subsection{Bioinformatics Tools Compilation and Review}

This section describes the tools used in the curation workflow
(Figure \ref{fig:curation-process}). In each step, the main up-to-date
programming tools, web services and applications, used for the extraction of
biodiversity data, are presented. These curation tools are listed, accompanied
with features such as extracted information, input format and their interface
in Table \ref{table-tools}.


\begin{table}[]
\LARGE
\resizebox{\textwidth}{!}{%
\renewcommand{\arraystretch}{2}%
\begin{tabular}{lllll}
\hline
    \textbf{Tool} & \textbf{Curation Step} & \textbf{Input} & \textbf{Interface} & \textbf{Reference} \\
    \hline
Global Names Recognition and Discovery & NER - Taxon names & User query, Free text, PDF or image & WA, API, CLI & Pyle (2016) \\
BOM (Biodiversity Observations Miner) & \begin{tabular}[c]{@{}l@{}}OCR\\ NER - Taxon names, Biotic interactions, Traits\end{tabular} & User query, Free text, PDF & WA, API & Muñoz et al. (2019) \\
TextAnnotator & NER - Generic Annotations & User query, Free text & WA & Abrami et al. (2021) \\
Pensoft Annotator & \begin{tabular}[c]{@{}l@{}}NER - Annotation of free text with ontology terms\\ Entity Mapping\end{tabular} & User query, Free text & WA, API & Dimitrova et al. (2020) \\
Taxon Finder & NER - Taxon names & User query, Free text & WA, API &  \\
EXTRACT & NER - Taxon names, Environments and Tissue & Free text & API, CLI & Pafilis et al. (2017) \\
TaxoNerd & NER - Taxon names & Free text, PDF, png & CLI & Le Guillarme and Thuiller (2022) \\
Stanford NER & NER - People, organisation, locality & Free text & CLI & Finkel et al. (2005) \\
Clear Earth & NER - Locality, unit, value, functional traits, taxon names & Free text & CLI & Thessen et al. (2018) \\
BioStor & \begin{tabular}[c]{@{}l@{}}Literature identification, \\ NER - geolocation\end{tabular} & Taxon names and other keywords & WA & Page (2011) \\
Marine Regions Gazetteer & Entity mapping & User input & WA, API & Claus et al. (2014) \\
Edinburgh geoparser & \begin{tabular}[c]{@{}l@{}}NER - geolocation\\ Entity mapping\end{tabular} & Free text & CLI & Alex et al. (2015) \\
Ontobee & Entity mapping & User input & WA & Xiang et al. (2011) \\
WoRMS taxon match & Entity mapping & Taxon list on comma separated / spreadsheet file & WA, CLI, API & WoRMS Editorial Board (2022) \\
worrms R package & \begin{tabular}[c]{@{}l@{}}Entity mapping\\ Data transformations\end{tabular} & Taxon list: comma / tab separated file & CLI, API & Chamberlain (2020) \\
Taxize R package & \begin{tabular}[c]{@{}l@{}}Entity mapping\\ Data transformations\end{tabular} & Taxon list: comma / tab separated file & CLI, API & Chamberlain and Szöcs (2013) \\
GloBI nomer tool & \begin{tabular}[c]{@{}l@{}}Entity mapping\\ Data transformations\end{tabular} & Tab separated file & CLI & Poelen and Salim (2022) \\
 &  &  &  &  \\
OpenRefine & Data transformations, Quality control & Spreadsheet files, Comma / tab separated files, XML, RDF, JSON, SQL database & GUI app & Verborgh and Wilde (2013) \\
LifeWatch Belgium \& EMODnet Biology QC tool & Quality control & IPT or a DwC-A file & WA &  \\
LifeWatch Belgium Data Services & Quality control, Entity mapping & Comma / tab separated file, spreadsheet excel file & WA &  \\
EMODnetBiocheck & Quality Control & IPT, comma / tab separated file & CLI & De Pooter and Perez-Perez (2019) \\
GBIF Data Validator & Quality control & Comma separated, IPT or a DwC-A file & WA, CLI, API &  \\
Obistools R package & Entity Mapping, Data transformations, Quality control & Free text, comma / tab separated file & CLI & Provoost et al. (2019) \\
IPT server nodes & Quality control, Data Upload & Comma / tab separated file & GUI app & Robertson et al. (2014) \\
GoldenGate-Imagine & OCR, NER, Entity mapping & PDF & GUI app & Sautter et al. (2007) \\
DECO & OCR, NER, Entity mapping, & PDF, png, free text & CLI & This work
\end{tabular}%
}
\caption{Functions, interface and curation step of the tools tested in this work.}
\label{table-tools}
\end{table}

   \subsubsection{Named Entity Recognition}
   The Global Names Recognition and Discovery\footnote{https://gnrd.globalnames.org/}
(GNRD) tool, within Global Names Architecture
\footnote{http://globalnames.org/} (GNA), is a web application used for the
recognition of scientific names. It can use files such as PDF, images or
Microsoft Office documents and one can still input URLs or even free-form text.
It supports OCR transformation from PDF files using the tool
Tesseract\footnote{\url{https://github.com/tesseract-ocr/tesseract}} and uses
the GNfinder\footnote{https://github.com/gnames/gnfinder} discovery engine, in
order to provide the list of names. It offers an API and can be installed
locally. GNA is also used by the BHL platform to locate taxonomic names within
the pages of its collections \cite{richard_improving_2020}.

The test performed on the \cite{WoRMS:SourceID:40714} six-page PDF template
provided 128 unique scientific names at species level, out of the 218
identified through the manual curation
(Figure \ref{fig:gnrd-screenshot})\todo{where to put the supplementary figures}. WoRMS Aphia IDs
\cite{vandepitte_fishing_2015,martin_miguez_european_2019} are widely used and
included in GNRD.

The Biodiversity Observation Miner \footnote{\url{https://fgabriel1891.shinyapps.io/biodiversityobservationsminer/}}
(BOM) is a web application based on R Shiny \footnote{https://shiny.rstudio.com/},
also available on GitHub \footnote{https://github.com/fgabriel1891/BiodiversityObservationsMiner},
that allows for the semi-automated discovery of biodiversity observations (e.g.
biotic interactions, functional or behavioural traits and natural history
descriptions) associated with the species scientific names \cite{10.3897/BDJ.7.e28737}.
It uses the GNfinder discovery engine through the R package
taxize \footnote{https://github.com/ropensci/taxize} \cite{chamberlain_taxize_2013}.
BOM is still under development (April 2022) and an OCR processed PDF file must
be used as input. The novelty of this tool is the provision of text snippets
(Figure \ref{fig:bom-screenshot})\todo{another S figure} and the co-occurrence of words, accompanied
with their count, to inform curators for terms that appear together in the
document.

TextAnnotator\footnote{http://www.textannotator.texttechnologylab.org/},
provided by the specialised information service BIOfid\footnote{https://biofid.de/en/},
is focused on information extraction about taxon names of vascular plants,
birds, moths and butterflies, location and time mentioned in German texts
\cite{driller_workflow_2018,driller_fast_2020}. This could be extended to
other environments, languages and taxonomic groups with the BIOfid Github
page\footnote{\url{https://github.com/FID-Biodiversity/BIOfid/tree/master/BIOfid-Dataset-NER}}
serving as the starting point. The TextAnnotator - in beta version - accepts
web pages or free text. Evidence of recent use of this tool was found in \cite{driller_fast_2020}.

The Pensoft Annotator\footnote{https://annotator.pensoft.net/} is another beta
web application that works with ontologies \cite{dimitrova_pensoft_2020}
(Figure \ref{fig:pensoft-annotator-screenshot})\todo{another s figure}. The Pensoft Annotator has Relation
Ontology\footnote{\url{https://github.com/oborel/obo-relations}} (RO) and ENVO
built in but it is extendable to any ontology with curation modifications for
stopwords. The character limitation, however, can be expanded upon
communication with the tool’s administrators.

Taxonfinder\footnote{http://taxonfinder.org/} is a web application for the
extraction of scientific names mentioned in web pages. It features an API that
was used in BHL for large scale annotations of taxonomic names until 2019,
when it was replaced by GNfinder \cite{richard_improving_2020}.

The most notable NER tool, with CLI, for taxon names is the Global Names Finder
(GNfinder) \cite{pyle_towards_2016,mozzherin_gnamesgnfinder_2022} which
provides fuzzy search and is the underlying engine of most biodiversity text
mining tools. It is in active development, deeming it a reliable tool for this
work. The main command line tool is gnfinder find which returns two arrays
(metadata and names). The metadata are the language, date of the execution of
the command and total number of words. The data have one entry per identified
string which contains the matched string, the returned name and the positional
boundaries in character sequence.

In order to simultaneously extract taxa, environment and tissue mentions, the
tool EXTRACT\footnote{https://extract.jensenlab.org/} \cite{pafilis_extract_2017}
implements the JensenLab tagger API \cite{jensen2016one} with advanced
dictionaries SPECIES-ORGANISMS\footnote{https://species.jensenlab.org} \cite{pafilis_species_2013},
ENVIRONMENTS\footnote{https://environments.jensenlab.org}
\cite{pafilis_environments_2015} and TISSUES\footnote{https://tissues.jensenlab.org/About}
\cite{palasca_tissues_2018}. It returns NCBI Taxonomy IDs \cite{schoch2020ncbi},
ENVO terms and BRENDA IDs\footnote{https://www.brenda-enzymes.org/},
respectively to a file with 3 columns: tagged text, entity type and term ID.
TaxoNERD \cite{le_guillarme_taxonerd_2022}, using Deep neural networks, scores
higher than other NER tools on taxon name recognition based on golden standard
corpora.

An important NER system is the Stanford NER\footnote{\url{https://nlp.stanford.edu/software/CRF-NER.html}}
\cite{finkel_incorporating_2005} which recognises locations, persons and
organisations in text. It has a generic scope but it can also assist in the
curation of biodiversity data. The general tokenisation and normalisation
procedures developed by the NLP Stanford team are the basis of many text mining
tools. Additionally, the ClearEarth \footnote{http://github.com/ClearEarthProject/ClearEarthNLP}
project \cite{thessen_automated_2018} can tag biotic and abiotic entities,
localities, units and values in text and is built using the ClearTK NLP
toolkit \footnote{http://cleartk.github.io/cleartk/} \cite{bethard_cleartk_2014}.
Upon installation it downloads multiple dictionaries and takes up to six
gigabytes of space. It relies on Stanford NLP and other dependencies and
provides a Python wrapper and a CLI.

A common constraint in historical documents is the lack of coordinates from the
sampling areas, so the data curator should provide the coordinates using the
toponyms given. There are tools that enable this procedure, such as Marine
Gazetteer. BioStor-Lite map \footnote{https://biostor.org/map.php}, which
contains automated geolocation annotation of BHL documents
\cite{page_text-mining_2019}, displays the points on the global map providing
the user the ability to search for additional documents with selected points on
the map or by drawing rectangles. The Edinburgh geoparser
\cite{alex_adapting_2015}, a command line tool, recognises places in text and
is one of very few tools to have this functionality. The Stanford NER system
has been used as well \cite{stahlman_geoparsing_2019} upon receiving training,
for geolocation recognition.

   \subsubsection{Entity Normalisation and Mapping}
   Mapping the information retrieved from the NER tools to different IDs is
crucial for cross-platform interoperability, ensuring a good output requires
the mapping services to be up to date.

Taxon names can have multiple IDs depending on the platform, taxonomy common
IDs, apart from the Linnaean system, are the LSID, NCBI taxonomy identifiers,
EOL identifiers etc. For marine species LSIDs based on Aphia IDs are the most
widely adopted.

Ontobee \footnote{https://www.ontobee.org/}, a web server that links
ontologies, is useful for the annotation of text to ontology IDs
\cite{xiang_ontobee_2011}. Curators can provide text snippets to Ontobee in
order to retrieve ontology terms regarding environmental features (e.g. ENVO
IDs), functional traits (e.g. PATO
IDs\footnote{\url{https://github.com/pato-ontology}} \cite{tan_pato-ontologypato_2022})
or other ontology terms of interest. Currently, the use of entire documents is
not recommended.

The WoRMS Taxon match\footnote{\url{http://www.marinespecies.org/aphia.php?p=match}}
tool matches the taxon list found against the World Register of Marine Species
(WoRMS) taxon LSID. Geographic regions are confirmed with the use of the
georeference tool developed for the Marine Gazetteer, users can enter the
location name in the gazetteer search field of the web interface and the
result’s output includes the region’s boundaries and the corresponding MRGID.

Most vocabulary servers provide APIs that map the different IDs. EMODnet Biology
has adopted LSIDs for marine species based on Aphia IDs from the
WoRMS vocabulary, which provides a dedicated API and an R package worrms
\cite{chamberlain_worrms_2020}. Additionally, the R package taxize
\cite{chamberlain_taxize_2013} provides taxon mapping capabilities across many
data sources (i.e. NCBI taxonomy, Integrated Taxonomic Information System,
Encyclopedia of Life, WoRMS). Functions like get\_eolid, get\_nbnid, get\_wormsid
can perform mapping across rows of the taxon name of the case study. In
addition, the GloBI \footnote{https://www.globalbioticinteractions.org/}
(Global Biotic Interactions) nomer tool
\footnote{https://github.com/globalbioticinteractions/nomer}
\cite{poelen_globalbioticinteractionsnomer_2022} can also be used as it
provides entity mapping functionality via CLI \cite{poelen_global_2014}.

   \subsubsection{Data Transformations}
   In this step, curators organise data according to the Darwin
Core\footnote{https://dwc.tdwg.org/} standard and extensions, such as extended
Measurement or Fact Extension\footnote{https://manual.obis.org}, resulting in
the creation of a Darwin Core Archive (see guidelines via the
link\footnote{\url{https://www.gbif.org/tool/81282/darwin-core-archive-assistant}})
with detailed sampling descriptors and terms based on controlled vocabularies.

When considering data transformations, curators tend to use GUI spreadsheet
applications like Microsoft Excel, Google Sheets and LibreOffice Calc.
OpenRefine \footnote{\url{https://openrefine.org/}} is a free, open source software that
handles messy data and provides their transformation in various
ways \cite{ham_openrefine_2013}. The software’s main goal is to provide data
cleaning, fixing and analysing while also enhancing the interconnection between
different datasets \cite{verborgh_using_2013}.

Automation can be used for this transformation through CLI tools like the R
tidyverse\footnote{https://www.tidyverse.org} package suite, Python
pandas\footnote{https://pandas.pydata.org} library and AWK programming
language\footnote{https://en.wikipedia.org/wiki/AWK}, among others. These tools
support fast and scalable tabular and text data handling, manipulations,
merging and filtering. The choice of tools depends on the users’ familiarity,
expertise and operating system.

   \subsubsection{Quality Control}
Prior to publishing the dataset it is important to perform sanity checks and
quality checks to ensure that the data comply with the Darwin Core
Standards \cite{vandepitte_fishing_2015}. LifeWatch- EMODnetBiology QC
tool\footnote{https://rshiny.lifewatch.be/BioCheck/} allows the use of the IPT
URL or the dataset’s DwC-A files and provides a list of the quality issues
encountered, according to the EMODnet Biology criteria, as an output. It is
available as a Web Application interface, based on RShiny, and as a R
package\footnote{https://github.com/EMODnet/EMODnetBiocheck} \cite{de_pooter_emodnetbiocheck_2019}.
LifeWatch Belgium Data Services’\footnote{\url{https://www.lifewatch.be/data-services/}}
has similar functionalities, providing a compilation of data services from
plain text and spreadsheet files as input. The GBIF Data
Validator\footnote{\url{http://gbif.org/tools/data-validator}} combines all the
above mentioned options, in terms of input, and provides a detailed summary of
issues encountered in data and metadata. Open Refine, is equipped with a few
extensions that can also check for taxon names and reconcile them.

The Obistools\footnote{https://github.com/iobis/obistools} R package \cite{provoost_iobisobistools_2019},
the basis of the LifeWatch-EMODnetBiology QC tool, can be used to check the
coordinate boundaries and calculate centroids in cases where the exact location
is unknown. It also checks for dates’ formats and events. It has comprehensive
documentation and is in active development.

   \subsubsection{Upload to Database}
The last step of the curation process is the publication of the standards’
compliant formatted data, which is facilitated by the Integrated Publishing
Toolkit\footnote{https://www.gbif.org/ipt} (IPT) software platform \cite{robertson_gbif_2014}.
Curators create an IPT resource entry with the aforementioned data and
associated metadata, which is then uploaded to an IPT instance, e.g. the
MedOBIS\footnote{\url{https://www.lifewatchgreece.eu/?q=content/medobis}}
Repository \cite{arvanitidis_medobis_2006}. In the case of MedOBIS, the IPT is
subsequently harvested and made available by the central OBIS\footnote{https://manual.obis.org}
system, thus being a strong example and supporter of the ‘collect once, use
many times’ concept.

   \subsubsection{One-Stop-Shop Tools}
The main all-in-one GUI computer program is Golden-GATE-
imagine\footnote{\url{https://github.com/plazi/GoldenGATE-Imagine}}, an updated
version of GoldenGATE editor \cite{sautter_semi-automated_2007}. This tool
supports OCR, NER and entity mapping, as described in the various steps of the
curator’s workflow by providing annotations on PDF backed up by ontologies. It
was developed by Plazi in 2015 and was last updated in 2016. Several recent
biodiversity data related publications still report the use of it although it
has not been updated since that time
\cite{10.3897/biss.3.37078,rivera-quiroz_extracting_2019,10.3897/biss.4.59178}.
Due to its open source nature, Golden-Gate-imagine can be further developed by
any interested parties, as exemplified in GNfinder.

\begin{table}[ht]
\large
\resizebox{\textwidth}{!}{%
\begin{tabular}{lllll}
\hline
\textbf{OS} & \textbf{Source code - running time} & \textbf{Container - running time (minutes)} & \textbf{CPU} & \textbf{RAM (GB)} \\
\hline
macOS Catalina 10.15.7 & 28 minutes & Docker - 33’ & Intel(R) Core(TM) i5-4258U CPU @ 2.40GHz & 8 \\
Linux Ubuntu 18.04.5 LTS (Bionic Beaver) & 20 minutes & Docker - 27’ & Intel(R) Pentium(R) Dual-Core CPU T4200 @ 2.00GHz & 4 \\
Linux Debian server 4.9.0-8-amd64 & --- & Singularity - 20’ & Intel(R) Xeon(R) Silver 4114 CPU @ 2.20GHz & 4
\end{tabular}%
}
\caption{The platforms where the CLI workflow was tested.Please note that running time can be affected by internet speed and stability due to API calls. The workflow uses open source tools and software libraries that are distributed across the major platforms; Linux, Mac and Windows.}
\label{table-CLI}
\end{table}

   \subsection{DECO: A Biodiversity Data Curation Programming Workflow}
A CLI workflow named DECO developed to demonstrate the advantages of the CLI
approach, is available via this GitHub repository\footnote{\url{https://github.com/lab42open-team/deco}}.
DECO has connected different tools of the programming curation steps
\ref{fig:curation-process}. The execution is via a single command with a
user-provided PDF file and the output are the taxon names and records from
WoRMS API, taxonomy NCBI IDs and ENVO terms from the Environmental Ontology.
Complementary tools (i.e. Ghostscript\footnote{https://www.ghostscript.com/index.html},
jq\footnote{https://stedolan.github.io/jq/} and ImageMagick\footnote{https://imagemagick.org/index.php})
and UNIX commands are also called in a single Bash script which unifies
the workflow. In order to simplify the setup procedure of the workflow a Docker
container and a Singularity container were developed that include all the
dependencies and the code. The code and both containers have been tested on
Ubuntu, Mac and Debian server (Table \ref{table-CLI}). For a larger corpus of biodiversity
historical data the recommendation is to use the Singularity container in a
remote server or a High Performance Computing (HPC) cluster.

% DECO DISCUSSION
\section{Discussion}
\label{sec:deco-discussion}

   \subsection{Data Rescue Landscape}
   The huge difference between rescued historical marine datasets uploaded on
OBIS and the available digital items on BHL holdings reflects the challenges
faced by curators and the minimal attention paid by the wider community,
when compared to other data rescue activities (e.g. specimens, oceanographic
data, etc.). Many publications lack basic metadata such as location, date,
purpose or method of sampling. Tracing this information is limited as the data
providers may (a) have forgotten these details, (b) be retired or (c) be
deceased \cite{michener_nongeospatial_1997}.

The project ‘Census of Marine Life’ included, among its initial objectives, the
rescue of historical marine data. Since then, there have been ongoing efforts
within the EMODnet Biology project and LifeWatch Research Infrastructure, among
others. Similarly, initiatives like Global Oceanographic Data Archaeology and
Rescue \footnote{\url{https://www.ncei.noaa.gov/products/ocean-climate-laboratory/global-oceanographic-data-archaeology-and-rescue}}
(GODAR), Oceans Past Initiative\footnote{https://oceanspast.org} (OPI) and
RECovery of Logbooks And International Marine data
\footnote{https://icoads.noaa.gov/reclaim/} (RECLAIM) \cite{wilkinson_recovery_2011}
rescue data from ship logs for oceanographic, climate and biodiversity data.
More effort is however needed, as exemplified by museum specimen collections
and herbaria digitisation \cite{mora_how_2011,wheeler_mapping_2012}.
The museum specimen collections and herbaria digitisation has multiple projects
and infrastructures like Distributed System of Scientific
Collections\footnote{https://www.dissco.eu} (DiSSCo), Innovation and
consolidation for large scale digitisation of natural
heritage\footnote{https://icedig.eu/} (ICEDIG), Integrated Digitized
Biocollections\footnote{https://www.idigbio.org/} (iDigBio) and Biodiversity
Community Integrated Knowledge Library (BiCIKL) \cite{penev_biodiversity_2022}.
Similar attention is required to rescue marine biodiversity data from
historical documents that can contribute to a more complete global biodiversity
synthesis \cite{heberling_j_mason_data_2021}.

In the last few years, an upsurge in web applications development regarding the
enhancement of biodiversity data digitisation has been observed. This is an
indication of the need for such initiatives. Advancements in the field of OCR,
text mining and information technology promise semi-automation and acceleration
of the curator’s work, which could transform the biodiversity curation field
into an -omics like, interdisciplinary field that requires complementary skills
of document handling, web technologies and text mining, to name but a few.

   \subsection{Interface Remarks}
   Web applications provide the advantage of visual aids (e.g. highlights of
NER terms), which improve the evaluation easiness and intuitiveness when using
their graphical interfaces. Emerging web development technologies like R Shiny,
Flask \footnote{https://flask.palletsprojects.com/} and
Django\footnote{https://www.djangoproject.com/} among others, have simplified
the processes of web application development. These applications are powerful
and effective in most cases but are siloed in functionality and extendability,
they also have many software dependencies which increase instability, when not
maintained in the long term.

CLI tools are a powerful way to implement scalable, reproducible and replicable
workflows: scalable because the same code can be applied to multiple files
(e.g. in this case, the various documents); reproducible and replicable because
the code can be executed multiple times and with different types of documents,
respectively. Furthermore, they usually have additional functionalities that
have not been implemented in their web application counterparts. The
difficulties regarding CLI tools’ dependency and portability are being resolved
with the rise of containerised applications which include all system
requirements and are distributed through web repositories like Docker
Hub\footnote{https://hub.docker.com/}, the downside is that without
interactiveness they are cumbersome when assisting the curation process.

   \subsection{Sustainability}
   Tool usability relies on active development and continuous support and
debugging. Sustainability is considered the main issue regarding the tools’
functionality. An example is EnvMine \cite{tamames_envmine_2010}, a promising
2010 cutting edge tool which is no longer available. One-stop-shop purpose
software applications for domain specific usage, like GoldenGate, are very
helpful but require more effort to stay up to date with the integrated tools.
Other tools are often out of date, as active development and contribution to
reporting issues in open-source repositories, such as Github, is lacking, thus
becoming obsolete and unsupported in only a few years from their first release.

   \subsection{Curation Step-Wise Remarks}
   The curators’ role is invaluable in the data rescue process, as their domain
specific expertise is far from becoming entirely automated. There are plenty of
available digitised historical documents that are not curated in web libraries,
such as BHL, the Belgian Marine
Bibliography\footnote{\url{https://www.vliz.be/en/belgian-marine-bibliography}},
Web of Science \footnote{https://www.webofknowledge.com}, Wiley Online
Library \footnote{https://onlinelibrary.wiley.com} and Taylor \& Francis
Online \footnote{https://www.tandfonline.com/}, among others \cite{kearney_its_2019}.
BHL provides “OCRed” documents and there are plenty of other tools that can
tackle this process which are reviewed elsewhere \cite{10.3897/rio.6.e58030},
however OCR is a crucial limiting step in the workflows, in terms of the
information transformed from image to text, because there are many cases that
lead to mispelled or lost text; especially the case with handwritten text and
poor quality images \cite{lyal_digitising_2016}.

Information extraction can be performed both on a small and a large scale.
Named Entities are what most text mining tools extract. Taxon names recognition
is the main function of the majority of the current tools and has matured
significantly over the past decade, especially through the integration of
multiple platforms with the GNA \cite{pyle_towards_2016}. Environments and
geolocations have strong background data, Environment Ontology terms (retrieved
with the EXTRACT tool) and GeoNames\footnote{http://www.geonames.org}/
Marineregions gazetteers, respectively. Geolocation mining, in particular, has
not been adapted in biodiversity curation but there are generic tools
(e.g. mordecai\footnote{https://github.com/openeventdata/mordecai} -
\cite{halterman2017mordecai} that are able to be trained with gazetteers to
extract approximate localities from text. Also extraction of sample location
from maps is possible by first geolocating the historic map in Geographic
Information Systems \cite{jenny_studying_2011} and then using computer vision
to find the locations’ coordinates \cite{10.1145/2557423}. Characteristics of
taxa, i.e. phenotypic traits, associated physico-chemical variables, units and
the use of semantics to describe relations, are still under standardisation
\cite{thessen_transforming_2020} and NER prototypes have been made with
ClearEarth and Pensoft Annotator, for example.

Entity mapping has also seen an important development because there are many
open public APIs for vocabularies like those used in WoRMS, and Marine Regions
and aggregators such as GBIF and OBIS, among others, and in some cases software
packages (mostly in the R programming language). The task for Publication has
its dedicated applications and tools with the CLI tools being able to perform
quality control and deliver a preferred on-the-fly format.

   \subsection{DECO}
   The CLI scientific workflow assembled in this paper, DECO, is a demonstration
of EMODnet Biology’s vision for biodiversity data rescue using programming
tools. To the best of our knowledge, this is the first task-driven CLI that
brings together state-of-the-art image processing, OCR tools, text mining
technologies and Web APIs, in order to assist curators. By using programming
interface and Command Line Tools the workflow is scalable, customisable and
modular, meaning that more tools can be incorporated to, e.g. include the
entities mentioned in the previous section. It is fast, may be used on a
personal computer, and is available as a Docker and a Singularity container.
The containerised versions of the workflow simplify the installation procedure
and increase its stability, scalability and portability because they include
all the necessary dependencies. This CLI scientific workflow promises a faster
and high throughput processing that could be applied to any type of data, not
only historical, thus contributing to the overall digitisation of biodiversity
knowledge.

   \subsection{Future Outlook}
   Progress has been made in the advancement of the historical data rescue
process, from digitisation platforms to standards, services and publication
\cite{beja_chapter_2022}. To bridge the gap between tools and curators
requires effort on both ends; namely the data curators and the tool developers.
It is recommended that curators are trained in basic programming skills from
which they and the historical data rescue process in general would benefit in
the long term \cite{10.12688/f1000research.25413.2}. Regarding software
development, important features are highlighted, like the use of multiple
ontologies in Pensoft Annotator. This is a direction which should be further
expanded to all entities of interest. Multidisciplinary cooperation between
scientific communities and partners of tools, ontologies and databases is
needed to accomplish this task \cite{bowker_biodiversity_2000}. An important
example was set by GNA which advanced scientific names recognition
significantly. In addition, the co-occurrence feature, that was present in
Biodiversity Observation Miner, once expanded to other entities and associated
with a scoring scheme will be a state-of-the-art text mining application that
goes beyond NER to actually infer relations. The rise of deep neural networks
is promising as well in all different tasks of Information Extraction, as seen
in TaxoNERD \cite{le_guillarme_taxonerd_2022}. Lastly, the community is
pushing to Semantic Publishing, FAIR completeness of new data and new taxonomic
publishing guidelines to eliminate the need of text mining and curation in
current publications \cite{penev_openbiodiv_2019,fawcett_digital_2022}.

The implementation of crowdsourced curation within citizen science projects for
the historical biodiversity data is encouraged \cite{clavero_mine_2014,arnaboldi_text_2020,10.12688/f1000research.25413.2}.
Practices like this are already in place in the digitisation of natural history
collections and have been proved fruitful \cite{ellwood_accelerating_2015}.
EMODnet Biology’s Phase IV will launch such a citizen science project for
historical documents curation during the second half of 2022. Approaches from
other fields of science that handle historical and old data, such as history,
linguistics, archaeology would provide useful insights for the text mining of
historical biodiversity data.


   \subsection{Concluding Remarks}
   Historical marine biodiversity data provide important and significant
snapshots of the past that can help understand the current status of ocean
ecosystems and predict future trends in face of the climate crisis. There is a
wealth of historical documents that have been digitised yet, most of their data
have not been rescued or published in online systems. To accelerate the tedious
data rescue process it is essential that more curators become engaged, and
tools for Information Extraction and Publication get improved to satisfy their
needs. Tools like DECO and GoldenGATE demonstrate possible future directions
for one-stop-shop applications for command line and graphical user interfaces,
respectively. Research Infrastructures can play a pivotal role towards this
goal. Last but not least, the community and funding bodies should prioritise
the data rescue of these invaluable documents before their decay and inevitable
loss.


\section*{DATA AVAILABILITY STATEMENT}

DECO is available here:\url{https://github.com/lab42open-team/deco}. Historical
marine literature analysis is here:
\href{https://github.com/savvas-paragkamian/historical-marine-literature}{https://github.com/savvas-paragkamian/historical-marine-literature}.
BHL, EMODnet Biology and OBIS data are available for download here
\href{https://about.biodiversitylibrary.org/tools-andservices/developer-and-data-tools/}{https://about.biodiversitylibrary.org/tools-andservices/developer-and-data-tools/}
and \url{https://www.emodnetbiology.eu/toolbox/en/download/occurrence/explore}
and here \url{https://obis.org/manual/access/}, respectively. The digitised
document of the “Report on the Mollusca andRadiata of the Aegean Sea, and on
their distribution, considered as bearing on Geology. 13th Meeting of the
British Association for the Advancement of Science, London, 1844” is available
here: \url{https://www.biodiversitylibrary.org/page/12920789}. The curated
dataset of the case study is available here (version 1.9 and above):
\url{http://ipt.medobis.eu/resource?r=mollusca_forbes}.

\section*{AUTHOR CONTRIBUTIONS}

Conceptualisation: CA, EP, and VG. Wrote first draft of the manuscript: SP, GS,
DM, CP, CA, EP, and VG. Revised the manuscript: all. Web applications testing:
SP, GS, ME, RP. Programming tools testing: SP, HZ, and EP. Code development and
containerisation: SP and HZ. Work Package Leaders: VG and DM. Project
coordinator: JB. All authors contributed to the article and approved the
submitted version.

\section*{FUNDING}
This work was supported by EMODnet Biology Phase III
(EASME/ EMFF/2016/-1.3.1.2/Lot 5/SI2.750022 and EASME/EMFF/2017/1.3.1.2/02/SI2.789013)
and Phase IV (EMFF/2019/1.3.1.9/Lot6/SI2.837974). The European Marine
Observation and Data Network (EMODnet) is financed by the European Union under
Regulation (EU) No 508/2014 of the European Parliament and of the Council of 15
May 2014 on the European Maritime and Fisheries Fund. SP was supported also by
EMODnet Biology Phase IV and for different parts of his work he was supported
from the project “Centre for the study and sustainable exploitation of Marine
Biological Resources (CMBR)” (MIS 5002670), which is implemented under the
Action “Reinforcement of the Research and Innovation Infrastructure,” funded by
the Operational Programme “Competitiveness, Entrepreneurship and Innovation”
(NSRF 2014–2020) and co-financed by Greece and the EU (European Regional
Development Fund). GS received support from EMODnet Biology Phase III and Phase
IV. SP and HZ received support from the Hellenic Foundation for Research and
Innovation(HFRI) and the General Secretariat for Research and Innovation
(GSRI), under grant agreement No. 241 (PREGO project). DM and VG have received
support from LifeWatchGreece Research Infrastructure \cite{arvanitidis_lifewatchgreece_2016}
and “Centre for the study and sustainable exploitation of Marine Biological
Resources (CMBR)” (MIS 5002670), which is implemented under the Action
“Reinforcement of the Research and Innovation Infrastructure,” funded by the
Operational Programme “Competitiveness, Entrepreneurship and Innovation”
(NSRF 2014-2020) and co-financed by Greece and the European Union (European
Regional Development Fund). LB received support by EMODNET Biology, Phase IV.
The work of LV is funded by Research Foundation - Flanders (FWO) as part of the
Belgian contribution to LifeWatch. For different aspects of his work HZ
received support from ELIXIR-GR: Managing and Analysing Life Sciences Data
(MIS: 5002780) which is co-financed by Greece and the European Union - European
Regional Development Fund. CA received support from LifeWatch ERIC. LifeWatch
ERIC funded the publication fees. The funders had no role in study design, data
collection and analysis, decision to publish, or preparation of the manuscript.


