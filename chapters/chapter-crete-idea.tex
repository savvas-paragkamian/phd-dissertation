% --------------------------------------------------
% 
% This chapter is for Crete system ecology
% 
% --------------------------------------------------


\chapter{Towards a Cretan soil biodiversity data system}
\label{cha:crete-idea}

\section{Introduction}\label{intro_idea}

Ecosystem ecology studies the interactions of organisms with their surroundings
as a whole \parencite{van-dyne1966ecosystems}. These interactions are complex and hard to interpret and quantify.
Macroscopically they represent the flows of energy and materials in the total environment.
The implications of this approach to ecosystems are that multiple types of disciplines 
are intersected. For example, soils are defined by their physical characteristics,
i.e grain size, clay content, their chemistry, i.e pH, elements compositions, and 
their biodiversity, i.e plants, arthropods and microbes \parencite{vogel2022}. Humans have influenced 
and/or exploited almost all the environments of earth with their activities.
Hence, understanding ecosystems requires a synthesis which in the digital era 
is becoming more and more plausible. 

There are multiple terms that describe the digital representation of ecosystems.
One of the earliest is "digital earth" which mentioned in the late 90s \parencite{Goodchild_2012}.
This notion of the virtual earth was initially realised with satellite imaging,
on-land sensors and the relevant software. Space, an area on Earth, is
the starting point of ecosystems functioning which is influenced by geography,
climate, geology, biodiversity and human management across scales \parencite{Zarnetske2019_towards}. The coupling of all these
fields, and the data they provide, with spatial information leads to the
digital twin of ecosystems. All these data can be considered as different layers
of an Earth system data cube \parencite{mahecha2020earth}. The data cube representation
allows for multivariate modeling and ecosystem level predictions \parencite{mahecha2020earth}.

Studying Earth as a whole and thus generating data at the global scale, all at once,
is part of a new field called planetary biology. Smaller scales of such integrating systems exist like
the Island Digital Ecosystem Avatar \parencite{Davies2016}. Islandscapes 
is also an interesting approach which defines the holistic ecosystem of the island
and the interface of land and sea \parencite{Vogiatzakis_land_2017}. Also islands
due to their smaller scale and confined nature are useful virtual ecological labs. They harbour 20\% of 
biodiversity even though they cover less than 7\% of land surface area. Also they are under major threats 
that diminish their biodiversity \parencite{fernandez-palacios2021scientists}.
Further focusing on soil, could assist bringing the natural and human activities 
together in the so-called soilscape \parencite{LAGACHERIE2001105}.

There are multiple tangible outcomes of integrating all these data gather together about a
specific place and biome. First, it allows for sampling biases and gaps identification. Second, 
it enables testing new hypotheses about unforeseen associations of variables. Third, 
it forms a foundation of the current state of knowledge to assist decision making 
processes about conservation, ecosystem services management and comparisons for 
future states. Building such systems is also a cultural treasure for society because
they will present the uniqueness of each place.

Biodiversity is represented in ecosystems as point occurrences of each taxon. The
DarwinCore standards have assisted a lot the homogeneity of the data. GBIF is a 
global aggregator that contains such information about all ecosystems from 
different resources \parencite{noauthor_gbif_nodate}. It is crucial that each sampling point has rich metadata. 
Sequences from eDNA studies are also part of biodiversity and are stored in 
databases like ENA, JGI IMG and GenBank.
From the bacterial point of view, metagenomic samples are abundant in these 
databases and there are also curated databases that classify them in some baseline functionality. 
There is also a wealth of knowledge about biodiversity in the literature. 
Databases like PubMed, Goolge Scholar, Dimensions and Biodiversity Heritage Library 
are world class providers with different content.

Examples of platforms that host open data soil data are the 
European Soil Data Centre \parencite{Panagos2022} and the World Soil Information
Service (WoSIS) of the ISRIC \parencite{Batjes2024}. Apart from samplings and literature,
spatial data are openly available terrestrial ecosystems.
To name a few, there openly available climatic, land cover, desertification risk, aridity, soil type, normalized
vegetation index, bedrock geological formations data.

In this chapter, the focus is system's approach of the soil ecosystems of the island of Crete.
First, the literature about Crete is aggregated and summarised. Second, biodiversity data
from GBIF and IUCN \parencite{iucn2024} are collected and third a spatial data cube is created. 
Having all these data joined about Crete shows the culmination of intensive 
research of the Cretan environments over the centuries and provokes for new
hypotheses and conservation priorities.


\section{Materials and Methods}\label{integration_methods}

\subsection{Literature}\label{crete-literature}

Crete is an unambiguous name, this fact allows for simple keyword searches to discover 
the relevant literature in multiple platforms. Given various possible references to Crete,i.e \textit{Crete, Kriti, Kreta, and Cretan},
it is essential to account for these variations during the keyword search process.

The corpus of PubMed was downloaded via the ftp protocol and the E-utilities API, details of which can be found on the PubMed Download page.
The ftp method generates multiple XML files, which were later converted into
files with tab-separated values. These files were then compressed. Each .tsv.gz file
features six distinct fields in this format: PMID, DOI, Authors, Journal, Volume:Pages, Year, Title, Abstract.
Pubmed was searched with the aforementioned keywords using the DIG tool \parencite{fanini2021coupling}.

The Dimensions platform has a user interface specifically tailored for academic literature searches.
It is the open source equivalent of Web of Science.
With this interface, a query for \textit{Crete, Kriti, Kreta, and Cretan} within
the title and abstract fields was requested yielded the articles, each annotated with the respective Field of Research.

Google scholar was also searched through the \textit{serpapi} API using
a python script. The same keywords were used as in the previous searches.

Historical literature related to the biodiversity of Crete has been explored
using the BHL data model and schema, conducting searches on titles, items, and subjects.
Items represent the digitised documents, while subjects are assigned to each title.
The schema includes a page table with page-level information.
A comprehensive archive of OCR texts was downloaded, totaling approximately
40GB compressed and 300GB uncompressed, that includes 62 million OCR pages from 292,000 unique documents.
To ensure a thorough analysis, search was performed through all pages since
relevant information about Crete may be embedded within the documents
themselves, and not necessarily in the title or summaries.

\subsection{Samplings}\label{crete_samplings}

The samplings are distinguished as sequences, biodiversity and soil, using ENA, GBIF and WoSIS respectively.
The ENA API was used with a POST request containing the bounding box of Crete as a query \parencite{Yuan2023}. This 
query returned a list with the sample identifiers. Using this list and a custom python script the API
was invoked with GET requests to retrieve all the available metadata for each sample.
GBIF has the in the online user interface the functionality of selecting the area of interest with a bounding box and requesting
the data \parencite{noauthor_gbif_nodate}. The request was then approved and was downloaded. All species 
from IUCN that are in Crete were downloaded from the website \parencite{iucn2024}. IUCN provides the 
interactive maps selection functionality. 
WoSIS has a dedicated Web Feature Service (WFS) that is specialised for spatial data \parencite{Batjes2024}. 
Using the available tutorial and a adjusted R script the samplings and metadata of 
soil in Crete were retrieved.
Edaphobase was also searched for sampling information from Crete. Edaphobase is
a curated database of museum records that aims to include all aspects of soil biodiversity \parencite{BURKHARDT20143}.

In addition, the JGI IMG GOLD database contains many samples which were downloaded from \href{https://gold.jgi.doe.gov/downloads}{the webservice}.
This path of Public Studies / Biosamples / SPs / APs / Organisms was selected in the excel file format option. 
GOLD didn't have any unique samples when compared with ENA.

\subsection{Spatial data}\label{crete_spatial}

Maps were downloaded and cropped from multiple resources.
In terms of managing spatial data, geometry, and transformations, the following
R packages were used:
sf package version 1.0-14 \parencite{Pebesma2023}, for polygons, and the terra
package version 1.7-55 \parencite{hijmans2024terra}, for rasters.

The land use categories were obtained from the CORINE Land Cover (CLC) 2018
version v.2020.20u1 \parencite{CLC2023}.
The Historic Land Dynamics Assessment (HILDA+) dataset \parencite{winkler2021global}
facilitated the exploration of human pressures, such as changes in land usage and agriculture,
allowing the calculation of land use alterations from 1998 to 2018.

The protected areas of Crete were included as well from the wdpar package \parencite{Hanson2022}.
In WorldClim 2.0, twelve global climatic variables, such as the annual mean
temperature and annual precipitation, were included \parencite{Fick2017}.
The dataset on the Environmentally Sensitive Areas Index to desertification (ESAI)
in Greece \parencite{KARAMESOUTI2018266} was also employed, alongside the
Global Aridity Index Database \parencite{zomer2022version}.
The geological formations of Crete were derived from the geoportal of the
Decentralized Administration of Crete. These maps were created by the Crinno-Emeric Group
project \url{https://geoportal.apdkritis.gov.gr/gis/apps/storymaps/stories/19690f65abbe4e8ab0141b2fe7261a8c}.
Furthermore, the Harmonised World Soil Database v2 was integrated for soil mapping units and taxonomic soil classification \parencite{fao2023}.

\subsection{Tools and code}\label{Coding environment}
Visualisation was implemented with ggplot2 \parencite{wickham_ggplot2_2016} and pheatmap \parencite{Kolde2019}.
The environment we worked had Python 3.11.4, R version 4.3.2 \parencite{rcoreteam}.
Finally, computations were performed on HPC infrastructure of HCMR \parencite{zafeiropoulos_0s_2021}.

Scripts about data integration are available in
\href{https://github.com/savvas-paragkamian/crete-data-integration}{Crete data integration repository}.
Code is structured to be reproducible and interoperable.

\section{Results}\label{crete_idea_results}

\subsection{Historical and Contemporary literature}

Regarding PubMed the search process yielded 1556 unique abstracts. These were retrieved with only a handful of false positives.
The majority of these abstracts pertain to biomedical topics. Out of those, 170 originate from environmental sciences.

\subsection{Biodiversity and samplings}

Worldwide projects of microbiome studies have collected one or two topsoil
samples from Crete \parencite{Vasar2022, Labouyrie2023, Bahram2018, Orgiazzi2018}.
Some have focused on soil fungi \parencite{Mikryukov2023, Davison2021, Tedersoo2021}
and other soil eukaryotes \parencite{Aslani2022}.
The only thorough microbiome study of a soil ecosystem in Crete, to our knowledge,
is in the north west part of the island, the Koiliaris Critical Zone Observatory \parencite{tsiknia2014}.
Apart from the soil microbiome, soil physical and chemical properties has been
investigated by global and European projects like the Forum of European Geological Surveys
(FOREGS) \parencite{nerc19017}, the Geochemical Mapping of Agricultural and Grazing Land
Soil in Europe (GEMAS) \parencite{REIMANN2018302} and the Soil Profile Analytical
Database for Europe (SPADE) \parencite{Hiederer2006}.
LTER map

\begin{figure}[h] 
    \centering\includegraphics[width=\columnwidth]{crete_integration_ena_terrestrial.png}
    \caption{Available terrestrial metagenomic samples from ENA in Crete.}
    \label{fig:isd_crete_ena}
\end{figure}


\begin{figure}[h] 
    \centering\includegraphics[width=\columnwidth]{crete_integration_map_wosi_soil.png}
    \caption{Soil samples from Crete that are uploaded in WoSIS.}
    \label{fig:isd_crete_wosis}
\end{figure}

GBIF map 

Edaphobase has in total, 26 resources with 17 different sampling points covering 120 distinct taxa.



\subsection{Maps}

Multiple maps for Crete.


\section{Discussion}\label{crete_idea_discussion}

Compare with Samothrace \parencite{noll2024insights}.

\section{Conclusion}

Focusing in a confined area provides unique opportunities for data integration.
And as complex systems ecologists have stated bringing together information across
scales \parencite{brown2004METABOLIC} and functions leads to emergent properties that weren't possible to 
predict before \parencite{smith2016Origin}. The techniques to integrate are multiple in terms of statistics and modelling. But,
the advent of Large Language Models revolutionised the data interpretation and 
user interaction. In ecology and conservation this is very promising \parencite{doi2024biodiversity}.
In my opinion, the presented knowledge base of Crete would be a suitable 
case study to investigate these significant advancements to tighten the integration and form a 
user application.

