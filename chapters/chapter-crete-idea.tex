% --------------------------------------------------
% 
% This chapter is for Crete system ecology
% 
% --------------------------------------------------


\chapter{Towards a Cretan soil biodiversity data system}
\label{cha:crete-idea}

\section{Introduction}\label{intro_idea}

Ecosystem ecology studies the interactions of organisms with their surroundings
as a whole \parencite{van-dyne1966ecosystems}. These interactions are complex and hard to interpret and quantify.
Macroscopically they represent the flows of energy and materials in the total environment.
The implications of this approach to ecosystems are that multiple types of disciplines 
are intersected. For example, soils are defined by their physical characteristics,
i.e grain size, clay content, their chemistry, i.e pH, elements compositions, and 
their biodiversity, i.e plants, arthropods and microbes \parencite{vogel2022}. Humans have influenced 
and/or exploited almost all the environments of earth with their activities.
Hence, understanding ecosystems requires a synthesis which in the digital era 
is becoming more and more plausible. 

There are multiple terms that describe the digital representation of ecosystems.
One of the earliest is "digital earth" which mentioned in the late 90s \parencite{Goodchild_2012}.
This notion of the virtual earth was initially realised with satellite imaging,
on-land sensors and the relevant software. Space, an area on Earth, is
the starting point of ecosystems functioning which is influenced by geography,
climate, geology, biodiversity and human management across scales \parencite{Zarnetske2019_towards}. The coupling of all these
fields, and the data they provide, with spatial information leads to the
digital twin of ecosystems. All these data can be considered as different layers
of an Earth system data cube \parencite{mahecha2020earth}. The data cube representation
allows for multivariate modeling and ecosystem level predictions \parencite{mahecha2020earth}.

Studying Earth as a whole and thus generating data at the global scale, all at once,
is part of a new field called planetary biology. Smaller scales of such integrating systems exist like
the Island Digital Ecosystem Avatar \parencite{Davies2016}. Islandscapes 
is also an interesting approach which defines the holistic ecosystem of the island
and the interface of land and sea \parencite{Vogiatzakis_land_2017}. Also islands
due to their smaller scale and confined nature are useful virtual ecological labs. They harbour 20\% of 
biodiversity even though they cover less than 7\% of land surface area. Also they are under major threats 
that diminish their biodiversity \parencite{fernandez-palacios2021scientists}.
Further focusing on soil, could assist bringing the natural and human activities 
together in the so-called soilscape \parencite{LAGACHERIE2001105}.

There are multiple tangible outcomes of integrating all these data gather together about a
specific place and biome. First, it allows for sampling biases and gaps identification. Second, 
it enables testing new hypotheses about unforeseen associations of variables. Third, 
it forms a foundation of the current state of knowledge to assist decision making 
processes about conservation, ecosystem services management and comparisons for 
future states. Building such systems is also a cultural treasure for society because
they will present the uniqueness of each place.

Biodiversity is represented in ecosystems as point occurrences of each taxon. The
DarwinCore standards have assisted a lot the homogeneity of the data. GBIF is a 
global aggregator that contains such information about all ecosystems from 
different resources \parencite{noauthor_gbif_nodate}. It is crucial that each sampling point has rich metadata. 
IUCN redlist \parencite{iucn2024} is also an important resource of species occurrences,
habitats, threats and conservation status of taxa. Redlists contain 
manually curated content by experts of each taxonomic group which is updated
after some years depending on the country and group.
Sequences from eDNA studies are also part of biodiversity and are stored in 
databases like ENA, JGI IMG and GenBank.
From the bacterial point of view, metagenomic samples are abundant in these 
databases and there are also curated databases that classify them in some baseline functionality. 
There is also a wealth of knowledge about biodiversity in the literature. 
Databases like PubMed, Google Scholar, Dimensions and Biodiversity Heritage Library 
are world class providers with different content.

Examples of platforms that host open data soil data are the 
European Soil Data Centre \parencite{Panagos2022} and the World Soil Information
Service (WoSIS) of the ISRIC \parencite{Batjes2024}. Regarding the latter, a unified
platforms called SoilGrids provides open access to many predicted maps \parencite{poggio-soil-7-217-2021}.
Apart from samplings and literature,
spatial data are openly available terrestrial ecosystems.
To name a few, there openly available climatic, land cover, desertification risk, aridity, soil type, normalized
vegetation index, bedrock geological formations data.

In this chapter, the focus is system's approach of the soil ecosystems of the island of Crete.
First, the literature about Crete is aggregated and summarised. Second, samplings and biodiversity data
are collected and third a spatial data cube with maps is created. 
Having all these data joined about Crete shows the culmination of intensive 
research of the Cretan environments over the centuries and provokes for new
hypotheses and conservation priorities.


\section{Materials and Methods}\label{integration_methods}

This section presents the methods and data that are required to create
a Cretan soil biodiversity data system, Figure \ref{fig:crete_soil_logo}. First, the literature regarding 
Crete is downloaded and analysed. Then sampling data from different 
resources about sequences, biodiversity occurrences and soil samples 
are retrieved. Lastly, the spatial data are incorporated as different 
layers of the whole island.

\begin{figure}[hbt!] 
    \centering\includegraphics[width=0.7\columnwidth]{crete_soil_integration_logo.png}
    \caption{The schematic representation of the different types of data needed to create the Crete Soil System.}
    \label{fig:crete_soil_logo}
\end{figure}

\subsection{Literature}\label{crete-literature}

Crete is an unambiguous name, this fact allows for simple keyword searches to discover 
the relevant literature in multiple platforms. Given various possible references to Crete,i.e \textit{Crete, Kriti, Kreta, and Cretan},
it is essential to account for these variations during the keyword search process.

The corpus of PubMed was downloaded via the ftp protocol and the E-utilities API, details of which can be found on the PubMed Download page.
The ftp method generates multiple XML files, which were later converted into
files with tab-separated values. These files were then compressed. Each .tsv.gz file
features six distinct fields in this format: PMID, DOI, Authors, Journal, Volume:Pages, Year, Title, Abstract.
Pubmed was searched with the aforementioned keywords using the DIG tool \parencite{fanini2021coupling}. To further 
analyse the results, the MeSH terms were retrieved using the \href{https://www.ncbi.nlm.nih.gov/books/NBK25497/}{E-utilities API}.
MeSH terms were also further analysed with custom script to extract 
parent categories from the \href{https://www.nlm.nih.gov/databases/download/mesh.html}{xml global file}.

The Dimensions platform has a user interface specifically tailored for academic literature searches.
It is the open source equivalent of Web of Science.
With this interface, a query for \textit{Crete, Kriti, Kreta, and Cretan} within
the title and abstract fields was requested yielded the articles, each annotated with the respective Field of Research.

Google scholar was also searched through the \textit{serpapi} API using
a python script. The same keywords were used as in the previous searches.

Historical literature related to the biodiversity of Crete has been explored
using the BHL data model and schema, conducting searches on titles, items, and subjects.
Items represent the digitised documents, while subjects are assigned to each title.
The schema includes a page table with page-level information.
A comprehensive archive of OCR texts was downloaded, totaling approximately
40GB compressed and 300GB uncompressed, that includes 62 million OCR pages from 292,000 unique documents.
To ensure a thorough analysis, search was performed through all pages since
relevant information about Crete may be embedded within the documents
themselves, and not necessarily in the title or summaries.

\subsection{Samplings}\label{crete_samplings}

The samplings are distinguished as sequences, biodiversity and soil, using ENA, GBIF and WoSIS respectively.
The ENA API was used with a POST request containing the bounding box of Crete as a query \parencite{Yuan2023}. This 
query returned a list with the sample identifiers. Using this list and a custom python script the API
was invoked with GET requests to retrieve all the available metadata for each sample.
GBIF has incorporated in the online user interface, the functionality of selecting the area of interest with a bounding box and then requesting
the data \parencite{noauthor_gbif_nodate}. The request was first approved and then it was downloaded. 
IUCN redlist data were downloaded from the web interface \parencite{iucn2024}.
Advanced filters were selected, for Land Regions option \textit{Kriti} was opted 
along with global geographical scope. This search has a permalink on the user's profile.

WoSIS has a dedicated Web Feature Service (WFS) that is specialised for spatial data \parencite{Batjes2024}. 
Using the available tutorial and a adjusted R script the samplings and metadata of 
soil in Crete were retrieved.
Edaphobase was also searched for sampling information from Crete. Edaphobase is
a curated database of museum records that aims to include all aspects of soil biodiversity \parencite{BURKHARDT20143}.

In addition, the JGI IMG GOLD database contains many samples which were downloaded from \href{https://gold.jgi.doe.gov/downloads}{the webservice}.
This path of Public Studies / Biosamples / SPs / APs / Organisms was selected in the excel file format option. 
GOLD didn't have any unique samples when compared with ENA.

\subsection{Spatial data}\label{crete_spatial}

Maps were downloaded and cropped from multiple resources.
In terms of managing spatial data, geometry, and transformations, the following
R packages were used:
sf package version 1.0-14 \parencite{Pebesma2023}, for polygons, and the terra
package version 1.7-55 \parencite{hijmans2024terra}, for rasters.

The land use categories were obtained from the CORINE Land Cover (CLC) 2018
version v.2020.20u1 \parencite{CLC2023}.
The Historic Land Dynamics Assessment (HILDA+) dataset \parencite{winkler2021global}
facilitated the exploration of human pressures, such as changes in land usage and agriculture,
allowing the calculation of land use alterations from 1998 to 2018.

The protected areas of Crete were included as well from the wdpar package \parencite{Hanson2022}.
In WorldClim 2.0, twelve global climatic variables, such as the annual mean
temperature and annual precipitation, were included \parencite{Fick2017}.
The dataset on the Environmentally Sensitive Areas Index to desertification (ESAI)
in Greece \parencite{KARAMESOUTI2018266} was also employed, alongside the
Global Aridity Index Database \parencite{zomer2022version}.
The geological formations of Crete were derived from the geoportal of the
Decentralized Administration of Crete. These maps were created by the Crinno-Emeric Group
project \url{https://geoportal.apdkritis.gov.gr/gis/apps/storymaps/stories/19690f65abbe4e8ab0141b2fe7261a8c}.
Furthermore, the Harmonised World Soil Database v2 was integrated for soil mapping units and taxonomic soil classification \parencite{fao2023}.

\subsection{Tools and code}\label{Coding environment}
Visualisation was implemented with ggplot2 \parencite{wickham_ggplot2_2016} and pheatmap \parencite{Kolde2019}.
Taxon names resolve and taxonomic classification was facilitated by taxize R package \ref{chamberlain_taxize_2013}.
The working environment had Python 3.11.4, R version 4.3.2 \parencite{rcoreteam}.
Finally, computations were performed on HPC infrastructure of HCMR \parencite{zafeiropoulos_0s_2021}.

Scripts about data integration are available in
\href{https://github.com/savvas-paragkamian/crete-data-integration}{Crete data integration repository}.
Code is structured to be reproducible and interoperable.

\section{Results}\label{crete_idea_results}

\subsection{Contemporary literature}

Regarding PubMed, the search process yielded 1556 unique abstracts about Crete.
These were retrieved with only a handful of false positives.
The majority of these abstracts pertain to biomedical topics.
Using the Medical Subject Headings (MeSH) terms, 291 abstracts didn't have associated terms. 
There were 3301 associated MeSH terms of the abstracts retrieved. 
The most used term is "Human" appearing in 907 abstracts. The geographical 
place Greece is also mentioned in 1100 abstracts. Genetic phenomena
and Behavior mechanisms are listed in 500 abstracts, each. 
While infections and organic chemicals are in 200 abstracts each. These terms 
can appear with overlaps in abstracts.
Out all the original set of abstracts, the 177, 11\%, originate from environmental sciences.
From these, Phylogeny was assigned to 24, Ecosystem to 21, Environmental Monitoring to 16
and Conservation of Natural Resources in 8 abstracts.

Dimensions web application includes a broad range of disciplines publications.
Regarding Crete there are 2,675 publications in History, Heritage, and Archaeology,
making it the most prevalent discipline in this platform.
Earth Sciences follow with 2,256 publications, while Education and Historical
Studies contribute 1,973 and 1,925 publications, respectively.
The exploration in Biological Sciences also constitutes of
Crete's academic interest, with 1,539 papers published about the island.

Google scholar contains books and articles in the results. The search through
the API resulted in 910 documents. 
The research is summarised as follows:

\begin{itemize}
    \item Biodiversity with 25\%,
    \item History with 22\%,
    \item Archaeological 21\%,
    \item Sociology is represented with 15\% publications,
    \item Folklore studies also are present, with 13\% of publications.
\end{itemize}

\subsection{Historical literature}
Regarding the Biodiversity Heritage Library and historical literature, it was
observed that 25,812 documents contained fewer than five mentions of keywords related to Crete.
Conversely, a smaller set of 3,696 documents exhibited five or more mentions of such keywords.
Items were filtered if they had less than five keyword hits related to Crete.
Consequently, 35,130 pages were extracted representing the most commonly referenced Crete-related content in the document corpus.
Of the filtered 35,130 pages, a content analysis was carried out to identify the presence of taxon names.
BHL employs the gnfinder tool on each OCRed page \parencite{mozzherin_gnamesgnfinder_2022}.
The results showed that 23,391 pages (66.61\%) contained taxon names within 3,000 distinct items.
From these, 890, contained more that 50 different taxon names.
Additionally, 75,405 unique taxon names were detected across these pages, with a total of 214,215 occurrences. 
The phyla with most taxa are summarised in Table \ref{table:crete_bhl_phyla}. 
The items with taxon names were further analyzed based on the titles mentioned
in the Biodiversity Heritage Library (BHL). It was found that 1,879 items had titles in various languages, with the following distribution:

Czech (CZE): 1
Danish (DAN): 1
Hungarian (HUN): 1
Japanese (JPN): 1
Portuguese (POR): 1
Russian (RUS): 1
Spanish (SPA): 1
Norwegian (NOR): 2
Swedish (SWE): 2
Romanian (RUM): 3
Undetermined (UND): 11
Dutch (DUT): 12
Latvian (LAT): 16
Italian (ITA): 31
Chinese (CHI): 34
Multilingual (MUL): 37
French (FRE): 105
German (GER): 268
English (ENG): 1,350

The oldest document was published in 1554 and there are 1482 items published before 1960 containing thousands of taxa occurrences.

\begin{longtable}{llll}
    \caption{Number of taxa per Phylum as mentioned in documents that offer information about Crete. Note that some documents contain information from multiple places. \label{table:crete_bhl_phyla}} \\
\textbf{Animalia} & \textbf{Taxa} & \textbf{Plantae} & \textbf{Taxa} \\
\endfirsthead
%
\endhead
%
Annelida          & 291           & Anthocerotophyta & 11            \\
Arthropoda        & 21916         & Bryophyta        & 650           \\
Brachiopoda       & 189           & Charophyta       & 45            \\
Bryozoa           & 138           & Chlorophyta      & 83            \\
Chordata          & 7353          & Marchantiophyta  & 283           \\
Cnidaria          & 444           & Rhodophyta       & 77            \\
Ctenophora        & 16            & Tracheophyta     & 26971         \\
Echinodermata     & 224           &                  &               \\
Hemichordata      & 14            &                  &               \\
Mollusca          & 4271          &                  &               \\
Nematoda          & 83            &                  &               \\
Nemertea          & 15            &                  &               \\
Platyhelminthes   & 134           &                  &               \\
Porifera          & 251           &                  &               \\
Rotifera          & 51            &                  &              
\end{longtable}

\subsection{Biodiversity and samplings}

GBIF as an biodiversity aggregator contains a most open data occurrences of species. 
Regarding Crete, it contains 116,643 occurrences of 7065 unique species,
10189 unique taxa (Figure \ref{fig:crete-samplings}). The data
have been provided by 196 different publishers, e.g Institutions, Museum Collections, Platforms etc.
The majority of occurrences, 98571, are originated from citizen science platforms, i.e 
Cornell Lab of Ornithology, iNaturalist.org,Observation.org, Pl@ntNet and naturgucker.de.
These occurrences were removed from because they are not submitted by experts.
There is a single occurrence of 1684, the oldest in Crete on the GBIF platform. Before 
1960 there are 1021 unique taxa with 2051 occurrences submitted by 62 publishers (Figure \ref{ig:crete_taxa_gbif}).
Edaphobase has in total, 26 resources with 17 different sampling points covering 120 distinct taxa.
A part of it is included in GBIF, yet it provides some additional functionalities. 

\begin{figure}[hbt!] 
    \centering\includegraphics[width=0.7\columnwidth]{crete_gbif_taxa_accumulation_classification.png}
    \caption{Taxa identifications in Crete across the years. Data from GBIF.}
    \label{fig:crete_taxa_gbif}
\end{figure}

IUCN redlist resource download contains multiple comma-separated files that 
complement the assessment of each taxon. 


Regarding the genomic and metagenomic samples from Crete, ENA hosts all the public 
available data and metadata. There are 92 studies with 1205 samples.
From these, the 530 samples are terrestrial metagenomic samples as shown in
Figure \ref{fig:crete-samplings} C. 
The most samples and studies are in the soil metagenome with 164 samples in 7 studies, without the
ISD project presented in Chapter \ref{cha:isd-crete-soil}.
Insect metagenome has 108 samples from 2 studies
and plant metagenomes contains 69 samples in 2 studies.
Then follows air metagenome with 61 samples during 3 studies.

\begin{figure}[hbt!] 
    \centering\includegraphics[width=\columnwidth]{crete_map_crete_samples.png}
    \caption[Samplings in Crete]{Samplings in Crete. A. GBIF occurrences in Crete.
    B. Edaphobase samplings in Crete. C. ENA metagenomic samples in Crete (without ISD). D. WoSIS soil samples in Crete.}
    \label{fig:crete-samplings}
\end{figure}

WoSIS contains soil samplings about physical and chemical characteristics. Regarding 
Crete, it has 27 sampling locations. These data were collected during 3 EU projects, namely,
EU-FOREGS, EU-GEMAS and EU-SPADE.

\subsection{Maps}

Crete is km\textsuperscript{2} in area size. It is a mosaic, as the ten
terrestrial environmental layers indicate in Figure \ref{fig:crete_data_cube_map}.
In terms of aridity, the largest area falls under the Semi-Arid class with 6244.44 km\textsuperscript{2},
followed by Dry sub-humid with 1323.28 km\textsuperscript{2}, and Humid with 537.20 km\textsuperscript{2}, Table \ref{table:crete_data_cube_area}.

\begin{figure}[hbt!] 
    \centering\includegraphics[width=0.93\columnwidth]{crete_data_cube_maps.png}
    \caption[Crete data cube]{The ten layers of the terrestrial environments of Crete.
    A. The digital elevation map. B. The Corine Land Cover EEA, Level 2.
    C. The geology of Crete. D. The Harmonised World Soil dataset, version 2.
    E. The Historic Land Dynamics Assessment dataset. F. The different protection areas of the island.
    G. The mean surface temperature of WorldClim2. H. The mean precipitation of  WorldClim2.
    I. The aridity index of soil, from the Global Aridity Index v3. J. The desertification risk assessment}
    \label{fig:crete_data_cube_map}
\end{figure}

Crete has undergone significant land-use changes in the past 40 years. In total, 250.23 km\textsuperscript{2}
of land transitioning to urban, while 56.57 of urban fabric has moved to various other land types.
Cropland experienced the extensive changes, shifting 648.72 km\textsuperscript{2}
units to pasture/rangeland, 163.42 km\textsuperscript{2} to forest.
Meanwhile, forests increased during that period by 365 km\textsuperscript{2}, Table \ref{table:crete_data_cube_area}.

Key transitions include:
\begin{itemize}
    \item Urban expansion, with the most significant development occurring from agricultural land.
    \item Conversion of croplands to various uses, notably to pastures, forests, and unmanaged grasslands.
    \item Transitions from pastures to urban areas and different types of vegetation.
    \item Forest developments, including both expansion and contraction in various regions.
    \item Water bodies have a considerable area but no change in the total area reported.
\end{itemize}

The geological map of Crete reveals a complex stratigraphy,characterized by carbonate
rocks. Namely these include the limestone of Tripolis (1252.97 km\textsuperscript{2}),
Plattenkalk (1346.89 \textsuperscript{2}), Limestone of Pindos (247.97 \textsuperscript{2}).
The island also has a Neogenic section (Mk, 811.96; Mm.I, 1614.25). The
Quaternary alluvium (Q.al, 910.75) and other sedimentary rocks (Ph-T, 1012.08; f, 117.83; fo, 317.59; ft, 275.55)
contributing to the diverse geological features of Crete.

HWSD (Harmonized World Soil Database) results reveal a diverse range of soil types.
Predominant among these are Lithic Leptosols (4777 km\textsuperscript{2}),
representing 61.11\% of the sampled soils.
These are typically shallow soils on hard rocky outcrops.
Calcaric Regosols, accounting for 2250.80 instances (27.82\%),
are another category. These are shallow, light-coloured soils, frequently
found on karst landscapes.
Calcaric Fluvisols, Chromic Luvisols, and Eutric Cambisols were also
identified, with 598.40 (7.29\%), 767.04 (9.53\%), and 97.92 (1.21\%) occurrences
respectively. These soils are typically associated with alluvial deposits, have
a clay-enriched subsoil, and possess a high base saturation.

The highest degree of protection of Crete habitats was observed in the Special Areas
of Conservation (Habitats Directive) with 2371.42 km\textsuperscript{2}.
It is followed by the Wildlife Refuge (610.70 km\textsuperscript{2}),
and Special Protection Area (Birds Directive) at 1261.78 km\textsuperscript{2}.
Furthermore, the Core zone of the National (Woodland) Park, i.e Samaria Gorge, covers 47.56 km\textsuperscript{2}.
In summary, the protection directives of Crete's natural areas are span several categories and cover 2900 km\textsuperscript{2}
of Crete with some conservation sites overlapping.

High-risk areas are predominantly represented by fragile regions with varying
degrees of desertification risk. The fragile regions classified as F3 (high fragility)
encompass an area of approximately 1,118 km\textsuperscript{2},
indicating a significant level of vulnerability to desertification processes.
Furthermore, the critical areas, which are classified based on their potential
threat level, include low (C1) with 769.70 km\textsuperscript{2}, medium (C2)
with 623.50 km\textsuperscript{2}, and high (C3) with 232.20 km\textsuperscript{2}.
Area coverage of land-use management, conservation efforts, geology and
desertification risks in Crete, are presented in Table \ref{table:crete_data_cube_area}.

\begin{longtable}{ll}
    \caption{Area cover, in km\textsuperscript{2}, of the different categories per spatial data layer of Crete. \label{table:crete_data_cube_area}} \\
\textbf{hilda+ transition 1978-2018}               & \textbf{area}    \\
\endfirsthead
%
\endhead
%
urban (stable)                                     & 250.23           \\
urban to cropland                                  & 5.06             \\
urban to pasture/rangeland                         & 4.03             \\
urban to unmanaged grass/shrubland                 & 1.01             \\
cropland to urban                                  & 56.57            \\
cropland (stable)                                  & 1,193.61         \\
cropland to pasture/rangeland                      & 648.72           \\
cropland to forest                                 & 163.42           \\
cropland to unmanaged grass/shrubland              & 5.05             \\
pasture/rangeland to urban                         & 64.47            \\
pasture/rangeland to cropland                      & 145.34           \\
pasture/rangeland (stable)                         & 4,694.29         \\
pasture/rangeland to forest                        & 205.83           \\
pasture/rangeland to unmanaged grass/shrubland     & 127.28           \\
pasture/rangeland to sparse/no vegetation          & 1.01             \\
forest to cropland                                 & 1.01             \\
forest to pasture/rangeland                        & 23.22            \\
forest (stable)                                    & 162.33           \\
unmanaged grass/shrubland to urban                 & 10.08            \\
unmanaged grass/shrubland to cropland              & 41.40            \\
unmanaged grass/shrubland to pasture/rangeland     & 271.59           \\
unmanaged grass/shrubland to forest                & 20.20            \\
unmanaged grass/shrubland (stable)                 & 10.09            \\
sparse/no vegetation to cropland                   & 9.11             \\
sparse/no vegetation to pasture/rangeland          & 84.87            \\
sparse/no vegetation to forest                     & 1.01             \\
sparse/no vegetation (stable)                      & 6.05             \\
water                                              & 164.57           \\
\textbf{CLC 2}                                     &                  \\
Arable land                                        & 88.29            \\
Artificial, non-agricultural vegetated areas       & 20.68            \\
Forests                                            & 299.93           \\
Heterogeneous agricultural areas                   & 1102.96          \\
Industrial, commercial and transport units         & 39.77            \\
Inland waters                                      & 6.74             \\
Mine, dump and construction sites                  & 10.05            \\
Open spaces with little or no vegetation           & 410.66           \\
Pastures                                           & 59.25            \\
Permanent crops                                    & 2367.70          \\
Scrub and/or herbaceous vegetation associations    & 3797.83          \\
Urban fabric                                       & 110.63           \\
                                                   &                  \\
\textbf{Protected areas}                           & \textbf{area}    \\
Aesthetic Forest                                   & 0.17             \\
Sites of Community Importance (Habitats Directive) & 0.34             \\
Game breeding station                              & 1.02             \\
Controlled hunting area                            & 11.59            \\
Core zone in National (Woodland) Park              & 47.56            \\
UNESCO-MAB Biosphere Reserve                       & 88.65            \\
Protected Forest                                   & 417.74           \\
Wildlife Refuge                                    & 610.70           \\
Special Protection Area (Birds Directive)          & 1261.78          \\
Special Areas of Conservation (Habitats Directive) & 2371.42          \\
total protected (no overlap)                       & 2900.59          \\
total protected                                    & 4810.96          \\
\textbf{Aridity index class}                       & \textbf{area}    \\
Semi-Arid                                          & 6244.44          \\
Dry sub-humid                                      & 1323.28          \\
Humid                                              & 537.20           \\
                                                   &                  \\
\textbf{Soil (HWSD2)}                              & \textbf{area}    \\
Calcaric Fluvisols                                 & 598.40           \\
Calcaric Regosols                                  & 2250.80          \\
Chromic Luvisols                                   & 767.04           \\
Eutric Cambisols                                   & 97.92            \\
Lithic Leptosols                                   & 4777.00          \\
NA                                                 & 4.08             \\
                                                   &                  \\
\textbf{elevation}                                 & \textbf{area}    \\
(0,200{]}                                          & 2227.85          \\
(200,400{]}                                        & 2192.21          \\
(400,600{]}                                        & 1584.84          \\
(600,800{]}                                        & 815.99           \\
(800,1000{]}                                       & 484.21           \\
(1000,1200{]}                                      & 355.34           \\
(1200,1400{]}                                      & 249.44           \\
(1400,1600{]}                                      & 150.83           \\
(1600,1800{]}                                      & 91.55            \\
(1800,2000{]}                                      & 75.50            \\
(2000,2200{]}                                      & 39.69            \\
(2200,2400{]}                                      & 12.23            \\
(2400,2600{]}                                      & 0.21             \\
                                                   &                  \\
\textbf{geology}                                   & \textbf{area}    \\
-                                                  & 0.57             \\
J-E Plattenkalk                                    & 1346.89          \\
K-E Limestone of Pindos                            & 247.97           \\
K.k Limestone of Tripolis                          & 1252.97          \\
K.m Carbonaceous Allochthonous                     & 12.95            \\
Mk Neogenic                                        & 811.96           \\
Mm.I Neogenic                                      & 1614.25          \\
Ph-T Phylites - Chalazites                         & 1012.08          \\
Q.al Quaternary alluvium                           & 910.75           \\
T.br  Limestone of Tripolis                        & 299.58           \\
f Schale                                           & 117.83           \\
fo Schale of Pindos                                & 317.59           \\
ft Flysch of Tripolis                              & 275.55           \\
o Ophilite Complex Allochthonous                   & 93.94            \\
                                                   &                  \\
\textbf{Desertification risk}                      & \textbf{area}    \\
N non-affected                                     & 159.10           \\
F2 fragile (medium)                                & 1900.60          \\
F1 fragile (low)                                   & 1483.50          \\
P potential                                        & 1556.60          \\
C2 critical areas (medium)                         & 623.50           \\
Other areas                                        & 283.80           \\
F3 fragile (high)                                  & 1118.00          \\
C1 critical areas (low)                            & 769.70           \\
C3 critical areas (high)                           & 232.20           \\
\textbf{Crete total}                               & \textbf{8345}




\end{longtable}


\section{Discussion}\label{crete_idea_discussion}

The wide range of languages and year of historical publications illustrates the
global impact and interdisciplinary nature Cretan historical biodiversity literature.
These results indicate Crete's rich archaeological and historical past,
combined with its diverse geological features, have made it an attractive subject
for multidisciplinary research. Further studies will undoubtedly continue to
unearth more insights into Crete's intriguing history and natural environment.

Yet a lot of work is needed to rescue the hidden biodiversity knowledge in
historical literature, natural history collections and herbaria \parencite{Paragkamian2022}.
In Crete this is apparent from the 2 orders of magnitude difference in the documents
in GBIF before 1960 and the imaged documents in Biodiversity Heritage Library. The 
difference in taxa are in 2 orders in magnitude less in GBIF. These data 
require specialised curators to rescue and is crucial to include for 
further comparisons and information homogeneity. Of course there must be 
a lot of documents with important information not even imaged, thereby under 
critical threats to become lost.

Worldwide projects of microbiome studies have collected one or two topsoil
samples from Crete \parencite{Vasar2022, Labouyrie2023, Bahram2018, Orgiazzi2018}.
Some have focused on soil fungi \parencite{Mikryukov2023, Davison2021, Tedersoo2021}
and other soil eukaryotes \parencite{Aslani2022}.
The only thorough microbiome study of a soil ecosystem in Crete, to our knowledge,
is in the north west part of the island, the Koiliaris Critical Zone Observatory \parencite{tsiknia2014}.

\begin{figure}[hbt!] 
    \centering\includegraphics[width=\columnwidth]{crete_map_crete_lter.png}
    \caption[LTERs in Crete]{Long term monitoring stations in Crete}
    \label{fig:crete-lter}
\end{figure}

Apart from the soil microbiome, soil physical and chemical properties has been
investigated by global and European projects like the Forum of European Geological Surveys
(FOREGS) \parencite{nerc19017}, the Geochemical Mapping of Agricultural and Grazing Land
Soil in Europe (GEMAS) \parencite{REIMANN2018302} and the Soil Profile Analytical
Database for Europe (SPADE) \parencite{Hiederer2006}. These data with other data
not publicly available have led to the creation of the \href{https://esdac.jrc.ec.europa.eu/content/soil-map-greece-0}{soil map} of Crete 
and Greece in general. These maps are not as complete as the geological maps (Figure \ref{fig:crete_data_cube_map} C), 
and that is the reason of the coarse mapping of the Harmonized World Soil Database map of Crete (Figure \ref{fig:crete_data_cube_map} D).
Soil type results highlight the diversity of soil types on Crete, with a
dominance of Lithic Leptosols and Calcaric Regosols, which may be linked to the
region's geomorphological features (karst, limestone).

In addition to all these wealth of data, there are long term monitoring stations in 
Crete, the most in Greece, Figure \ref{fig:crete-lter}.
Artificial reefs monitoring (ARMS-MBON) with 2 stations \parencite{obst2020arms},
European marine omics biodiversity observation network with one station \parencite{santi2023emobon},
LTER-Greece network with 3 stations \parencite{Skoulikidis2021lter} and the POSEIDON system with 2 stations \parencite{ntoumas2022}. 
These are marine (5) and terrestrial (3) station collecting biodiversity data and abiotic data.
Focusing in a confined area provides unique opportunities for data integration.
For example in Samothrace \parencite{noll2024insights} a 15 year long study 
showed practices to help sustain the socioecological system.
Crete would be a great example of the digital representation of it's ecosystems
and the human activities in order to tighten the relations with the society and 
protect vulnerable areas.

As complex systems ecologists have stated bringing together information across
scales \parencite{brown2004METABOLIC} and functions leads to emergent properties that weren't possible to 
predict before \parencite{smith2016Origin}. The techniques to integrate are multiple in terms of statistics and modelling. But,
the advent of Large Language Models revolutionised the data interpretation and 
user interaction. In ecology and conservation this is very promising \parencite{doi2024biodiversity}.
The presented knowledge base of Crete would be a suitable 
case study to investigate these significant advancements to tighten the integration and form a 
user application. In addition, it put on front what is known and what is missing 
in order to prioritise the next steps. The data integration would serve a useful 
source to perform statistical analysis and enrich datasets.

