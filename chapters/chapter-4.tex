% --------------------------------------------------
% 
% This chapter is for Crete system ecology
% 
% --------------------------------------------------


\chapter{Island Sampling Day and Crete soil microbial interactome}
\label{cha:crete-soil}

%\textbf{Citation:} \\ 


% ISD ABSTRACT
\section{Abstract}
Ecosystem functioning is an integral part of the sciences of climate change and
conservation. Microbes are known for their versatility, abundance and influence
on ecosystem functions, but a more thorough analysis of ecosystems should
include a more complex assembly of organisms. For example, plant-arthropod-soil
microbiome interactions are important associations that have historically been
ignored. A synthesized knowledge base of biodiversity, in terms of ecological
and remote-sensing data remains a major challenge. Many worldwide studies have
been published regarding soil microbiome ecosystems, though there are many gaps
to cover the complexity of functioning and biodiversity in the local scale.
Islands can be important case studies for this integration, from micro to macro,
with Crete being a great example. Here, we utilize the Island Sampling Day
Crete 2016 microbial 16S rRNA gene amplicon data and metadata, integrated with
soil, faunistic and remote sensing data, to decipher the drivers of ecosystem
function of the island. Cretan macroecology has been studied for centuries for
its diverse and unique geology, fauna and flora. In addition, Crete as a
continental island, presents a distinct natural and evolutionary history with
high contrasts in vegetation cover, climatic conditions and geology. The Island
Sampling Day Crete 2016 project (co-hosted by the Genomic Standards Consortium
(GSC) and the Institute of Marine Biology, Biotechnology and Aquaculture (IMBBC-HCMR)),
has collected microbial 16s amplicon data from 72 distinct, with ecosystem
diversity topsoil sites from all around Crete, accompanied by
FAIR (Findable, Accessible, Interoperable and Reproducible) data by design.
With this island wide study the GSC put the genomic standards in action. The
preliminary results indicate a notable influence of the pH and the elevation
over the island's microbiota. In particular, sites in higher altitudes found to
be inhabited by a more diverse number of microorganisms, a pattern commonly
seen in several faunistic groups, such as arthropods. These data along with the
open access data regarding arthropod fauna, flora and the vast data of remote
sensing and biodiversity, provide the basis to identify major drivers of
biodiversity, to evaluate hotspots and contribute to foreknowledge of
threatened ecosystems.

% ISD INTRODUCTION
\section{Introduction}
\label{sec:crete-soil-intro}

% ISD METHODS
\section{Methods}
\label{sec:crete-soil-method}

% ISD RESULTS
\section{Results}
\label{sec:crete-soil-results}

% ISD DISCUSSION
\section{Discussion}
\label{sec:crete-soil-discussion}

% 
