% --------------------------------------------------
% 
% This chapter is for Crete system ecology
% 
% --------------------------------------------------


\chapter{Crete soil ecosystem integration}
\label{cha:crete-soil}


%\textbf{Citation:} \\ 


% ISD ABSTRACT
\section{Abstract}

% ISD INTRODUCTION
\section{Introduction}
\label{sec:crete-soil-intro}

Soil ecosystems are the cornerstone of terrestrial functioning.
Biodiversity interactions are between all domains of life which form
multilevel associations. Bacteria, archaea, unicellular eukaryotes, nematodes,
earthworms, arthropods, molluscs, plants, mammals; all occur in unison and 
influence the ecosystems they inhabit with their abundance, biomass \citep{bar2018biomass} and metabolism.
The plant-insect-soil ecosystem is starting to be studied as a whole to discover
important associations with practical implications such as plant resistance 
to insect attack \citep{plant-insect-soil2023}.


Bacteria \citep{Delgado-Baquerizo-atlas}

Nematodes \citep{vandenHoogen2019}

Earthworms \citep{Phillips2021}

Arthropods \citep{milo-arthropods}

Springtails \citep{potapov2023Globally}

Plant traits and soil microbiome interaction \citep{beugnon2022Abiotic}

All of these life forms occur side by side and influence on another. This is visible in the 
top of the mountains \citep{winkler2018side}

% ISD METHODS
\section{Methods}
\label{sec:crete-soil-method}

% DECO RESULTS
\section{Results}
\label{sec:crete-soil-results}

% DECO DISCUSSION
\section{Discussion}
\label{sec:crete-soil-discussion}

% 
