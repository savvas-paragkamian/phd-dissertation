% --------------------------------------------------
% 
% This chapter is the introduction
% 
% --------------------------------------------------


\chapter{Introduction}
\label{cha:intro}

% Integration of the general approach of the PhD


\section{Ecosystem ecology}
\label{sec:intro-ecosystem}

Ecosystem ecology principles \citep{Chapin_Matson_Vitousek_2011}

\section{Text mining}
\label{sec:text-mining}

Text as input

History and contemporary literature as corpus. 

Information extraction. 

Metadata normalisation.



\section{Spatial Dimention}
\label{sec:crete-spatial}

Spatial dimention is shaping the biodiversity

Along with time, these dimentions are the cornestone of 
biological questions. 


Remote sensing revolution

\section{Micro and Macro}
\label{sec:crete-micro-macro}

Biodiversity and ecosystems contain life across the tree of life, from
bacteria, to nematoda, to arthropods, to plants, to birds, to mammals to 
humans. 

How to acquire such data, traditional biodiversity sampling 

metabarcoding, 
metagenomics,

\section{Intergation}
\label{sec:crete-integration}

The integration of biodiversity knowledge in one place is a longstanding
goal in ecological research \citep{Walter_2012}. The synthesis of multiple
data types and datasets across the globe has enabled 
holistic approaches to crutial scientific and sociatal questions \citep{heberling_j_mason_data_2021}.

\subsection{Data and tools}
\label{sec:data-tools}

Digital Earth
The concept of Digital Earth, first coined in Al Gore’s book entitled 
“Earth in the Balance” (Gore 1992), was further developed in a speech
written for Gore at the opening of the California Science Center in 1998.

Digital Twin

Island Digital Ecosystem Avatars (IDEA)

Flow of information though web channels of communication.

\subsection{Methods of integration}
\label{sec:meth-int}

Apart from data, ecosystem holistic understanding requires integrative methods.
Multilayer networks \citep{marine-multilayers}

Metabolism is a scale independant biological process spanning from chemical reactions
to biogeochemical cycles \citep{hall2018understanding}. 

Information theory is a conceptual framework to bring together agents,
their interactions and the flow of information through these interactions \citep{oconnor-information-ecology}.


\subsection{Islands as models}
\label{sec:island-model}

Why islands

Crete 

Historical and contemporary ecological research
