% --------------------------------------------------
% 
% This chapter is the introduction
% 
% --------------------------------------------------


\chapter{Introduction}
\label{cha:intro}

% Integration of the general approach of the PhD

\section{Ecosystem ecology}
\label{sec:intro-ecosystem}

Ecosystem ecology principles \citep{Chapin_Matson_Vitousek_2011} and terrestrial
ecology. 

Comprehension of ecosystem function is one of the pinnacles of ecology and
requires quantitative and conceptual advances \citep{Chapin_Matson_Vitousek_2011}.
Ecosystem function refers to the processes that occur in an ecosystem, their
interconnectance, the agents that carry them out
and their relationship with the environment. Biodiversity
\citep{hooperEFFECTSBIODIVERSITYECOSYSTEM2005, loreau2001Biodiversity}
and community structure and dynamics \citep{gonze2018Microbial,morris2020Linking}
are known parameters of ecosystem function. This holistic approach must be taken
into account when tackling complex systems such as ecosystems. Studying and
combining these parameters covers multiple fields of ecological science;
metabolic ecology, biogeochemistry, community and population ecology and
environmental ecology. Adding to this complexity, ecosystem behaviour cannot be
explained or predicted with just vast amounts of data because interconnectance
and interdependence lead to emergent phenomena in spatial and temporal scales.
Currently we are experiencing a shift in understanding ecosystem function
because the accumulated knowledge from ecological theories is quantified and
supported with experimental data.

Large scale -omics experiments have introduced a vast quantity of data from the
environment regarding sequences, proteins and small molecules \citep{shaffer2022Standardized}.
System biology has been established as a means to study community ecology interactions spanning
from the molecular level to the community level \citep{raes2008molecular}.
One area that has emerged from omics analyses is microbial ecology and
consequently the metabolic processes of ecosystems \citep{perez_garcia2016Metabolic}
which unify all living scales, their diversity and complex dynamics \citep{smith2016Origin}.
Microbes are claimed to be one of the major drivers of ecosystem metabolic
processes of all kinds \citep{falkowski2008microbial,hall2018understanding} and
their percentage in total biomass is enormous, comparable to plants \citep{bar2018biomass}.
These discoveries make microbes and microbiomes perfect candidates when
studying ecosystem function \citep{klitgord2011Ecosystems,widder2016Challenges}.

Microbial ecology brought together all scales of ecosystem functioning, from
molecules to biogeochemical cycles \citep{hall2018understanding,kempes2012Growth,raes2011molecular}.
Several large scale sampling projects have been launched like the Earth
Microbiome Project, Tara Oceans, Global Ocean Sampling Expedition (GOS), and
many small ones\todo{add refs about the projects}.
This explosion of metagenomic and microbiome data was followed by the creation
of many repositories and databases, some project specific and some that
collected many different sources. The metadata richness of metagenomic data
varies depending on the sampling and it may include location data, habitat type
and environmental parameters like pH, humidity and nutrients. Metadata
deficiency of data is a limiting step for the ecosystem function analysis
which, in the short-term can be overcomed with data integration solutions from
alternative sources (for example satellite data for the sampling region). In
the long-run, initiatives like data FAIRification will improve their
interoperability. Nevertheless, microbiome data can be integrated with microbe
specific data of metabolism and pathways using ontologies like KEGG, Gene
Ontology and REACTOME and with habitat description from the environment
ontology (EnvO). Bringing together all the above requires web technologies and
data enrichment methodologies which are only recently starting to be
implemented in microbial ecology \citep{jiang2016Microbiome}. This integration
will create maps of microbes, metabolism and environments. 

\section{Text mining}
\label{sec:text-mining}

Text as input

History and contemporary literature as corpus. 

Information extraction. 

Metadata normalisation.

However, this mapping derived from well structured data forms like databases
and ontologies will cover only a small part of the current knowledge. Most
scientific information is scattered in journal publications which consist of
text that is unstructured. Hence, text mining is needed to synthesize
information from publications \citep{jensen2006Literature}. Microbiome and
human diseases research as stated in \citep{badal2019Challenges} have been
greatly advanced by text mining. This is still in its infancy in microbial
ecology. Bringing together text mining associations and maps derived from
databases will provide a huge knowledge base consisting of microbes, their
metabolism and their habitat. The creation of the knowledge base is the first
step towards relation deciphering between these data types. \todo{add PREGO and other similar platforms}

\section{Micro and Macro}
\label{sec:crete-micro-macro}

Biodiversity and ecosystems contain life across the tree of life, from
bacteria, to nematoda, to arthropods, to plants, to birds, to mammals to 
humans. 

How to acquire such data, traditional biodiversity sampling 

metabarcoding, 
metagenomics,

Network analysis based on the knowledge base will quantify the relations
between microbes, their metabolism and ecosystem
function \citep{graham2016Microbes,muller2018Using, perez_garcia2016Metabolic}.
Microbial interactions can be inferred from the aforementioned data as shown
in \citep{machado2021Polarization} based on their habitat and metabolism.
These community dynamics can be used to build ecosystem level metabolic
networks \citep{perez_garcia2016Metabolic} that influence nutrient cycling in
ecosystems \citep{bauer2018Network}. Another approach is by using species as a
container of pathways and then examining the functions accumulated in the
ecosystem \citep{loucaDecouplingFunctionTaxonomy2016}. Both approaches have
been applied to specific ecosystems with applicable results but relying on a
global knowledge base will expand ecosystem function analysis into many
different ecosystems that have been sampled.

Microbes are known for their versatility, abundance and influence on ecosystem
functions. Yet, for a more complete analysis all forms of life must be included,
especially plants \citep{thompson2012Food}. But viruses as well because they
change the metabolic processes of microbes and are highly important in
ecosystem functions \citep{hurwitz2016Viral}. Another restriction is the fact
that the bulk of analyses regarding microbial data are using associations
through different categories of data. Associations are very useful and can
provide important insights but more rigorous and explanatory methodologies
regarding bioenergetics \citep{kempes2012Growth}, population dynamics
\citep{gonze2018Microbial}, ecosystem stability \citep{berdugo2020Global} and
metabolic ecology \citep{brown2004METABOLIC} should be incorporated as well to
advance our understanding of ecosystem function.

\section{Spatial Dimention}
\label{sec:crete-spatial}

Spatial dimention is shaping the biodiversity

Along with time, these dimentions are the cornestone of 
biological questions. 


Remote sensing revolution

Use these data to monitor resilience \citep{Lenton2022resilience}

\section{Intergation}
\label{sec:crete-integration}

Microbial ecology challenges the views on the relationships of ecology and evolution.

Conceptual challenges in microbial community ecology \citep{prosser2020Conceptual}

Investigating the eco-evolutionary response of microbiomes to environmental change \citep{martiny2023Investigating}

Opportunities to advance the synthesis of ecology and evolution \citep{loreau2023Opportunities}


The integration of biodiversity knowledge in one place is a longstanding
goal in ecological research \citep{Walter_2012}. The synthesis of multiple
data types and datasets across the globe has enabled 
holistic approaches to crutial scientific and sociatal questions \citep{heberling_j_mason_data_2021}.

Need for policy regarding Earth System Science for Global
Sustainability \citep{reid2010earth}.

\subsection{Data and tools}
\label{sec:data-tools}

Current biodiversity knowledge is in mulitple digital forms. 
Taxa names, traits and metadata are mainly in matrix data
structures. Taxa spatial distributions are in spatial data
forms like polygons or points. In additions taxa information
is available in sequences in fasta format. Abiotic data like 
temperature, pH, nutrients are also stored in matrix
and spatial formats. Audiovisual files contain information
of species and ecosystems in images, audio and video formats.
In addition, mathematical models contail mechanistic information
about many different aspects of ecology and algorithms, in the 
form of code, as well. All the above are communicated and dissimated through the
literature in the form of text.

Digital Earth
The concept of Digital Earth, first coined in Al Gore’s book entitled 
“Earth in the Balance” (Gore 1992), was further developed in a speech
written for Gore at the opening of the California Science Center in 1998.

Digital Twin

Island Digital Ecosystem Avatars (IDEA)

Global Earth Observation System of Systems (GEOSS).

GEO BON (Biodiversity Observation Network) and Soil BON

Earth system

Flow of information though web channels of communication.

A user friendly interface web portal can provide all this wealth of data
regarding the multifaceted ecosystem variables. 

Currently there are many global databases with distinct and overlapping 
scope like GBIF, OBIS, iDigBio, EOL etc. \citep{feng2022Review}. 
The contents of these databases covers taxonomy, spatial extend, traits, 
connections to speciments, species literature, distributions,
IUCN status.

These portals in order to remain relavant and updated require 
continuous funding and development. Thus, Research Infrastructures (RI) are 
the most suitable organisational models to sustain these portals. Examples 
of such RIs are :

\begin{itemize}

    \item eLTER: European long-term ecosystem, critical zone and socio-ecological systems research infrastructure 
    \item LifeWatch ERIC
    \item Other \todo{fill other examples}

\end{itemize}

Yet, a synthesised knowledge base of biodiversity, in terms of ecological and
remote-sensing data remains a major challenge \citep{feng2022Review}. The challenge
is mainly on the different data standards used but also in the implementation of 
data exchange models and synchronisation.

\subsection{Methods of integration}
\label{sec:meth-int}

Apart from data and their hosting, ecosystem holistic understanding requires
integrative methods. Macrosystems approach uses interactions of local parameters
such as biological, geophysical and sociacultural and explores their infuence in
the macro scale \citep{heffernan2014}.

Multi omics and communing ecology \citep{jurburg2022community}

Network inferrence and analysis for biogeochemical cycling \citep{jameson2023Network}

Multilayer networks \citep{marine-multilayers}

Metabolism is a scale independant biological process spanning from chemical reactions
to biogeochemical cycles \citep{hall2018understanding}. Bioenergetic models on food 
webs can provide information about energy exchanges in the food webs \citep{valdovinos2023bioenergetic}.

Information theory is a conceptual framework to bring together agents,
their interactions and the flow of information through these interactions \citep{oconnor-information-ecology}.

Structural Equation models

Causality

\subsection{Values and motivation}
\label{sec:values}


Ecosystem functioning is an integral part of the sciences of climate change
and conservation \citep{cavicchioli2019scientists}. These fields undertake the
sustainability of ecosystems such as coral reefs and tropical forests, when
faced with habitat loss, biodiversity loss, pollution and generally any change
in the natural environment conditions. Monitoring these ecosystems is crucial
and environmental data analysed using network theory will facilitate real time
inspection and inspire immediate action \citep{derocles2018Biomonitoring}.
Ecosystem services, on the other hand, aim at utilising ecosystem resources
and functions for all sorts of human activities ranging from goods supplied
from agriculture and farming \citep{alvarez-silva2017Compartmentalized} to
recreational and educational purposes. In addition, the one health concept has
shown that microbiomes are the for human health and prosperity
\citep{banerjee2023Soil, lehmann2020concept}. Yet, the pressure poised from human activities
is reducing the ecosystem services \citep{rillig2023Increasing}.

Last but not least, terraformation
projects explore ecosystem function with a bottom-up approach for engineering
systems to become habitable for humans, like planet Mars, or transforming
Earth's collapsed ecosystems to new habitable states
\citep{conde-pueyo2020Synthetic}. Using the data and methods described in this
article will enhance our understanding in each of these areas. Knowledge
integration from data, metadata and text sources combined with ecological and
metabolic theory will expand our knowledge and eventually lead to new
knowledge regarding the aforementioned fields. A soil monitoring framework 
is overdue to track and inform the sociatey about the health of 
soil ecosystems \citep{guerra2021tracking}.

\section{Islands as models}
\label{sec:island-model}

Why islands
Many worldwide studies regarding soil ecosystems are being implemented, yet
there are many gaps to cover the complexity of functioning and biodiversity
\citep{guerra2020Blind}. Knowledge is lacking for specific taxonomic groups and
ecosystems for example the Sahara desert and belowground fauna.
These gaps are crutial to be studied in order to
improve our understanding of soil ecosystems \citep{cameron2018Global}. 
Islands can be important case studies for this integration.

\subsection{Crete}

Location and natural history

Historical and contemporary ecological research

Microbiome

\section{Aim of this research}
\label{sec:aim}
