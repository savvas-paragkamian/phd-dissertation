% --------------------------------------------------
% 
% This chapter is the introduction
% 
% --------------------------------------------------


\chapter{Introduction}
\label{cha:intro}

% Integration of the general approach of the PhD

Comprehension of ecosystem function is one of the pinnacles of ecology and
requires quantitative and conceptual advances. Ecosystem function refers to the
processes that occur in an ecosystem, their interconnectance, the agents that
carry them out and their relationship with the environment. Biodiversity \citep{hooperEFFECTSBIODIVERSITYECOSYSTEM2005, loreau2001Biodiversity}
and community structure and dynamics \citep{gonze2018Microbial,morris2020Linking}
are known parameters of ecosystem function. This holistic approach must be taken
into account when tackling complex systems such as ecosystems. Studying and
combining these parameters covers multiple fields of ecological science;
metabolic ecology, biogeochemistry, community and population ecology and environmental ecology. Adding to this complexity, ecosystem behaviour cannot be explained or predicted with just vast amounts of data because interconnectance and interdependence lead to emergent phenomena in spatial and temporal scales. Currently we are experiencing a shift in understanding ecosystem function because the accumulated knowledge from ecological theories is quantified and supported with experimental data. The aim of this review is to illustrate the state of the art methods of data and text mining and how to interpret them in order to decipher ecosystem functions.

Large scale -omics experiments have introduced a vast quantity of data from the environment regarding sequences, proteins and small molecules. System biology has been established as a means to study community ecology interactions spanning from the molecular level to the community level (Raes and Bork, 2008). One area that has emerged from omics analyses is microbial ecology and consequently the metabolic processes of ecosystems (Perez-Garcia et al., 2016) which unify all living scales, their diversity and complex dynamics (Smith and Morowitz, 2016). Microbes are claimed to be one of the major drivers of ecosystem metabolic processes of all kinds (Falkowski et al., 2008; Hall et al., 2018) and their percentage in total biomass is enormous, comparable to plants (Bar-On et al., 2018). These discoveries make microbes and microbiomes perfect candidates when studying ecosystem function (Klitgord and Segrè, 2011; Widder et al., 2016).

Microbial ecology brought together all scales of ecosystem functioning, from molecules to biogeochemical cycles (Hall et al., 2018; Kempes et al., 2012; Raes et al., 2011). Several large scale sampling projects have been launched like the Earth Microbiome Project, Tara Oceans,  Global Ocean Sampling Expedition (GOS),  and many small ones. This explosion of metagenomic and microbiome data was followed by the creation of many repositories and databases, some project specific and some that collected many different sources. The metadata richness of metagenomic data varies depending on the sampling and it may include location data, habitat type and environmental parameters like pH, humidity and nutrients. Metadata deficiency of data is a limiting step for the ecosystem function analysis which, in the short-term can be overcomed with data integration solutions from alternative sources (for example satellite data for the sampling region). In the long-run, initiatives like data FAIRification will improve their interoperability. Nevertheless, microbiome data can be integrated with microbe specific data of metabolism and pathways using ontologies like KEGG, Gene Ontology and REACTOME and with habitat description from the environment ontology (EnvO). Bringing together all the above requires web technologies and data enrichment methodologies which are only recently starting to be implemented in microbial ecology (Jiang and Hu, 2016). This integration will create maps of microbes, metabolism and environments. 

However, this mapping derived from well structured data forms like databases and ontologies will cover only a small part of the current knowledge. Most scientific information is scattered in journal publications which consist of text that is unstructured. Hence, text mining is needed to synthesize information from publications (Jensen et al., 2006). Microbiome and human diseases research as stated in (Badal et al., 2019) have been greatly advanced by text mining. This is still in its infancy in microbial ecology. Bringing together text mining associations and maps derived from databases will provide a huge knowledge base consisting of microbes, their metabolism and their habitat. The creation of the knowledge base is the first step towards relation deciphering between these data types. 

Network analysis based on the knowledge base will quantify the relations between microbes, their metabolism and ecosystem function (Graham et al., 2016; Muller et al., 2018; Perez-Garcia et al., 2016). Microbial interactions can be inferred from the aforementioned data as shown in (Machado et al., 2020) based on their habitat and metabolism. These community dynamics can be used to build ecosystem level metabolic networks (Perez-Garcia et al., 2016) that influence nutrient cycling in ecosystems (Bauer and Thiele, 2018). Another approach is by using species as a container of pathways and then examining the functions accumulated in the ecosystem (Louca et al., 2016). Both approaches have been applied to specific ecosystems with applicable results but relying on a global knowledge base will expand ecosystem function analysis into many different ecosystems that have been sampled.

In this review we relied mostly on microbes because of their versatility, abundance and influence on ecosystem functions. For a more complete analysis all forms of life must be included, especially plants (Thompson et al., 2012). But viruses as well because they change the metabolic processes of microbes and are highly important in ecosystem functions (Hurwitz and U’Ren, 2016). Another restriction is the fact that the bulk of analyses regarding microbial data are using associations through different categories of data. Associations are very useful and can provide important insights but more rigorous and explanatory methodologies regarding bioenergetics (Kempes et al., 2012), population dynamics (Gonze et al., 2018), ecosystem stability (Berdugo et al., 2020) and metabolic ecology (Brown et al., 2004) should be incorporated as well to advance our understanding of ecosystem function.

Ecosystem functioning is an integral part of the sciences of climate change and conservation (Cavicchioli et al., 2019). These fields undertake the sustainability of ecosystems such as coral reefs and tropical forests, when faced with habitat loss, biodiversity loss, pollution and generally any change in the natural environment conditions. Monitoring these ecosystems is crucial and environmental data analysed using network theory will facilitate real time inspection and inspire immediate action (Derocles et al., 2018). Ecosystem services, on the other hand, aim at utilising ecosystem resources and functions for all sorts of human activities ranging from goods supplied from agriculture and farming (Alvarez-Silva et al., 2017) to recreational and educational purposes. Last but not least, terraformation projects explore ecosystem function with a bottom-up approach for engineering systems to become habitable for humans, like planet Mars, or transforming Earth's collapsed ecosystems to new habitable states (Conde-Pueyo et al., 2020). Using the data and methods described in this article will enhance our understanding in each of these areas. In our view, knowledge integration from data, metadata and text sources combined with ecological and metabolic theory will expand our knowledge and eventually lead to new knowledge regarding the aforementioned fields.

\section{Ecosystem ecology}
\label{sec:intro-ecosystem}

Ecosystem ecology principles \citep{Chapin_Matson_Vitousek_2011} and terrestrial
ecology. 

\section{Text mining}
\label{sec:text-mining}

Text as input

History and contemporary literature as corpus. 

Information extraction. 

Metadata normalisation.



\section{Spatial Dimention}
\label{sec:crete-spatial}

Spatial dimention is shaping the biodiversity

Along with time, these dimentions are the cornestone of 
biological questions. 


Remote sensing revolution

\section{Micro and Macro}
\label{sec:crete-micro-macro}

Biodiversity and ecosystems contain life across the tree of life, from
bacteria, to nematoda, to arthropods, to plants, to birds, to mammals to 
humans. 

How to acquire such data, traditional biodiversity sampling 

metabarcoding, 
metagenomics,

\section{Intergation}
\label{sec:crete-integration}

The integration of biodiversity knowledge in one place is a longstanding
goal in ecological research \citep{Walter_2012}. The synthesis of multiple
data types and datasets across the globe has enabled 
holistic approaches to crutial scientific and sociatal questions \citep{heberling_j_mason_data_2021}.

\subsection{Data and tools}
\label{sec:data-tools}

Digital Earth
The concept of Digital Earth, first coined in Al Gore’s book entitled 
“Earth in the Balance” (Gore 1992), was further developed in a speech
written for Gore at the opening of the California Science Center in 1998.

Digital Twin

Island Digital Ecosystem Avatars (IDEA)

Flow of information though web channels of communication.

\subsection{Methods of integration}
\label{sec:meth-int}

Apart from data, ecosystem holistic understanding requires integrative methods.
Multilayer networks \citep{marine-multilayers}

Metabolism is a scale independant biological process spanning from chemical reactions
to biogeochemical cycles \citep{hall2018understanding}. 

Information theory is a conceptual framework to bring together agents,
their interactions and the flow of information through these interactions \citep{oconnor-information-ecology}.

Structural Equation models

Causality

\subsection{Islands as models}
\label{sec:island-model}

Why islands

Crete 

Historical and contemporary ecological research
