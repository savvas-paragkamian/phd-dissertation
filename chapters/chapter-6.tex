% --------------------------------------------------
% 
% This chapter is for Crete system ecology
% 
% --------------------------------------------------


%\chapter{Towards a Cretan soil biodiversity data model}
\chapter{Exploring the soil microbiome coupling with Crete island data cube}
\label{cha:crete-soil}

%\textbf{Citation:} \\ 


% ISD ABSTRACT
%\section{Abstract}
%    Microbes are known for their versatility, abundance
%    and influence on soil ecosystem functioning.
%    A synthesized knowledge base of microbial biodiversity, in terms of
%    ecological and remote-sensing data remains a major challenge.
%    Many worldwide studies have been published regarding soil
%    microbiome ecosystems, though there are still many blind spots.
%    Islands can be important case studies for this integration for more resolute and dense samplings.
%    Here, we utilize the Island Sampling Day Crete 2016 microbial 16S rRNA gene
%    amplicon data, integrated with soil and remote
%    sensing data, to decipher the drivers of ecosystem function of the island.
%    The Island Sampling Day Crete 2016 project has collected 144 topsoil samples
%    from 72 sites, capturing a lot of this diversity, accompanied by FAIR
%    (Findable, Accessible, Interoperable and Reproducible) data by design. 
%    Cretan macroecology has been studied for centuries for its diverse  and endemic
%    fauna and flora.
%    In addition, Crete has been considered as a miniature continent with high contrasts in
%    vegetation cover, elevation, climatic conditions. 
%    We show that, higher altitudes in Crete found to
%    be inhabited by a more diverse number of microorganisms, a pattern commonly
%    seen in several faunistic groups, such as arthropods.
%    The integration of the spatial data with state of the art methods enabled warning signals
%    in pristine and grazing ecosystems.
%    These results along with the
%    climatic and desertification index influences on the soil microbiome of Crete,
%    provide the basis to identify major drivers of biodiversity, to evaluate hotspots
%    and contribute to foreknowledge of threatened ecosystems.
%
\section{Introduction}\label{intro_integration}

Soil ecosystems are the cornerstone of terrestrial habitats, biodiversity and henceforth human activities.
Soils are characterised by multiple properties; chemical, physical and biological that 
form complex interdependent interactions. Biodiversity of soils covers
all forms of life, fauna, flora, bacteria, archaea, fungi, viruses. 
Bacteria and archaea are considered major drivers for the functionality of soil.
They influence and are influenced by their environment and their community structure 
defies their macroscopic functionality \parencite{Bahram2018}.
Global soil microbiome studies have been employed to decipher soil microbiome
compositions \parencite{thompson2017a-communal, Delgado-Baquerizo-atlas, Labouyrie2023},
functions \parencite{Bahram2018} and biogeography \parencite{Martiny2006, guerra2020Blind}.
These results showed the remarkable diversity in soils yet there are blind spots \parencite{guerra2020Blind}
and these sampling are sparse when considering samples per area density. One of most resolute
study is by \parencite{Karimi2020} which exemplified the 
vast complexity of soil bacterial communities and the requirement of
dense samplings and isolated systems \parencite{Dini-Andreote2021}.

Data and metadata of these samplings are stored in different databases, yet 
great effort from these distinct communities have led to establishing standards
to enable FAIR data \parencite{wilkinson2016the-fair}. For amplicon sequences the Genome Standards
Consortium \parencite{Field2011} has established the MIMARKS \parencite{yilmaz2011minimum}
standards among others, and has been a advocate for open and unrestricted data \parencite{Amann2019}.
Examples of rich metadata and platforms of hosting open data and digital soil maps are the 
European Soil Data Centre \parencite{Panagos2022} and the World Soil Information
Service (WoSIS) of the ISRIC \parencite{Batjes2024}. Apart from samplings,
spatial data are openly available terrestrial ecosystems.
Climatic data, land cover, desertification risk, aridity, soil type, normalized
vegetation index, bedrock geological formations. From the bacterial point of view, 
there curated databases that classify in some baseline functionality. 

Crete is a continental forarc island \parencite{ali2016}, fifth largest island of the Mediterranean (8350 km\textsuperscript{2}),
and a Mediterranean biodiversity hotspot \parencite{myers2000biodiversity}.
The island of Crete has been studied since the classical times for its'
fauna \parencite{Sidiropoulos_Polymeni_Legakis_2017,Anastasiou2018Tenebrionid}, flora \parencite{Krimbas_2005} and ecosystems \parencite{Grove1993}.
Crete is home to the only endemic mammal of Greece, the Cretan shrew (\textit{Crocidura zimmermanni}),
more than 350 endemic arthropods \parencite{bolanakis2024} and 183 endemic plants \parencite{Kougioumoutzis2020}
among them a tree \textit{Zelkova abelicea}. Multifaceted factors have shaped the
biodiversity of the island, for example the sharp elevation gradient \parencite{trigas2013elevational, FAZAN2017},
the complex evolutionary history \parencite{POULAKAKIS2002} and the human - nature
interactions over thousands of years \parencite{Vogiatzakis2008_med, Sfenthourakis2017}.
The major threats of human activities are becoming apparent in the island's ecosystems,
like desertification \parencite{KARAMESOUTI2018266}, intensive grazing \parencite{JouffroyBapicot2016},
climate change \parencite{Kougioumoutzis2020,Vogiatzakis2016} and habitat loss \parencite{ISPIKOUDIS1993259}.
Yet the topsoil microbial diversity of Crete has been unexplored.

Disentangling the soil ecosystem functioning requires an holistic and 
multidisciplinary approach \parencite{vogel2022}. The integration of the aforementioned
data is needed to understand the biogeochemical cycles along with biodiversity interactomes 
that have been characterised as a driver of community composition soil
functioning \parencite{GUSEVA2022108604}.
Regarding the latter there are still big challenges to infer actual microbial
interactions remain \parencite{Faust2021}. All this work is needed in order to meet
UN and EU soil goals for the 2030 and 2050 for healthy soils \parencite{LAL2021e00398}.

In this study we ask: what the differences in the microbiome communities in different land use types?
Are there any climatic, geological, elevational, aridity or functional factors that affected the differences?
How the interactome changes over different land use types and climate?
Are there any distinctions between arid regions of Crete?
To address these questions, we integrate multiple types of data and methods to decipher hidden 
signals of the Island Sampling Day soil microbiome data. In total, 144 samples from
72 sites (2 samples per site) of Crete are used with their metadata.
Warning signals are also identified for various types of ecosystems.


\section{Materials and Methods}\label{integration_methods}

\begin{figure}[t] 
    \centering\includegraphics[width=\columnwidth]{crete_integration_soil_microbiome}
    \caption{Workflow of this study. Data integration of ISD data with multiple types of spatial data. Then, a threefold analysis of function annotations, network analysis and differential abundance. All these data and methods are used to focus on specific taxa, ecosystems and threats.}
    \label{fig:workflow}
\end{figure}

\subsection{Island Sampling Day: Crete}\label{isd_data}

\subsection{Crete data cube}\label{spatial_data}

The compilation of Crete data cube has multiple spatial data layers of global,
European, Greek or Cretan scale. 
The Copernicus CORINE Land Cover has 3 layers resolution that classify the land
use and cover in shapefile format \parencite{CLC2023}. 
WorldClim 2.0 contains global climatic data for 12 variables, e.g annual mean
temperature and annual precipitation \parencite{Fick2017}.
The Environmentally Sensitive Areas Index to desertification (ESAI) 
of Greece dataset \parencite{KARAMESOUTI2018266} were utilised. Additionally the 
Global Aridity Index and Potential Evapotranspiration Database \parencite{zomer2022version} was include.
Geological formations shapefiles were downloaded from the geoportal of
Decentralized Administration of Crete, which were developed by
Crinno-Emeric Group project\footnote{\url{https://geoportal.apdkritis.gov.gr/gis/apps/storymaps/stories/19690f65abbe4e8ab0141b2fe7261a8c}}.
The Harmonised World Soil database v2 was incorporated for the soil mapping units and 
the soil taxonomic classification \parencite{fao2023}.
Handling and analysis of these data was done with the sf and terra R packages \parencite{Pebesma2023}.

\subsection{Integrative Analysis and Annotations}\label{int_analysis}
The network inference was facilitated with FlashWeave 0.19.2 \parencite{Tackmann2019}.
To use FlashWeave we reduced 
the abundance table of the ASVs to keep the ones at the genus or species level.
%In addition, we filtered these taxa that appeared in SOSO samples and had more than 
%SOSO mean relative abundance.
The subsequent network analyses were carried out
with the igraph R package \parencite{Csardi2006}.

For taxa function annotation we used the manually curated FAPROTAX database and python script \parencite{loucaDecouplingFunctionTaxonomy2016}.
Whereas for the differential abundance analysis with ANCOM-BC2 R package \parencite{Lin2023}.
Numerical ecology analyses,e.g diversity indices, NMDS, PERMANOVA we calculated
with the vegan R package \parencite{oksanen2024vegan}.
For PCoA ordination we used ape R packege \parencite{Paradis2004} and UMAP python library\parencite{mcinnes2018umap-software}.

\subsection{Tools}\label{Coding environment}
PEMA for OTU inference \parencite{zafeiropoulos2020pema}
U-CIE R package for coloring 3 dimensional data \parencite{Koutrouli2022}

Visualisation was implemented with ggplot2 \parencite{wickham_ggplot2_2016} and pheatmap \parencite{Kolde2019}.
The environment we worked had Python 3.11.4, R version 4.3.2 \parencite{rcoreteam}
and Julia language version 1.9.3 \parencite{Julia-2017}in Julia language version 1.9.3 \parencite{Julia-2017}.
Finally, computations were performed on HPC infrastructure of HCMR \parencite{zafeiropoulos_0s_2021}.

\subsection{Data and Code}
The documentation and scripts developed for this study are available in
\href{https://github.com/savvas-paragkamian/crete_soil_microbiome/}{Crete soil microbiome github repository}.
This repository contains all the necessary scripts for the data retrieval,
filtering and ASV inference, taxonomy assignment, data integration of spatial data, 
functional annotation and the subsequent analyses and visualisation.
Additional scripts about data integration are available in
\href{https://github.com/savvas-paragkamian/crete-data-integration}{Crete data integration}.
Code is structured to be reproducible and interoperable.

\section{Results}\label{integration_results}


\subsection{Samplings}



\subsection{Data cube}

Crete data cube.


\subsection{Soil microbiome}\label{soil_microbiome}

\begin{sidewaystable*}
    \caption{Summary of the different spatial layers in Crete in terms of total area, number of samples and microbial diversity.\label{table:data_cube_summary}}
\begin{tabular*}{\textwidth}{@{\extracolsep{\fill}}llllllll@{\extracolsep{\fill}}}
\tabcolsep=0pt%
class                                           & area & category             & samples & taxa richness & asv richness & mean shannon & sd shannon \\
Arable land                                     & 88   & CLC LABEL2           & 4       & 1518           & 6178          & 4.73          & 0.17        \\
Artificial, non-agricultural vegetated areas    & 21   & CLC LABEL2           & NA      & NA             & NA            & NA            & NA          \\
Forests                                         & 300  & CLC LABEL2           & 4       & 1803           & 8663          & 4.91          & 0.18        \\
Heterogeneous agricultural areas                & 1103 & CLC LABEL2           & 31      & 13701          & 59581         & 4.83          & 0.24        \\
Industrial, commercial and transport units      & 40   & CLC LABEL2           & 4       & 1760           & 6517          & 4.79          & 0.32        \\
Inland waters                                   & 7    & CLC LABEL2           & NA      & NA             & NA            & NA            & NA          \\
Mine, dump and construction sites               & 10   & CLC LABEL2           & NA      & NA             & NA            & NA            & NA          \\
Open spaces with little or no vegetation        & 411  & CLC LABEL2           & 7       & 3132           & 14568         & 4.85          & 0.21        \\
Pastures                                        & 59   & CLC LABEL2           & 4       & 1609           & 7535          & 4.85          & 0.14        \\
Permanent crops                                 & 2368 & CLC LABEL2           & 22      & 9999           & 43833         & 4.88          & 0.18        \\
Scrub and/or herbaceous vegetation associations & 3798 & CLC LABEL2           & 58      & 24394          & 110127        & 4.76          & 0.29        \\
Urban fabric                                    & 111  & CLC LABEL2           & 4       & 1765           & 9828          & 4.78          & 0.19        \\
-                                               & 1    & Geology              & NA      & NA             & NA            & NA            & NA          \\
J-E                                             & 1347 & Geology              & 22      & 8995           & 41730         & 4.75          & 0.25        \\
K-E                                             & 248  & Geology              & NA      & NA             & NA            & NA            & NA          \\
K.k                                             & 1253 & Geology              & 27      & 11277          & 51737         & 4.72          & 0.25        \\
K.m                                             & 13   & Geology              & NA      & NA             & NA            & NA            & NA          \\
Mk                                              & 812  & Geology              & 13      & 5618           & 24633         & 4.79          & 0.2         \\
Mm.I                                            & 1614 & Geology              & 12      & 5259           & 22854         & 4.86          & 0.15        \\
Ph-T                                            & 1012 & Geology              & 14      & 6720           & 29742         & 4.92          & 0.19        \\
Q.al                                            & 911  & Geology              & 34      & 15580          & 64851         & 4.91          & 0.27        \\
T.br                                            & 300  & Geology              & 2       & 625            & 2839          & 4.4           & 0.24        \\
f                                               & 118  & Geology              & 2       & 720            & 4753          & 4.5           & 0.01        \\
fo                                              & 318  & Geology              & 2       & 825            & 4606          & 4.65          & 0.04        \\
ft                                              & 276  & Geology              & 10      & 4062           & 19085         & 4.79          & 0.17        \\
o                                               & 94   & Geology              & NA      & NA             & NA            & NA            & NA          \\
N                                               & 163  & Desertification Risk & NA      & NA             & NA            & NA            & NA          \\
F2                                              & 1945 & Desertification Risk & 53      & 23404          & 103956        & 4.82          & 0.26        \\
F1                                              & 1518 & Desertification Risk & 22      & 8817           & 41020         & 4.7           & 0.23        \\
P                                               & 1593 & Desertification Risk & 36      & 16256          & 73256         & 4.86          & 0.19        \\
C2                                              & 638  & Desertification Risk & 6       & 2113           & 9105          & 4.59          & 0.22        \\
Other areas                                     & 290  & Desertification Risk & NA      & NA             & NA            & NA            & NA          \\
F3                                              & 1144 & Desertification Risk & 14      & 5900           & 25362         & 4.82          & 0.28        \\
C1                                              & 788  & Desertification Risk & 7       & 3191           & 14131         & 4.93          & 0.25        \\
C3                                              & 238  & Desertification Risk & NA      & NA             & NA            & NA            & NA          \\
Semi-Arid                                       & 6336 & Aridity class        & 98      & 43064          & 184930        & 4.83          & 0.25        \\
Dry sub-humid                                   & 1343 & Aridity class        & 32      & 13649          & 65134         & 4.79          & 0.21        \\
Humid                                           & 545  & Aridity class        & 8       & 2968           & 16766         & 4.55          & 0.19       
\end{tabular*}
\end{sidewaystable*}


\subsection{Communities}\label{communities}
Microbial beta diversity differently associated with physical and chemical
features of ISD. PCoA 1 (describing 14\% of the variance) was largely driven by
elevation (F=23, p < 0.001). PCoA 2, which explained 12\% of the variance,
was significantly associated with soil moisture (F=90, p < 0.001).
PCoA 2 also positively correlated with both organic carbon and nitrogen.

Mean annual temperature, elevation and total nitrogen are statistically significant variables for
community dissimilarity across samples (PERMANOVA).

Network of associations (FlashWeave, sensitive) after filtering prevalent ASV (mean normalised relative
abundance>0.001, samples > 2) led to 7,455 ASVs and
29,282 associations (787 are negative).

\subsubsection{Functions}\label{functions}
PREGO


The functional annotation was performed with FAPROTAX. Potential human pathogens 
appear concentrated in Richtis gorge. Most plant pathogens occur in an agricultural 
field in south Rethymnon. 

\begin{figure}[t] 
    \centering\includegraphics[width=\columnwidth]{crete_integration_functions_faprotax}
\caption{Heatmap of the FAPROTAX functions relative abundances per sample.}
    \label{fig:isd_functions_faprotax}
\end{figure}

\subsubsection{Significant taxa}\label{sig_taxa}
Differential abundance (ANCOM-BC2) showed that there are 294 taxa that are significantly different
with Annual mean temperature and elevation. The geological rock type doesn't distinguish taxa
across the island. \textit{Rhodococcus equi} (pathogen of foal or similar) is significantly abundant in
pastures (Corine Land Cover), and has been found in 16 samples.

\section{Discussion}\label{integration_discussion}

Multiple works have emerged the past 5 years about enumerating all
biodiversity \parencite{Anthony2023} and deciphering the biogeochemical 
processes and interactions of fauna, flora and microbes in global
studies \parencite{Fry2019, Crowther2019,GRANDY201640,Delgado-Baquerizo2020} and
in mountain peaks soil microbiomes \parencite{Adamczyk2019}. One of our profound
results is that soil bacterial biodiversity is very complex, even sites a few meters apart can differ
significantly in their community composition, Figure \ref{fig:isd_site_locations}.
This is a fact that is sometimes neglected in worldwide studies and the island biogeography
paradigm can assist to remove clutter.

Apart from the profound diversity in soils, there is also high speciation and uniqueness. 
As shown in Figure \ref{fig:isd_fig2_taxonomy}, most ASVs occur in 1 or 2 or 3 samples.
This is different when compering with the ocean. When using the deepest taxonomic
level of ASVs is possible to identify the specialists and generalists \parencite{Barberan2012}. 
In addition, focusing on the phyla, we see a pattern looking like a phase transition, from 
rare phyla to phyla that dominate all samples, Figure \ref{fig:isd_fig2_taxonomy} C. Lastly, in Figure \ref{fig:isd_fig2_taxonomy} D,
and \ref{fig:isd_top_phyla_samples}, we found some distinct top phyla profiles of samples.

Elevation gradients of biodiversity are known since Humboldt's work \parencite{Rahbek2019} 
yet these patterns remain elusive regarding the soil microbiome \parencite{Looby2020, Siles2023}.
Mostly because it's difficult to isolate other co-founding effects \parencite{Nottingham2018}.
In our results elevation showed important distinction of taxa and of diversity. Yet more work is 
needed to explore the Asterousia mountain transect in isolation to avoid indirect influences of 
other variables.

Crete's ecosystems are mostly semi-arid, whereas in the mountain ranges there 
are areas classified as dry sub-humid and humid. Parent material and influences
on soil functions and bacterial communities have been documented. Here, as shown in Table \ref{table:data_cube_summary},
Quaternary-alluvial sediments (Q.al) and Phyllite-Quartzite series (Ph-T) hold the most diversity 
and richness. Maybe because they are mostly found in riverbeds. Critical areas for 
desertification (C2) hold the most diverse samples. Regarding the land cover the most
rich sample is near the HCMR building, a high touristic area close to the beach. Yet, forests
hold the highest Shannon diversity index.

Richtis gorge (highly popular) has alarmingly high values for human pathogens, sulfur respiration and
nitrate reduction. This gorge has water all year long, rare in the gorges of Crete,
and is the highest touristic attraction of eastern Crete. It's name means "throw" and 
there is a local rumor that people threw unwanted stuff throughout the centuries. 
Nevertheless, it is an important freshwater ecosystem which is not included in 
any protection regimen or legislation.
The potential functions of the bacteria in Richtis gorge are an alarming signal which needs to be further investigated.
Another important finding is the statistically significant presence of \textit{Rhodococcus equi} (pathogen of foal or similar)
in pastures. It is one of the most common causes of pneumonia in foals which
become infected by inhaling dust or soil particles contaminated with the bacterium.

\section{Conclusion}

Deciphering and validating the results presented here requires future work.
Even though amplicon studies in soil should be interpreted with caution \parencite{alteio2021} they 
can act as early warning signals towards public health concerns \parencite{banerjee2023Soil}.
In addition, the immediate release and availability of these data is crucial for 
taking action.
The pillar of data integration is the unrestricted open data across disciplines and 
the open source software.
"A holistic perspective on soil architecture is needed as a key to soil functions" \parencite{philippot2024the-interplay}, is 
an important statement for future soil projects.
Shotgun metagenomics and metatranscriptomics can unleash the functional potential of
topsoil along with other advancements like long reads sequencing. Higher resolution
samplings using grid system will enhance the resolution and also the resampling of
ISD sites in different time points will provide additional insights to the complex soil 
functions and expand the positive and negative associations in soils \parencite{Liu2024}.
Lastly, global hotspots \parencite{Guerra2022} and soil ecosystem conservation is needed as 
a whole and expanding current protection of specific species \parencite{guerra2021tracking}.
This along with policy \parencite{KONINGER2022} across countries \parencite{Putten2023}
without borders and the implementation of legislation in Greece \parencite{SCHISMENOS2022100035} is 
imperative.


