% --------------------------------------------------
% 
% This chapter is for general conclusions
% 
% --------------------------------------------------

\chapter{Conclusions}
\label{cha:conclusions}

The work presented here has combined different approaches of contemporary 
ecological questions.
Regarding microbial diversity on the global scale, the available 
knowledge was harmonised using literature and data mining methods in chapter \ref{cha:prego}.
On the local scale,
in chapter \ref{cha:crete-idea}, the Crete soil model was assembled 
with different types of data, from literature, to biodiversity to spatial data. 
Crete has long been established as a model island for evolution, desertification, biogeographical patterns etc.
This system will assist future analyses of the most studied region of Greece as 
it brings disparate open data all in one place.

A new chapter of the well-studied biodiversity of Crete was initiated, 
the soil microbial diversity.
As expected, the bacterial diversity of the island is very heterogeneous 
with patterns across soil moisture, elevation, land use, mean temperature and bedrock geology.
The large scale of ISD and the confined space of Crete allowed for more resolute 
correlations of diversity. At the same time it removes aspects like latitudinal factors
of the global scale studies which in some cases cannot be explained biologically. 
Seasonality effects are also avoided because all samples were collected in the same day, a critical
feature of this study.
One important result of the ISD analysis is the difference of microbial 
communities from samples a few meters apart. This is a important result
that often is not explicitly mentioned in soil microbial studies. This heterogeneity 
makes finding patterns extremely difficult, a fact that demonstrates that
there are important processes of microbial functioning and
dispersal in soil that are not deciphered yet.

Arthropods have been neglected from most conservation frameworks, biogeochemical models
and microbial studies. With the work presented here,
based on the Crete soil system (chapter \ref{cha:crete-idea})
and the endemic arthropods assessment (chapter \ref{sec:arthropods-intro}),
the foundations are established to move forward both 
conservation and basic research on ecosystem modeling with 
data from the most diverse phylum of terrestrial fauna.
Expert curation was applied for the compilation of historical and contemporary
literature along with specimens from NHMC for the endemic Cretan arthropods occurrences.
After the compilation of the dataset it became clear that a conservation analysis
was a priority because the majority of species were predicted to be threatened.

Literature and data mining 
methodologies are also very useful to rescue historical biodiversity data which are
indispensable as demonstrated in chapter \ref{sec:arthropods-intro}.
The gap of historical data rescue and curator tools was narrowed
through the DECO workflow. DECO streamlined a collection of 
of tools and standards aiming to assist the curators' process. From the comparison
of such tools it became clear that human curation is a transversal undertaking in all steps.
Tools like DECO, are very useful to curators for combining all tools for their process 
to improve quality control.

There is a wealth of available data and tools as demonstrated in 
multiple chapters in this PhD. 
The previous generations of scientists were the first to 
compile an enormous amount of biodiversity data.
Yet a large portion of these data remains unpublished and stored in legacy formats,
making them vulnerable to destruction and inevitable loss.
This PhD has focused on rescue of this data and bringing them to contemporary standards and digital media,
a task that is a challenge of current generations. 
An additional challenge is to bring together and analysing historical data
and contemporary data under the ecosystem approach.

The basic conceptual challenges remain of systems ecology. 
Some of these can be formulated as : What are the causes of ecosystem collapse?
What is needed for a sustainable future?
How will climate warming change life on Earth?
These clear questions require collaboration across scientific fields. 
Interdisciplinary collaboration requires effort from all sides to 
achieve effective communication. 
The exemplary collaboration of science, enterprise and society was
illustrated during the pandemic of COVID-19 in 2020 \parencite{ioannidis2021the-rapid,lee2021scientific}. 
This level of synergy is needed to tackle the environmental issues
as well, like sustainable agriculture.
Different academic cultures can be integrated to form a scientific transculturalism,
the process of integration of the three cultures, variance, coarse-graining, and exactitude \parencite{Enquist_2024}.
These cultures can be vaguely described as natural history, numerical ecology and complex systems ecology, respectively.
An important step to bring these cultures together is communication and openness across scientists.
These gaps must be eliminated soon to reach predictive ecology goals \parencite{mouquet_review_2015}.

Soil health,
and one health in general, is key to avoid the disastrous projections of
current practices in agriculture and industries \parencite{banerjee2023Soil}.
For example, 19\% of soils of Crete are under desertification risk \parencite{KARAMESOUTI2018266}.
The integrative analysis of ISD Crete revealed some health potential risks in 
natural ecosystems like Richtis gorge. This gorge has the highest abundance of 
potential human pathogens of all soil samples of Crete. This is alarming of 
the human activities that take place in this ecosystem like intensive tourism 
and agriculture. Richtis gorge in not protected by any framework, chapters \ref{cha:crete-idea} and \ref{cha:crete-soil}.

More than 60\% of
soils in Europe are considered unhealthy leading to erosion, degradation, contamination,
and disruption of global nutrient cycles \parencite{commission2020caring}.
Economically, it is estimated that due to soil degradation about 50 billion euro per year
are lost in the European Union. 
Yet, soil, the foundation of 
terrestrial ecosystems, is not protected with legislation across Europe.
Since 2020, after the devastating realisations about soil, the European Parliament 
is in the process of structuring a soil monitoring and resilience law to achieve 
the goal of having healthy soils by 2050. As of writing this Thesis, the EU 
parliament has just voted the \href{https://www.europarl.europa.eu/thinktank/en/document/EPRS_BRI(2024)757627}{soil monitoring law}
while Trialoge, Second reading and Adoption steps are still in progress to make it final.

While this PhD is mostly on bioinformatics, field sampling was organised
for the second sampling expedition for ISD Crete.
While analysing the ISD Crete 2016 the sampling replicability was put into action with a 
new sampling in July of 2022.
Using the same protocols and locations, 29 people from HCMR, NHMC, UOC Biology
Department and citizen scientists where split in 10 teams and went sampling. 
The goal of this sampling was to collect a second time point of the same locations
to decipher the metagenomic content of soil. This was a voluntary work supported 
by the SUPP GEN project of HCMR. The DNA extraction and shipment was carried out 
by HCMR and sequencing by the Joint Genome Initiative. DNA extracted by the 72 locations 
is going to be sequenced using whole genome sequencing. 
This project is part of the Island Microbiome Encyclopedia (ISME) consortium.
This is one of the few large scale metagenomic soil projects in
Europe \parencite{nayfach2021a-genomic, ma2023a-genomic}.

Findable, Accessible, Interoperable and Reusable (FAIR) data and reproducible analyses
are the only way for science to assist the society and convince stakeholders for action. 
Open data are the important first step, but a lot more work is required to reach 
the FAIR status of data.
This Thesis is doing exactly that using Crete as a model of application. Across the chapters it is shown how 
integrating multiple types of data can help identify signals of health concerns and 
pressured ecosystems.

The outcome of this bioinformatic PhD is one software tool (\hyperref[cha:deco]{DECO}),
one knowledge base (\hyperref[cha:crete-idea]{Crete soil system}), one online
database (\hyperref[cha:prego]{PREGO}) and
two novel data analyses (\hyperref[cha:arthropods]{Endemic arthropods} and \hyperref[cha:isd-crete-soil]{ISD Crete}).
The code produced during the aforementioned projects consists of more that 12 thousand lines of code 
and is under open access licences and reproducible by design to allow for full transparency and assist similar research projects. 
