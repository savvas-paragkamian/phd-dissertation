% --------------------------------------------------
% 
% This chapter is for Cretan endemic Arthropods
% 
% --------------------------------------------------


\chapter{Cretan Endemic Arthropods: distribution, threats and hotspots}
\label{cha:arthropods}


This work is based on the finding of \textcite{bolanakis2024}.
\textbf{Citation:} \\ 
Giannis Bolanakis, Savvas Paragkamian, Maria Chatzaki, Nefeli Kotitsa, Liubitsa Kardaki and Apostolos Trichas.\\
The conservation status of the Cretan Endemic Arthropods under Natura 2000 network,
Shared co-first authorship.

Accepted the 21st of May 2024 in the Journal of Biodiversity and Conservation.

DOI: \href{https://doi.org/10.21203/rs.3.rs-2671168/v1}{10.21203/rs.3.rs-2671168/v1}\footnote{
   For author contributions and supplementary material please refer to the relevant sections. 
   This is a modified version of the published version,in terms of relevance, coherence and formatting.}



% INTRODUCTION
\section{Introduction}
\label{sec:arthropods-intro}

Functioning of arthropods has been investigating since the seventies \parencite{rosswall1997energetical}.


%% need to modify
Hotspot definitions vary from quantitative methods to experts opinions and curation.
In quantitative methods, grid size and shape influences the determination of
the areas of interest such as hotspots and key biodiversity areas \parencite{hurlbert2007species,nhancale2011the-influence}.
Choosing the size of the grid is not trivial \parencite{mo2019influences} and is dependent
on the conservation goals \parencite{margules2000systematic}. In the past decade,
there have been major advances for conservation standards, guidelines,
frameworks and tools available to be put into action \parencite{bongaarts2019ipbes}.


%%%%% old
Numbering approximately 7,000 species [extrapolated from Fauna Europaea
\parencite{jong2014fauna} and \textcite{legakis2018}], the Arthropods of Crete, Greece,
have been studied for almost two centuries \parencite{Anastasiou2018Tenebrionid}.
Only 135 of these species (1.9\%) have been assessed in IUCN Red List as of
this publication, making Arthropods the third most evaluated group of the
island, behind vascular plants (291) and land mollusks (165).

Habitat loss and degradation occurs throughout Crete as a result of urban,
agricultural and touristic development. This is a major issue since habitat
loss is a major threat in Europe for many Arthropod groups,
e.g. Butterflies \parencite{VanSwaaycommission2010european}, Bees \parencite{nieto2014},
Orthoptera \parencite{hochkirch2016} and Saproxylic Beetles \parencite{Calix_2018}.
Climate change is predicted to induce scarcer yet more intense precipitation,
increase of drought locally \parencite{koutroulis2011spatiotemporal} and shrinkage as well as
possible shifts to the rainfall period \parencite{koutroulis2013impact}. Groups
associated with fresh water could be deeply impacted from the locally increased
drought and the increase in need of water for irrigation and domestic use,
e.g. Odonata \parencite{kalkman2010}, which has become harsher due to the
increase of agriculture and land use \parencite{tzanakakis2020challenges}. Stock raising
(sheep and goats) has always been an important aspect of Cretan life and
economy \parencite{rackham1996the-making}. Overgrazing impacts severely soil erosion,
soil moisture and vegetation \parencite{kosmas2015exploring,orestis2015exploring}. All the
above contribute to a worrying trend for Crete, i.e. the higher percentage of
Threatened endemic Arthropods when compared with the respective European
groups (Appendix Figure \ref{fig:arthropods-figS1}).

Using over 100 publications (as of 2020) and 733 NHMC sampling events
(\href{https://doi.org/10.5281/zenodo.10635645}{Supplementary Material}), we assembled a dataset of 343 taxa (species and subspecies),
with 4,924 records across 1,569 distinct sites of Crete. The taxa
are distributed to eleven orders, with Coleoptera having the most taxa (206)
and Chilopoda and Scorpiones the least (two).

In this study I aim to a) identify cretan arthropods available information
b) compare with the the endemics of Crete
c) their relation with the anthropogenic pressures in these sites.
To do so, I assembled the accumulated knowledge of 
online databases.
Secondly, we committed to the Findable, Accessible, Interoperable and Reproducible (FAIR) principles \parencite{wilkinson2016the-fair}
for data and code (see \href{https://doi.org/10.5281/zenodo.10635645}{Supplementary Material}).


% METHODS
\section{Methods}
\label{sec:arthropods-method}
   


    \subsection{Species occurrences}
    \label{subsec:arthropods-data-assemblage}

Based on the above criteria, the selected groups are:
Araneae, Chilopoda, Coleoptera, Diplopoda, Heteroptera, Hymenoptera (Chrysididae, Formicidae, Symphyta), Lepidoptera (Geometridae), Odonata, Orthoptera, Scorpiones and Trichoptera.

We curated the bibliography and the NHMC collection to assemble taxa
occurrences. The bibliography used
contains both historical and contemporary published material.
The coordinate reference system we used for all location data is WGS84 - EPSG:4326.

From here on, we refer to both species
and subspecies as “taxa”.
    
All species 
from IUCN that are in Crete were downloaded from the website \parencite{iucn2024}. IUCN provides the 
interactive maps selection functionality. 

Criterion B, i.e. on the Extent of Occurrence (EOO) and
Area of Occupancy (AOO).

Criterion B is the widely used for Arthropods \parencite{cardoso2011adapting,carpaneto2015a-red-list},
because the majority of Arthropods taxa are missing neccessary information of the other criteria (A, C, D and E).
Criterion B could overestimate the danger of Arthropods \parencite{cardoso2011adapting},
which should always be taken into consideration.

    
    \subsection{Grids and hotspots}
    \label{subsec:arthropods-ehs-kbas}

In order to avoid biases concerning the grid cell size, the same
pipeline was tested with cells of different size (4 x 4, 8 x 8 and 10 x 10 km).
For the subsequent analyses we opted for the 10 x 10 km grid (see section \ref{subsec:arthropods-grids})
which is also the EEA reference grid, the standard for the reporting format
(Groups of Experts, 2017) of the Resolution No. 8 (2012) of the Standing Committee
to the Bern Convention on the Emerald Network of Areas of Special Conservation Interest (ASCI).
Moreover, the EHs of the various cell sizes are aggregated in the same areas.
We made the same treatment for each of the selected groups separately.
We redefined EHs as the 10\% of the grid cells with max overlap of the orders
to check for biases towards more speciose orders (e.g. Coleoptera).


    \subsection{Spatial overlaps}
    \label{subsec:arthropods-spatial}
I retrieved land use categories from the CORINE Land Cover, CLC 2018 version
v.2020-20u1 \parencite{CLC2023}.
To evaluate the current land use I used the CORINE Land Cover
and to examine the human pressure (change in land use, agriculture), I used the
Historic Land Dynamics Assessment (HILDA+) dataset \parencite{winkler2021global} to
estimate the change of land use the from 1998 to 2018. Furthermore, I examined
the overlap of the AOO.


    \subsection{Code}
    \label{subsec:arthropods-tools}

    We performed the analyses using the R Statistical Software \parencite{rcoreteam},
the visualization using the ggplot2 R package \parencite{wickham_ggplot2_2016}. The figures
created are colored using the colorblind-friendly 'Okabe-Ito' palette \parencite{ichihara2008color}.
We calculated EOO and AOO using the ConR R package \parencite{dauby2017conr:} and PACA
using custom scripts. For the spatial data handling, transformations and
geometry we used the sf v1.0-14 \parencite{pebesma2018simple} and terra v1.7-55 R packages \parencite{hijmans2024terra}.
Adaptive grid is created using the quadtree R package \parencite{friend2023quadtree}.
Jaccard similarity was calculated with the vegan 2.6-4 R package \parencite{oksanen2024vegan}.
All scripts are reproducible by design and available in this 
\href{https://github.com/savvas-paragkamian/arthropoda_assessment_crete}{GitHub repository}.

Data, scripts and results of the analysis are available and documented \href{https://github.com/savvas-paragkamian/arthropods_assessment_crete}{here} 

% RESULTS
\section{Results}
\label{sec:arthropods-results}

   \begin{figure}[ht]
      \centering
      \includegraphics[width=\textwidth,height=\textheight,keepaspectratio]{figures/arthropods-FigS1.png}
      \caption[Comparisons of Threatened Endemics in Crete, Greece, Europe and the World]{Comparisons of Threatened Endemics in Crete, Greece, Europe and the World for seven Arthropod groups (top to bottom, data aggregated from the IUCN web resource). Values are absolute (proportion in parentheses). Also, note that there are 3 threatened species of Scorpiones in the World Red List column and 5 not threatened species in the Trichoptera column.}
      \label{fig:arthropods-figS1}
   \end{figure}

    \subsection{Grid cell size}
    \label{subsec:arthropods-grids}
The grid cell size 10 x 10 km is the most suitable for our study since our
dataset - being compiled from numerous different sources and sampling efforts -
is rather coarse and inhomogenous for a smaller cell size.
The unique taxa of the EHs of each grid is distributed as follows: 10 km=283,
8 km=278, 4 km=293, adaptive cells=267, with the 4km grid covering most endemic
species. The 4 km grid mostly highlighted areas known for their tourist/recreational activities,
indicating that it is more sensitive to sampling intensity.
Focusing on sampling we applied the adaptive grid size with quadtrees resulting
in 157 grids with 8 km length, 38 with 4 km and 74 with 2 km (Appendix Figure \ref{fig:arthropods-figS5}).
This indicates the preference of larger cells for the majority of our dataset
even though a small percent of regions has higher density of sampling.
The highest overlap among all grids is between the 10 km and 8 km reaching
57\% (Appendix Table \ref{table:arthropods-tableS2}). Finally, the 10 km grid has more taxa
per cell and is a reference grid system.
Based on our analysis and interoperability and reproducibility aims we choose
the 10 km EEA reference grid for the EHs and candidate KBAs inference.
Nevertheless, we also performed the WEGE analysis for KBAs using the adaptive
grid, yielding practically the same areas as the 10km grid minus Zakros (Figure \ref{fig:arthropods-fig5}).
The same pipeline can not be done with EHs for they require a fixed cell size.

   \begin{figure}[ht]
      \centering
      \includegraphics[width=\textwidth,height=\textheight,keepaspectratio]{figures/arthropods-Fig_quads.png }
      \caption[Adaptive grid size based on sampling using quadtrees]{Adaptive grid size based on sampling using quadtrees. A. The proportion of threatened species on all the cells, 157 grids with 8 km length, 38 with 4 km and 74 with 2 km. B. The endemic taxa density with quad cells. C. WEGE with quad cells.}
      \label{fig:arthropods-figS5}
   \end{figure}

   \begin{figure}[ht]
      \centering
      \includegraphics[width=\textwidth,height=\textheight,keepaspectratio]{figures/arthropods-fig_crete_sampling_intensity_order.png}
      \caption[Sampling intensity for every order]{Sampling intensity (i.e number of samples) in each 10 km grid for all samples and for each arthropod order.}
      \label{fig:arthropods-figS7}
   \end{figure}


\begin{table}[]
    \caption{Different grid size hotspots overlap. The top 10\% of cells with most species are considered as hotspots. All units are in km\textsuperscript{2}.}
\begin{tabular}{cccccc}
\textbf{Grids}              & \textbf{1 km\textsuperscript{2}} & \textbf{4 km\textsuperscript{2}} & \textbf{8 km\textsuperscript{2}} & \textbf{10 km\textsuperscript{2}} & \textbf{Adaptive cell size} \\
\textbf{1 km\textsuperscript{2}}              & \textbf{126}   & 72             & 55             & 53              & 41                          \\
\textbf{4 km\textsuperscript{2}}              & -              & \textbf{918}   & 503            & 497             & 349                         \\
\textbf{8 km\textsuperscript{2}}              & -              & -              & \textbf{1317}  & 800             & 691                         \\
\textbf{10 km\textsuperscript{2}}             & -              & -              & -              & \textbf{1400}   & 646                         \\
\textbf{Adaptive cell size} & -              & -              & -              & -               & \textbf{1312}              
\end{tabular}
\label{table:arthropods-tableS2}
\end{table}


\begin{table}[]
    \caption{Corine Land Cover LEVEL 1 and 2 categories area and overlaps with protected areas and hotspots/KBAs}
\resizebox{\textwidth}{!}{%
\begin{tabular}{lllllll}
LABEL1                        & LABEL2                                          & Total area (km\textsuperscript{2}) & \begin{tabular}[c]{@{}l@{}}Natura2000\\ (km\textsuperscript{2})\end{tabular} & \begin{tabular}[c]{@{}l@{}}Wildlife\\ (km\textsuperscript{2})\end{tabular} & \begin{tabular}[c]{@{}l@{}}Hotspots\\ (km\textsuperscript{2})\end{tabular} & WEGE KBAs (km\textsuperscript{2}) \\
Agricultural areas            & Arable land                                     & 88.29            & 34.11                                                      & 1.03                                                     & 26.90                                                    & 26.91           \\
Agricultural areas            & Heterogeneous agricultural areas                & 1102.96          & 130.00                                                     & 19.71                                                    & 93.42                                                    & 100.17          \\
Agricultural areas            & Pastures                                        & 59.25            & 15.11                                                      & 1.81                                                     & 8.50                                                     & 9.71            \\
Agricultural areas            & Permanent crops                                 & 2367.70          & 122.71                                                     & 34.87                                                    & 146.30                                                   & 164.33          \\
Artificial surfaces           & Artificial, non-agricultural vegetated areas    & 20.68            & NA                                                         & NA                                                       & NA                                                       & NA              \\
Artificial surfaces           & Industrial, commercial and transport units      & 39.77            & 0.65                                                       & 0.97                                                     & 0.33                                                     & 0.33            \\
Artificial surfaces           & Mine, dump and construction sites               & 10.05            & 0.57                                                       & 1.15                                                     & 0.32                                                     & 0.32            \\
Artificial surfaces           & Urban fabric                                    & 110.63           & 2.63                                                       & 1.45                                                     & 3.54                                                     & 3.92            \\
Forest and semi natural areas & Forests                                         & 299.93           & 199.20                                                     & 60.36                                                    & 163.62                                                   & 146.67          \\
Forest and semi natural areas & Open spaces with little or no vegetation        & 410.66           & 322.04                                                     & 46.58                                                    & 143.27                                                   & 98.52           \\
Forest and semi natural areas & Scrub and/or herbaceous vegetation associations & 3797.83          & 1519.02                                                    & 438.63                                                   & 785.19                                                   & 813.61          \\
Water bodies                  & Inland waters                                   & 6.74             & 2.50                                                       & 1.80                                                     & 0.99                                                     & 0.31           
\end{tabular}%
}
\label{table:arthropods-tableS3}
\end{table}

   \begin{figure}[ht]
      \centering
      \includegraphics[width=\textwidth,height=\textheight,keepaspectratio]{figures/arthropods-Fig4.png}
      \caption[AOO, EOO and N2K overlaps per order]{A. Locations, EOO and AOO of all orders. Each dot represents one taxon and boxplots show the mean value and the first and third quantiles. The y axis is in log10 scale. B. Proportion of overlap of AOO with Natura 2000 areas per Order. Each dot is a taxon with its respective proportion of AOO overlap. The horizontal line of the boxplot shows the average, and the box shows the 1st and 3rd quantiles of the values.}
      \label{fig:arthropods-fig4}
   \end{figure}

    \subsection{Human Intervention}
    \label{subsec:arthropods-human-intervention}


\begin{table}[]
    \caption{The land use transitions in the 20 year period (1998-2018) from the HILDA+ dataset.}
\resizebox{\textwidth}{!}{%
\begin{tabular}{lllll}
HiLDA+ transitions                             & Crete (km\textsuperscript{2}) & Natura2000 (km\textsuperscript{2}) & EHs (km\textsuperscript{2}) & WEGE KBAs (km\textsuperscript{2}) \\
urban (stable)                                 & 326         & 46               & 14        & 8               \\
urban to pasture/rangeland                     & 3           & NA               & NA        & NA              \\
cropland to urban                              & 42          & 1                & NA        & NA              \\
cropland (stable)                              & 1314        & 186              & 71        & 63              \\
cropland to pasture/rangeland                  & 565         & 74               & 32        & 28              \\
cropland to forest                             & 178         & 40               & 20        & 33              \\
cropland to unmanaged grass/shrubland          & 3           & 1                & NA        & NA              \\
cropland to sparse/no vegetation               & 3           & 3                & 3         & NA              \\
pasture/rangeland to urban                     & 13          & NA               & NA        & NA              \\
pasture/rangeland to cropland                  & 82          & 15               & 3         & 3               \\
pasture/rangeland (stable)                     & 5124        & 2594             & 1198      & 1210            \\
pasture/rangeland to forest                    & 11          & 6                & 3         & 3               \\
pasture/rangeland to unmanaged grass/shrubland & 137         & 73               & 38        & 33              \\
pasture/rangeland to sparse/no vegetation      & 2           & 2                & 2         & 1               \\
forest to pasture/rangeland                    & 34          & 33               & 25        & 20              \\
forest (stable)                                & 364         & 182              & 174       & 177             \\
forest to unmanaged grass/shrubland            & 2           & 1                & 2         & 2               \\
unmanaged grass/shrubland (stable)             & 1           & NA               & NA        & NA              \\
sparse/no vegetation (stable)                  & 2           & 2                & 2         & NA              \\
water                                          & 165         & 78               & 5         & 3              
\end{tabular}%
}
\label{table:arthropods-tableS4}
\end{table}



   \begin{figure}[ht]
      \centering
      \includegraphics[width=\textwidth,height=\textheight,keepaspectratio]{figures/arthropods-Fig5.png}
      \caption[Land Use of Crete and the changes]{A. CORINE Land Cover, LEVEL 2 of the Copernicus system. B. HILDA+ Land use change for the years 1998-2018.}
      \label{fig:arthropods-fig5}
   \end{figure}


% DISCUSSION
\section{Discussion}
\label{sec:arthropods-discussion}

    \subsection{Endemicity Hotspots}
    \label{subsec:arthropods-Endemicity-Hotspots}

Mountains host a great amount of Earth’s biodiversity, being a main driver for
the birth of species \parencite{antonelli2018geological,noroozi2018hotspots,rahbek2019building,Rahbek2019}
and a crucial frontier for their fate \parencite{steinbauer2018accelerated,urban2018escalator}.
Crete is not an exception to this trend \parencite{kougioumoutzis2020plant,trigas2013elevational}.
Our results conform to that, since the EHs are gathered primarily in the major
Cretan mountains. Lefka Ori and Dikti are the sites with the most
EHs, in agreement with studies focused on vascular plants \parencite{dimitrakopoulos2004questioning,kougioumoutzis2020plant}.
\textcite{sfenthourakis2001hotspots}, employing invertebrate groups, also recovered these mountains as EHs.

Islands are biodiversity sanctuaries \parencite{whittaker2007island}, and
so are mountains \parencite{rahbek2019humboldts}. Our work advocates for approaches that
treat islands and mountains under a holistic perspective. The combination of
the two provides a complex biogeographical interplay governing the forces of
speciation, preservation and extinction of biodiversity \parencite{steinbauer2016topography-driven}.
This synergistic effect of mountains-islands has also been recovered in other
areas such as the Balearic islands \parencite{guardiola2023are-mediterranean}.

    \subsection{Species assessment}
    \label{subsec:arthropods-species-assessment-disc}

Arthropods with wider ranges that are not assessed as Threatened under
criterion B, are not necessarily Least Concern and should not be neglected.
Arthropod communities can be affected by the reduction of the abundance of
common and abundant species that offer important functions to the biocommunity.
Wide range does not guarantee high abundance (even though this is true for many
taxa) and even common species can be threatened \parencite{habel2018vanishing,klink2023disproportionate}.

The inclusion of Arthropod taxa in protected areas is often insufficient, with
Arthropods experiencing declines inside the protected areas \parencite{borges2005ranking,chowdhury2023protected,harry2019protected,rada2019protected}.
In fact, even when certain Arthropod groups are adequately included in N2K,
there are gaps and omissions \parencite{sanchez-fernandez2008are-the-endemic,verovnik2011is-the-natura}.
At a global level 75\% of Insects are not sufficiently covered by protected
areas \parencite{chowdhury2023three-quarters}. Crete stands in an intermediate position,
following the general trend of Greece’s N2K adequacy, being the best covered
area at a national level \parencite{kougioumoutzis2021plant,spiliopoulou2021the-natura}.
However, there are some clear gaps regarding certain taxa, encouraging more
locally focused conservation policies complementary to N2K. For example actions
need to be taken for KBAs that fall outside N2K like Kritsa and Zakros.

Biases towards Arthropods cause their poor coverage by protected
areas \parencite{chowdhury2023protected,damen2013protected,delso2021protected}. These
biases derive from geography, size, color and charisma \parencite{cardoso2012habitats,mammola2020towards,wang2021out-of-sight},
and even from political/economic reasons \parencite{dias-silva2021protected}. For example,
the strongest driver for a conservation program funding within the European
Union is the online popularity \parencite{mammola2020towards}. The unpopularity of
Arthropods has begun to change \parencite{wagner2021insect}, especially through citizen
science, which is a trend we should build on to properly conserve the Arthropods.

    \subsection{Human Intervention in Arthropods’ EHs}
    \label{subsec:arthropods-human-intervention-ehs}
Human activities account for almost 20\% of the EHs. The primary human activity
in the EHs is agriculture (~19.6\%). Agricultural intensification is one of the
most important drivers of Arthropods’ decline \parencite{bruhl2019biodiversity,habel2019agricultural,raven2021agricultural}.
Moreover, threats associated with agriculture are the number one threat for
Insect species inside protected areas in Europe \parencite{chowdhury2023protected}.
Nevertheless, regarding change in land use, there is a somewhat equal
transition trend from cropland to forest and vice versa inside EHs and KBAs
(Appendix Table \ref{table:arthropods-tableS4}). This means that while some sites are being
degraded others may recover. More research within EHs and KBAs is essential in
order to quantify the impact (negative or positive) of these transitions to the
endemic Arthropods. A vast amount of cropland has been transformed to pasture
lands (Appendix Table \ref{table:arthropods-tableS4}) which requires further examination,
since grazing has both positive [eg. on Gnaphosidae (Spiders) communities \parencite{kaltsas2019overgrazed}]
and negative effects [e.g. Carabidae (Coleoptera) \parencite{kaltsas2013ground}].
The reduction of croplands could be interpreted under the general trend of
urbanization (Appendix Table \ref{table:arthropods-tableS4}), which nevertheless occurs
outside EHs and KBAs, but a shift towards montane areas especially under new
forms of tourism could deeply impact the sites of conservation importance.


