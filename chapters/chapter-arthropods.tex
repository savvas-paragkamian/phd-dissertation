% --------------------------------------------------
% 
% This chapter is for Cretan endemic Arthropods
% 
% --------------------------------------------------


\chapter{Cretan Endemic Arthropods: distribution, hotspots and threats}
\label{cha:arthropods}

% INTRODUCTION
\section{Introduction}
\label{sec:arthropods-intro}

Functioning of arthropods has been investigating since the seventies \parencite{rosswall1997energetical}.

Importance of arthropods in functioning
\parencite{GRANDY201640} beyond microbes
\parencite{COLEMAN2024131}

\parencite{Fry2019}
\parencite{briones2014soilfauna}

Biomass of arthropods and specific orders 
\parencite{milo-arthropods}
\parencite{bar2018biomass}

soil dwelling arthropods

soil health and arthropods
\parencite{insects11010054}
Crete data since 19th century. 

To being able to study first they have to be conserved.

protected areas, IUCN assessments and hotspots and Key Biodiversity Areas


%% need to modify
Hotspot definitions vary from quantitative methods to experts opinions and curation.
In quantitative methods, grid size and shape influences the determination of
the areas of interest such as hotspots and key biodiversity areas \parencite{hurlbert2007species,nhancale2011the-influence}.
Choosing the size of the grid is not trivial \parencite{mo2019influences} and is dependent
on the conservation goals \parencite{margules2000systematic}. In the past decade,
there have been major advances for conservation standards, guidelines,
frameworks and tools available to be put into action \parencite{bongaarts2019ipbes}.


%%%%% old
Numbering approximately 7,000 species [extrapolated from Fauna Europaea
\parencite{jong2014fauna} and \textcite{legakis2018}], the Arthropods of Crete, Greece,
have been studied for almost two centuries \parencite{Anastasiou2018Tenebrionid}.
Only 135 of these species (1.9\%) have been assessed in IUCN Red List as of
this publication, making Arthropods the third most evaluated group of the
island, behind vascular plants (291) and land mollusks (165).

Habitat loss and degradation occurs throughout Crete as a result of urban,
agricultural and touristic development. This is a major issue since habitat
loss is a major threat in Europe for many Arthropod groups,
e.g. Butterflies \parencite{VanSwaaycommission2010european}, Bees \parencite{nieto2014},
Orthoptera \parencite{hochkirch2016} and Saproxylic Beetles \parencite{Calix_2018}.
Climate change is predicted to induce scarcer yet more intense precipitation,
increase of drought locally \parencite{koutroulis2011spatiotemporal} and shrinkage as well as
possible shifts to the rainfall period \parencite{koutroulis2013impact}. Groups
associated with fresh water could be deeply impacted from the locally increased
drought and the increase in need of water for irrigation and domestic use,
e.g. Odonata \parencite{kalkman2010}, which has become harsher due to the
increase of agriculture and land use \parencite{tzanakakis2020challenges}. Stock raising
(sheep and goats) has always been an important aspect of Cretan life and
economy \parencite{rackham1996the-making}. Overgrazing impacts severely soil erosion,
soil moisture and vegetation \parencite{kosmas2015exploring,orestis2015exploring}. All the
above contribute to a worrying trend for Crete, i.e. the higher percentage of
Threatened endemic Arthropods when compared with the respective European
groups.

The bibliography and the NHMC collection was curated to assemble taxa
occurrences. The bibliography used
contains both historical and contemporary published material.
Using over 100 publications (as of 2020) and 733 NHMC sampling events
(\href{https://doi.org/10.5281/zenodo.10635645}{data online availability}), a dataset of 343 taxa (species and subspecies) was assembled,
with 4,924 records across 1,569 distinct sites of Crete. The taxa
are distributed to eleven orders, with Coleoptera having the most taxa (206)
and Chilopoda and Scorpiones the least (two).
Based on the criteria described in \textcite{bolanakis2024} the data resulted in the
following orders:
Araneae, Chilopoda, Coleoptera, Diplopoda, Heteroptera, Hymenoptera (Chrysididae, Formicidae, Symphyta), Lepidoptera (Geometridae), Odonata, Orthoptera, Scorpiones and Trichoptera.


In this study I aim to a) identify cretan arthropods available information
b) compare with the endemics of Crete
c) their relation with the anthropogenic pressures in these sites.
To do so, I assembled the accumulated knowledge of 
online databases.


% METHODS
\section{Methods}
\label{sec:arthropods-method}
   
    \subsection{Data retrieval}
    \label{subsec:arthropods-land-use}

The occurrences of the endemic arthropods of Crete were downloaded from the
\href{https://zenodo.org/records/10635645}{Zenodo repository}. This dataset
is a manually compiled dataset from the Natural History Museum of Crete \parencite{bolanakis2024}.

Retrieval and compilation of information on arthropod species' taxonomy,
global distributions, and IUCN status was performed using public databases.
Taxa from IUCN that appear in Crete were downloaded from the website \parencite{iucn2024}. IUCN provides the 
interactive maps selection functionality. 
The dataset of endemic species assessments from a TSV file and several IUCN Red
List datasets for Greece, endemic Greek species, Europe, and the world.
Each Red List dataset, was processed to identify threatened species
(those categorized as Critically Endangered, Endangered, or Vulnerable).
They were grouped by taxonomic order and threat status, and a source identifier was created.
All arthropod species names from the IUCN Red List were used to retrieve their GBIF IDs
and resolve their species names using the GBIF Backbone Taxonomy.
This taxonomy was used to harmonise and summarise all different datasets from IUNC

Reference grids were downloaded as shapefiles from \href{https://www.eea.europa.eu/en/datahub/datahubitem-view/3c362237-daa4-45e2-8c16-aaadfb1a003b}{EEA reference grid}.
The European Environment Agency grids are the standard reporting format of the Bern Convention
on the Emerald Network of Areas of Special Conservation Interest (ASCI).

Landscape changes are mostly due to human activities in Crete.
The Historic Land Dynamics Assessment (HILDA+) dataset \parencite{winkler2021global}
is a global analysis to identify land use changes using remote sensing and 
historical data. Crete was cropped from this dataset and the land use change was 
summarised for 20 years period, from 1998 to 2018. This period has the most 
high quality data baselines. The dataset is in raster format of 1 km\textsuperscript{2}
resolution. Each year has one raster file with values the ids of the land use. 
Using the 1998 and 2018 baselines, they were combined to calculate the 
transitions of each cell from one id to another. Some cells have undergone 
multiple transitions during this period which are neglected in the analysis.

    \subsection{Taxa distributions}
    \label{subsec:arthropods-eoo-aoo}

Endemic arthropod distributions were estimated using the criterion B of IUCN. 
Three metrics are important in the criterion B, the Extent of Occurrence (EOO),
the Area of Occupancy (AOO) and 
number of locations of a species that are 10km apart.
AOO function input is the shapefile of a species occurrences,
a baseline map (Crete), a shapefile of overlap areas (protected areas and land use maps),
Initially, each species' AOO is calculated, converting the data to a spatial feature
and calculating its area in square kilometers. 
AOO function was set to use 2x2 km grids that it maps on the area after 1000
repeats. The placement of the grid with maximum squares is selected and this AOO 
is selected for a taxon. Hence, AOO rasterized taxon occurrences to an optimum 
2x2km grid to estimate its' distribution.

The Extent of Occurrence (EOO) calculation is the convex hull polygon of the occurrences 
of a taxon. There are cases where this metric is overlapping with marine areas due 
to the nature of the polygon. Hence this metric tends to overestimate the 
total area of distribution of a taxon.
The EOO for a single species is calculated if there are more than three occurrences.
Together, these calculation provide a comprehensive method for calculating and
assessing the AOO and EOO of arthropods, including their overlaps with protected
areas, as part of the IUCN assessment process. This approach allows for detailed
spatial analysis and visualization, facilitating the conservation status evaluation of arthropod species.


    \subsection{Grid assay}
    \label{subsec:arthropods-ehs-kbas}

    Assessments of hotspots of endemic and threatened species of Cretan arthropods
ware carried out using various grid sizes, including 1, 4, 8, and 10 square kilometers, and
adaptive resolution algorithms such as quadtree. 
Endemicity Hotspots (EH) used here are the local biodiversity hotspots which contain
the 10\% of the richest (in endemic taxa) grid cells.
The testing of different grid sizes and the adaptive grid is helpful to avoid 
spatial scale biases and sampling biases.
Arthropod order biases were also investigated to examine whether some species
rich taxonomic groups drive the hotspots (e.g. Coleoptera). Hence, endemicity 
hotspots were also created for each order and their overlap was measured.

To facilitate this analysis, the essential data, including endemic species occurrences, Crete shapefiles
are used. For each grid size, occurrence
data were transformed and joined to calculate distinct species and sampling efforts,
and correlations between species richness and sampling intensity.
The 1km\textsuperscript{2} and 10 km\textsuperscript{2} grids are the 
EEA reference grid.
These grids are in shapefile format and their handling involves spatial joins,
summarizing species occurrences, and calculating sampling intensity,
followed by Pearson correlation assessment.

The 4 and 8 km\textsuperscript{2} grids utilize manually created rasters to map occurrences and
sampling efforts, with similar summarization and correlation steps.
The adaptive resolution grid is based on quadtrees, generating spatial polygons
and assessing species richness and endemicity hotspots.
The minimum cell length was set to 1 km\textsuperscript{2} and maximum to 10 km\textsuperscript{2}.
The algorithm split threshold was set to 4, i.e. if a grid has more than 4 samplings
it is split.
All various grid sizes and the adaptive quadtree approach, provide a comprehensive assessment of biodiversity hotspots on Crete.


    \subsection{Code}
    \label{subsec:arthropods-tools}

The coordinate reference system WGS84 - EPSG:4326 was used throughout the analysis.
Data, like the EEA reference grid were transformed to EPSG:4326 prior analysis.

All calculations, spatial data handling and data retrieval were implemented using the R Statistical Software \parencite{rcoreteam}.
Ggplot2 package was used for bar plots, heatmaps, boxplots and maps \parencite{wickham_ggplot2_2016}.
EOO was calculated with a custom function and AOO using the ConR R package \parencite{dauby2017conr:}.
Spatial analysis involved polygon and raster data handling, using the sf v1.0-14 \parencite{pebesma2018simple}
and terra v1.7-55 R packages \parencite{hijmans2024terra}, respectively.
Quadtree R package was used to infer the adaptive resolution grids \parencite{friend2023quadtree}.
Code is reproducible and available in this 
\href{https://github.com/savvas-paragkamian/arthropoda_assessment_crete}{GitHub repository}.


% RESULTS
\section{Results}
\label{sec:arthropods-results}

    \subsection{EOO and AOO}
    \label{subsec:arthropods-sampling}

   \begin{figure}[htp!]
      \centering
      \includegraphics[width=\textwidth,height=\textheight,keepaspectratio]{figures/arthropods-FigS1.png}
      \caption[Comparisons of Threatened Endemics in Crete, Greece, Europe and the World]{Comparisons of Threatened Endemics in Crete, Greece, Europe and the World for seven Arthropod groups (top to bottom, data aggregated from the IUCN web resource). Values are absolute (proportion in parentheses). Also, note that there are 3 threatened species of Scorpiones in the World Red List column and 5 not threatened species in the Trichoptera column.}
      \label{fig:arthropods-figS1}
   \end{figure}


   \begin{figure}[htp!]
      \centering
      \includegraphics[width=\textwidth,height=\textheight,keepaspectratio]{figures/arthropods_aoo-eoo_order.png}
      \caption[AOO, EOO relationship per order]{The relationship of AOO and EOO of species of different orders. Each dot is a species. }
      \label{fig:arthropods-eoo-aoo}
   \end{figure}


    \subsection{Hotspot size selection}
    \label{subsec:arthropods-grids}


   \begin{figure}[htp!]
      \centering
      \includegraphics[width=\textwidth,height=\textheight,keepaspectratio]{figures/arthropods-fig_crete_sampling_intensity_order.png}
      \caption[Sampling intensity]{Sampling intensity in Crete (i.e number of samples) in each 10 km grid for all samples and for each arthropod order.}
      \label{fig:arthropods-sampling-intesity}
   \end{figure}




The grid cell size 10 x 10 km is the most suitable for our study since our
dataset - being compiled from numerous different sources and sampling efforts -
is rather coarse and inhomogenous for a smaller cell size.
The unique taxa of the EHs of each grid is distributed as follows: 10 km=283,
8 km=278, 4 km=293, adaptive cells=267, with the 4km grid covering most endemic
species. The 4 km grid mostly highlighted areas known for their tourist/recreational activities,
indicating that it is more sensitive to sampling intensity.
Focusing on sampling we applied the adaptive grid size with quadtrees resulting
in 157 grids with 8 km length, 38 with 4 km and 74 with 2 km (Figure \ref{fig:arthropods-figS5}).
This indicates the preference of larger cells for the majority of our dataset
even though a small percent of regions has higher density of sampling.
The highest overlap among all grids is between the 10 km and 8 km reaching
57\% (Table \ref{table:arthropods-tableS2}). Finally, the 10 km grid has more taxa
per cell and is a reference grid system.
Based on our analysis and interoperability and reproducibility aims we choose
the 10 km EEA reference grid for the EHs and candidate KBAs inference.
Nevertheless, we also performed the WEGE analysis for KBAs using the adaptive
grid, yielding practically the same areas as the 10km grid minus Zakros (Figure \ref{fig:arthropods-fig5}).
The same pipeline can not be done with EHs for they require a fixed cell size.


   \begin{figure}[htp!]
      \centering
      \includegraphics[width=\textwidth,height=\textheight,keepaspectratio]{figures/arthropods_crete_multiple_grids_hotspots.png}
      \caption[Comparisons of different grid sizes]{Comparisons of different grid sizes to identify hotspots.}
      \label{fig:arthropods-different-hotposts}
   \end{figure}

   \begin{figure}[htp!]
      \centering
      \includegraphics[width=\textwidth,height=\textheight,keepaspectratio]{figures/arthropods-fig_grid_stat.png}
      \caption[Comparisons of proportions of endemics across grid sizes hotspots]{Comparisons of different grid sizes. A. The proportion of endemics. B. The Jaccard similarity across hotspots with the same grid size.}
      \label{fig:arthropods-different-hotposts-stat}
   \end{figure}


   \begin{figure}[htp!]
      \centering
      \includegraphics[width=\textwidth,height=\textheight,keepaspectratio]{figures/arthropods-Fig_quads.png }
      \caption[Adaptive grid size based on sampling using quadtrees]{Adaptive grid size based on sampling using quadtrees.}
      \label{fig:arthropods-figS5}
   \end{figure}

\begin{table}[]
    \caption{Different grid size hotspots overlap. The top 10\% of cells with most species are considered as hotspots. All units are in km\textsuperscript{2}.}
\begin{tabular}{cccccc}
\textbf{Grids}              & \textbf{1 km\textsuperscript{2}} & \textbf{4 km\textsuperscript{2}} & \textbf{8 km\textsuperscript{2}} & \textbf{10 km\textsuperscript{2}} & \textbf{Adaptive cell size} \\
\textbf{1 km\textsuperscript{2}}              & \textbf{126}   & 72             & 55             & 53              & 41                          \\
\textbf{4 km\textsuperscript{2}}              & -              & \textbf{918}   & 503            & 497             & 349                         \\
\textbf{8 km\textsuperscript{2}}              & -              & -              & \textbf{1317}  & 800             & 691                         \\
\textbf{10 km\textsuperscript{2}}             & -              & -              & -              & \textbf{1400}   & 646                         \\
\textbf{Adaptive cell size} & -              & -              & -              & -               & \textbf{1312}              
\end{tabular}
\label{table:arthropods-tableS2}
\end{table}


   \begin{figure}[ht]
      \centering
      \includegraphics[width=\textwidth,height=\textheight,keepaspectratio]{figures/arthropods-fig_crete-hotspots_order.png}
      \caption[Hotspots of every order]{Endemicity hotspots across orders.}
      \label{fig:arthropods-hotspots-order}
   \end{figure}

    \subsection{Land use changes}
    \label{subsec:arthropods-human-intervention}


\begin{table}[]
    \caption{The land use transitions from 1998 to 2018, using the HILDA+ dataset.}
\resizebox{\textwidth}{!}{%
\begin{tabular}{llll}
HiLDA+ transitions                             & Crete (km\textsuperscript{2}) & Natura2000 (km\textsuperscript{2}) & EHs (km\textsuperscript{2}) \\
urban (stable)                                 & 326         & 46               & 14        \\
urban to pasture/rangeland                     & 3           & NA               & NA        \\
cropland to urban                              & 42          & 1                & NA        \\
cropland (stable)                              & 1314        & 186              & 71        \\
cropland to pasture/rangeland                  & 565         & 74               & 32        \\
cropland to forest                             & 178         & 40               & 20        \\
cropland to unmanaged grass/shrubland          & 3           & 1                & NA        \\
cropland to sparse/no vegetation               & 3           & 3                & 3         \\
pasture/rangeland to urban                     & 13          & NA               & NA        \\
pasture/rangeland to cropland                  & 82          & 15               & 3         \\
pasture/rangeland (stable)                     & 5124        & 2594             & 1198      \\
pasture/rangeland to forest                    & 11          & 6                & 3         \\
pasture/rangeland to unmanaged grass/shrubland & 137         & 73               & 38        \\
pasture/rangeland to sparse/no vegetation      & 2           & 2                & 2         \\
forest to pasture/rangeland                    & 34          & 33               & 25        \\
forest (stable)                                & 364         & 182              & 174       \\
forest to unmanaged grass/shrubland            & 2           & 1                & 2         \\
unmanaged grass/shrubland (stable)             & 1           & NA               & NA        \\
sparse/no vegetation (stable)                  & 2           & 2                & 2         \\
water                                          & 165         & 78               & 5         
\end{tabular}%
}
\label{table:arthropods-tableS4}
\end{table}


% DISCUSSION
\section{Discussion}
\label{sec:arthropods-discussion}

    \subsection{Endemicity Hotspots}
    \label{subsec:arthropods-Endemicity-Hotspots}

Mountains host a great amount of Earth’s biodiversity, being a main driver for
the birth of species \parencite{antonelli2018geological,noroozi2018hotspots,rahbek2019building,Rahbek2019}
and a crucial frontier for their fate \parencite{steinbauer2018accelerated,urban2018escalator}.
Crete is not an exception to this trend \parencite{kougioumoutzis2020plant,trigas2013elevational}.
Our results conform to that, since the EHs are gathered primarily in the major
Cretan mountains. Lefka Ori and Dikti are the sites with the most
EHs, in agreement with studies focused on vascular plants \parencite{dimitrakopoulos2004questioning,kougioumoutzis2020plant}.
\textcite{sfenthourakis2001hotspots}, employing invertebrate groups, also recovered these mountains as EHs.

Islands are biodiversity sanctuaries \parencite{whittaker2007island}, and
so are mountains \parencite{rahbek2019humboldts}. Our work advocates for approaches that
treat islands and mountains under a holistic perspective. The combination of
the two provides a complex biogeographical interplay governing the forces of
speciation, preservation and extinction of biodiversity \parencite{steinbauer2016topography-driven}.
This synergistic effect of mountains-islands has also been recovered in other
areas such as the Balearic islands \parencite{guardiola2023are-mediterranean}.

    \subsection{Criterion B preliminary assessment}
    \label{subsec:arthropods-species-assessment-disc}

Criterion B is the widely used for Arthropods \parencite{cardoso2011adapting,carpaneto2015a-red-list},
because the majority of Arthropods taxa are missing necessary information of the other criteria (A, C, D and E).
Criterion B could overestimate the danger of Arthropods \parencite{cardoso2011adapting},
which should always be taken into consideration.

Arthropods with wider ranges that are not assessed as Threatened under
criterion B, are not necessarily Least Concern and should not be neglected.
Arthropod communities can be affected by the reduction of the abundance of
common and abundant species that offer important functions to the biocommunity.
Wide range does not guarantee high abundance (even though this is true for many
taxa) and even common species can be threatened \parencite{habel2018vanishing,klink2023disproportionate}.

The inclusion of Arthropod taxa in protected areas is often insufficient, with
Arthropods experiencing declines inside the protected areas \parencite{borges2005ranking,chowdhury2023protected,harry2019protected,rada2019protected}.
In fact, even when certain Arthropod groups are adequately included in N2K,
there are gaps and omissions \parencite{sanchez-fernandez2008are-the-endemic,verovnik2011is-the-natura}.
At a global level 75\% of Insects are not sufficiently covered by protected
areas \parencite{chowdhury2023three-quarters}. Crete stands in an intermediate position,
following the general trend of Greece’s N2K adequacy, being the best covered
area at a national level \parencite{kougioumoutzis2021plant,spiliopoulou2021the-natura}.
However, there are some clear gaps regarding certain taxa, encouraging more
locally focused conservation policies complementary to N2K. For example actions
need to be taken for KBAs that fall outside N2K like Kritsa and Zakros.

Biases towards Arthropods cause their poor coverage by protected
areas \parencite{chowdhury2023protected,damen2013protected,delso2021protected}. These
biases derive from geography, size, color and charisma \parencite{cardoso2012habitats,mammola2020towards,wang2021out-of-sight},
and even from political/economic reasons \parencite{dias-silva2021protected}. For example,
the strongest driver for a conservation program funding within the European
Union is the online popularity \parencite{mammola2020towards}. The unpopularity of
Arthropods has begun to change \parencite{wagner2021insect}, especially through citizen
science, which is a trend we should build on to properly conserve the Arthropods.

    \subsection{Human Intervention in Arthropods’ EHs}
    \label{subsec:arthropods-human-intervention-ehs}
Human activities account for almost 20\% of the EHs. The primary human activity
in the EHs is agriculture (~19.6\%). Agricultural intensification is one of the
most important drivers of Arthropods’ decline \parencite{bruhl2019biodiversity,habel2019agricultural,raven2021agricultural}.
Moreover, threats associated with agriculture are the number one threat for
Insect species inside protected areas in Europe \parencite{chowdhury2023protected}.
Nevertheless, regarding change in land use, there is a somewhat equal
transition trend from cropland to forest and vice versa inside EHs and KBAs
(Table \ref{table:arthropods-tableS4}). This means that while some sites are being
degraded others may recover. More research within EHs and KBAs is essential in
order to quantify the impact (negative or positive) of these transitions to the
endemic Arthropods. A vast amount of cropland has been transformed to pasture
lands (Table \ref{table:arthropods-tableS4}) which requires further examination,
since grazing has both positive [eg. on Gnaphosidae (Spiders) communities \parencite{kaltsas2019overgrazed}]
and negative effects [e.g. Carabidae (Coleoptera) \parencite{kaltsas2013ground}].
The reduction of croplands could be interpreted under the general trend of
urbanization (Table \ref{table:arthropods-tableS4}), which nevertheless occurs
outside EHs and KBAs, but a shift towards montane areas especially under new
forms of tourism could deeply impact the sites of conservation importance.


