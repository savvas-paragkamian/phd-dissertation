% --------------------------------------------------
% 
% This chapter is for PREGO
% 
% --------------------------------------------------


\chapter{Harmonising the literature and metagenomic resources to infer microbial, environmental and functional relations}
\label{cha:prego}


\section{Introduction}
\label{sec:prego-intro}

The rapid advancements in genomic technology have significantly impacted the
all fields of biology, creating a need for the establishment
of genomic standards to ensure the reliability and reproducibility of research
findings \parencite{Field2011}. As genomic datasets grow exponentially, the importance of these standards
in genomic research cannot be overstated. Standards are essential for enabling
the comparison of results across different studies, facilitating data sharing and
collaboration, and ultimately driving the progression of genomic medicine \parencite{vangay2021microbiome}.
However, the potential of genomic data is often limited by the lack of accompanying
metadata.

Data devoid of metadata are essentially a black box, with no contextual
information to interpret the results or to assess their relevance \parencite{michener_nongeospatial_1997}.
This issue is
prevalent in genomics, where the complexity and diversity of the data
necessitate comprehensive metadata for accurate analysis \parencite{vangay2021microbiome}.
Without proper metadata, it becomes challenging to assess data quality,
reproducibility, and to ensure that research conclusions are robust.
The emergence of metagenomic resources has added a new dimension to genomic
research, with a myriad of databases providing a wealth of data on microbial
communities from various environments.
Examples of public resources are MGnify \parencite{mitchell2020mgnify},
JGI/IMG \parencite{chen2021img} and MG-RAST \parencite{wilke2015restful}). Each one 
has it's one pipeline to analyse user submitted environmental sequences. MGnify
directly imports data from ENA sequence database.
These resources are invaluable for
understanding the genetic diversity and functional potential of microbial
ecosystems. Adding to the complexity, the sheer volume of data and the varying formats in which
they are presented can make comparing and combining these resources a daunting task. 

Metagenomic mining approaches are required to leverage vast databases and analyze
microorganisms from diverse environments \parencite{delmont2011metagenomic}.
Mining allows scientists to extract and interpret genetic material 
from multiple resources thus providing a comprehensive view of the microbiome.
Metadata associated with each sample in these databases, such as source information and collection conditions,
contribute significantly to the contextual understanding and statistical interpretation of the data.
Enable researchers to make meaningful comparisons across different studies.
Furthermore, the application of metagenomic mining to these rich databases can help
uncover novel genes, metabolic pathways, and potential therapeutic targets \parencite{ma2023a-genomic},
thus driving advancements in health, agriculture, and environmental sustainability.
To accomplish this, named entity recognition techniques have become essential for
efficiently harmonising relevant information from metadata.

Named entity recognition is a term identification process. It is based upon
vocabularies and ontologies that provided multiple versions of terms and 
hierarchy relationships. Environment Ontology (ENVO) \parencite{buttigieg2016environment} 
is an important ontology that has classified environments and metagenomic 
resources suggest it's usage in the metadata in fields like biome and 
environmental feature. Molecular functions and biological processes are two separate 
ontologies, part of Gene Ontology \parencite{ashburner2000gene, gene2021gene} that
are widely used in bioinformatics and are open source. 
Regarding functions and pathways, the KEGG database \parencite{kanehisa2000kegg} is the most used resource
in microbial ecology, yet it requires a licence to implement it in a new resource.
Taxa names are also extracted from free text based on different taxonomies. In microbiology,
NCBI taxonomy \parencite{schoch2020ncbi} is convenient to use and implement because every taxon 
is represented by sequences. Nevertheless, LPSN (List of Prokaryotic names with Standing in Nomenclature) \parencite{parte2020list}
is a curated nomenclature that has high accuracy. Based on some of the aforementioned 
systems it is possible to extract entity names from metadata fields and free text in general 
in order to homogenise metadata and data fields.

Apart from metagenomic data, scientific literature contains invaluable information
about microbes, their functions and habitats. This information is usually hidden 
or captured by experts on the fields. 
Text mining methods have been long assisted in extracting knowledge from 
the scientific literature \parencite{jensen2006Literature}. 
By leveraging text mining, researchers can more readily identify and use the rich information that
is often hidden within the text, thereby accelerating discoveries in this
rapidly evolving field \parencite{badal2019Challenges}. Text mining is a multiple step
process \parencite{10.5555/1199003}. It begins with establishing a corpus with 
documents specifically formatted for the tools used. PubMed \parencite{roberts2001pubmed}
holds more that 40 million abstracts through MEDLINE. These data are accessible 
and open access, therefore useful for text mining applications. A downside is the 
domain of publications because it doesn't cover most of ecological and environmental 
journals. Then these documents are analysed with Named Entity Recognition and a 
score of co-occurrences between entities is produced. This lies in the principle 
that the highest the co-mention of terms the higher the probability there is 
a biological meaning of this association \parencite{jensen2006Literature}.
Text mining approaches have been very successful in unearthing novel 
associations of proteins and deceases \parencite{pletscher2015diseases}.

PREGO, the resource developed and presented in \textcite{microorganisms10020293}
as part of this PhD (see \ref{fig:prego_paper}), is a hypothesis generation platform designed to integrate the 
available knowledge about microbes and their processes and environments. 
It is a global platform, first of it's kind that combines metagenomic data with literature
using an established text mining methodology. 
In this chapter, my contributions in the development of PREGO are presented along with the analysis of global 
soil microbiome knowledge.

\section{Methods}
\label{sec:prego-methods}

   \subsection{Pipeline and structure}
   \label{subsec:prego-pipeline}

PREGO's pipeline can be summarised in 6 steps as presented in Figure \ref{fig:prego-pipeline}.
In steps 1 and 2, the web resources are harvested from 3 different types of data,
literature, environmental samples and genome annotations. 
In step 3, the Named Entity Recognition system of the Jensen Lab Tagger is employed \parencite{jensen2016one}
to identify taxa, environments and processes. In addition, in genome annotations
the processes entities are mapped to Gene Ontology terms with custom scripts. 
During the step 4, the co-occurrences of terms is calculated and scored,
each type of resource has it's own scoring scheme. 
This creates a global association network with taxa, processes and environments as 
nodes, co-occurrences as edges and score values as edge weights. 
Lastly, this knowledge graph is uploaded and provided in a web interface 
and through Application Programming Interface.

   \begin{figure}[hbt!]
      \centering
      \includegraphics[width=0.85\columnwidth]{figures/prego_analysis.png}
      \caption[PREGO analysis methodology]{
         PREGO methodology: retrieval of 3 types of open access data resources, literature, environmental DNA samples, and genomic annotations. 
         Named Entity Recognition and co-occurrences bring together associations from organisms, environments and processes. 
         These associations has a score and are provided is a Web interface and an API. Figure modified version of \parencite{microorganisms10020293}.
      }
      \label{fig:prego-pipeline}
   \end{figure}

The knowledge graph in Figure \ref{fig:prego-pipeline} step 5, contains
nodes categorised in three entity types: \textit{Process}, \textit{Environment}, and \textit{Organism}. 
Organisms, i.e taxa, are the microbial NCBI Taxonomy Ids (Bacteria, Archaea, and unicellular eukaryotes).
All environments are mapped to terms of Environmental Ontology. 
Each process corresponds to either a Biological process (GObp) or a Molecular function (GOmf) identifier of Gene Ontology. 

PREGO's knowledge base is structured into three channels, depended on the category of the data type.
\textit{Literature} channel is about the associations extracted by abstracts and full open access text of scientific articles.
The \textit{Annotated Genomes and Isolates} channel contains genome annotations and their harmonised metadata.
The \textit{Environmental Samples} channel integrates metagenomic analyses from amplicon and shotgun sequencing studies. 
Hence, in this channel, the taxonomic and functional profiles and their metadata are harmonised and the associations are extracted.

\subsection{Dictionary of entities}
\label{subsec:prego-dict}

PREGO's entities are 4 types and each term corresponds to specific identifier. 
NCBI taxonomy identifiers for taxa, Environmental Ontology for environments and
Gene Ontology for Biological Processes (GObp) and Molecular Functions (GOmfs).
Gene Ontology was selected because it has a Creative Commons Attribution 4.0 License
and because it has been mapped to many other resource identifiers.
These terms, their names and synonyms comprise the PREGO dictionary. In addition,
some terms have been removed because of their uniqueness and/or ubiquitousness which reduce the precision of text mining systems.
These terms comprise the blocklist, a separate file in the dictionary.
The ORGANISMS \parencite{pafilis2013species} and ENVIRONMENTS \parencite{pafilis2015environments} are evaluated dictionaries
whereas Gene Ontology Biological Process and Molecular Function are in experimental stage.
The dictionary is 
available for download in the Jensen Lab text mining \href{https://jensenlab.org/resources/textmining/#dictionaries}{resources}. 
NCBI contains all taxonomy, therefore it was filtered for bacteria, archaea and
for the unicellular eukaryotes. Due to unicellular complex taxonomy a manually curated
list was populated with these taxa.

Some resources use KEGG orthology terms or Uniprot50 ids to indicate functions. 
Hence, a mapping KEGG orthology to GOmf and Uniprot50 to GOmf was carried out via UniProtKB mapping files (see \texttt{idmapping.dat} and \texttt{idmapping\_selected.tab} files). 

\subsection{Scoring associations}
\label{scoring}

In the vast knowledge base of PREGO, scoring is crucial to sort 
the results. Ranking the association based on their trustworthiness and
relevance to the user's query is what makes platforms useful to users.
Hence, scoring is an indispensable component of data platforms 
that respond to queries.

PREGO responses with relevant and probable associations are the user query
through the three channels of information. 
Each channel has a tailored scoring scheme depending on the data structure to
rank the associations of terms.
All associations are scored and normalised in the interval $(0,5]$ to maintain consistency. 
The scoring starts with the contingency table (Table~\ref{table:pregoA1}) of two random variables, $X$ and $Y$.
For example, $X = 1$ is a NCBI id and $Y = 1$ represents an GObp term id. 
The $c_{1,1}$ is the co-mention count of these two terms, $X = 1$ and $Y = 1$, i.e., the joint frequency. 
The marginals, $c_{1,.}$ and $c{.,1}$ for $x$ and $y$ respectively, are the counts for entity type id. 
These frequencies are used differently depending on the scoring indices.
Each scoring scheme, must be evaluated and benchmarked because there is not a 
single perfect measure. 

\begin{table}[ht]
   \centering
   \begin{tabular}{c|llll}
    & \multicolumn{4}{l}{Y = y} \\ \cline{2-5} 
   \multirow{4}{*}{X = x} &  & Yes & No & Total \\ \cline{3-5} 
    & \multicolumn{1}{l|}{Yes} & $c_{x,y}$ & $c_{x,0}$ & $c_{x,.}$ \\
    & \multicolumn{1}{l|}{No} & $c_{0,y}$ & $c_{0,0}$ & $c_{0,.}$ \\
    & \multicolumn{1}{l|}{Total} & $c_{.,y}$ & $c_{.,0}$ & $c_{.,.}$
   \end{tabular}
   \caption[PREGO contingency table between two terms]{Contingency table of co-occurrences between entities $X = x$ and $Y = y$. 
   $c_{x,y}$ is the count of the
co-occurrence of these entities. $c_{x,.}$ is the count of the $x$ with all the
entities of $Y$ typy. Conversely, $c_{.,y}$ is the count of $y$ with all the entities of $X$ type (e.g., taxonomy}
   \label{table:pregoA1}
\end{table}

The Literature channel scores associations of each pair of entities based on their co-occurrence in each document, paragraph, and sentence.
Environmental Samples channel scores associations based on abundance tables and metadata. Hence, the number of samples that entities co-occur. 
Thus, it has two depth levels, microbes with metadata, and sample identifiers.
The Genome Annotation and Isolates channel has predetermined values of scores
based on the resource and the level of manual curation which contain the most trustworthy associations. 

   \subsection{Literature}
   \label{subsec:prego-tm}

   \subsubsection{Corpus}
The \textit{Literature} channel of PREGO contains the associations between 
entities as extracted by PubMeD®. Abstracts and full text articles were retrieved
from MEDLINE® and PubMed Central® Open Access Subset (PMC OA Subset) \parencite{sayers2021database}, respectively. 
NCBI, which hosts these data, has a powerful File Transfer Protocol (FTP) service 
that is used to periodically retrieve new published documents. The download time 
is around 5 hours.
The retrieved files are XML files, compressed. These raw files are the are converted
to tabular format while maintaining the following information:

\begin{itemize}
    \item PMID DOI
    \item Authors
    \item Journal.volume:pages
    \item year
    \item title
    \item Abstract
\end{itemize}

Each file contains ~30000 lines which, sum to 33 million abstracts. Some
abstracts are duplicated across files, so in order to avoid bias in tagging,
a script keeps only the latest occurrence of duplicated abstracts.
The PMC Open Access Subset has two directories: oa comm and oa noncomm.
They are separated because they contain different licences. For commercial usage,
articles in the oa comm directory are allowed because they have CC BY and CC0 licenses.
For non-commercial usage, oa noncomm can be used which has full text documents
licensed under all Creative Commons license types with the exclusion of CC BY and CC0)
and oa comm directories. 

   \subsubsection{Tagger}

Text mining is on the corpus using the dictionary through the EXTRACT tagger \parencite{pafilis2016extract, jensen2016one}.
The tagger is implemented in C++ and first loads all entities of the dictionary
and then it recognises the co-occurrences in the corpus and returns them with a score. 
The tagger output files, the pubmed tsv files and dictionary files are used by
the 6 perl scripts, Figure \ref{fig:prego-textmining-structure}. More specifically, 
the create documents script reads all the tsv files of pubmed and stores all abstracts
in a single file while simultaneously selecting the latest abstract of the duplicated ones. 
The total duration is 211 minutes. The maximum amount of memory used is 6gb and the tagger
is running using 8 threads.
   
\begin{figure}[hbt!]
      \centering
      \includegraphics[width=0.85\columnwidth]{figures/prego_textmining-structure.png}
      \caption[PREGO analysis methodology]{
         Structure of the Literature channel with multiple scripts working together.
         The initial data are the Dictionary and Pubmed. Each one has preprocessing steps. 
         The Text mining steps are the last steps to create the database pairs file 
         with all associations and scores.
      }
      \label{fig:prego-textmining-structure}
\end{figure}

\subsubsection{Score}

Literature channel adopts the scoring scheme of STRING 9.1 \parencite{franceschini2012string}
and COMPARTMENTS \parencite{binder2014compartments}. This scoring has three steps. 
First, for each co-occurrence of a pair of entities, a weighted count is
calculated because of the complexity of the text.

\begin{equation}
   c_{x,y} = \sum_{k=1}^{n}{w_d \delta_{dk}(x,y) +w_p \delta_{p,k}(x,y) + w_s \delta_{sk}(x,y)}
   \label{eq:prego-score-1}
\end{equation}

Different weights are used for each part of the document ($k$) for which both
entities have been co-mentioned, $w_d = 1$ for the weight for the whole document
level, $w_p = 2$ for the weight of the paragraph level, and $w_s = 0.2$ for the
same sentence weight. 
The delta functions equal to one (Equation~\ref{eq:prego-score-1}) if the co-occurrence
exists and is zero if it doesn't. Therefore, the weighted count becomes higher as
the entities are mentioned in the same paragraph and even higher when in the same sentence.
Subsequently, the score is calculated with the following equation:

\begin{equation}
   score_{x,y} = c_{x,y}^a (\frac{c_{x,y} c_{.,.}}{c_{x, .}c_{.,y}})^{1-a}
   \label{eq:prego-score-2}
\end{equation}

where $a = 0.6$ is a weighting factor, and the $c_{x,.}$, $c_{.,.}$, 
$c_{.,y}$ are the weighted counts as shown in Table~\ref{table:pregoA1} estimated using the same Equation~\ref{eq:prego-score-2}. 
This weighting factor has been optimized and benchmarked in multiple 
applications of text mining~\parencite{franceschini2012string, binder2014compartments, pletscher2015diseases}. 
The value of Equation~\ref{eq:prego-score-2} has been noted that is sensitive to
the increasing size of the number of documents (MEDLINE PubMed—PMC OA).
Therefore, to obtain a more robust measure, the value of the score is transformed to $z$-score. 
This transformation is presented in detail in the COMPARTMENTS resource \parencite{binder2014compartments}. 
Finally, the confidence score is the $z$-score divided by two. Cases in which
the scores exceed the (0,4] interval are capped to a maximum of 4 out of 5 to
reflect the uncertainty of the text mining pipeline.

   \subsection{Environmental Samples}
   \label{subsec:prego-envsamples}

Environmental Samples channel contains associations extracted by environmental DNA studies, 
namely amplicon and shotgun metagenomic analyses. These data were downloaded with custom API clients
of MGnify \parencite{mitchell2020mgnify} and MG-RAST \parencite{wilke2015restful} repositories.
From these repositories the taxonomic and functional profiles were downloaded along with their 
studies metadata. Both amplicon and shotgun metagenomic have taxonomic profiles, while 
only the latter include also functional profiles.
KEGG Orthology terms and/or Uniref identifiers of functional profiles are mapped to GOmf terms 
prior to the analysis with the custom mapping scripts, see \ref{subsec:prego-dict}.

PREGO processes these data per sample, so per sample the profiles are associated with the metadata. 
To harmonise the data and metadata names, organisms, environments, processes, functions, the EXTRACT tagger and the dictionary is used, as described previously. 
This process creates a entity pairs co-occurrence file for all samples. 
These associations are subsequently scored. The score is based on the count of
samples the entity of interest co-occurs with specific sample metadata or profile (function or taxonomic). 

More specifically, for every term $x$, the number of samples it is present is calculated as the background 
and the count of samples of the co-occurrent term (metadata background) $c.,y$ (see Table~\ref{table:pregoA1}). 
Each association between terms $x$,$y$ is found in $c_{x,y}$ samples. 
These calculations are performed separately for each resource.
The equation of the score is:

\begin{equation}
   score_{x,y} = 2.0*{\frac{\sqrt{c_{x,y}}}{c_{.,y}^{0.1}}}
\end{equation}

This is asymmetric because the denominator is the marginal of the associated entity. 
Thus, the value is reduced as the marginal of $y$ is increasing, i.e., the number of samples that $y$ is found. 
Conversely, it promotes pairs of associations in which the number of samples of 
the association are similar to the marginal of $y$. 
The exponents on the numerator and denominator equal to $0.5$ and 
to $0.1$, respectively, in order to reduce the rapid increase of score.
Lastly, the value of the score is capped in the range $(0,4]$.

   \subsection{Annotated Genomes and Isolates}
   \label{subsec:prego-isolates}

Annotated genomes and isolates are the manually curated data of PREGO's knowledge base.
This fact makes this channel the most trustworthy because every single species/strain have manually curated metadata. 
JGI-IMG \parencite{chen2021img, mukherjee2021genomes} has millions of genes that 
correspond to isolated genomes, SAGs and MAGs. These annotations and their corresponding metadata
were collected using web scrapping because even though the data are open there isn't an API client for these data.
The metadata about environments were harmonised using the EXTRACT tagger leading to organisms—environments co-occurrences.
The KEGG ids from the annotations were mapped to GOmf terms and were used for the organisms—processes associations.
   
The Struo pipeline \parencite{de2020struo}, which is based on the Genome Taxonomy DataBase (GTDB) (v.03-RS86) \parencite{parks2020complete} was also used.
This pipeline contains organisms—processes associations. 
UniRef50 annotations were changed to GOmf terms with custom mapping. 
GTDB genome taxonomy was mapped to the corresponding NCBI taxa. 
Struo's associations were assigned a confidence level of four out of five. 
This score reflects the high-quality of these data and metadata.
   
Lastly, BioProject data were used in PREGO from the NCBI FTP/e-utils services \parencite{sayers2021database}. 
The BioProject ids that has been assigned a PubMed abstract, a unicellular taxon, and Genome sequencing as data type were filtered.
The text mining pipeline described above, was used to extract associations of the
assigned taxon with the other entities in the abstracts. 
The BioProject derived associations were assigned the score of three (out of five)
because of the implementation of the text mining pipeline.

\subsection{Behind the scenes}
\label{deamons}

PREGO's pipeline and hosting is running on a 64 GB RAM DELL R540, 20 core, Debian server.
The server has 2 terabyte SSD storage for high responsiveness. 
All code is versioned with the git versioning tool to keep history and 
resolve bugs. The development cycles of PREGO are based on DevOps practices that
combine development and operations and enable the efficient deployment of software services.
The Continuous Development and Continuous Integration (CD/CI) method proved to be
important aspect during the developed of PREGO in terms of multiple testing,
resolving bugs and effective collaborative coding. 

The pipeline of PREGO is streamlined and integrated to be executed regularly,
spanning from once a month to six-month cycles, ensuring that changes of resources are
continuously integrated into PREGO. These cycles are called daemons in PREGO and 
each channel has a dedicated script that executes all the code while 
maintaining backups, Figure \ref{fig:devops}.


\begin{figure}[hbt!]
   \centering
   \includegraphics[width=95mm]{figures/figure_A1_PREGO_daemons.png}
   \caption[PREGO DevOps]{Scripts called daemons are executing the PREGO methodology in cycles depending on the updates of the databases used. Figure adopted by \parencite{microorganisms10020293}.}
   \label{fig:devops}
\end{figure}


   \subsection{Code} 
Code produced for PREGO are available under BSD 2-Clause “Simplified” License.
Scripts, where third party libraries have been used, are subject to their individual licenses.
API calls were implemented in Python 3. Daemons and FTP downloads were written in Bash and 
mappings and statistics in GNU AWK.
   
   \begin{itemize}
      \item prego\_gathering\_data 
      \href{https://github.com/lab42open-team/prego_gathering_data}{github.com/lab42open-team/prego\_gathering\_data}
      \item prego\_daemons \href{https://github.com/lab42open-team/prego_daemons}{github.com/lab42open-team/prego\_daemons}
      \item prego\_mappings \href{https://github.com/lab42open-team/prego_mappings}{github.com/lab42open-team/prego\_mappings} 
      \item prego\_statistics \href{https://github.com/lab42open-team/prego_statistics}{github.com/lab42open-team/prego\_statistics}
   \end{itemize}

Additional software and curated lists along with their individual license are:
   \begin{itemize}
      \item tagger:	\href{https://github.com/larsjuhljensen/tagger}{https://github.com/larsjuhljensen/tagger}, BSD 2-Clause "Simplified" License
      \item mamba: \href{https://github.com/larsjuhljensen/mamba}{https://github.com/larsjuhljensen/mamba}, BSD 2-Clause "Simplified" License 
      \item tagger dictionary:  \href{https://download.jensenlab.org/}{https://download.jensenlab.org/} and there in: \\
      \href{https://download.jensenlab.org/prego_dictionary.tar.gz}{https://download.jensenlab.org/prego\_dictionary.tar.gz}, CC-BY 4.0 license
   \end{itemize}

\section{Results}
\label{sec:prego-results}

   \subsection{PREGO Contents}
   \label{subsec:prego-contents}

The data sources of PREGO can be categorized into six types: abstracts, full articles,
isolates, annotated genomes, markergene samples, and metagenomic samples, Figure \ref{fig:prego-summary}.
In terms of metadata availability, JGI IMG, Struo, BioProject, and MG-RAST provide
metadata for their respective data types.
BioProject provides metadata for annotated genomes with abstracts, which could be useful for researchers seeking comprehensive information.
The number of items varies significantly across sources, with MEDLINE and PubMed
having the largest collection of items (33 million) and PubMed Central OA Subset
having a significant collection of full articles (2.7 million).
JGI IMG, Struo, BioProject, and MG-RAST have a smaller but still substantial number of isolates, annotated genomes, and markergene samples.
In terms of licenses, the majority of the sources have open licenses,
such as CC0, CC BY-SA 4.0, CC-BY, and NLM Copyright, which allows for the free
use and sharing of the data. This is an essential aspect of data sharing and
collaboration in the scientific community.

   \begin{figure}[hbt!]
      \centering
      \includegraphics[width=\columnwidth]{figures/PREGO_summary_channels.png}
      \caption[PREGO contents summary]{
         Resources, entities and associations summary in each channel of information of PREGO.}
      \label{fig:prego-summary}
   \end{figure}

Regarding the entity types, PREGO's knowledge base contains 364,000 taxa
(NCBI Taxonomy has 620,000 Bacteria, Archaea, and microbial eukaryotes). 
About 258,000 taxa are at the species level. 
All environment Ontology terms are found at about 1000 terms. Regarding Gene ontology, 15,000 biological process terms and 7.900 molecular function term are present.
PREGO's knowledge base of entities and associations are a multipartite network
with entities as nodes and co-occurrences as links with score as weight.

\subsection{Bulk download}
\label{bulk-download}

All the knowledge base of PREGO is available for download programmatically.
In Table~\ref{table:prego-appD-1} the hyperlinks are provided, one per channel, along with md5sum files for sanity checks.
Data are compressed to tar.gz format so expect an order of magnitude higher file size when decompressed.

   \begin{table}[ht]
      

      \begin{tabular}{lll}
      \toprule
      Channel Link & md5sum & Size (in GB) \\ \midrule

      \href{https://prego.hcmr.gr/download/literature.tar.gz}{Literature} & \href{https://prego.hcmr.gr/download/literature.tar.gz.md5}{literature.tar.gz.md5} & 5.4 \\

      \href{https://prego.hcmr.gr/download/environmental\_samples.tar.gz}{Environmental samples} & 
      \href{https://prego.hcmr.gr/download/environmental\_samples.tar.gz.md5}{environmental\_samples.tar.gz.md5}
      & 0.69 \\

      \href{https://prego.hcmr.gr/download/annotated\_genomes\_isolates.tar.gz}{Annotated genomes} &
      \href{https://prego.hcmr.gr/download/annotated\_genomes\_isolates.tar.gz.md5}{annotated\_genomes\_isolates.tar.gz.md5} & 0.26 \\ \bottomrule
      \end{tabular}
      \caption[PREGO Bulk download.]{Bulk download links and md5sum files.}
      \label{table:prego-appD-1}
   \end{table}



%%%%%%%%%%%%%%%%%%%%%%%%%%%%%%%% soil %%%%%%%%%%%%%%%%%%%%%%%%%%%%
   \subsection{PREGO about soil}
   \label{subsec:prego-soil}




\section{Discussion}
\label{sec:prego-discussion}

   \subsection{One stop shop}
   \label{subsec:prego-contents-disc}

PREGO approach is based upon four principles. First, open access data and
metadata integration along with literature are the pillar of the current
available knowledge to microbiologists. Second, the associations of microorganisms,
environments and processes are key for the deciphering of ecosystem function.
Third, the modular structure and regular cycles of updates are fundamental to
keep up with the rapid advances of microbial ecology. And last but not least,

The harmonisation of information from multiple types of resources and of different
entities is a major issue in science today. PREGO approached this problem from 
the Natural Language Processing perspective offering a global integration 
of microbial knowledge. To accomplish this some strategic choices were made, 
namely using the same dictionary of terms for all resources and used 
the EXTRACT tagger to identify the terms in free text and metadata.
PREGO manages to a great proportion of microbial taxa, even unicellular eukaryotes, in the knowledge base.
Because different channels, PREGO extracts associations in the species and higher taxonomic levels.
Higher level taxonomic associations with environments and processes are present
which are useful for classification processes and versatility of taxa hypotheses.

One important caveat of this approach is the difficulty to 
identify false positives, i.e associations that are apparent but without 
biological meaning. This is because, in the current version, the text mining
pipeline is based on Named Entity Recognition without taking into account 
the context. This is an important limitation that Large Language Models 
are promising.

PREGO provides all this knowledge base in multiple 
ways and is implemented to be easy to use and openly accessible to all with it’s graphical and programming user interface.
A dedicated web interface is useful to users for single queries
and exploration of literature and/or samples. The API is useful and 
robust for multiple calls and, not for the faint of heart, PREGO is 
available for bulk download. 

   \subsection{Platform comparison}
   \label{subsec:prego-similar-platforms}


Based on the table \ref{table:prego4}, BacDive seems to be the most versatile
platform among the four, offering a wide range of features. It has a high level
of manual curation, with environment-taxa associations, process/function-taxa
associations, and phenotypic data available. Additionally, BacDive provides
spatial coordinates, an application programming interface but limited bulk download of data.
It has a high score for manual curation, indicating that it is well-maintained and regularly updated with accurate information.
As for the presence of various types of data, BacDive appears to be the platform
with the most comprehensive dataset, offering original and integrated data from
various sources. Its high scores for environment-taxa associations,
process/function-taxa associations, and phenotypic data suggest that it contains a broad range of microbial data.

PREGO appears to have some notable benefits. Specifically, it has a high score
for literature integration, which suggests that it is able to integrate microbial
data from various sources and publications. Additionally, PREGO is capable of
processing and storing data on environment-process/function associations, which
could be useful for researchers studying microbial processes and functions.
Another benefit of PREGO is its ability to facilitate bulk downloads of data,
making it easier for researchers to access and analyze large datasets.
This feature can be particularly useful for big data analysis and machine learning applications.
It is worth noting that PREGO's scores for other features, such as manual
curation and environment-taxa associations, are relatively low compared to
BacDive and Web of Microbes. However, its strengths in literature integration
and bulk data download may make it a useful platform for researchers with specific needs.


   \begin{table}[ht]

      \begin{adjustwidth}{-0.75cm}{}

      \begin{tabular}{@{}lllll@{}}
      
      \toprule
      Functionality & BacDive & Web of Microbes & NMDC & PREGO \\ \midrule
      manual curation & high & high & intermediate & low \\ 

      literature integration & limited & no & no & yes \\

      environment—taxa associations & yes & yes & yes & yes \\

      \begin{tabular}[c]{@{}l@{}}environment—process/\\ function associations\end{tabular} & no & no & no & yes \\

      process/function—taxa associations & yes & yes & yes & yes \\
      phenotypic data & yes & no & no & no \\

      data origin & \begin{tabular}[c]{@{}l@{}} original \\integration \end{tabular} & original & \begin{tabular}[c]{@{}l@{}} original \\integration \end{tabular} & integration \\

      spatial coordinates & yes & no & yes & no \\

      application programming interface & yes & no & yes & yes \\

      bulk download & limited & yes & yes & yes \\ \bottomrule

      \end{tabular}
      \end{adjustwidth}
      \caption[Feature comparison between PREGO and other similar platforms]{Feature comparison among platforms that facilitate knowledge discovery and integration of microbial data.}
      \label{table:prego4}
   
   \end{table}      


   \subsection{Soil microbiome}
   \label{subsec:prego-soil}


%   \begin{table}[ht]
%      \begin{tabular}{@{}cccrccc@{}}
%      \toprule
%      \textbf{Channel} & \textbf{Source} & \multicolumn{2}{c}{\textbf{Taxonomy}} & \textbf{\begin{tabular}[c]{@{}c@{}}Environ- \\ ments\end{tabular}} & \textbf{\begin{tabular}[c]{@{}c@{}}Biological \\ Processes\end{tabular}} & \textbf{\begin{tabular}[c]{@{}c@{}}Molecular \\ Functions\end{tabular}} \\ \midrule
%      \multirow{3}{*}{Literature} & \multirow{3}{*}{\begin{tabular}[c]{@{}c@{}}MEDLINE \\ PubMed - \\ PMC OA\end{tabular}} & Strains & 8,929 & \multirow{3}{*}{1,077} & \multirow{3}{*}{15,079} & \multirow{3}{*}{7,318} \\
%      &  & Species & 240,377 &  &  &  \\
%      &  & Total & 342,506 &  &  &  \\
%      \multirow{9}{*}{\begin{tabular}[c]{@{}c@{}}Environ- \\ mental \\ samples\end{tabular}} & \multirow{3}{*}{\begin{tabular}[c]{@{}c@{}}MG-RAST \\ amplicon\end{tabular}} & Strains & 1,392 & \multirow{3}{*}{162} & \multirow{3}{*}{-} & \multirow{3}{*}{-} \\
%      &  & Species & 4,324 &  &  &  \\
%      &  & Total & 5,859 &  &  &  \\
%      & \multirow{3}{*}{\begin{tabular}[c]{@{}c@{}}MG-RAST \\ metagenome\end{tabular}} & Strains & 2,522 & \multirow{3}{*}{258} & \multirow{3}{*}{-} & \multirow{3}{*}{3,839} \\
%      &  & Species & 4,406 &  &  &  \\
%      &  & Total & 7,157 &  &  &  \\
%      & \multirow{3}{*}{\begin{tabular}[c]{@{}c@{}}MGnify \\ amplicon\end{tabular}} & Strains & 2 & \multirow{3}{*}{216} & \multirow{3}{*}{11} & \multicolumn{1}{l}{\multirow{3}{*}{-}} \\
%      &  & Species & 1,471 &  &  & \multicolumn{1}{l}{} \\
%      &  & Total & 2,955 &  &  & \multicolumn{1}{l}{} \\
%      \multirow{9}{*}{\begin{tabular}[c]{@{}c@{}}Annotated\\Genomes \&\\Isolates\end{tabular}} & \multirow{3}{*}{JGI IMGisolates} & Strains & 2,398 & \multirow{3}{*}{241} & \multirow{3}{*}{-} & \multirow{3}{*}{3,670} \\
%      &  & Species & 11,203 &  &  &  \\
%      &  & Total & 13,849 &  &  &  \\
%      & \multirow{3}{*}{STRUO} & Strains & 6 & \multirow{3}{*}{-} & \multirow{3}{*}{-} & \multirow{3}{*}{2,789} \\
%      &  & Species & 19,289 &  &  &  \\
%      &  & Total & 19,325 &  &  &  \\
%      & \multirow{3}{*}{BioProject} & Strains & 5,754 & \multirow{3}{*}{309} & \multirow{3}{*}{626} & \multirow{3}{*}{-} \\
%      &  & Species & 3,373 &  &  &  \\
%      &  & Total & 9,393 &  &  &  \\
%      \multirow{3}{*}{Total} & \multirow{3}{*}{All} & Strains & 12,840 & \multirow{3}{*}{1,090} & \multirow{3}{*}{15,091} & \multirow{3}{*}{7,971} \\
%      &  & \multicolumn{1}{l}{} & \multicolumn{1}{l}{} &  &  &  \\
%      &  & \multicolumn{1}{l}{} & \multicolumn{1}{l}{} &  &  &  \\ \bottomrule
%      \end{tabular}
%
%      \caption[The entities of PREGO after the NER and mapping of every source]{
%         The entities of PREGO after the NER and mapping of every source. 
%         Counts of distinct entities of Taxa, Environments (ENVO terms), Biological Processes (Gene Ontology Biological process) and Molecular Function (Gene Ontology Molecular Function).
%      }
%      \label{table:prego2}
%
%   \end{table}


%   % ASSOCIATIOS TABLE - table:prego3
%   \begin{sidewaystable}
%      \begin{tabular}{@{}cccccrcr@{}}
%      \toprule
%      \textbf{Channel} & \textbf{Source} & \textbf{\begin{tabular}[c]{@{}c@{}}Environments \\ - \\ Processes\end{tabular}} & \textbf{\begin{tabular}[c]{@{}c@{}}Environments \\ - \\ Functions\end{tabular}} & \textbf{Taxonomy} & \multicolumn{1}{c}{\textbf{\begin{tabular}[c]{@{}c@{}}Taxa \\ -\\  Environments\end{tabular}}} & \textbf{\begin{tabular}[c]{@{}c@{}}Taxa \\ - \\ Processes\end{tabular}} & \multicolumn{1}{c}{\textbf{\begin{tabular}[c]{@{}c@{}}Taxa \\ -\\ Function\end{tabular}}} \\ \midrule
%      \multirow{3}{*}{Literature} & \multirow{3}{*}{\begin{tabular}[c]{@{}c@{}}MEDLINE \\ PubMed - \\ PMC OA\end{tabular}} & \multirow{3}{*}{883,997} & \multirow{3}{*}{422,579} & Strains & 69,968 & \multicolumn{1}{r}{590,630} & 384,079 \\
%      &  &  &  & Species & 778,877 & \multicolumn{1}{r}{3,501,635} & 1,961,920 \\
%      &  &  &  & Total & 1,669,608 & \multicolumn{1}{r}{7,969,310} & 4,613,827 \\
%      \multirow{9}{*}{\begin{tabular}[c]{@{}c@{}}Environmental \\ samples\end{tabular}} & \multirow{3}{*}{\begin{tabular}[c]{@{}c@{}}MG-RAST \\ amplicon\end{tabular}} & \multirow{3}{*}{-} & \multirow{3}{*}{-} & Strains & 13,645 & \multirow{3}{*}{-} & \multicolumn{1}{c}{\multirow{3}{*}{-}} \\
%      &  &  &  & Species & 39,007 &  & \multicolumn{1}{c}{} \\
%      &  &  &  & Total & 53,439 &  & \multicolumn{1}{c}{} \\
%      & \multirow{3}{*}{\begin{tabular}[c]{@{}c@{}}MG-RAST \\ metagenome\end{tabular}} & \multirow{3}{*}{-} & \multirow{3}{*}{620,846} & Strains & 262,106 & \multirow{3}{*}{-} & 8,626,328 \\
%      &  &  &  & Species & 103,913 &  & 10,715,548 \\
%      &  &  &  & Total & 372,301 &  & 19,950,096 \\
%      & \multirow{3}{*}{\begin{tabular}[c]{@{}c@{}}MGnify \\ amplicon\end{tabular}} & \multirow{3}{*}{-} & \multirow{3}{*}{-} & Strains & 18 & - & \multicolumn{1}{l}{} \\
%      &  &  &  & Species & 30,122 & \multicolumn{1}{r}{351} & \multicolumn{1}{c}{-} \\
%      &  &  &  & Total & 111,976 & \multicolumn{1}{r}{2,097} & \multicolumn{1}{l}{} \\
%      \multirow{9}{*}{\begin{tabular}[c]{@{}c@{}}Annotated Genomes \\ and Isolates\end{tabular}} & \multirow{3}{*}{\begin{tabular}[c]{@{}c@{}}JGI IMG\\ isolates\end{tabular}} & \multirow{3}{*}{-} & \multirow{3}{*}{-} & Strains & 8,229 & \multirow{3}{*}{-} & 3,461,693 \\
%      &  &  &  & Species & 42,141 &  & 13,216,559 \\
%      &  &  &  & Total & 50,888 &  & 16,821,850 \\
%      & \multirow{3}{*}{STRUO} & \multirow{3}{*}{-} & \multirow{3}{*}{-} & Strains & \multicolumn{1}{c}{\multirow{3}{*}{-}} & \multirow{3}{*}{-} & 1,803 \\
%      &  &  &  & Species & \multicolumn{1}{c}{} &  & 4,070,195 \\
%      &  &  &  & Total & \multicolumn{1}{c}{} &  & 4,079,312 \\
%      & \multirow{3}{*}{BioProject} & \multirow{3}{*}{-} & \multirow{3}{*}{-} & Strains & 3,263 & \multicolumn{1}{r}{7,473} & \multicolumn{1}{l}{} \\
%      &  &  &  & Species & 4,187 & \multicolumn{1}{r}{4,294} & \multicolumn{1}{l}{} \\
%      &  &  &  & Total & 7,641 & \multicolumn{1}{r}{12,169} & \multicolumn{1}{l}{} \\
%      \multirow{3}{*}{Total} & \multirow{3}{*}{All} & \multirow{3}{*}{883,997} & \multirow{3}{*}{1,043,425} & Strains & 357,229 & \multicolumn{1}{r}{598,103} & 12,473,903 \\
%      &  &  &  & Species & 998,247 & \multicolumn{1}{r}{3,506,280} & 29,964,222 \\
%      &  &  &  & Total & 2,265,853 & \multicolumn{1}{r}{7,983,576} & 45,465,085 \\ \cmidrule(l){5-8} 
%      \end{tabular}
%      \caption[Associations among the PREGO entities]{
%         The associations between entities of PREGO after co-occurrence analysis: The supported entity types of associations are Environments—Biological Processes, Environments—Molecular Functions, Taxa—Environments, Taxa—Biological Processes, Taxa—Molecular Functions.
%      }
%      \label{table:prego3}
%   \end{sidewaystable}

