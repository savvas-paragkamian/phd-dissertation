% --------------------------------------------------
% 
% This chapter is for ISD Crete
% 
% --------------------------------------------------


\chapter{The conservation status of the Cretan Endemic Arthropods under Natura 2000 network}
\label{cha:arhropods}


\textbf{Citation:} \\ 
Giannis Bolanakis, Savvas Paragkamian, Maria Chatzaki, Nefeli Kotitsa, Liubitsa Kardaki and Apostolos Trichas. 

DOI: \href{https://doi.org/10.21203/rs.3.rs-2671168/v1}\footnote{
   For author contributions and supplementary material please refer to the relevant sections. 
   This is a modified version of the published version,
   in terms of relevance, coherence and formatting.
   }



% ABSTRACT
\section{Abstract}

Arthropod decline has been globally and locally documented, yet they are still
not sufficiently protected. Crete (Greece), a Mediterranean biodiversity
hotspot, is a continental island renowned for its diverse geology, ecosystems
and endemicity of flora and fauna, with continuous research on its Arthropod
fauna dating back to the 19th century. Here we investigate the conservation
status of the Cretan Arthropods using Preliminary Automated Conservation
Assessments (PACA) and the overlap of Cretan Arthropod distributions with the
Natura 2000 protected areas. Moreover we investigate their endemicity hotspots
and propose candidate Key Biodiversity Areas. In order to perform these
analyses, we assembled occurrences of the endemic Arthropods in Crete located
in the collections of the Natural History Museum of Crete together with
literature data. These assessments resulted in 75\% of endemic Arthropods as
potentially threatened. The hotspots of endemic taxa and the candidate Key
Biodiversity Areas are distributed mostly on the mountainous areas where the
Natura 2000 protected areas have great coverage. Yet human activities have
significant impact even in those areas, while some taxa are not sufficiently
covered by Natura 2000. These findings call for countermeasures and conservation actions.


% INTRODUCTION
\section{Introduction}
\label{sec:arthropods-intro}

In the Anthropocene, the need to tackle biodiversity loss is urgent (Johnson et al. 2017; Meng et al. 2021).
Arthropods include more than 78\% of the described animal taxa (Zhang 2013)
numbering approximately 7 million terrestrial species (Stork 2018). Many recent
studies highlight the decline of Insect (Cardoso et al. 2020; Wagner 2020; Raven and Wagner 2021),
Spider (Potapov et al. 2019; Branco and Cardoso 2020) and Myriapod biodiversity (Karam-Gamael et al. 2018; Iniesta et al. 2023).
For instance, Hallmann et al. (2017) estimate a 75\% reduction of flying Insect
biomass in Germany in the last 27 years. Klink et al. (2020) yielded a 9\% per
decade decline in Insect abundance. Sánchez-Bayo and Wyckhuys (2019) estimate
the possible extinction of 40\% of Insect species in the near future (but see
Wagner (2019) for a critique). Consequently Samways (2019) speaks about the “Fall of Insects”.
Yet, the actions taken for their conservation are deemed as insufficient in
global and local scale (Cardoso et al. 2012; D’Amen et al. 2013; Chowdhury et al. 2023).

Numbering approximately 7,000 species [extrapolated from Fauna Europaea
(Jong et al. 2014) and Legakis et al. (2018)], the Arthropods of Crete, Greece,
have been studied for almost two centuries (Anastasiou et al. 2018).
Only 135 of these species (1.9\%) have been assessed in IUCN Red List as of
this publication, making Arthropods the third most evaluated group of the
island, behind vascular plants (291) and land mollusks (165). The low
evaluation percentage is a common motif for Arthropods, hindered by the lack of
data (Cardoso et al. 2011a, b; Cardoso et al. 2012; Wagner et al. 2021) and
charisma of the Arthropods themselves (Cardoso 2012; Wang et al. 2021), leading
to knowledge shortfalls (see Hortal et al. 2015).

Crete is located between three continents (Europe, Africa, Asia), in a
well established global biodiversity hotspot (Myers et al. 2000) of the
Mediterranean basin. Isolated from the rest of the Aegean and the continental
Greece for more than 5 million years (Fassoulas 2018), with a complex geological
and climatic history and long-term human presence (Rackham and Moody 1996),
Crete has developed a species rich biodiversity with high endemism
(Médail and Quézel 1997; Chatzaki et al. 2015; Sfenthourakis and Schmalfuss 2018; Vardinoyannis et al. 2018).
It is a special biogeographical entity for various taxonomic groups: Buprestidae (Mühle et al. 2000),
Tenebrionidae (Fattorini 2006a, 2008), Cerambycidae (Vitali and Schmitt 2017),
Orthoptera (Willemse et al. 2023), Vascular Plants (Kougioumoutzis et al. 2017)
and snails (Vardinoyannis et al. 2018). Moreover, Crete presents the highest
percentage of threatened species of the IUCN assessed Greek fauna and flora
(12\%) (Spiliopoulou et al. 2021) and is the hottest Mediterranean island for
plant endemism (Médail 2017). The biogeographical and conservational
significance of Crete thus becomes apparent.

Arthropod decline is the result of multiple - synergistically acting - causes (Cardoso et al. 2020; Wagner 2020; Wagner et al. 2021).
Habitat loss (Cardoso et al. 2020; Wagner 2020; Wagner et al. 2021),
agricultural intensification (Habel et al. 2019; Raven and Wagner 2021),
urbanization (Wagner et al. 2021), pollution/pesticides (Brühl and Zaller 2019; Cardoso et al. 2020)
and climate change (Cardoso et al. 2020; Harvey et al. 2022) are the major
drivers of this decline. Crete complies with this global trend.

Habitat loss and degradation occurs throughout Crete as a result of urban,
agricultural and touristic development. This is a major issue since habitat
loss is a major threat in Europe for many Arthropod groups,
e.g. Butterflies (van Swaay et al. 2010) Bees (Nieto et al. 2014),
Orthoptera (Hochkirch et al. 2016) and Saproxylic Beetles (Cálix et al. 2018).
Climate change is predicted to induce scarcer yet more intense precipitation,
increase of drought locally (Koutroulis et al. 2011) and shrinkage as well as
possible shifts to the rainfall period (Koutroulis et al. 2013). Groups
associated with fresh water could be deeply impacted from the locally increased
drought and the increase in need of water for irrigation and domestic use,
e.g. Odonata (Kalkman et al. 2010), which has become harsher due to the
increase of agriculture and land use (Tzanakakis et al. 2020). Stock raising
(sheep and goats) has always been an important aspect of Cretan life and
economy (Rackham and Moody 1996). Overgrazing impacts severely soil erosion,
soil moisture and vegetation (Kairis et al. 2015; Kosmas et al. 2015). All the
above contribute to a worrying trend for Crete, i.e. the higher percentage of
Threatened endemic Arthropods when compared with the respective European
groups (Supplementary Material 2, Figure 1).

The largest structure of biodiversity conservation in Crete is the Natura 2000
network (N2K). N2K is the only regional assemblage of protected areas
worldwide (Crofts 2014). Operating throughout European Union (EU) since 1992,
the N2K is the alloy of two EU directives, The Birds Directive (Council Directive 79/409/EEC, 1979)
and the Habitats and Species Directive (HSD) (The Council Directive 92/43/EEC, 1992).
The Arthropods are linked with the HSD. None of the Cretan endemic Arthropods
are listed in the annex of the HSD (driven by taxonomical, geographical and
other biases - Cardoso 2012).

Crete has by far the highest percentage of overlap between threatened species’
ranges (flora and fauna) and N2K in Greece (Spiliopoulou et al. 2021).
Sfenthourakis and Legakis (2001) investigated the N2K overlap in Crete with
land mollusks, Orthoptera, Carabidae, Tenebrionidae and Oniscidea, and found
that four out of five endemicity hotspots in Crete (Dia islet, Lefka Ori,
Psiloritis and Dikti massifs) reside in N2K. Kougioumoutzis et al. (2021a, b)
found a great number of endemicity hotspots and threat-spots of Greek vascular
plants in the Cretan mountains with significant overlap with the N2K.
In contrast Dimitrakopoulos et al. (2004) focusing on vascular plants,
recovered small percentages of overlap between plant endemicity / threat-spots and N2K.
Overall, Crete seems to be under an adequate protection regime (Kougioumoutzis et al. 2021b; Spiliopoulou et al. 2021),
however, the aforementioned studies do not focus on Arthropods, leaving space
for a more close up research for their conservation status.

In this study we aim to a) identify cretan endemicity hotspots (EHs)
b) investigate for candidate Key Biodiversity Areas (KBAs)
c) examine the overlap of EHs, KBAs and threatened taxa with the N2K areas and
d) their relation with the anthropogenic pressures in these sites (Figure 1).
To do so, we assembled the accumulated knowledge of the past 200 years of
entomological research in Crete with the collections of NHMC for 11 Arthropod groups:
Araneae, Scorpiones, Chilopoda, Diplopoda, Coleoptera, Heteroptera,
Hymenoptera, Lepidoptera, Odonata, Orthoptera and Trichoptera.
Secondly, we committed to the Findable, Accessible, Interoperable and Reproducible (FAIR) principles (Wilkinson et al., 2016)
for data and code (see Supplementary Material). Hence Crete could become Greece’s
spearhead in meta-analyses concerning Arthropods. Thus, we contribute to the
ongoing discussion concerning the global conservation status of Arthropods from
the perspective of a continental island, rich in endemic species.


% METHODS
\section{Methods}
\label{sec:arthropods-method}
   
    \subsection{Taxa selection criteria}
    \label{subsec:arthropods-taxa-selection}

We aggregated data of Cretan endemic Arthropod groups (NHMC collections and bibliography). The taxa included should satisfy the following criteria:

There should be at least one authoritative work on the group for Crete which
should make clear remarks about the group’s taxonomic dynamics, so that future
taxonomical or systematic works on the taxon would not severely affect our inferences.
In essence, we selected groups whose biodiversity is well studied and we do not
expect significant changes in their number of species for Crete. 

The geographic information about the endemic species distribution should be in
a form convertible to coordinates (i.e., either in coordinates or with a
precise locality). The included coordinates are precise, while the converted
ones do not exceed a radius of certainty more than 2 km.

The group should not be dominated by cavernicolous species. We opted to exclude
cave dwelling fauna, since it is a special system, governed by different
biogeographical and ecological processes. Thus, groups like terrestrial Isopods
were excluded for their high percentage of cavernicolous species (Schamlfuss et al. 2004; Sfenthourakis and Schmalfuss 2018).

Based on the above criteria, the selected groups are:
Araneae, Chilopoda, Coleoptera, Diplopoda, Heteroptera, Hymenoptera (Chrysididae, Formicidae, Symphyta), Lepidoptera (Geometridae), Odonata, Orthoptera, Scorpiones and Trichoptera.

    \subsection{Data assemblage}
    \label{subsec:arthropods-data-assemblage}

We curated the bibliography and the NHMC collection to assemble taxa
occurrences (Supplementary Material 1; Figure 2C). The bibliography used
contains both historical and contemporary published material. Bibliographic
records include: 1) Author of the article 2) Species name and 3) Locality coordinates.
For the records without coordinates, coordinates approximating the site were
given based on the locality and description given. NHMC specimens (over 2
million Arthropods) have been primarily collected by pitfall trapping as part
of MSc, PhD studies and environmental monitoring programs, over the last 40
years (details on trapping protocols are discussed more extensively in Salata et al. 2020b and Willemse et al., 2023).
NHMC data include: 1) NHMC Field Code, 2) Species name and 3) Locality.
The coordinate reference system we used for all location data is WGS84 - EPSG:4326.

We opted for an integrated approach including as many Arthropod taxa as
possible. Thus, our dataset is inhomogeneous. Different Orders and/or Families
require different sampling methods, while there has been an inconsistent
historical interest for various groups. For example Staphylinidae (Coleoptera)
are systematically studied in the last 30 years in Crete, while the research in
Carabidae (Coleoptera) dates back to the 19th century. Sampling methods vary
even within groups. Orthoptera have been classically sampled by net or hand
from the beginning of the 20th century, while in the last 30 years, when NHMC
started studying the Cretan biodiversity, there are numerous specimens that
have been captured with pitfall traps (Willemse et al. 2023).

Different subspecies of the same species were treated separately as in Fattorini (2006b); Dimitrakopoulos et al. (2004); Fattorini and Baselga (2012),
because the distinction between species and subspecies is usually arbitrary and
unstable, hence excluding subspecies could lead to the neglection of important
conservational or evolutionary units. From here on, we refer to both species
and subspecies as “taxa”.
    

    \subsection{Taxa assessments}
    \label{subsec:arthropods-taxa-assessments}
For the taxa assessment we used the Preliminary Automated Conservation Assessment
(PACA) pipeline (Stévart et al. 2019). PACA is an approximation of the IUCN
assessment based on Criterion B, i.e. on the Extent of Occurrence (EOO) and
Area of Occupancy (AOO) and cannot be used as a replacement of a full IUCN
assessment. Some important differences between PACA and a full IUCN assessment
are that PACA always assumes a continuous decline of the species’ habitat
quality and automatizes some processes that require the assessors' engagement
(e.g. defining locations). PACA is a useful tool to obtain a preliminary image
regarding a taxa assemblage of an area, in the absence of a thorough IUCN assessment.
Moreover PACA can be really useful for datasets that have incorporated subspecies,
which enjoy less attention from IUCN mainly for taxonomic reasons, i.e., the
lack of consensus on the subspecies as a biological entity. Criterion B is the
one most widely used for Arthropods (Cardoso et al. 2011a, b; Carpaneto et al. 2015),
since most Arthropod groups lack the data for the other criteria (A, C, D and E),
i.e., mainly population size and trend information or quantitative analyses.
Criterion B could overestimate the danger of Arthropods (Cardoso et al. 2011a),
which should always be taken into consideration.

In order to identify locations (as defined by PACA), we used the European
Environment Agency (EEA) reference grid with a 10 x 10 km grid cell to assign
occurrences to locations. All the occurrences of a taxon that reside in a
10 x 10 km cell constitute one location. The PACA are estimated as shown in Table 1.

Subsequently, we converted the PACA categories to the respective IUCN ones (Stévart et al. 2019).

\begin{table}
\centering
\caption{The PACA and potential IUCN categories based on the number of locations and EOO or AOO.}
\begin{tabular}{p{0.2\linewidth} | p{0.2\linewidth} | p{0.2\linewidth} | p{0.2\linewidth}}
\textbf{Potential IUCN categories}     & \textbf{PACA categories}         & \textbf{\# locations} & \textbf{EOO (km2) OR AOO (km2)}  \\
Potentially Vulnerable (VU)            & Potentially Threatened (PT)      & =10                   & 20000 OR 2000                    \\
Potentially Endangered (EN)            & Likely Threatened (LT)           & =5                    & 5000 OR 500                      \\
Potentially Critically Endangered (CR) & Likely Threatened (LT)           & 1                     & 100 OR 10                        \\
Other                                  & Potentially Not Threatened (PNT) & rest                  & rest                            
\end{tabular}
\label{arthropods-paca}
\end{table}
    
    \subsection{Endemicity Hotspots (EHs), and Key Biodiversity Areas (KBAs)}
    \label{subsec:arthropods-ehs-kbas}

Hotspot definitions vary from quantitative methods to experts opinions and curation.
In quantitative methods, grid size and shape influences the determination of
the areas of interest such as hotspots and key biodiversity areas (Hurlbert and Jetz, 2007, Nhancale and Smith, 2011).
Choosing the size of the grid is not trivial (Mo et al., 2019) and is dependent
on the conservation goals (Margules and Pressey, 2000). In the past decade,
there have been major advances for conservation standards, guidelines,
frameworks and tools available to be put into action (IPBES 2019).

We defined the EHs as the 10\% of the grid cells with the highest number of
endemic taxa. In order to avoid biases concerning the grid cell size, the same
pipeline was tested with cells of different size (4 x 4, 8 x 8 and 10 x 10 km).
For the subsequent analyses we opted for the 10 x 10 km grid (see section 3.1)
which is also the EEA reference grid, the standard for the reporting format
(Groups of Experts, 2017) of the Resolution No. 8 (2012) of the Standing Committee
to the Bern Convention on the Emerald Network of Areas of Special Conservation Interest (ASCI).
Moreover, the EHs of the various cell sizes are aggregated in the same areas (Figure 3A).
We made the same treatment for each of the selected groups separately (Supplementary Material 2, Figure 2).
We redefined EHs as the 10\% of the grid cells with max overlap of the orders
to check for biases towards more speciose orders (e.g. Coleoptera) (Supplementary Material 2, Figure 3).

For the investigation of KBAs (IUCN 2016) we used the WEGE index (Farooq et al. 2020)
on the same grid as hotspots based on the PACA assessments and the distributions
of the taxa. WEGE can be used to indicate candidate KBAs or prioritize already
existing KBAs (given the limited resources available for conservation) but does
not replace a throughout KBAs assessment (Farooq et al. 2020).

    \subsection{Spatial overlaps}
    \label{subsec:arthropods-spatial}
We compared and evaluated the overlap of EHs and KBAs with protected areas and
land use categories. The N2K data were downloaded from the European Environment
Agency portal and filtered for the Habitats Directive and Crete spatial extent.
We retrieved land use categories from the CORINE Land Cover, CLC 2018 version
v.2020-20u1 (Copernicus Land Monitoring Service, 2023).
To evaluate the current land use of yielded EHs we used the CORINE Land Cover
and to examine the human pressure (change in land use, agriculture), we used the
Historic Land Dynamics Assessment (HILDA+) dataset (Winkler et al. 2021) to
estimate the change of land use the from 1998 to 2018. Furthermore, we examined
the overlap of the AOO of each taxon with the N2K.


    \subsection{Tools and scripts}
    \label{subsec:arthropods-tools}
We performed the analyses using the R Statistical Software (v4.3.2; R Core Team 2023),
the visualization using the ggplot2 R package (Wickham, 2016). The figures
created are colored using the colorblind-friendly 'Okabe-Ito' palette (Ichihara et al., 2009).
We calculated EOO and AOO using the ConR R package (Dauby et al. 2017) and PACA
using custom scripts. For the spatial data handling, transformations and
geometry we used the sf v1.0-14 (Pebesma 2018) and terra v1.7-55 R packages (Hijmans 2023).
WEGE index is calculated with the WEGE R package (Farooq et al. 2020).
Adaptive grid is created using the quadtree R package (Friend, 2023).
Jaccard similarity was calculated with the vegan 2.6-4 r package (Oksanen et al. 2022).
All scripts are reproducible by design and available in this 
\href{https://github.com/savvas-paragkamian/arthropoda_assessment_crete}{GitHub repository}.

% RESULTS
\section{Results}
\label{sec:arthropods-results}

Using over 100 publications (as of 2020) and 733 NHMC sampling events
(Supplementary Material 1), we assembled a dataset of 343 taxa (species and subspecies),
with 4,924 records across 1,569 distinct sites of Crete (Figure 2C). The taxa
are distributed to eleven orders, with Coleoptera having the most taxa (206)
and Chilopoda and Scorpiones the least (two) (Table 3).

    \subsection{Grid cell size}
    \label{subsec:arthropods-grids}
The grid cell size 10 x 10 km is the most suitable for our study since our
dataset - being compiled from numerous different sources and sampling efforts -
is rather coarse and inhomogenous for a smaller cell size (Figure 3A).
The unique taxa of the EHs of each grid is distributed as follows: 10 km=283,
8 km=278, 4 km=293, adaptive cells=267, with the 4km grid covering most endemic
species. The 4 km grid mostly highlighted areas known for their tourist/recreational activities,
indicating that it is more sensitive to sampling intensity (Figure 3A).
Focusing on sampling we applied the adaptive grid size with quadtrees resulting
in 157 grids with 8 km length, 38 with 4 km and 74 with 2 km (Supplementary Material 2, Figure 5).
This indicates the preference of larger cells for the majority of our dataset
even though a small percent of regions has higher density of sampling.
The highest overlap among all grids is between the 10 km and 8 km reaching
57\% (Supplementary Material 2, Table 2). Finally, the 10 km grid has more taxa
per cell (Supplementary Material 2, Figure 6) and is a reference grid system.
Based on our analysis and interoperability and reproducibility aims we choose
the 10 km EEA reference grid for the EHs and candidate KBAs inference.
Nevertheless, we also performed the WEGE analysis for KBAs using the adaptive
grid, yielding practically the same areas as the 10km grid minus Zakros (Supplementary Material 2, Figure 5C).
The same pipeline can not be done with EHs for they require a fixed cell size.

    \subsection{EHs}
    \label{subsec:EHs}

The EHs cover 17\% of Crete (Table 2). They are aggregated in Lefka Ori, Dikti,
Psiloritis, Thrypti and Selino at the southwest of Chania (Figure 3B).
Four of the five areas are mountainous (the main massifs of Crete). Lefka Ori
and Dikti host the highest number of EHs. Psiloritis hosts one, the only EH in
central Crete. No satellite island of Crete is yielded as an EH, in spite of
their faunas being in essence a subset of the Cretan biodiversity, and Gavdos
islet having some single island endemic Arthropods. The reason behind this is a
purely numeric one. Compared to the yielded EHs, they have less endemic and local
endemic taxa. With finer grids the number of EHs increases (Figure 3A). Gavdos
islet is yielded as EH only in the 4 x 4 km grid. The finer grid is less strict
and appropriate for an inhomogeneous dataset as ours, although it does indeed
unveil areas that would be otherwise neglected. From now on we discuss our
results grounded in the 10 x 10 km grid, but refer to Figure 3A for the other
cell sizes.

The different orders display a variation in their respective EHs (Supplementary material 2, Figure 2).
Almost unanimously, they exhibit hotspots in one or more massifs,
with Trichoptera and Odonata being exceptions, driven from their need of inland waters.
When aggregated, the EHs of the different orders generally agree with EHs of
Arthropods as a whole (Supplementary Material 2, Figures 3, 4A). Only the
latter approach (Arthropods as a whole) is treated onwards.

\begin{table}[]
\caption{Overlap of Arthropod EHs, WEGE KBAs with N2K HSD, Wildlife Refuges and CORINE Land Cover (LEVEL1) areas of Crete. Areas are measured in km\textsuperscript{2}}
\begin{tabular}{lllll}
Type                          & Area       & \% of Crete & Overlap EHs (\%)                          & Overlap KBAs (\%)                  \\
Crete                         & 8347       & -           & -                                      & -                               \\
Endemic hotspots (EHs)        & 1400       & 17\%        & -                                      & 1200 (86\%)                     \\
WEGE KBAs                     & 1400       & 17\%        & 1200 (86\%)                            & -                               \\
Natura2000 SAC                & 2371       & 28\%        & 858 (61\%)                             & 736 (53\%)                      \\
Wildlife refuges              & 610        & 7\%         & 143 (10\%)                             & 136 (10\%)                      \\
Agricultural areas            & 3618       & 46\%        & 275 (20\%)                             & 301 (22\%)                      \\
Artificial surfaces           & 181        & 2\%         & 5 (0.4\%)                              & 4.56 (0.3\%)                    \\
Forest and semi natural areas & 4508       & 54\%        & 1092 (78\%)                            & 1059 (76\%)                     \\
Water bodies                  & 7          & 0.08\%      & 1 (0.07\%)                             & 0.3 (0.02\%)                   
\end{tabular}
\label{arthropods-overlaps}
\end{table}


    \subsection{Species Assessment}
    \label{subsec:arthropods-species-assessment}
    
According to the PACA analysis, 75\% of the taxa are Likely / Potentially
Threatened (from here on referred to as Threatened) and 25\% are assessed as
Near Threatened/Least Concern (Table 3). These percentages vary between the groups,
nevertheless there are some concrete patterns. For example, threatened
categories dominate most of the orders except Odonata and Orthoptera (Table 3).
Chilopoda and Scorpiones have no threatened taxa at all. Both of these orders
display a low endemic diversity (two species each). On the contrary,
Heteroptera have only threatened taxa, followed by Coleoptera (82\%),
Diplopoda (71\%), Hymenoptera (68\%), Geometridae (67\%) and Aranae (65\%).
Of course, the bias of Criterion B towards a more severe categorization
(Cardoso et al. 2011a) and the fact that we are using a preliminary assessment
advocate a conservative interpretation of our results which are explorative and not concrete assessments.

% \usepackage{color}
% \usepackage{array}
% \usepackage{longtable}
% \usepackage{array}

\begin{sidewaystable}
\caption{Number of taxa included in the dataset in total and per order. In addition, the mean AOO km\textsuperscript{2} for each taxon and its coverage by N2K are given (Standard Deviation in parentheses), as well as the percentages for PACA and IUCN categories. Categories for PACA: LT - Likely Threatened, PT - Potentially Threatened and LNT - Likely Not Threatened. Categories for IUCN: Threatened (sum of Critically Endangered, Endangered and Vulnerable) and NT/LC - Near Threatened/Least Concern.}
\begin{tabular}{lllllllllll}
Order                        & taxa & sites & occurrences & Mean AOO (sd)     & Mean AOO in N2K (sd)        & LT         & PT        & LNT       & Threatened & NT/LC     \\
All taxa                     & 343  & 1539  & 4924        & 58 (109)          & 27 (51)                  & 195 (57\%) & 62 (18\%) & 86 (25\%) & 257 (75\%) & 86 (25\%) \\
Araneae                      & 40   & 253   & 523         & 55 (63)           & 21 (19)                  & 16 (40\%)  & 10 (25\%) & 14 (35\%) & 26 (65\%)  & 14 (35\%) \\
Chilopoda                    & 2    & 234   & 269         & 552 (560)         & 223 (251)                & 0          & 0         & 2 (100\%) & 0          & 2 (100\%) \\
Coleoptera                   & 206  & 925   & 2584        & 50 (90)           & 26 (45)                  & 132 (64\%) & 36 (17\%) & 38 (18\%) & 168 (82\%) & 38 (18\%) \\
Diplopoda                    & 7    & 74    & 101         & 61 (88)           & 27 (43)                  & 4 (57\%)   & 1 (14\%)  & 2 (29\%)  & 5 (71\%)   & 2 (29\%)  \\
Heteroptera                  & 17   & 46    & 58          & 14 (13)           & 4 (3)                    & 14 (82\%)  & 3 (18\%)  & NA        & 17 (100\%) & NA        \\
Hymenoptera                  & 25   & 173   & 285         & 48 (70)           & 20 (30)                  & 14 (56\%)  & 3 (12\%)  & 8 (32\%)  & 17 (68\%)  & 8 (32\%)  \\
Lepidoptera                  & 9    & 40    & 65          & 30 (24)           & 13 (10)                  & 5 (56\%)   & 1 (11\%)  & 3 (33\%)  & 6 (67\%)   & 3 (33\%)  \\
Odonata                      & 3    & 30    & 49          & 63 (26)           & 12 (6)                   & 0          & 1 (33\%)  & 2 (67\%)  & 1 (33\%)   & 2 (67\%)  \\
Orthoptera                   & 20   & 363   & 579         & 122 (133)         & 61 (64)                  & 7 (35\%)   & 3 (15\%)  & 10 (50\%) & 10 (50\%)  & 10 (50\%) \\
Scorpiones                   & 2    & 240   & 254         & 518 (619)         & 240 (279)                & 0          & 0         & 2 (100\%) & 0          & 2 (100\%) \\
Trichoptera                  & 12   & 66    & 157         & 54 (38)           & 16 (10)                  & 3 (25\%)   & 4 (33\%)  & 5 (42\%)  & 7 (58\%)   & 5 (42\%) 
\end{tabular}
\label{arthropods-results}
\end{sidewaystable}



\subsection{Potential KBAs}
    \label{subsec:arthropods-potential-kbas}
    
The proposed KBAs recovered with WEGE cover 17\% of Crete (Table 2). They are
aggregated in the Cretan mountains (as EHs) in the far west (Selino) and far
east (Zakros) Crete (Figure 3C). WEGE utilizes the threat status and the
distributions of the taxa to rank potential KBAs (Farooq et al. 2020), thus it
is expected for yield areas with high congruence with EHs since many threatened
species are concentrated there. PACA assessment was carried out based on
Criterion B, which primarily uses the range of the species to estimate the
Threat category, thus it is prone to assess geographically restricted species
(e.g., in a mountain plateau) as Threatened (Figure 3B, C).

    \subsection{N2K overlap}
    \label{subsec:arthropods-n2k-overlap}

EHs display 61\% overlap with the N2K (Table 2, Figure 3B), mostly of which is
in the mountains (Figure 2C, 3B). The greatest overlap occurs in Psiloritis and
Dikti (Figure 3B) In contrast, the EHs outside the Cretan mountains
(Selino, Kritsa - near Dikti) display lower overlap with the respective
protected areas near them (Figure 3B).The areas of agreement of KBAs with N2K
are aggregated in the Cretan mountains, while Selino, Kritsa and Zakros are the
areas with the smallest overlap (Figure 3C). The average overlap of taxa AOO
with N2K is 52\% and increases to 55\% when only the threatened taxa are
considered (Table 3). Orthoptera have the highest average \% overlap (62.38\%),
while Odonata have the lowest (20.39\%) (Figure 4).
    
    \subsection{Human Intervention}
    \label{subsec:arthropods-human-intervention}


In order to evaluate the human impact in the yielded EHs and candidate KBAs we
used CORINE layers (Figure 5A). At LEVEL 1 of the classification, the dominant
habitat is “Forest and semi-natural areas”, covering ~ 76-78\% of the EHs/KBAs,
while agricultural areas also display coverage of 20-22\%. This exhibits the
presence of human activity in the EHs/KBAs (Table 2). Using the LEVEL 2 CORINE
layer, we acquired a more detailed image of the coverage. The dominant habitat
seems to consist of scrub and/or herbaceous vegetation  (56\% coverage),
forests (12\%), permanent crops (10\%) and open spaces with little or
no vegetation (10\%) (Supplementary Material 2, Table 3). With HILDA+ we
estimated some negative and some positive transitions within the EHs/KBAs in
the previous two decades (Supplementary material 2, Table 4). Around 10-12\% of
forest area has been transformed to cropland. Likewise 22-25\% of cropland has
been transformed to pastureland. On the other hand 15-26\% of cropland area has
been transformed to forest. Worryingly urban areas have increased for about 16.8\%,
although outside EHs and KBAs mainly at the expense of croplands and
pasturelands (Supplementary Material 2, Table 4). Finally, water areas remain
stable, albeit more research is needed to assess potential decreases in the
quality of this habitat, especially given the aggressive urban and touristic expansion.

% DISCUSSION
\section{Discussion}
\label{sec:arthropods-discussion}

    \subsection{Endemicity Hotspots}
    \label{subsec:arthropods-Endemicity-Hotspots}

Mountains host a great amount of Earth’s biodiversity, being a main driver for
the birth of species (Antonelli et al. 2018; Noroozi et al. 2018; Rahbek et al. 2019a, b)
and a crucial frontier for their fate (Steinbauer et al. 2018; Urban 2018).
Crete is not an exception to this trend (Trigas et al. 2013; Kougioumoutzis et al. 2020).
Our results conform to that, since the EHs are gathered primarily in the major
Cretan mountains (Figure 3B). Lefka Ori and Dikti are the sites with the most
EHs, in agreement with studies focused on vascular plants (Dimitrakopoulos et al. 2004; Kougioumoutzis et al. 2020).
Sfenthourakis and Legakis (2001), employing invertebrate groups, also recovered these mountains as EHs.

Only one EH was recovered for Psiloritis in this study. This could result from
Psiloritis’ position in the center of the island, with Lefka Ori and Dikti
filtering taxa moving from west and east, from its relatively smaller volume
(when compared with Lefka Ori), the intense human intervention, and the less
intense topography and relief compared to the other Cretan mountains (for the
importance of topography and relief in speciation - biodiversity see: Stuessy et al. 2006; Muellner-Riehl 2019; Igea and Tanentzap 2021).
Thrypti as EH is consistent with the aforementioned literature. Isolated in the
far east part of the island, Thrypti could be of a major conservational importance for Crete.
A novelty of our study is the relative importance (participating with more grid cells)
of Dikti when compared with the aforementioned studies, even though it is always obtained as an EH site. 

Dia islet, although obtained as an EH for invertebrate fauna in Sfenthourakis
and Legakis (2001), is not recovered as a hotspot for Arthropods in our study.
The island of Dia has indeed some importance for Arthropod taxa, such Isopods,
hosting some single island endemics (Schmalfuss et al. 2004), but its endemic
diversity is mostly driven by snails (Vardinoyannis pers. communication), which
are not treated here.

The accumulation of more EHs in the West (Lefka Ori and west of Lefka Ori) and
East (Dikti and Thrypti) Crete can be explained by their isolation today and in
the past, when Crete was divided in palaeo-islands during the Pliocene
(see Poulakakis et al. 2014 and Fassoulas 2018 for a review). Moreover, the
west and east parts of Crete function as “sinks'' for Balkan and Eastern
species respectively. The footprint of the Balkans and the Middle East in the
Cretan fauna is discussed in various studies (Vardinoyannis 1994; Trichas 1996; Chatzaki 2003; Trichas et al. 2020).
The “redness” of West and East Crete as endemic centers is also obtained in
other studies (e.g. Assing 2019; Kougioumoutzis et al. 2020). 

Islands are biodiversity sanctuaries (Whitaker and Fernández-Palacios 2007), and
so are mountains (Rhabek et al. 2019b). Our work advocates for approaches that
treat islands and mountains under a holistic perspective. The combination of
the two provides a complex biogeographical interplay governing the forces of
speciation, preservation and extinction of biodiversity (Steinbauer et al. 2016).
This synergistic effect of mountains-islands has also been recovered in other
areas such as the Balearic islands (Guardiola and Sáez 2023).

    \subsection{Species assessment}
    \label{subsec:arthropods-species-assessment-disc}

The species assessment from PACA showed that 75\% (Table 3) of the
taxa assessed are Potentially/Likely Threatened (hereafter referred as Threatened).
The variation between the different orders is not substantial (most of them
score above 50\% in Threatened taxa) (Table 3). Local endemic or restricted taxa
increase an order’s Threatened percentage. Chilopoda and Scorpiones, have zero
Threatened taxa. For Odonata we also recovered a low Threatened percentage
(33.3\% - 2 taxa) compared to the 100\% (2 taxa) of IUCN
(Supplementary Material 2, Figure 1). That is an artifact of the PACA
assessment, not taking into consideration population data and population
fragmentation. This factor, albeit an important aspect of criterion B for the
IUCN assessments, is excluded from PACA for it requires special treatment for
each taxon (Dauby et al. 2019). There are multiple reasons for the disagreement
between the two assessments. An IUCN assessment is an exhaustive, overall
assessment, performed by experts and focusing on each species separately.
A PACA assessment is a rather automated pipeline that allows researchers to
have a preliminary approach on understudied taxa and areas, but in no way an
alternative of a thorough IUCN assessment.

Arthropods with wider ranges that are not assessed as Threatened under
criterion B, are not necessarily Least Concern and should not be neglected.
Arthropod communities can be affected by the reduction of the abundance of
common and abundant species that offer important functions to the biocommunity.
Wide range does not guarantee high abundance (even though this is true for many
taxa) and even common species can be threatened (Habel and Schmitt 2018; Klink et al. 2023).

With 75\% Threatened taxa, Cretan Arthropods appear to be in better fate than
the Cretan vascular plants, assessed as Threatened in their totality
(Kougioumoutzis et al. 2020). This is most likely a result of the combined use
of Criteria A and B in the vascular plant assessment (Kougioumoutzis et al. 2020) -
something impossible for the Arthropods since their data are too coarse for the
utilization of criterion A. This dominant trend of Crete is also true for land
mollusks with 41.7\% of the Cretan endemics being Threatened (IUCN) compared to
the 20.5\% of Threatened endemics for Europe (Neubert et al. 2019). This is
particularly worrying given Crete’s significance as a biodiversity
hotspot (Myers et al. 2000; Médail, 2017) and the fact that it refers to single
island endemics. Cretan taxa display a worse trend not only compared to Europe (Supplementary Material 2, Figure 1),
but also when compared to Greece. For example, 46.1\% of the Greek endemic
vascular flora is recovered as Threatened according to Kougioumoutzis et al. (2021b),
compared to the 100\% of the Cretan endemic flora (Kougioumoutzis et al. 2020). 
    
    \subsection{KBAs}
    \label{subsec:arthropods-KBAs}
    
The candidate KBAs yielded by WEGE are gathered in Lefka Ori, Dikti, Thrypti,
Psiloritis, Selino and Zakros. WEGE analysis is stricter in evaluating
potential KBAs in the sense that Crete, having many endemic species, would be
qualified for KBA as a whole, triggering criteria A and B (IUCN 2016).
This is a weakness of KBAs highlighted by Farooq et al. (2023) that WEGE seems
to resolve (Farooq et al. 2020). Furthermore, the use of WEGE overcomes
obstacles in the ranking of areas for conservation such as the lack of robust
phylogenetic information regarding the taxa under focus (Farooq et al. 2020).

Our results are congruent with previous studies that enquire about EHs or
threat-spots in Crete (Dimitrakopoulos et al. 2004; Kougioumoutzis et al. 2020; Kougioumoutzis et al. 2021b).
The KBAs obtained here refer to Arthropods and are not mandatory in any way.
Other areas of Crete could be candidates as well. First of all, when it comes to
Arthropods, areas such as Gavdos islet are yielded as EHs with a different grid (Figure 3A).
Moreover, other areas may be important for other organisms. For example Asteroussia
are a KBA for Birds (Key Biodiversity Areas Partnership, 2024), while they are
also recovered as a potential climatic refuge for plants (Kougioumoutzis et al. 2020).
The essence is that KBAs should always be under inquiry grounded on the available
resources and will of the stakeholders and political authorities. From the
simple proposal of some KBAs to the implementation of a conservation plan there
are many steps to follow that do not all abide by quantifiable scientific
thresholds. Venter et al. (2018) found that KBAs have been selected in order to
avoid incorporating areas with agricultural activities, while there is a need
for mediation between national and global sites of conservation
interest (Kougioumoutzis et al. 2021b; Lim et al. 2023). In this international
and interdisciplinary questioning, the effective selection of candidate areas
is of great importance (Plumptre et al. 2024). Our work contributes to this
matter by highlighting the significance of island mountains as KBAs.

    \subsection{N2K overlap}
    \label{subsec:arthropods-N2K-overlap}

N2K has been characterized as the only protection structure that “has the
political chance to be implemented in the island” (Dimitrakopoulos et al. 2004).
The overlap of threatened taxa, EHs and KBAs with N2K is thus of major
conservation importance. 

Crete is by far the area of Greece with the highest mean complementary
percentage between threatened species distribution and N2K (Spiliopoulou et al. 2021).
Focusing on vascular endemic plants Kougioumoutzis et al. (2021b) also obtained
high complementarity between the endemicity/threat hotspots
(obtained with various indices) and the N2K. Our work contributes to this
discussion, exhibiting a high overlap between EHs and KBAs with N2K and
obtaining a satisfactory coverage of  EHs/KBAs by N2K (Table 2).
Additionally, N2K covers many areas of Crete (peninsulas, gorges, islets and
massifs) which, even though they are not yielded as EHs/KBAs, host a plethora
of endemic Arthropods.

We examined the overlap of each taxon’s AOO with N2K to obtain a more detailed
overview of its conservation status. The mean percentage of coverage was 52\%,
and increased to 55\% for the Threatened taxa. This percentage is close, albeit
lower, to the 62.3\% recovered from Spiliopoulou et al. (2021) for Crete. This
can be attributed to the innate differences of our datasets and methodologies.
We focused strictly on Arthropods, while Spiliopoulou et al. (2021) examined
all the species of Greece (flora and fauna) assessed in a Threatened category.
Moreover, we converted the PACA assessment to the respective IUCN category,
while Spiliopoulou et al. (2021) used the actual IUCN assessments. Despite
these methodological differences, another explanation could be that the
Arthropods are indeed in a worse conservation position than other groups, an
inference which rhymes with the ongoing global discussion around Arthropods’
decline (Chowdhury et al. 2022, 2023).

Orthoptera have the highest average overlap with N2K (62.38\%) (Figure 4B).
This is mainly caused by the genus Eupholidoptera which is responsible for a
great part of Orthopteran endemism in Crete (Willemse et al. 2023), which
differentiate areas mostly covered by N2K. Odonata and Trichoptera exhibit the
lowest average overlap (Figure 4B). A closer investigation towards the
freshwater species of Crete, especially those associated with seasonal streams
or ponds, is recommended. The overdrafting of Crete’s natural water reservoirs
and the aggressive urbanization and agricultural intensification could be a
hazard for smaller springs and streams. Kalkman et al. (2010) highlight the
need for a freshwater plan for the conservation of the Cretan dragonflies.
The ill fate of aquatic insects is a global phenomenon (Deacon et al. 2019; Roth et al. 2020; Dia-Silva et al. 2021),
although there are studies that recover more positive trends (Klink et al. 2020, but see also Desquilbet et al. 2020).

In our dataset, 29 (8.4\%) of the taxa have zero overlap with N2K. All of them
are Threatened. Additionally, 44 (17\%) of the Threatened taxa have less than
10\ overlap with N2K. The percentages (25.4\% in aggregate) of disagreement
obtained here are higher than those obtained from Spiliopoulou et al. (2021).
This becomes more acute since only nine Insect species out of 124 (7.2\%) that
were analyzed in Spiliopoulou et al. (2021) are excluded from the protected areas.

The inclusion of Arthropod taxa in protected areas is often insufficient, with
Arthropods experiencing declines inside the protected areas (Borges et al. 2005; Harry et al. 2019; Rada et al. 2019, Chowdhury et al. 2022).
In fact, even when certain Arthropod groups are adequately included in N2K,
there are gaps and omissions (Sánchez-Fernández et al. 2008; Verovnik et al. 2011).
At a global level 75\% of Insects are not sufficiently covered by protected
areas (Chowdhury et al. 2023). Crete stands in an intermediate position,
following the general trend of Greece’s N2K adequacy, being the best covered
area at a national level (Kougioumoutzis et al. 2021b; Spiliopoulou et al. 2021).
However, there are some clear gaps regarding certain taxa, encouraging more
locally focused conservation policies complementary to N2K. For example actions
need to be taken for KBAs that fall outside N2K like Kritsa and Zakros.

Biases towards Arthropods cause their poor coverage by protected
areas (D’Amen et al. 2013; Delso et al. 2021; Chowdhury et al. 2022). These
biases derive from geography, size, color and charisma (Cardoso 2012; Mammola et al. 2020; Wang et al. 2021),
and even from political/economic reasons (Dias-Silva et al. 2021). For example,
the strongest driver for a conservation program funding within the European
Union is the online popularity (Mammola et al. 2020). The unpopularity of
Arthropods has begun to change (Wagner et al. 2021), especially through citizen
science, which is a trend we should build on to properly conserve the Arthropods.

    \subsection{Human Intervention in Arthropods’ EHs}
    \label{subsec:arthropods-human-intervention-ehs}
Human activities account for almost 20\% of the EHs. The primary human activity
in the EHs is agriculture (~19.6\%). Agricultural intensification is one of the
most important drivers of Arthropods’ decline (Habel et al. 2019; Brühl and Zaller 2019; Raven and Wagner 2021).
Moreover, threats associated with agriculture are the number one threat for
Insect species inside protected areas in Europe (Chowdhury et al. 2022).
Nevertheless, regarding change in land use, there is a somewhat equal
transition trend from cropland to forest and vice versa inside EHs and KBAs
(Supplementary Material 2, Table 4). This means that while some sites are being
degraded others may recover. More research within EHs and KBAs is essential in
order to quantify the impact (negative or positive) of these transitions to the
endemic Arthropods. A vast amount of cropland has been transformed to pasture
lands (Supplementary Material 2, Table 4) which requires further examination,
since grazing has both positive [eg. on Gnaphosidae (Spiders) communities (Kaltsas et al.  2019)]
and negative effects [e.g. Carabidae (Coleoptera) (Kaltsas et al. 2013)].
The reduction of croplands could be interpreted under the general trend of
urbanization (Supplementary Material 2, Table 4), which nevertheless occurs
outside EHs and KBAs, but a shift towards montane areas especially under new
forms of tourism could deeply impact the sites of conservation importance.
    
    \subsection{Perspectives and Actions}
    \label{subsec:arthropods-perspectives-actions}

Arthropods are rarely approached as a whole, for biological and practical
reasons. The study of Arthropods is usually limited to a family or even to a
lower taxonomic level and to certain biogeographical areas (e.g. Borges et al. 2017, 2018).
Treatises tackling Arthropod issues in a wider scope are: Azores – Gaspar et al. (2010),
Atacama coast – Pizarro-Araya et al. (2021), Neotropical area - Barahona-Segovia and Zúñiga-Alonso (2021),
or meta-data studies (Klink et al. 2020; Chowdhury et al. 2023). In this study
we compiled a detailed and diverse dataset integrating different Arthropod groups.
Our goal was to obtain a holistic image of Crete’s Arthropods’ conservation
status and place it in the wider frame of the global issues of Arthropod conservation.

Crete follows the global pattern of island biodiversity, with the island biota
being under constant extinction pressure (Triantis et al. 2010; Fernández-Palacios et al. 2021).
All four main culprits for the impoverishing of island biota identified by
Fernández-Palacios et al. (2021) have an intense presence in Crete. The
lowlands of Crete are experiencing significant habitat loss due to urbanization
and transformation to olive tree cultivations. Natural resources are
overexploited - especially water reservoirs - mainly from agriculture and
aggressive touristic development. Invasive species have established populations
(D'Agata et al. 2009; Affre et al. 2010; Christopoulou et al. 2021) and the
impact of climate change is prominent. The aggregation of most of the endemic
Arthropods in the mountains renders them vulnerable not only due to their
insularity but adds extra pressure from mountain related processes. The lack of
space to retreat from climate change and their inability to outcompete with
lowland populations/species moving to higher elevations drives the extinction
of montane populations (Alexander et al. 2015; Steinbauer et al. 2018; Urban 2018; Yadav et al. 2018; Frishkoff et al. 2019).
Thus, the alloy of mountain-island can act not only as a driver for
biodiversity but also as the ground for its loss. Our work highlights the need
for a simultaneous evaluation of mountain and island driven phenomena inside
biodiversity hotspots, as is the Mediterranean basin.

For a better fate for the Cretan Arthropods under the global urgencies for
Arthropods’ conservation, we propose actions that could improve the
conservation status/framework of this special fauna:

1) The conservation situation inside the N2K should be examined to ensure the
correct implementation of the N2K goals and directives, especially given the
studies which have shown a significant decline of Arthropods inside protected
areas (Hallmann et al. 2017; Chwowdhury et al. 2022). This is also true for our
study, which demonstrates contradictory results regarding the human pressures
inside EHs and KBAs. Research efforts focused on the Arthropods species’
populations, abundances and communities will provide empirical data for the
interaction of human activities and the Arthropods of Crete. A multidisciplinary
study, utilizing molecular and geographical tools as well as the local
stakeholders in the spirit of Lehmann et al. (2021), would provide the much
needed research framework regarding the interaction dynamics of human activities
and Arthropods inside N2K in Crete.

2) The discussion for the expansion or optimization of already existing
protected areas like N2K, is imminent in the global bibliography (Chowdhury et al. 2023).
Ignoring important sites outside N2K would lead to neglect some threatened taxa,
and also encourage further human disturbance in unprotected areas (Borges et al. 2005).
Enquiries considering the incorporation of areas outside N2K to the network
and/or communication with local/centralized authorities and stakeholders to
form policies for the management of such areas could optimize the conservation
status of Cretan Arthropods. Despite the admittedly beneficial function of
protected areas in conservation targets such as the reducing of habitat loss (Geldmann et al. 2013),
the need of additional protection actions to tackle certain issues is highlighted (D’Amen et al. 2013; Hochkirch et al. 2013).
In essence, it is important for protected areas to be treated individually for
the achievement of different conservation goals instead of just complying with
a general protection trend. This issue is also brought up for the Greek reality (Dimitrakopoulos et al. 2004; Kougioumoutzis et al. 2020, 2021b).
Our study adds to this conversation, pointing to KBAs for Arthropods.

3) Educational/citizen science programs focused on the awareness of the local
communities towards the specificity and sensitivity of Cretan Arthropods could
build a social dynamic that would lighten Arthropods from the burden of
unpopularity (Wang et al. 2021). Given that Cretan Arthropods suffer from
biases related to their regional geographical position within the EU (Cardoso 2012),
the rise of awareness towards their threats and needs will improve their study
and conservation. Moreover, it would medicate the bias of Habitats Directive
towards central/northern European species and ground a more integrating
conservation approach within the EU.

4) PACA is utilized to map uncharted areas and biota that suffer from reduced
conservational focus. A thorough assessment of as many as possible of the
Cretan Arthropods under IUCN should be carried out and would provide a concrete
image of their threat status. This Herculean task is tackled in the upcoming
Red Data Book of Greece. Besides the value of a detailed IUCN assessment itself,
a Red Data Book will also provide a detailed dataset to test the effectiveness
and the limits of the PACA method, given its common use (e.g., Kougioumoutzis et al. 2021b; Iniesta et al. 2023).

5) Our study is a perfect example of the importance of contemporary research in
faunistics and taxonomy for conservation. Many core elements of our dataset
have been published only in the last five years (e.g., Assing 2019; Salata et al. 2020a).
In fact, 42.5\% (47 species) of the endemic Staphylinidae (Coleoptera) have
been described in 2019 (Assing et al. 2019). Knowledge shortfalls (Hortal et al. 2015)
regarding the Cretan Arthropods create an imperative need for basic taxonomic
and faunistic knowledge, i.e., the discovery of new taxa (tackling Linnean shortfall),
and for the better understanding of the species’ distributions (tackling Wallacean shortfall).
Thus, more funding should be focused on faunistic data assemblage studies.
In contrast to Garnett and Christidis (2017), we believe that taxonomy does not
hinder conservation biology, but instead makes conservation possible,
since when unaware of the existence of a species (whether species are
considered as real entities or not - see Raposo et al. 2017), it is impossible
to protect it. Therefore, trailing the voices of those who advocate for a
better incorporation of taxonomy in conservation (Dubois 2003; de Carvalho et al. 2007; Boero 2010; Andreone et al. 2022)
while acknowledging its innate value (Engel et al. 2021), we passionately call
for an extensive taxonomic and faunistic scrutiny of Cretan Arthropod biodiversity.

\section{Conclusions}
\label{sec:arthropods-conclusions}

The high percentage of potentially/likely Threatened taxa recovered (75\%),
points to immediate need for conservation actions and policies concerning Crete,
as well as a robust assessment of their threat status. These results are
worrying under the light of the “Insect Apocalypse”. Even though none of the
Cretan Arthropods was considered when the N2K was designed for Crete, N2K
appears to be an adequate conservation network for Cretan Arthropods. The EHs
and KBAs recovered here are the “usual suspects” also obtained in other studies
with different datasets. Lefka Ori, Psiloritis, Dikti, Thrypti, Selino and
Zakros are identified as EHs and KBAs for the Cretan Arthropods.  A point of
contradiction recovered here is the double role of an island-mountain system to
the birth and loss of biodiversity. Another contradiction is the one regarding
human activity and N2K coverage of the EHs/KBAs. Therefore, we suggest
multidisciplinary research efforts and policies that are not restricted to
scientific practice but welcome the participation of local communities to
achieve a better perspective for Cretan Arthropods.

\section*{Acknowledgements}
We would like to thank Katerina Vardinoyannis (Curator of Invertebrates - excluding Arthropods, in NHMC) and
Manolis Nikolakakis for their help in construction and design of the databases
used here. Moreover, we are also deeply obliged to professor Moysis Mylonas,
for his crucial theoretical remarks and advice and Leonidas Maroulis (PhD candidate - University of Cete)
for his insights regarding our methodology. Also, we would like to thank all
researchers and students of the NHMC and University of Crete who methodically
collected, sorted and identified Arthropod specimens during their studies in
various NHMC projects. Finally, we are deeply indebted to the comments of the
two anonymous reviewers, which have greatly improved our work.

\section*{Dedication}
We dedicate this work to Volker Assing (1956 - 2022). His research in Cretan
Staphylinidae has been remarkable in quantity and quality. Over the course of
numerous articles, he managed to exhibit and highlight the taxonomical and
biogeographical importance of Cretan rove beetles, by describing numerous new
species and unraveling interesting distributional patterns. His work is of most
significance for the conservation of this special fauna.

\section*{Data availability}
Data, scripts and results of the analysis are available and documented \href{https://github.com/savvas-paragkamian/arthropods_assessment_crete}{here}
Supplementary Material are available at the Zenodo repository \href{https://doi.org/10.5281/zenodo.10635645}{here}
Supplementary Material 1 is the Supplementary-material-1.xlsx file which
contains the occurrences as compiled from the literature and the specimens of the
NHMC. In addition all references of the literature are included in a separate sheet.
Supplementary Material 2 is the Supplementary-material-2.docx which
contains five (5) supplementary figures and four (4) supplementary tables.
Supplementary Material 3 is the
Supplementary-material-3-hilda-crete-1998-2018.mp4 timelapse video of yearly
changes of Land Use based on HILDA+ dataset 

\section*{Author contributions}
Conceptualization: AT, GB;
Data curation: AT, GB, LK, MC, NK;
Formal Analysis: SP; Methodology: GB, SP; Software: SP; Supervision: AT;
Validation: AT, GB, NK; Visualization: SP, AT; Writing – original draft: GB;
Writing – review - editing: All authors reviewed the manuscript.

\section*{Funding}

SP was funded by the 3rd H.F.R.I.
(Hellenic Foundation for Research and Innovation) Scholarships for
PHD Candidates (no. 5726). GB was funded by ELKE.uoc scholarship (no.11526).

