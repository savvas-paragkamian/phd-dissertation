\chapter{Curriculum Vitae}
\label{app:cv}

\section{Personal information}

\textbf{Savvas Paragkamian}

\begin{enumerate}

    \item[] s.paragkamian@hcmr.gr. +30 6934008102. Irakleio, Crete, Greece
    \item[] \underline{\href{https://github.com/savvas-paragkamian}{GitHub}}
    \item[] \underline{\href{https://orcid.org/0000-0002-8508-2521}{ORCID}}
    \item[] Birth: 27 January 1992
\end{enumerate}
  
\section{Education and training}

\begin{description}
    \item[December 2019 -- TBA] Phd Candidate on "Deciphering the relation of microbiome metabolic functions with environmental biogeochemical processes".
    University of Crete, Department of Biology. Institute of Marine Biology, Biotechnology and Aquaculture (IMBBC).
Doctoral Advisory Committee: \underline{\href{https://www.imbb.forth.gr/imbb-people/en/sarris-home}{Prof. P. F. Sarris}}, \underline{\href{http://lab42open.hcmr.gr}{Dr E. Pafilis}}, \underline{\href{https://cosynet.auth.gr/didaskontes-instructors/}{Prof. I. Antoniou}}
  
    \item[2015--2017] Master Program on Complex Systems and Networks. Grade 9.58 (Excellent).
    \underline{\href{https://doi.org/10.13140/RG.2.2.32339.63520}{Master's Thesis}}: Exploring the centrality - leathality rule in signed protein networks.
    Aristotle University, Thessaloniki (Greece), Supervisors: Prof. I. Antoniou, Prof. S. Sgardelis.

    \item[2010--2015] Undergraduate studies in Biology, Grade 7.30 (Very Good),
        \underline{\href{http://dx.doi.org/10.13140/RG.2.2.25628.74886}{Thesis}}:
        Menzerath-Altmann's law in human gene families at the gene - exon level and their evolutionary history.
        University of Crete, Irakleion (Greece), Supervisor: Prof. C. Nikolaou.
  
\end{description}

\section{Scholarship and Award}

\begin{description}
    \item[PhD Scholarship 2022 - 2024] 3rd call for Doctorate Scholarships (no. 5726). Hellenic Foundation for Research and Innovation and General Secretariat for Research and Innovation.

    \item[Dawn Field Award 2024]
        \underline{\href{https://genomicsstandardsconsortium.github.io/GSC24-Tucson/pages/keynotes/}{Dawn Field Award}} for Outstanding Contributions to Genomic Standards by the
        Genome Standards Consortium for my contributions to the genomics and island microbiome research communities.

\end{description}

\section{Teaching Experience}

\begin{description}
    \item[Jan 2020 -- Feb 2020] \underline{\href{https://www.geo-in.eu}{GEO-IN}}: Geotourism in insular geoparks. Workshop for the general public and entrepreneurs. Interreg V-A Cooperation Program Greece - Cyprus 2014-2020. 15 hours.

    \item[Nov 2018 -- Jan 2019] {\underline{\href{https://savvas-paragkamian.github.io/network_science_workshop/}{Network Science Workshop}}}. Bioinformatics MSc Program, University of Crete. 15 hours.
  
\end{description}

\section{Personal skills}

\begin{description}
    \item[Languages.] Greek (mother tongue), English (C2 level) and French (A2 level)
   
    \item[Programming languages.]
            \textbf{Proficient:} R, AWK. \textbf{Intermediate:} Python, Bash. \textbf{Basic:} Julia, Perl

    
    \item[Programming skills.]
        amplicon microbiome worklow (from sequences to ecology), complex network analysis, species hotspots and assessments,
        geospatial analysis, text mining, Application Programming Interface (API) calls, data enrichment and data integration,
        statistical and mathematical modelling

  
    \item[Databases - Ontologies.]
        \textbf{Biodiversity:} GBIF, OBIS, IUCN, MICROBEATLAS, EDAPHOBASE

        \textbf{Taxonomy:} GBIF Taxonomy, WORMS, Fauna Europaea, NCBI Taxonomy, GTDB, SILVA

        \textbf{Sequences:} ENA, MGnify, GenBank, Ensembl

        \textbf{Literature:} PubMed, Biodiversity Heritage Library, Google Scholar, Dimensions

        \textbf{Spatial - Remote Sensing:} Copernicus, NASA MODIS, ESDAC

        \textbf{Other:} JGI IMG/M, KEGG, STRING, FAPROTAX, BioGRID, EnvO, Gene Ontology
    
    \item[Environment - Software.]
        Unix, Conda, Tmux, R Studio, Tidyverse, Bioconductor, QGIS, r-spatial, Vim, Git, Markdown,
        Latex, High Performance Computing (slurm), Docker, Mac, Linux, Windows

    \item[Field Sampling.] 
        \textbf{Soil microbiome} sampling. Co-organised the Island Sampling Day Crete 2022 with 29 participants.
        Hellenic Canter for Marine Research

        \textbf{Cave microbiome} sampling from the deepest cave of Greece - Gourgouthakas Cave (depth -1100 meters). University of Crete

        \textbf{Arthropod} sampling from caves. Hellenic Institute of Speleological Research (INSPEE-HISR)

        \textbf{Raptor birds} catching from nests on cliffs using rope access techniques.
            Bonelli's eagle, Golden eagle and \textit{Gypaetus barbatus}. Host Institution: Natural History Museam of Crete
%        \item Dodecanese and Cycledes islands, Bonelli's eagle 2023 expendition.
%        Monitoring, capturing and rigging of Bonelli's eagle pullets with Tethys speedboat.
%        In total, 500 nautical miles in 8 Aegean islands capturing/rigging 14 pullets - Host Institute: Natural History Museam of Crete
    
    \item[Outdoor activities and photography.] Certified instructor (moniteur) in canyoning by the French Federation of Speleology. 
        Caving explorer and instructor. 
        Advanced Scuba Open Water Diver.
        Speedboat operator (Hellenic Coast Guard).
        Certified Unmanned aerial vehicles operator (A1-A2-A3).
        Assistant in equipment, production logistics of the acclaimed photographer 
      \underline{\href{https://heinztrollphotography.com}{Heinz Troll}} in 2018-2023.
  
\end{description}

\section{Research Experience}

\begin{description}

    \item[CCMRI] Climate Change Metagenomic Record Index (CCMRI): a dynamic researcher-notification and sample classification system
  
    Funding Source: 2nd call to Support Faculty Members and Researchers. Hellenic Foundation for Research and Innovation and General Secretariat for Research and Innovation
    Institute: IMBBC - HCMR
    Principal Investigator: Dr. E. Pafilis
    Role: proposal dissertation team member, 05- 2021
    

    \item[EMODnet Biology Phase III and IV] Text mining on marine taxa and traits (10-2021 to 03-2022). EU's maritime and fisheries fund
        \textbf{Host Institution:} IMBBC.
        \textbf{PI:} Ms D. Mavraki

    \item[PREGO]
        Development of \textcolor{teal}{\href{http://prego.hcmr.gr/}{PREGO}}, a literature and data integration platform to elucidate ecosystem functioning:
        associating organisms and environments with processes (01-2020 to 12-2021).
        Hellenic Foundation for Research and Innovation.
        \textbf{Host Institution:} IMBBC.
        \textbf{PI:} Dr E. Pafilis

    \item[CMBR] Centre for the study and sustainable exploitation of Marine Biological Resources (CMBR) is an integrative large-scale Greek Research Infrastructure (RI) of the National Roadmap for RI’s.

    Funding Source: Operational Programme "Competitiveness, Entrepreneurship and Innovation" (NSRF 2014-2020) and co-financed by Greece and the European Union (European Regional Development Fund).

    Institute: IMBBC - HCMR

    Coordinator: Dr. A. Magoulas

    Role: Project member from 01-2020 to 09-2021

    \item[Conservation of the Cave Fauna of Greece]
        Development of \textcolor{teal}{\href{https://database.inspee.gr/}{Cave Fauna of Greece (CFG) Database}} (07-2017 to 05-2019). 
        MAVA Foundation pour la nature, WWF Greece.
        \textbf{Host Institution:} INSPEE-HISR. 
        \textbf{PI:} K. Paragamian

    \item[EU Rural Development - LEADER] Co-organised 3 projects
        to document (description, map, audiovisual), improve rigging and engage local communities regarding the canyons of Crete.
        \textbf{Host Institution:} INSPEE-HISR.
        \textbf{PI:} K. Paragamian and S. Paragkamian
%    \begin{ecvitemize}
%        \item \textbf{Paragkamian, S.} and K. Paragamian. 2015. \underline{\href{https://www.inspee.gr/what-we-do/publications/ha-gorge-exploring-a-spectacular-hidden-landscape/}{Ha Gorge}}: Exploring a Spectacular Hidden Landscape. Irakleio, Crete, Greece. 64pp. ISBN: 978-618-82232-1-9
 %       \item Paragamian K., \textbf{S. Paragkamian} and I. Nikoloudakis. 2014. \underline{\href{https://www.inspee.gr/what-we-do/publications/canyons-a-valuable-hidden-world/}{Canyons}}: a valuable hidden world. Hellenic Institute of Speleological Research. Irakleio, Crete, Greece. 24pp.
%        \item Natural Park of Psiloreitis, Elements and views of the area. 2018. Exhibition created by \textbf{INSPEE}, funded by AKOMM S.A. (LEADER approach).
%    \end{ecvitemize}
\end{description}

\section{Presentations}
  \begin{description}
      \item[10th International Conference of MIKROBIOKOSMOS] \textbf{Savvas Paragkamian} \textit{et al.}.
From Micro to Macro, Defining Crete's Macroecology based on its soil Microbial Interactome.
Larisa, Greece, 12/2023
%\ecvitem{Presentation}{\textbf{Savvas Paragkamian} \textit{et al.,} Lynn Schriml, Johanna B Holm, Stephanie Yarwood, Melanthia Stavroulaki, Panagiotis F. Sarris, Georgios Kotoulas, Antonios Magoulas, Evangelos Pafilis. From Micro to Macro, Defining Crete's Macroecology based on its soil Microbial Interactome. 10th International Conference of MIKROBIOKOSMOS, Larisa, Greece, 30/11 - 2/12 2023}

\item[EMBO Satellite Workshop on Ecology Informatics] \textbf{Savvas Paragkamian}.
    Integrating data and methods: what can we learn about the soil ecosystems of Crete.
Irakleio, Greece, 5/2023

\item[9th International Conference of MIKROBIOKOSMOS] \textbf{Savvas Paragkamian} \textit{et al.}. 
PREGO (Process, Environment, Organism), mining literature and -omics (meta)data to associate microorganisms, biological processes, and environment types. 
Athens, Greece, 12/2021
%
%\ecvitem{Presentation}{\textbf{Savvas Paragkamian}, Haris Zafeiropoulos, Stylianos Ninidakis, Georgios A. Pavlopoulos, Lars Juhl Jensen, Evangelos Pafilis (2021). 
%PREGO (Process, Environment, Organism), mining literature and -omics (meta)data to associate microorganisms, biological processes, and environment types. 
%9th International Conference of MIKROBIOKOSMOS, Athens, Greece, 16-19 December 2021}

%    \ecvitem{Poster}{Haris Zafeiropoulos, \textbf{Savvas Paragkamian}, Christina Pavloudi, George Tsamis, Stylianos Ninidakis, Ioulia Santi,  Georgios A. Pavlopoulos, Lars Juhl Jensen, Evangelos Pafilis (2021). PREGO (Process, Environment, Organism), mining literature and -omics (meta)data to associate microorganisms, biological processes, and environment types. WORLD MICROBE FORUM, 20-14 JUNE 2021. POSTER ID:WMF21-0899}

\item[FEMS ONLINE CONFERENCE ON MICROBIOLOGY] Haris Zafeiropoulos, \textbf{Savvas Paragkamian}, Christina Pavloudi, George Tsamis, Stylianos Ninidakis, Ioulia Santi,  Georgios A. Pavlopoulos, Lars Juhl Jensen, Evangelos Pafilis (2020). Mining literature and -omics (meta)data to associate microorganisms, biological processes, and environment types. 28-31 October 2020.

\item[24th International Conference on Subterranean Biology] Paragamian K., \textbf{Paragkamian S.} (2018). Diversity and conservation of the cave fauna of Crete (Greece). 20-24th August 2018, University of Aveiro, Portugal. ARPHA Conference Abstracts 1: e29836. \url{https://doi.org/10.3897/aca.1.e29836}
  
    %\ecvitem{Poster}{ Paragamian K., Poulinakis M., \textbf{Paragkamian S.} and Nikoloudakis I. (2018). A comprehensive database for the cave fauna of Greece. 24th International Conference on Subterranean Biology, 20-24th August 2018, University of Aveiro, Portugal. ARPHA Conference Abstracts 1: e29843. \url{https://doi.org/10.3897/aca.1.e29843}}
  
    %\ecvitem{Poster}{ \textbf{Paragkamian S.}, Nikolaou, C., Sgardelis S. and Antoniou I. (2017). The centrality--lethality rule in signed protein interaction networks. In Hellenic Bioinformatics 10th Conference. 6--9 September 2017, Forth, Irakleio, Greece. DOI: \url{https://doi.org/10.13140/RG.2.2.32339.63520}}
  
    \item[Proceedings of the 29th Greek Statistical conference] Batziou, E., Gravanis, M., Karadimos, P., Koulikas, D., Mpatziakas, A., \textbf{Paragkamian S.}, Xanthopoulou, G. (2016). Comparison of correlation measures for deletion of brain network connectivity alterations during epileptiform discharges. pp. 245-259, 4-7 May 2016, ISBN: 978-618-80672-9-5. Naousa: Hellenic Statistical Institute

    %\ecvitem{Poster}{ \textbf{Paragkamian S.} and Nikolaou, C. (2015). Menzerath - Altmann's law in human gene families at the gene - exon level and their evolutionary history. In 10th Conference of the Hellenic Society for Computational Biology and Bioinformatics. 9-11 October 2015, Athens, Greece. DOI: \url{http://dx.doi.org/10.13140/RG.2.2.25628.74886}}
\item[Other] I have participated in other 2 presentations and presented 5 posters in international conferences and workshops.

\end{description}

\begin{refsection}
\nocite{*}
\printbibliography[keyword=own,title={My published works}]
\end{refsection}
%  \nocite{*}
